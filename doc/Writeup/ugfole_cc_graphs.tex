\section{Congruence closure graphs}
We define a congruence closure graph data structure that forms the basis of our verification algorithm. Each node in the graph represents an equivalence class of terms.
The data structure presented here is a variant on the common representation for congruence closure, dating back as far as \cite{DowneySethiTarjan} and earlier, but instantiated per CFG-node, and with edges added between instances to communicate equalities. We show our basic version here in order to fix the vocabulary and as a basis for the full version that we show in section \ref{section:ugfole:propagation}.

Our data structure represents a set of terms and a set of ground equalities.
The set of represented terms includes at least all the sub-terms that occur in the ground equalities.
The set of terms is also sub-term closed and also, for each term, the entire equivalence class (EC) of the term w.r.t. the congruence defined by the set equalities, is represented.

\subsection{Data Structure}
A basic EC-graph is composed of a set of nodes that represent equivalence classes of terms. We call each such node a \newdef{\GT{}} (ground term equivalence class). Each \GT{} is composed of a set of ground function application equivalence classes - \newdef{\GFAs} - where each \GFA{} is composed of a function and a tuple of \GTs{}. A \GFA{} represents an atomic equivalence class for the EC-graph.\\
The EC-graph includes a set of constants (the roots of the graph) and a map from each GT to to its super-terms - the GTs that contain a \GFA{} with a tuple that contains the GT. The basic data structure is summarized in figure \ref{fig_basic_ECGraph}.
The main difference between our representation and the more common representation is that we have an object (\GT{}) that represents an entire EC, while many congruence closure algorithms only have \GFAs{} and use equality edges between \GFAs{} to specify equality, selecting one \GFA{} to be a representative for each EC. 
The main reason we have an object per EC is that it makes our data structure independent of its construction order, 
and hence we can share more parts of an EC-graph between CFG-nodes and join operations are easier, in addition, performance is less sensitive to the order in which graph operations are performed.

\noindent
The set of terms represented by each node in the graph is defined recursively as follows:

\smallskip

\noindent
\m{\terms{GT(gfas)} \triangleq \bigcup\limits_{gfa \in gfas} \terms{gfa}}\\
\m{\terms{GFA(f,tt)} \triangleq \s{\fa{f}{t} \mid \bigwedge\limits_i t_i \in \terms{tt[i]}}}

\noindent
The \lstinline|ECGraph| object includes a map of constants (the roots of the terms) which maps a constant function symbol to the \GT{} in which it occurs, and a map from each \GT{} to its super-terms.\\
The set of \GTs{} in a graph is defined using the super-terms map, with the roots being the constants:

\smallskip

\noindent
\m{GTs(g) \triangleq \bigcup\limits_{c \in g.constants.values} GTs(g,c)}\\
\m{GTs(g,t) \triangleq \s{t} \cup \bigcup\limits_{s \in g.superTerms[t]} GTs(g,s)}

\bigskip

\begin{figure}
\begin{lstlisting}
class GT(gfas : Set[GFA])
	
class GFA(f : Function,tt : Tuple[GT])

class ECGraph
	method makeTerm(f:Function,tt:Tuple[GT]) : GT
	method assumeEqual(gt0,gt1:GT)

	constants:Map[Function,GT]
	superTerms:Map[GT,Set[GT]]
\end{lstlisting}
\caption{Basic EC graph data structure\\
\GTs{} and \GFAs{} are defined in Scala style to simplify the notation.\\
A \GT{} represents an EC of ground terms.\\
A \GFA{} represents an AEC (atomic EC) of ground terms.\\
}
\label{fig_basic_ECGraph}
\end{figure}

\subsubsection*{Invariant}
The invariant of the \lstinline|ECGraph| g is composed of several parts:\\
The constants map maps a constant function to its term:\\
\m{\forall t \in GTs(g), f \cdot GFA(f,()) \in t.gfas \Leftrightarrow g.constants[f]=t }

\noindent
The super-term map is closed:\\
\m{\ran{g.constants} \subseteq \dom{g.superTerms}}\\
\m{\forall t \in \dom{g.superTerms} \cdot g.superTerms[t] \subseteq \dom{g.superTerms}}

\noindent
The graph is congruence and transitive closed:\\
\m{\forall s,t \in GTs(g) \cdot \terms{s} \cap \terms{t} = \emptyset}


\subsubsection*{Operations}
The basic EC graph supports two operations:

\bigskip

\noindent
\lstinline|makeTerm(f:Function,tt:Tuple[GT]) : GT| - this operation receives a tuple of graph nodes (\m{tt \subseteq GTs(g)}) and a function with the appropriate arity and returns the GT in the graph that includes the \GFA{} \m{GFA(f,tt)}.
If there is no such GT, a singleton GT - \m{GT(SET(GFA(f,tt)))} - is added to graph, updating the data structure accordingly.

\bigskip

\noindent
\lstinline|assumeEqual(gt0,gt1:GT)| - this operation merges the \GTs{} gt0 and gt1 and performs congruence closure until the ECGraph invariant holds. Note that the classic E-graph used by SMT solvers uses a union-find data structure (as in e.g. \cite{NelsonOppenUnionFind}) which is generally more efficient, however, in our context our version has some advantages, as described below.

\bigskip

\noindent
While the operation of \lstinline|makeTerm| is rather straightforward for the basic ECGraph, \lstinline|assumeEqual| requires some more attention, as it forms the basis of all graph updates, hence we give a pseudo-code in figures \ref{fig_basic_ECGraph_assumeEqual},\ref{fig_basic_ECGraph_mergeOne}.

\begin{figure}
\begin{lstlisting}
method assumeEqual(gt0,gt1:GT)
	requires gt0,gt1 $\in$ GTs(this)
	
	if (gt0=gt1)
		return;
	
	var mergeQ := new Queue[GT]
	var mergeMap := new Map[GT,GT]
	enqueueMerge(gt0,gt1)
	
	while (!mergeQ.isEmpty)
		mergeOne(mergeQ.dequeue)
		
method enqueueMerge(gt0,gt1:GT)
	var sgt = transitiveMerge(gt0)
	var tgt = transitiveMerge(gt1)
	if (sgt!=tgt)
		mergeMap[sgt]:=tgt
		
method transitiveMerge(gt)
	while gt$\in$mergeMap.keys
		gt := mergeMap[gt]
\end{lstlisting}
\caption{Basic EC graph congruence closure code\\
The algorithm maintains a queue \lstinline|mergeQ| of \GTs{} that need to be merged,\\
and a map \lstinline|mergeMap| that encodes which \GT{} is merged to which \GT{}, 
it is maintained essentially as in the union-find algorithm.\\
The \lstinline|transitiveMerge| method traverses the map and returns the last element of the chain that starts at its argument.\\
The \lstinline|mergeOne| method merges two \GTs{}, updates the relevant maps and enqueues any equalities implied by congruence closure - it is detailed in figure \ref{fig_basic_ECGraph_mergeOne}.
}
\label{fig_basic_ECGraph_assumeEqual}
\end{figure}
\begin{figure}
\begin{lstlisting}
method mergeOne(gt:GT)
	var target := transitiveMerge(gt)
	
	//merge GFAs
	target.gfas.add(gt.gfas)
	foreach (gfa in gt.gfas)
		if (gfa.tt==())
			constants[gfa.f] := target
		else foreach (sgt in gfa.tt)
			superTerms[sgt].remove(gt)
			superTerms[sgt].add(target)
			
	//update super terms
	foreach (sgt $\in$ superTerms[gt])
		foreach (gfa $\in$ sgt.gfas)
			if (gt $\in$ gfa.tt)
				sgt.gfas.remove(gfa)
				var newGFA := gfa[gt$\mapsto$target]
				if ($\exists$gt2$\in$superTerms[target]$\cdot$newGFA$\in$gt2.gfas)
					enqueueMerge(target,gt2)
				else
					superTerms[target].add(sgt)
					sgt.gfas.add(gfa)
\end{lstlisting}
\caption{Basic EC graph single merge step\\
The method \lstinline|mergeOne| merges one pair of \GTs{} and updates the maps and super-terms accordingly.\\
If a super-term update causes the super-term's EC to overlap another EC, they are enqueued for a merge.\\
After the \lstinline|mergeOne| method, \lstinline|gt| is effectively removed from the graph.\\
The syntax \lstinline|gfa[gt$\mapsto$target]| means replacing each occurrence if \lstinline|gt| with \lstinline|target|.
}
\label{fig_basic_ECGraph_mergeOne}
\end{figure}

\noindent
The main operation of the \lstinline|assumeEqual| method maintains a map (\lstinline|mergeMap|) of merge edges which is maintained as an acyclic directed graph with out-degree one. 
Each \GT{} is merged with the \GT{} at the end of its chain in \lstinline|mergeMap|.
When merging two \GTs{}, the sets of \GFAs{} of the \GTs{} are merged, and all the \GFAs{} of the super-terms of the source \GT{} are updated to point to the target \GT{}. If any such updated \GFA{} already occurs in the graph in another \GT{}, we enqueue a merge of the other \GT{} with the target.

\subsubsection*{Properties}
The main property of the basic EC-Graph is that each term added to it (using \lstinline|makeTerm|) is represented in it, and that it represents a congruence that is exactly the congruence defined by the set of equalities added to it (using \lstinline|assumeEqual|).

\noindent
\textbf{Equations represented by the graph:}\\
There is not necessarily a unique minimal set of equations that represents the congruence represented in the graph - for any congruence relation and any set of equalities that define the congruence, adding the equalities to the graph in any order will result in the same graph - for example, adding \m{a=b} and then \m{f(b)=f(c)} will give the exact same graph as adding \m{f(a)=f(c)} and then {a=b}.\\
In order to define a canonical set of equations represented in the graph, we assume an arbitrary total order on terms, where we only require that deeper terms are larger. Using this order we can define the set of equations of a graph by defining a representative for each \GT{}:\\
\m{rep(GT(s)) \triangleq min_{u \in s} rep(u)}\\
\m{rep(GFA(f,tt)) \triangleq f(rep(\tup{tt}))}\\
\m{rep(\tup{t})_i \triangleq rep(t_i)}\\
We also define the representative \GFA{} of a \GT{} by using the order on \GFA{} representatives:\\
For a pair of  \GFAs{} \fa{f}{t},\fa{g}{s}, \m{\fa{f}{t}<\fa{g}{s}} iff $\rep{\fa{f}{t}}<\rep{\fa{g}{s}}$.\\
Using this ordering we define:\\
\m{repgfa(GT(s)) \triangleq min_{u \in s} u}\\
Because deeper terms are larger, the representatives are well defined.\\
The canonical set of equations is defined first for one GT{}:\\
\m{\eqs{u} \triangleq \s{s=t \mid t = rep(u) \land \exists v \in u.gfas \cdot u\neq v \land s=rep(v)}}\\
And for the EC-graph:\\
\m{\eqs{g} \triangleq \bigcup\limits_{u \in g} \eqs{u}}


\subsubsection*{Lazy representation for a congruence}
As we propagate equality information lazily in the CFG, we use EC-Graphs to represent an approximation of a congruence - the idea is that, for each pair of terms represented in the graph, the terms are equal in the congruence iff they are represented by the same \GT{}, while for a term not represented in the graph, there may be more equalities in the congruence we are approximating - for example:\\
In the graph constructed from the following sequence of operations:\\
\lstinline|a:=makeTerm(a,()); b:=makeTerm(b,());|\\
\lstinline|c:=makeTerm(c,()); assumeEqual(a,b)|\\
We know that \m{a=b,f(a)=f(b)} hold in the congruence we are approximating and we also know that \m{a=c,b=c} do not hold, 
but we do not know whether \m{f(a)=f(c)} or \m{g(b)=g(c)} or \m{a=f(a)} hold.

\bigskip
\noindent
Given a set of ground equalities and a pair of terms, the EC-graph can be used to decide whether the set of equalities implies equality between the two terms. This is done using the following auxiliary methods:\\
This method \lstinline|makeTerm(t:Term) : GT| simply constructs the term t in the graph bottom up, adding necessary \GTs{}:
\begin{lstlisting}
makeTerm($\fa{f}{s}$:Term) : GT
	return makeTerm(f,$[i\mapsto$makeTerm(s$_i$)])
\end{lstlisting}
The expression $[i\mapsto$makeTerm(s$_i$)] denotes constructing a tuple where the i-th element is the result of the call \lstinline|makeTerm(s$_i$)| (the order of the calls is unimportant in our data structure).\\
The method \lstinline|assume(s=t : Equality)| is used to assume equalities on terms rather than on \GTs{} - defined as:
\begin{lstlisting}
assume(s=t:Equality)
	assumeEqual(makeTerm(s),makeTerm(t))
\end{lstlisting}

\bigskip
\noindent
\textbf{Eager use of axioms:}
A simple way to check whether a set of axioms entails an equality \m{s=t} is to use \lstinline|s0:=makeTerm(s);t0:=makeTerm(t)| to add the two terms to an empty graph, and then use
\lstinline|assume(u=v)| for each equality \m{u=v} in the set of axioms, and return \lstinline|makeTerm(s)==makeTerm(t)|.\\
Note that the result of makeTerm(s) might change after an \lstinline|assumeEqual| - if we keetp the \lstinline|mergeMap| in the data structure rather than just for each \lstinline|merge| call, we can avoid the second calls for \lstinline|makeTerm(s),makeTerm(t)| by returning \lstinline|transitiveMerge(s0)==transitiveMerge(t0)|.

\bigskip
\noindent
\textbf{Lazy use of axioms:}
We can avoid adding all the axioms by only adding axioms where the equated terms are represented in the graph - the algorithm is shown in figure \ref{fig_lazy_congruence}.\\
The algorithm repeatedly adds axioms to the graph for all terms represented in the graph, until none are left.\\
Note that \lstinline|g.assume(l=r)| might add represented terms to the graph even if \m{l,r} are already represented in the graph - for example, if \m{\terms{g} = \s{a,b,f(a)}} and we call
\lstinline|g.assume(a=b)|, in the resulting graph \m{g'} \m{\terms{g'} = \s{a,b,f(a),f(b)}}.

\textbf{Correctness:}
We argue here the algorithm above is correct - that is, that the algorithm returns true iff \m{axioms \models s=t}.\\
We use the notation \m{[t]_g} for the EC-node (\GT{}) that represents the term \m{t} in the EC-graph g, if \m{t \in \terms{g}}.\\
We extend this notation to tuples - e.g. \m{[\tup{s}]_g} is the tuple of \GTs{} where the i-th element is \m{[s_i]_g} and also to \GFAs{} - \m{[f([\tup{s}]_g)]} is the \GT{} that contains the \GFA{} \m{f([\tup{s}]_g)]} if there is such a \GFA{} in g.\\
We assume the EC-graph is implemented correctly, so that the graph represents the congruence defined by the set S and the set of terms represented at the graph are all the terms that were added to the graph and their ECs under the congruence in the graph. 
Specifically, we assume that if the set S of equalities was \lstinline|assumed| in g, and for the terms s,t, \m{s,t \in \terms{g}}, then \m{[s]_g=[t]_g}.

The argument for soundness - that is, \true{} is returned only if \m{axioms \models s=t} - comes directly from the soundness of the EC-graph operations - we only add to the graph axioms from \lstinline|axioms|.

The argument for completeness - that is, if \m{axioms \models s=t} then \true{} is returned - can be shown by strong induction on the depth of a derivation tree of \m{s=t} from \lstinline|axioms| in the calculus \m{\mathbf{CC_I}}.\\
As \m{\mathbf{CC_I}} is complete (it includes all instances of axioms that define a congruence relation, as per the definition),
if \m{axioms \models s=t} then \m{axioms \vdash_{\mathbf{CC_I}} s=t}.

\noindent
We proceed by strong induction on the derivation tree in \m{\mathbf{CC_I}} of \m{s=t} from \m{axioms}.\\
The induction hypothesis is that for a derivation depth k, if \m{s' \in \terms{g}} and \m{s'=t'} is derivable from \m{axioms} in \m{\mathbf{CC_I}} in a derivation of at most k steps, then \m{t' \in \terms{g}} and \m{[s']_g = [t']_g}.\\
We perform case analysis on the root of the derivation tree:\\
\textbf{Axioms:} If \m{s \in \terms{g}} then the assignment to \lstinline|ns| adds the axiom \m{s=t} to \lstinline|ns| and hence to \lstinline|S|, hence \lstinline|S|\m{\models s=t} and so \m{[s]_g=[t]_g}.\\
\textbf{Reflexivity:} Immediate - \m{[s]_g=[s]_g}.\\
\textbf{Transitivity:} By definition, there is some u s.t. \m{axioms \models s=u,u=t} and their derivation is of depth at-most k-1. By i.h. \m{u \in \terms{g}} and hence again by i.h. \m{[s]_g=[u]_g=[t]_g}\\
\textbf{Congruence Closure:} If \m{s=\fa{f}{u},t=\fa{f}{v}}, then for each i we have a derivation of \m{u_i=v_i} of depth at most k-1.
Because the EC-graph is sub-term closed, for each i, \m{u_i \in \terms{g}} and hence by i.h. also \m{v_i \in \terms{g}} and \m{[u_i]_g=[v_i]_g}. 
As the EC-graph is congruence closed and contains, for a represented terms, all terms in its EC under the congruence defined by the graph, we get that \m{\fa{f}{v} \in \terms{[\fa{f}{u}]_g} \subseteq \terms{g}} and hence \m{[\fa{f}{v}]_g = [\fa{f}{u}]_g}\\
\QED




\begin{figure}
\begin{lstlisting}
checkEntailment(s,t : Term,axioms : Set[Clause]) : Boolean
	var S := new Set[Clause]()
	var g := new EC graph
	g.makeTerm(s)
	g.makeTerm(t)
	do 
		variant: |axioms \ S|
		invariant $\m{\eqs{g} \Leftrightarrow S}$
		
		var ns: = $\s{l=r \in axioms \setminus S \mid l \in \terms{g}}$
		foreach $\m{l=r \in ns}$//Axiom closure
			g.assume(l=r)
	until ns=$\emptyset$
	return g.makeTerm(s)==g.makeTerm(t)
\end{lstlisting}
\caption{Graph based lazy entailment checking\\
The algorithm simply selects all equations on terms the graph represents and \lstinline|assume|s them in the graph, until no such axioms remain.\\
We discuss later how to select the new axioms to add (build \lstinline|ns|) efficiently.
}
\label{fig_lazy_congruence}
\end{figure}





\subsection{Notation}
We use a graphical notation for presenting some examples with EC-graphs.\\
The notation diverges somewhat from the presentation of the algorithm as we sometimes represent tuples of equivalence classes in the notation - this is done to improve clarity.
A legend for the notation is given in figure \ref{fig_ec_graph_legend}.

\begin{figure}
%\framebox[0.5\textwidth]{
\begin{tikzpicture}
	
  \node[gttn]  (1)               {$()$};
  \node[gtn]   (2)  [above =1cm of 1] {\m{a,b}};
%  \node[rgttn] (3)  [above =1cm of 2] {\m{\left(\svb{a}{b}\right)}};
%  \node[rgtn]  (4)  [above =1cm of 3] {\m{f\left(\svb{a}{b}\right)}};

	\node[gl,align=left] (1l) [below = 0cm of 1] {
		Graph for \term{a=b}: \\ 
		\terms{g}  = \s{a,b} \\ 
		\gfasA{g} = \s{a(),b()} \\ 
%		\gtts{g}   = \s{()} \\
%		\rgtts{g}  = \s{(\s{a,b})} \\
%		\rgfas{g}  = \s{f(\s{a,b})}\
	};
				
	\draw[gfa] (2) to[bend right] node[el,anchor=east] {\m{a}} (1);
	\draw[gfa] (2) to[bend left]  node[el,anchor=west] {\m{b}} (1);

%	\draw[rgtt] (3) to node[rl] {\m{0}} (2);
%	\draw[rgfa] (4) to node[rl] {\m{f}} (3);

	\node (legendAnchor) [right = 3cm of 2] {};
%	\node (legendAnchorl) [right = 2cm of legendAnchor] {};
	\node[gtn,  below = 0.2 of legendAnchor] (gtn)   {x,y};
	\node[right = 0.95cm of gtn] (gtnl) {A \GT{} node - an EC of terms};
	\node[gttn, below = 0.2 of gtn]          (gttn)  {(x)};
	\node[right = 1.05cm of gttn] (gttnl) {A tuple-EC node - a tuple of \GTs{}};
%	\node[rgttn,below = 0.2 of gttn ]        (rgttn) {\scriptsize an \rgtt node};
%	\node[rgtn, below = 0.2 of gtn  ]        (rgtn)  {\scriptsize an \rgfa};

	\node (gfal) [below left  = 0.2 and 0 of gttn] {};
	\node (gfar) [right      = of gfal] {A \GFA{} edge};
	\draw[gfa] (gfal) to  node[el,anchor=south] {\m{f}} (gfar);

%	\node (rgfal) [below      = 0.2 of gfal] {};
%	\node (rgfar) [right      = of rgfal] {An \rgtt edge};
%	\draw[rgfa] (rgfal) to  node[rl,anchor=south] {\m{f}} (rgfar);

	\node (sgttl) [below    = 0.2  of gfal] {};
	\node (sgttr) [right    = of sgttl] {A tuple \term{i}-th member edge};
	\draw[sgtt] (sgttl) to node[el,anchor=south] {\term{i}} (sgttr);
%	\node (rgttl) [below    = 0.2  of sgttl] {};
%	\node (rgttr) [right    =      of rgttl] {An \rgtt \term{i}-th member edge};
%	\draw[rgtt] (rgttl) to node[rl,anchor=south] {\term{i}} (rgttr);

\end{tikzpicture}
%}
\caption{Equivalence class graph notation\\
\GTs{} are represented using circles and \GT{} tuples are represented using rectangles\\
Inside the circles and rectangles we enumerate some of the members (terms) of each EC for clarity,
but as some ECs are infinite, this list is not always complete.\\
Single arrows represent function application edges while indexed double arrows 
represent membership in the i-th position of a tuple.\\
We use the empty tuple as the base of each graph.
}
\label{fig_ec_graph_legend}
\end{figure}

\noindent 
In figures \ref{ec_graph_example_binary_tuple} we show some graphs with binary tuples.
For an EC-graph g, and a term \m{t \in \terms{g}}, we use \m{[t]_g} for the \GT{} (EC-node) that represents t in g - it is unique by the graph invariant. When an EC-node represents more than one term, we sometimes list more than one term for emphasis - for example if, in a graph g, \m{a=b}, then the EC for a can be written as \m{[a]_g,[b]_g,[a,b]_g} all of which are equivalent. When the graph is clear from the context we drop the g subscript. When we have numbered CFG-nodes, e.g. \m{p_0,p_1}, we sometimes use e.g. \m{[a]_0,[b]_1} to denote the \GT{} representing a in \m{g_{p_0}}, b in \m{g_{p_1}}, resp, and also \m{g_0} for \m{g_{p_0}}.\\
The graph in \ref{ec_graph_example_binary_tuple1} is constructed by\\
\lstinline|makeTerm(a,());makeTerm(b,());makeTerm(f,([a],[b]))| or \\
\lstinline|makeTerm(b,());makeTerm(a,());makeTerm(f,([a],[b]))| \\
the results are identical.\\
\ref{ec_graph_example_binary_tuple2} is obtained from \ref{ec_graph_example_binary_tuple1} by \lstinline|makeTerm(f,([b],[a]))|.
we can see how \GT{} nodes can be shared - the nodes \m{[a],[b]} participate in two tuples. \\
\ref{ec_graph_example_binary_tuple3} is obtained from either \ref{ec_graph_example_binary_tuple1} or \ref{ec_graph_example_binary_tuple2} by \lstinline|assumeEqual([a],[b])| - the results are identical.
We see one \GT{} - [a,b] - which occurs twice in a tuple - hence the tuple-EC contains four tuples. 
\ref{ec_graph_example_binary_tuple4} is obtained from \ref{ec_graph_example_binary_tuple2} by \lstinline|assumeEqual([f(a)],[f(b)])| - it cannot be obtained from \ref{ec_graph_example_binary_tuple3} as all of our operations are monotonic in the set of represented equalities.

\begin{figure}
\centering
\begin{subfigure}[t]{0.48\textwidth}
	\framebox[\textwidth]{
	\raisebox{0pt}[0.4\textheight][0pt]
	{
		\begin{tikzpicture}
			\node(O) at (0,0){};
			\node[gttn] (1)  at (2,2.5)                      {$()$};
			\node[gtn,label=center:\small{a}]  (2)  [above left  =0.8cm and 1cm of 1] {\phantom{b}};
			\node[gtn,label=center:\small{b}]  (3)  [above right =0.8cm and 1cm of 1] {\phantom{b}};
			\node[gttn] (4)  [above =2.5cm of 1]               {\m{(a,b)}};
			\node[gtn]  (5)  [above =1cm of 4]               {\m{f(a,b)}};

			\node[gl,align=left] (1l) [below = 0cm of 1] {
				Graph for the clause set $\emptyset$: \\ \terms{n} = \s{a,b,f(a,b)} \\ \gfasA{n} = \s{a(),b(),f([a,b])}
			};
						
			\draw[gfa] (2) to node[el,anchor=east] {\term{a}} (1);
			\draw[gfa] (3) to node[el,anchor=west] {\term{b}} (1);
			\draw[gfa] (5) to node[el] {\term{f}} (4);

			\draw[sgtt] (4) to[out=270,in=90] node[el,anchor=south] {\term{0}} (2);
			\draw[sgtt] (4) to[out=270,in=90] node[el,anchor=south] {\term{1}} (3);

%		\caption{The result of \\
%		\lstinline|makeTerm(a,())|\\
%		\lstinline|makeTerm(b,())|\\
%		\lstinline|makeTerm(f,([a],[b]))|}
		\end{tikzpicture}
	}}
	\caption{}
	\label{ec_graph_example_binary_tuple1}
\end{subfigure}
\begin{subfigure}[t]{0.48\textwidth}
	\framebox[\textwidth]{
	\raisebox{0pt}[0.4\textheight][0pt]
	{
		\begin{tikzpicture}
			\node(O) at (0,0){};
			\node[gttn] (1)   at (2,2.5)                     {$()$};
			\node[gtn,label=center:\small{a}]  (2)  [above left =0.8cm and 0.5cm of 1]  {\phantom{b}};
			\node[gtn,label=center:\small{b}]  (3)  [above right =0.8cm and 0.5cm of 1] {\phantom{b}};
			\node[gttn] (4)  [above =1.2cm of 2]               {\m{(a,b)}};
			\node[gttn] (5)  [above =1.2cm of 3]               {\m{(b,a)}};
			\node[gtn]  (6)  [above =1cm of 4]               {\m{f(a,b)}};
			\node[gtn]  (7)  [above =1cm of 5]               {\m{f(b,a)}};

			\node[gl,align=left] (1l) [below = 0cm of 1] {
				Graph for \s{}: \\
				\terms{n} = \s{a,b,f(a,b),f(b,a)} \\
				\gfasA{n} = \s{a(), b(), f([a],[b]),f([b],[a])}
			};
						
			\draw[gfa] (2) to node[el,anchor=east] {\term{a}} (1);
			\draw[gfa] (3) to node[el,anchor=west] {\term{b}} (1);
			\draw[gfa] (6) to node[el] {\term{f}} (4);
			\draw[gfa] (7) to node[el] {\term{f}} (5);

			\draw[sgtt] (4) to node[el,anchor=east] {\term{0}} (2);
			\draw[sgtt] (4.270) to[out=270,in=90] node[el,pos=0.3,anchor=south] {\term{1}} (3);

			\draw[sgtt] (5.270) to[out=270,in=90] node[el,pos=0.3,anchor=south] {\term{1}} (2);
			\draw[sgtt] (5) to node[el,anchor=west] {\term{0}} (3);

		\end{tikzpicture}
	}}
	\caption{}
	\label{ec_graph_example_binary_tuple2}
\end{subfigure}
\begin{subfigure}[t]{0.48\textwidth}
	\framebox[\textwidth]{
	\raisebox{0pt}[0.4\textheight][0pt]
	{
		\begin{tikzpicture}
			\node(O) at (0,0){};
			
			\node[gttn] (1)  at (2,2.5)                        {$()$};
			\node[gtn]  (2)  [above =1cm of 1]               {a,b};
			\node[gttn] (3)  [above =1cm of 2]               {\stackB{\m{(a,a),(a,b)}}{\m{(b,a),(b,b)}}};
			\node[gtn]  (4)  [above =4cm of 1]               {\stackB{\m{f(a,a),f(a,b)}}{\m{f(b,a),f(b,b)}}};

			\node[gl,align=left] (1l) [below = 0cm of 1] {
				Graph for \s{a=b}: \\
				$\begin{array}{ll}
				\terms{n} & =
					\left\{
						\begin{array}{l}
							\term{a,b, f(a,a),f(a,b),} \\
							\term{f(b,a),f(b,b)}
						\end{array}
					\right\} \\
				\gfasA{n} & = \s{a(),b(),f([a],[a])}
				\end{array}$
			};
						
			\draw[gfa] (2) to[bend right] node[el] {\term{a}} (1);
			\draw[gfa] (2) to[bend left] node[el,anchor=west] {\term{b}} (1);
			\draw[gfa] (4) to node[el] {\term{f}} (3);

			\draw[sgtt](3) to[bend right] node[el] {\term{0}} (2);
			\draw[sgtt](3) to[bend left] node[el,anchor=west] {\term{1}} (2);

		\end{tikzpicture}
	}}
	\caption{}
	\label{ec_graph_example_binary_tuple3}
\end{subfigure}
\begin{subfigure}[t]{0.48\textwidth}
	\framebox[\textwidth]{
	\raisebox{0pt}[0.4\textheight][0pt]
	{
		\begin{tikzpicture}
			\node(O) at (0,0){};
			
			\node[gttn] (1)  at (2,2.5)                        {$()$};
			\node[gtn,label=center:\small{a}]  (2)  [above left =0.8cm and 0.6cm of 1]  {\phantom{b}};
			\node[gtn,label=center:\small{b}]  (3)  [above right =0.8cm and 0.6cm of 1] {\phantom{b}};
			\node[gttn] (4)  [above =1.5cm of 2]               {\m{(a,b)}};
			\node[gttn] (5)  [above =1.5cm of 3]               {\m{(b,a)}};
			\node[gtn]  (6)  [above =4cm of 1]               {\m{f(a,b),f(b,a)}};

			\node[gl,align=left] (1l) [below = 0cm of 1] {
				Graph for \s{f(a,b)=f(b,a)}: \\
				\terms{n} = \s{a,b,f(a,b),f(b,a)} \\
				\gfasA{n} = \s{a(),b(),f([a],[b]),f([b],[a])}
			};
						
			\draw[gfa] (2) to node[el] {\term{a}} (1);
			\draw[gfa] (3) to node[el,anchor=west] {\term{b}} (1);
			\draw[gfa] (6) to node[el] {\term{f}} (4);
			\draw[gfa] (6) to node[el,anchor=west] {\term{f}} (5);

			\draw[sgtt] (4) to node[el,anchor=east] {\term{0}} (2);
			\draw[sgtt] (4) to[out=270,in=90] node[el,pos=0.3,anchor=south] {\term{1}} (3);

			\draw[sgtt] (5) to[out=270,in=90] node[el,pos=0.3,anchor=south] {\term{1}} (2);
			\draw[sgtt] (5) to node[el,anchor=west] {\term{0}} (3);

		\end{tikzpicture}
	}}
	\caption{}
	\label{ec_graph_example_binary_tuple4}
\end{subfigure}
\caption{Binary tuples}
\label{ec_graph_example_binary_tuple}
\end{figure}

\noindent
In figures \ref{ec_graph_example_cyclic} we show some graphs with cycles - note that some ECs are infinite.\\
The graph in \ref{ec_graph_example_cyclic1} is obtained by:\\
\lstinline|makeTerm(a,()); makeTerm(f,([a])); assumeEqual([a],[f(a)])|\\
The graph in \ref{ec_graph_example_cyclic2} is obtained by:\\
\lstinline|makeTerm(a,()); makeTerm(b,())|\\
\lstinline|makeTerm(f,([a],[b])); makeTerm(g,([a],[b]))|\\
\lstinline|assumeEqual([f(a,b)],[a]); assumeEqual([g(a,b)],b)|\\
Note that we can share the representation of tuple-ECs ( we use shared tuple-ECs in our implementation).\\
The graph in \ref{ec_graph_example_cyclic3} is obtained by:\\
\lstinline|makeTerm(a,()); makeTerm(b,())|\\
\lstinline|makeTerm(f,([a],[b])); makeTerm(g,([b],[a]))|\\
\lstinline|assumeEqual([f(a,b)],[a]); assumeEqual([g(b,a)],b)|

\begin{figure}[H]
\centering
\begin{subfigure}[t]{0.43\textwidth}
	\framebox[\textwidth]{
	\raisebox{0pt}[0.4\textheight][0pt]
	{
		\begin{tikzpicture}
			\node(O) at (0,0){};
			\node[gttn] (1)  at (2,2.5)                      {$()$};
			\node[gtn]  (2)  [above =1cm of 1]               {\stackC{a}{f(a)}{...}};
			\node[gttn] (3)  [above =1cm of 2]               {\stackB{(a)}{...}};

			\node[gl,align=left] (1l) [below = 0cm of 1] {
				Graph for\s{\term{a=f(a)}}: \\
				\terms{n} = \s{a,f^n(a)} \\
				\gfasA{n} = \s{a(), f([a])}
			};
						
			\draw[gfa] (2) to                            node[el] {\term{a}} (1);
			\draw[gfa] (2.270) to[out=240,in=120,looseness=2] node[el] {\term{f}} (3.90);

			\draw[sgtt](3) to node[el] {\term{0}} (2);

		\end{tikzpicture}
	}}
	\caption{}
	\label{ec_graph_example_cyclic1}
\end{subfigure}
\begin{subfigure}[t]{0.53\textwidth}
	\framebox[\textwidth]{
	\raisebox{0pt}[0.4\textheight][0pt]
	{
		\begin{tikzpicture}
			\node(O) at (0,0){};
			\node[gttn] (1)  at (2,2.5)                       {$()$};
			\node[gtn]  (2)  [above left =1cm and 0.2cm of 1]   {\stackB{a}{...}};
			\node[gtn]  (3)  [above right=1cm and 0.2cm of 1]   {\stackB{b}{...}};
			\node[gttn] (4)  [above      =2.9cm       of 1]   {\stackB{(a,b)}{...}};

			\node[gl,align=left] (1l) [below = 0cm of 1] {
				~Graph for\s{a=f(a,b),b=g(a,b)}: \\
				$\begin{array}{lll}
				\terms{n} & = &
					\left\{
						\begin{array}{l}
							\term{a,b, f(a,b),} \\
							\term{g(a,b),f(f(a,b),a),} \\
							\term{g(f(a,b),b),...} \\
						\end{array}
					\right\} \\
				\gfasA{n} & = & \m{\{a(),b(),f([a],[b]),}\\
				          &   & \m{g([a],[b])\}}
				\end{array}$				
			};
						
			\draw[gfa] (2.270) to[out=270,in=90] node[el] {\term{a}} (1);
			\draw[gfa] (3.270) to[out=270,in=90] node[el,anchor=west] {\term{b}} (1);
			\draw[gfa] (2.270) to[out=225,in=110,looseness=3] node[el] {\term{f}} (4.90);
			\draw[gfa] (3.270) to[out=315,in=70 ,looseness=3] node[el,anchor=west] {\term{g}} (4.90);

			\draw[sgtt] (4) to[out=270,in=90] node[el,anchor=south] {\term{0}} (2);
			\draw[sgtt] (4) to[out=270,in=90] node[el,anchor=south] {\term{1}} (3);

		\end{tikzpicture}
	}}
	\caption{}
	\label{ec_graph_example_cyclic2}
\end{subfigure}
\begin{subfigure}[t]{0.55\textwidth}
	\framebox[\textwidth]{
	\raisebox{0pt}[0.4\textheight][0pt]
	{
		\begin{tikzpicture}
			\node(O) at (0,0){};
			\node[gttn] (1)  at (2,2.5)                       {$()$};
			\node[gtn]  (2)  [above left =1cm and 0.2cm of 1]   {\stackC{a}{f(a,b)}{...}};
			\node[gtn]  (3)  [above right=1cm and 0.2cm of 1]   {\stackC{b}{g(b,a)}{...}};
			\node[gttn] (4)  [above      =1.3cm         of 2]   {$\stackB{(a,b)}{...}$};
			\node[gttn] (5)  [above      =1.3cm         of 3]   {$\stackB{(b,a)}{...}$};

			\draw[gfa] (2)     to[out=270,in=90] node[el] {\term{a}} (1);
			\draw[gfa] (3)     to[out=270,in=90] node[el,anchor=west] {\term{b}} (1);
			\draw[gfa] (2.270) to[out=225,in=135,looseness=1.5] node[el,anchor=east] {\term{f}} (4.90);
			\draw[gfa] (3.270) to[out=315,in=45, looseness=1.5] node[el,anchor=west] {\term{g}} (5.90);

			\draw[sgtt] (4) to node[el,anchor=east] {\term{0}} (2);
			\draw[sgtt] (4) to[out=270,in=90] node[el,pos=0.3,anchor=south] {\term{1}} (3);
			\draw[sgtt] (5) to[out=270,in=90] node[el,pos=0.3,anchor=south] {\term{1}} (2);
			\draw[sgtt] (5) to node[el,anchor=west] {\term{0}} (3);

			\node[gl,align=left] (1l) [below = 0cm of 1] {
				~Graph for\s{a=f(a,b),b=g(b,a)}: \\
				$\begin{array}{lll}
				\terms{n} & = &
					\left\{
						\begin{array}{l}
							\term{a,b, f(a,b),} \\
							\term{g(a,b),f(f(a,b),a),} \\
							\term{g(f(a,b),b),...} \\
						\end{array}
					\right\} \\
				\gfasA{n} & = & \m{\{a(),b(),f([a],[b]),}\\
				          &   & \m{g([b],[a])\}}
				\end{array}$				
			};
						
		\end{tikzpicture}
	}}
	\caption{}
	\label{ec_graph_example_cyclic3}
\end{subfigure}
\caption{Cyclic graphs}
\label{ec_graph_example_cyclic}
\end{figure}
