\showboxdepth=\maxdimen
\showboxbreadth=\maxdimen

\chapter{Unit ground first order logic with equality}\label{chapter:ugfole}
In this chapter we present an incremental, on-demand algorithm and data-structure for propagating equality information and verifying programs in the fragment of unit ground first order logic with equality - GFOLE. 
The algorithm is not intended as a stand alone verifier, but rather as a basis for the verification algorithms for stronger fragments.
The algorithm maintains a data-structure at each CFG-node that encodes some equality information and is updated on-demand with information from predecessors.
Other fragments can query the data structure on whether a ground equality holds at a given CFG-node, and can also add a derived equalities to the data-structure at a given CFG-node. The algorithm ensures that enough equality information is propagated to answer the queries that are provable in the fragment, including performing joins of congruence relations at join nodes, and the data-structure is updated accordingly. Equality propagation is done incrementally.
When another fragment derives an equality at some CFG-node, the algorithm ensures that this equality is propagated wherever it may affect the result of a previously queried equality, under some limitations.

Our data-structure at each CFG-node is similar to an E-graphs but ensures that the represented congruence relation is fully reduced.
In addition to a term-graph, the data-structure includes a relation between the graph of a CFG-node and those of its predecessors, 
used to communicate equality information.

We start by presenting our graph data structure. 
We continue by discussing the limitations of using EC-graphs (or sets of equalities) to represent post-conditions in the CFG.
Next we present the problem of lazy information propagation and present our algorithm for equality propagation for CFGs without joins.
In the next section we present our join algorithm as an extension to the join-less algorithm.
Lastly we discuss related work.

\section{Congruence closure graphs}
We define a congruence closure graph data structure that forms the basis of our verification algorithm. Each node in the graph represents an equivalence class of terms.
The data structure presented here is a variant on the common representation for congruence closure, dating back as far as \cite{DowneySethiTarjan} and earlier, but instantiated per CFG-node, and with edges added between instances to communicate equalities. We show our basic version here in order to fix the vocabulary and as a basis for the full version that we show in section \ref{section:ugfole:propagation}.

Our data structure represents a set of terms and a set of ground equalities.
The set of represented terms includes at least all the sub-terms that occur in the ground equalities.
The set of terms is also sub-term closed and also, for each term, the entire equivalence class (EC) of the term w.r.t. the congruence defined by the set equalities, is represented.

\subsection{Data Structure}
A basic EC-graph is composed of a set of nodes that represent equivalence classes of terms. We call each such node a \newdef{\GT{}} (ground term equivalence class). Each \GT{} is composed of a set of ground function application equivalence classes - \newdef{\GFAs} - where each \GFA{} is composed of a function and a tuple of \GTs{}. A \GFA{} represents an atomic equivalence class for the EC-graph.\\
The EC-graph includes a set of constants (the roots of the graph) and a map from each GT to to its super-terms - the GTs that contain a \GFA{} with a tuple that contains the GT. The basic data structure is summarized in figure \ref{fig_basic_ECGraph}.
The main difference between our representation and the more common representation is that we have an object (\GT{}) that represents an entire EC, while many congruence closure algorithms only have \GFAs{} and use equality edges between \GFAs{} to specify equality, selecting one \GFA{} to be a representative for each EC. 
The main reason we have an object per EC is that it makes our data structure independent of its construction order, 
and hence we can share more parts of an EC-graph between CFG-nodes and join operations are easier, in addition, performance is less sensitive to the order in which graph operations are performed.

\noindent
The set of terms represented by each node in the graph is defined recursively as follows:

\smallskip

\noindent
\m{\terms{GT(gfas)} \triangleq \bigcup\limits_{gfa \in gfas} \terms{gfa}}\\
\m{\terms{GFA(f,tt)} \triangleq \s{\fa{f}{t} \mid \bigwedge\limits_i t_i \in \terms{tt[i]}}}

\noindent
The \lstinline|ECGraph| object includes a map of constants (the roots of the terms) which maps a constant function symbol to the \GT{} in which it occurs, and a map from each \GT{} to its super-terms.\\
The set of \GTs{} in a graph is defined using the super-terms map, with the roots being the constants:

\smallskip

\noindent
\m{GTs(g) \triangleq \bigcup\limits_{c \in g.constants.values} GTs(g,c)}\\
\m{GTs(g,t) \triangleq \s{t} \cup \bigcup\limits_{s \in g.superTerms[t]} GTs(g,s)}

\bigskip

\begin{figure}
\begin{lstlisting}
class GT(gfas : Set[GFA])
	
class GFA(f : Function,tt : Tuple[GT])

class ECGraph
	method makeTerm(f:Function,tt:Tuple[GT]) : GT
	method assumeEqual(gt0,gt1:GT)

	constants:Map[Function,GT]
	superTerms:Map[GT,Set[GT]]
\end{lstlisting}
\caption{Basic EC graph data structure\\
\GTs{} and \GFAs{} are defined in Scala style to simplify the notation.\\
A \GT{} represents an EC of ground terms.\\
A \GFA{} represents an AEC (atomic EC) of ground terms.\\
}
\label{fig_basic_ECGraph}
\end{figure}

\subsubsection*{Invariant}
The invariant of the \lstinline|ECGraph| g is composed of several parts:\\
The constants map maps a constant function to its term:\\
\m{\forall t \in GTs(g), f \cdot GFA(f,()) \in t.gfas \Leftrightarrow g.constants[f]=t }

\noindent
The super-term map is closed:\\
\m{\ran{g.constants} \subseteq \dom{g.superTerms}}\\
\m{\forall t \in \dom{g.superTerms} \cdot g.superTerms[t] \subseteq \dom{g.superTerms}}

\noindent
The graph is congruence and transitive closed:\\
\m{\forall s,t \in GTs(g) \cdot \terms{s} \cap \terms{t} = \emptyset}


\subsubsection*{Operations}
The basic EC graph supports two operations:

\bigskip

\noindent
\lstinline|makeTerm(f:Function,tt:Tuple[GT]) : GT| - this operation receives a tuple of graph nodes (\m{tt \subseteq GTs(g)}) and a function with the appropriate arity and returns the GT in the graph that includes the \GFA{} \m{GFA(f,tt)}.
If there is no such GT, a singleton GT - \m{GT(SET(GFA(f,tt)))} - is added to graph, updating the data structure accordingly.

\bigskip

\noindent
\lstinline|assumeEqual(gt0,gt1:GT)| - this operation merges the \GTs{} gt0 and gt1 and performs congruence closure until the ECGraph invariant holds. Note that the classic E-graph used by SMT solvers uses a union-find data structure (as in e.g. \cite{NelsonOppenUnionFind}) which is generally more efficient, however, in our context our version has some advantages, as described below.

\bigskip

\noindent
While the operation of \lstinline|makeTerm| is rather straightforward for the basic ECGraph, \lstinline|assumeEqual| requires some more attention, as it forms the basis of all graph updates, hence we give a pseudo-code in figures \ref{fig_basic_ECGraph_assumeEqual},\ref{fig_basic_ECGraph_mergeOne}.

\begin{figure}
\begin{lstlisting}
method assumeEqual(gt0,gt1:GT)
	requires gt0,gt1 $\in$ GTs(this)
	
	if (gt0=gt1)
		return;
	
	var mergeQ := new Queue[GT]
	var mergeMap := new Map[GT,GT]
	enqueueMerge(gt0,gt1)
	
	while (!mergeQ.isEmpty)
		mergeOne(mergeQ.dequeue)
		
method enqueueMerge(gt0,gt1:GT)
	var sgt = transitiveMerge(gt0)
	var tgt = transitiveMerge(gt1)
	if (sgt!=tgt)
		mergeMap[sgt]:=tgt
		
method transitiveMerge(gt)
	while gt$\in$mergeMap.keys
		gt := mergeMap[gt]
\end{lstlisting}
\caption{Basic EC graph congruence closure code\\
The algorithm maintains a queue \lstinline|mergeQ| of \GTs{} that need to be merged,\\
and a map \lstinline|mergeMap| that encodes which \GT{} is merged to which \GT{}, 
it is maintained essentially as in the union-find algorithm.\\
The \lstinline|transitiveMerge| method traverses the map and returns the last element of the chain that starts at its argument.\\
The \lstinline|mergeOne| method merges two \GTs{}, updates the relevant maps and enqueues any equalities implied by congruence closure - it is detailed in figure \ref{fig_basic_ECGraph_mergeOne}.
}
\label{fig_basic_ECGraph_assumeEqual}
\end{figure}
\begin{figure}
\begin{lstlisting}
method mergeOne(gt:GT)
	var target := transitiveMerge(gt)
	
	//merge GFAs
	target.gfas.add(gt.gfas)
	foreach (gfa in gt.gfas)
		if (gfa.tt==())
			constants[gfa.f] := target
		else foreach (sgt in gfa.tt)
			superTerms[sgt].remove(gt)
			superTerms[sgt].add(target)
			
	//update super terms
	foreach (sgt $\in$ superTerms[gt])
		foreach (gfa $\in$ sgt.gfas)
			if (gt $\in$ gfa.tt)
				sgt.gfas.remove(gfa)
				var newGFA := gfa[gt$\mapsto$target]
				if ($\exists$gt2$\in$superTerms[target]$\cdot$newGFA$\in$gt2.gfas)
					enqueueMerge(target,gt2)
				else
					superTerms[target].add(sgt)
					sgt.gfas.add(gfa)
\end{lstlisting}
\caption{Basic EC graph single merge step\\
The method \lstinline|mergeOne| merges one pair of \GTs{} and updates the maps and super-terms accordingly.\\
If a super-term update causes the super-term's EC to overlap another EC, they are enqueued for a merge.\\
After the \lstinline|mergeOne| method, \lstinline|gt| is effectively removed from the graph.\\
The syntax \lstinline|gfa[gt$\mapsto$target]| means replacing each occurrence if \lstinline|gt| with \lstinline|target|.
}
\label{fig_basic_ECGraph_mergeOne}
\end{figure}

\noindent
The main operation of the \lstinline|assumeEqual| method maintains a map (\lstinline|mergeMap|) of merge edges which is maintained as an acyclic directed graph with out-degree one. 
Each \GT{} is merged with the \GT{} at the end of its chain in \lstinline|mergeMap|.
When merging two \GTs{}, the sets of \GFAs{} of the \GTs{} are merged, and all the \GFAs{} of the super-terms of the source \GT{} are updated to point to the target \GT{}. If any such updated \GFA{} already occurs in the graph in another \GT{}, we enqueue a merge of the other \GT{} with the target.

\subsubsection*{Properties}
The main property of the basic EC-Graph is that each term added to it (using \lstinline|makeTerm|) is represented in it, and that it represents a congruence that is exactly the congruence defined by the set of equalities added to it (using \lstinline|assumeEqual|).

\noindent
\textbf{Equations represented by the graph:}\\
There is not necessarily a unique minimal set of equations that represents the congruence represented in the graph - for any congruence relation and any set of equalities that define the congruence, adding the equalities to the graph in any order will result in the same graph - for example, adding \m{a=b} and then \m{f(b)=f(c)} will give the exact same graph as adding \m{f(a)=f(c)} and then {a=b}.\\
In order to define a canonical set of equations represented in the graph, we assume an arbitrary total order on terms, where we only require that deeper terms are larger. Using this order we can define the set of equations of a graph by defining a representative for each \GT{}:\\
\m{rep(GT(s)) \triangleq min_{u \in s} rep(u)}\\
\m{rep(GFA(f,tt)) \triangleq f(rep(\tup{tt}))}\\
\m{rep(\tup{t})_i \triangleq rep(t_i)}\\
We also define the representative \GFA{} of a \GT{} by using the order on \GFA{} representatives:\\
For a pair of  \GFAs{} \fa{f}{t},\fa{g}{s}, \m{\fa{f}{t}<\fa{g}{s}} iff $\rep{\fa{f}{t}}<\rep{\fa{g}{s}}$.\\
Using this ordering we define:\\
\m{repgfa(GT(s)) \triangleq min_{u \in s} u}\\
Because deeper terms are larger, the representatives are well defined.\\
The canonical set of equations is defined first for one GT{}:\\
\m{\eqs{u} \triangleq \s{s=t \mid t = rep(u) \land \exists v \in u.gfas \cdot u\neq v \land s=rep(v)}}\\
And for the EC-graph:\\
\m{\eqs{g} \triangleq \bigcup\limits_{u \in g} \eqs{u}}


\subsubsection*{Lazy representation for a congruence}
As we propagate equality information lazily in the CFG, we use EC-Graphs to represent an approximation of a congruence - the idea is that, for each pair of terms represented in the graph, the terms are equal in the congruence iff they are represented by the same \GT{}, while for a term not represented in the graph, there may be more equalities in the congruence we are approximating - for example:\\
In the graph constructed from the following sequence of operations:\\
\lstinline|a:=makeTerm(a,()); b:=makeTerm(b,());|\\
\lstinline|c:=makeTerm(c,()); assumeEqual(a,b)|\\
We know that \m{a=b,f(a)=f(b)} hold in the congruence we are approximating and we also know that \m{a=c,b=c} do not hold, 
but we do not know whether \m{f(a)=f(c)} or \m{g(b)=g(c)} or \m{a=f(a)} hold.

\bigskip
\noindent
Given a set of ground equalities and a pair of terms, the EC-graph can be used to decide whether the set of equalities implies equality between the two terms. This is done using the following auxiliary methods:\\
This method \lstinline|makeTerm(t:Term) : GT| simply constructs the term t in the graph bottom up, adding necessary \GTs{}:
\begin{lstlisting}
makeTerm($\fa{f}{s}$:Term) : GT
	return makeTerm(f,$[i\mapsto$makeTerm(s$_i$)])
\end{lstlisting}
The expression $[i\mapsto$makeTerm(s$_i$)] denotes constructing a tuple where the i-th element is the result of the call \lstinline|makeTerm(s$_i$)| (the order of the calls is unimportant in our data structure).\\
The method \lstinline|assume(s=t : Equality)| is used to assume equalities on terms rather than on \GTs{} - defined as:
\begin{lstlisting}
assume(s=t:Equality)
	assumeEqual(makeTerm(s),makeTerm(t))
\end{lstlisting}

\bigskip
\noindent
\textbf{Eager use of axioms:}
A simple way to check whether a set of axioms entails an equality \m{s=t} is to use \lstinline|s0:=makeTerm(s);t0:=makeTerm(t)| to add the two terms to an empty graph, and then use
\lstinline|assume(u=v)| for each equality \m{u=v} in the set of axioms, and return \lstinline|makeTerm(s)==makeTerm(t)|.\\
Note that the result of makeTerm(s) might change after an \lstinline|assumeEqual| - if we keetp the \lstinline|mergeMap| in the data structure rather than just for each \lstinline|merge| call, we can avoid the second calls for \lstinline|makeTerm(s),makeTerm(t)| by returning \lstinline|transitiveMerge(s0)==transitiveMerge(t0)|.

\bigskip
\noindent
\textbf{Lazy use of axioms:}
We can avoid adding all the axioms by only adding axioms where the equated terms are represented in the graph - the algorithm is shown in figure \ref{fig_lazy_congruence}.\\
The algorithm repeatedly adds axioms to the graph for all terms represented in the graph, until none are left.\\
Note that \lstinline|g.assume(l=r)| might add represented terms to the graph even if \m{l,r} are already represented in the graph - for example, if \m{\terms{g} = \s{a,b,f(a)}} and we call
\lstinline|g.assume(a=b)|, in the resulting graph \m{g'} \m{\terms{g'} = \s{a,b,f(a),f(b)}}.

\textbf{Correctness:}
We argue here the algorithm above is correct - that is, that the algorithm returns true iff \m{axioms \models s=t}.\\
We use the notation \m{[t]_g} for the EC-node (\GT{}) that represents the term \m{t} in the EC-graph g, if \m{t \in \terms{g}}.\\
We extend this notation to tuples - e.g. \m{[\tup{s}]_g} is the tuple of \GTs{} where the i-th element is \m{[s_i]_g} and also to \GFAs{} - \m{[f([\tup{s}]_g)]} is the \GT{} that contains the \GFA{} \m{f([\tup{s}]_g)]} if there is such a \GFA{} in g.\\
We assume the EC-graph is implemented correctly, so that the graph represents the congruence defined by the set S and the set of terms represented at the graph are all the terms that were added to the graph and their ECs under the congruence in the graph. 
Specifically, we assume that if the set S of equalities was \lstinline|assumed| in g, and for the terms s,t, \m{s,t \in \terms{g}}, then \m{[s]_g=[t]_g}.

The argument for soundness - that is, \true{} is returned only if \m{axioms \models s=t} - comes directly from the soundness of the EC-graph operations - we only add to the graph axioms from \lstinline|axioms|.

The argument for completeness - that is, if \m{axioms \models s=t} then \true{} is returned - can be shown by strong induction on the depth of a derivation tree of \m{s=t} from \lstinline|axioms| in the calculus \m{\mathbf{CC_I}}.\\
As \m{\mathbf{CC_I}} is complete (it includes all instances of axioms that define a congruence relation, as per the definition),
if \m{axioms \models s=t} then \m{axioms \vdash_{\mathbf{CC_I}} s=t}.

\noindent
We proceed by strong induction on the derivation tree in \m{\mathbf{CC_I}} of \m{s=t} from \m{axioms}.\\
The induction hypothesis is that for a derivation depth k, if \m{s' \in \terms{g}} and \m{s'=t'} is derivable from \m{axioms} in \m{\mathbf{CC_I}} in a derivation of at most k steps, then \m{t' \in \terms{g}} and \m{[s']_g = [t']_g}.\\
We perform case analysis on the root of the derivation tree:\\
\textbf{Axioms:} If \m{s \in \terms{g}} then the assignment to \lstinline|ns| adds the axiom \m{s=t} to \lstinline|ns| and hence to \lstinline|S|, hence \lstinline|S|\m{\models s=t} and so \m{[s]_g=[t]_g}.\\
\textbf{Reflexivity:} Immediate - \m{[s]_g=[s]_g}.\\
\textbf{Transitivity:} By definition, there is some u s.t. \m{axioms \models s=u,u=t} and their derivation is of depth at-most k-1. By i.h. \m{u \in \terms{g}} and hence again by i.h. \m{[s]_g=[u]_g=[t]_g}\\
\textbf{Congruence Closure:} If \m{s=\fa{f}{u},t=\fa{f}{v}}, then for each i we have a derivation of \m{u_i=v_i} of depth at most k-1.
Because the EC-graph is sub-term closed, for each i, \m{u_i \in \terms{g}} and hence by i.h. also \m{v_i \in \terms{g}} and \m{[u_i]_g=[v_i]_g}. 
As the EC-graph is congruence closed and contains, for a represented terms, all terms in its EC under the congruence defined by the graph, we get that \m{\fa{f}{v} \in \terms{[\fa{f}{u}]_g} \subseteq \terms{g}} and hence \m{[\fa{f}{v}]_g = [\fa{f}{u}]_g}\\
\QED




\begin{figure}
\begin{lstlisting}
checkEntailment(s,t : Term,axioms : Set[Clause]) : Boolean
	var S := new Set[Clause]()
	var g := new EC graph
	g.makeTerm(s)
	g.makeTerm(t)
	do 
		variant: |axioms \ S|
		invariant $\m{\eqs{g} \Leftrightarrow S}$
		
		var ns: = $\s{l=r \in axioms \setminus S \mid l \in \terms{g}}$
		foreach $\m{l=r \in ns}$//Axiom closure
			g.assume(l=r)
	until ns=$\emptyset$
	return g.makeTerm(s)==g.makeTerm(t)
\end{lstlisting}
\caption{Graph based lazy entailment checking\\
The algorithm simply selects all equations on terms the graph represents and \lstinline|assume|s them in the graph, until no such axioms remain.\\
We discuss later how to select the new axioms to add (build \lstinline|ns|) efficiently.
}
\label{fig_lazy_congruence}
\end{figure}





\subsection{Notation}
We use a graphical notation for presenting some examples with EC-graphs.\\
The notation diverges somewhat from the presentation of the algorithm as we sometimes represent tuples of equivalence classes in the notation - this is done to improve clarity.
A legend for the notation is given in figure \ref{fig_ec_graph_legend}.

\begin{figure}
%\framebox[0.5\textwidth]{
\begin{tikzpicture}
	
  \node[gttn]  (1)               {$()$};
  \node[gtn]   (2)  [above =1cm of 1] {\m{a,b}};
%  \node[rgttn] (3)  [above =1cm of 2] {\m{\left(\svb{a}{b}\right)}};
%  \node[rgtn]  (4)  [above =1cm of 3] {\m{f\left(\svb{a}{b}\right)}};

	\node[gl,align=left] (1l) [below = 0cm of 1] {
		Graph for \term{a=b}: \\ 
		\terms{g}  = \s{a,b} \\ 
		\gfasA{g} = \s{a(),b()} \\ 
%		\gtts{g}   = \s{()} \\
%		\rgtts{g}  = \s{(\s{a,b})} \\
%		\rgfas{g}  = \s{f(\s{a,b})}\
	};
				
	\draw[gfa] (2) to[bend right] node[el,anchor=east] {\m{a}} (1);
	\draw[gfa] (2) to[bend left]  node[el,anchor=west] {\m{b}} (1);

%	\draw[rgtt] (3) to node[rl] {\m{0}} (2);
%	\draw[rgfa] (4) to node[rl] {\m{f}} (3);

	\node (legendAnchor) [right = 3cm of 2] {};
%	\node (legendAnchorl) [right = 2cm of legendAnchor] {};
	\node[gtn,  below = 0.2 of legendAnchor] (gtn)   {x,y};
	\node[right = 0.95cm of gtn] (gtnl) {A \GT{} node - an EC of terms};
	\node[gttn, below = 0.2 of gtn]          (gttn)  {(x)};
	\node[right = 1.05cm of gttn] (gttnl) {A tuple-EC node - a tuple of \GTs{}};
%	\node[rgttn,below = 0.2 of gttn ]        (rgttn) {\scriptsize an \rgtt node};
%	\node[rgtn, below = 0.2 of gtn  ]        (rgtn)  {\scriptsize an \rgfa};

	\node (gfal) [below left  = 0.2 and 0 of gttn] {};
	\node (gfar) [right      = of gfal] {A \GFA{} edge};
	\draw[gfa] (gfal) to  node[el,anchor=south] {\m{f}} (gfar);

%	\node (rgfal) [below      = 0.2 of gfal] {};
%	\node (rgfar) [right      = of rgfal] {An \rgtt edge};
%	\draw[rgfa] (rgfal) to  node[rl,anchor=south] {\m{f}} (rgfar);

	\node (sgttl) [below    = 0.2  of gfal] {};
	\node (sgttr) [right    = of sgttl] {A tuple \term{i}-th member edge};
	\draw[sgtt] (sgttl) to node[el,anchor=south] {\term{i}} (sgttr);
%	\node (rgttl) [below    = 0.2  of sgttl] {};
%	\node (rgttr) [right    =      of rgttl] {An \rgtt \term{i}-th member edge};
%	\draw[rgtt] (rgttl) to node[rl,anchor=south] {\term{i}} (rgttr);

\end{tikzpicture}
%}
\caption{Equivalence class graph notation\\
\GTs{} are represented using circles and \GT{} tuples are represented using rectangles\\
Inside the circles and rectangles we enumerate some of the members (terms) of each EC for clarity,
but as some ECs are infinite, this list is not always complete.\\
Single arrows represent function application edges while indexed double arrows 
represent membership in the i-th position of a tuple.\\
We use the empty tuple as the base of each graph.
}
\label{fig_ec_graph_legend}
\end{figure}

\noindent 
In figures \ref{ec_graph_example_binary_tuple} we show some graphs with binary tuples.
For an EC-graph g, and a term \m{t \in \terms{g}}, we use \m{[t]_g} for the \GT{} (EC-node) that represents t in g - it is unique by the graph invariant. When an EC-node represents more than one term, we sometimes list more than one term for emphasis - for example if, in a graph g, \m{a=b}, then the EC for a can be written as \m{[a]_g,[b]_g,[a,b]_g} all of which are equivalent. When the graph is clear from the context we drop the g subscript. When we have numbered CFG-nodes, e.g. \m{p_0,p_1}, we sometimes use e.g. \m{[a]_0,[b]_1} to denote the \GT{} representing a in \m{g_{p_0}}, b in \m{g_{p_1}}, resp, and also \m{g_0} for \m{g_{p_0}}.\\
The graph in \ref{ec_graph_example_binary_tuple1} is constructed by\\
\lstinline|makeTerm(a,());makeTerm(b,());makeTerm(f,([a],[b]))| or \\
\lstinline|makeTerm(b,());makeTerm(a,());makeTerm(f,([a],[b]))| \\
the results are identical.\\
\ref{ec_graph_example_binary_tuple2} is obtained from \ref{ec_graph_example_binary_tuple1} by \lstinline|makeTerm(f,([b],[a]))|.
we can see how \GT{} nodes can be shared - the nodes \m{[a],[b]} participate in two tuples. \\
\ref{ec_graph_example_binary_tuple3} is obtained from either \ref{ec_graph_example_binary_tuple1} or \ref{ec_graph_example_binary_tuple2} by \lstinline|assumeEqual([a],[b])| - the results are identical.
We see one \GT{} - [a,b] - which occurs twice in a tuple - hence the tuple-EC contains four tuples. 
\ref{ec_graph_example_binary_tuple4} is obtained from \ref{ec_graph_example_binary_tuple2} by \lstinline|assumeEqual([f(a)],[f(b)])| - it cannot be obtained from \ref{ec_graph_example_binary_tuple3} as all of our operations are monotonic in the set of represented equalities.

\begin{figure}
\centering
\begin{subfigure}[t]{0.48\textwidth}
	\framebox[\textwidth]{
	\raisebox{0pt}[0.4\textheight][0pt]
	{
		\begin{tikzpicture}
			\node(O) at (0,0){};
			\node[gttn] (1)  at (2,2.5)                      {$()$};
			\node[gtn,label=center:\small{a}]  (2)  [above left  =0.8cm and 1cm of 1] {\phantom{b}};
			\node[gtn,label=center:\small{b}]  (3)  [above right =0.8cm and 1cm of 1] {\phantom{b}};
			\node[gttn] (4)  [above =2.5cm of 1]               {\m{(a,b)}};
			\node[gtn]  (5)  [above =1cm of 4]               {\m{f(a,b)}};

			\node[gl,align=left] (1l) [below = 0cm of 1] {
				Graph for the clause set $\emptyset$: \\ \terms{n} = \s{a,b,f(a,b)} \\ \gfasA{n} = \s{a(),b(),f([a,b])}
			};
						
			\draw[gfa] (2) to node[el,anchor=east] {\term{a}} (1);
			\draw[gfa] (3) to node[el,anchor=west] {\term{b}} (1);
			\draw[gfa] (5) to node[el] {\term{f}} (4);

			\draw[sgtt] (4) to[out=270,in=90] node[el,anchor=south] {\term{0}} (2);
			\draw[sgtt] (4) to[out=270,in=90] node[el,anchor=south] {\term{1}} (3);

%		\caption{The result of \\
%		\lstinline|makeTerm(a,())|\\
%		\lstinline|makeTerm(b,())|\\
%		\lstinline|makeTerm(f,([a],[b]))|}
		\end{tikzpicture}
	}}
	\caption{}
	\label{ec_graph_example_binary_tuple1}
\end{subfigure}
\begin{subfigure}[t]{0.48\textwidth}
	\framebox[\textwidth]{
	\raisebox{0pt}[0.4\textheight][0pt]
	{
		\begin{tikzpicture}
			\node(O) at (0,0){};
			\node[gttn] (1)   at (2,2.5)                     {$()$};
			\node[gtn,label=center:\small{a}]  (2)  [above left =0.8cm and 0.5cm of 1]  {\phantom{b}};
			\node[gtn,label=center:\small{b}]  (3)  [above right =0.8cm and 0.5cm of 1] {\phantom{b}};
			\node[gttn] (4)  [above =1.2cm of 2]               {\m{(a,b)}};
			\node[gttn] (5)  [above =1.2cm of 3]               {\m{(b,a)}};
			\node[gtn]  (6)  [above =1cm of 4]               {\m{f(a,b)}};
			\node[gtn]  (7)  [above =1cm of 5]               {\m{f(b,a)}};

			\node[gl,align=left] (1l) [below = 0cm of 1] {
				Graph for \s{}: \\
				\terms{n} = \s{a,b,f(a,b),f(b,a)} \\
				\gfasA{n} = \s{a(), b(), f([a],[b]),f([b],[a])}
			};
						
			\draw[gfa] (2) to node[el,anchor=east] {\term{a}} (1);
			\draw[gfa] (3) to node[el,anchor=west] {\term{b}} (1);
			\draw[gfa] (6) to node[el] {\term{f}} (4);
			\draw[gfa] (7) to node[el] {\term{f}} (5);

			\draw[sgtt] (4) to node[el,anchor=east] {\term{0}} (2);
			\draw[sgtt] (4.270) to[out=270,in=90] node[el,pos=0.3,anchor=south] {\term{1}} (3);

			\draw[sgtt] (5.270) to[out=270,in=90] node[el,pos=0.3,anchor=south] {\term{1}} (2);
			\draw[sgtt] (5) to node[el,anchor=west] {\term{0}} (3);

		\end{tikzpicture}
	}}
	\caption{}
	\label{ec_graph_example_binary_tuple2}
\end{subfigure}
\begin{subfigure}[t]{0.48\textwidth}
	\framebox[\textwidth]{
	\raisebox{0pt}[0.4\textheight][0pt]
	{
		\begin{tikzpicture}
			\node(O) at (0,0){};
			
			\node[gttn] (1)  at (2,2.5)                        {$()$};
			\node[gtn]  (2)  [above =1cm of 1]               {a,b};
			\node[gttn] (3)  [above =1cm of 2]               {\stackB{\m{(a,a),(a,b)}}{\m{(b,a),(b,b)}}};
			\node[gtn]  (4)  [above =4cm of 1]               {\stackB{\m{f(a,a),f(a,b)}}{\m{f(b,a),f(b,b)}}};

			\node[gl,align=left] (1l) [below = 0cm of 1] {
				Graph for \s{a=b}: \\
				$\begin{array}{ll}
				\terms{n} & =
					\left\{
						\begin{array}{l}
							\term{a,b, f(a,a),f(a,b),} \\
							\term{f(b,a),f(b,b)}
						\end{array}
					\right\} \\
				\gfasA{n} & = \s{a(),b(),f([a],[a])}
				\end{array}$
			};
						
			\draw[gfa] (2) to[bend right] node[el] {\term{a}} (1);
			\draw[gfa] (2) to[bend left] node[el,anchor=west] {\term{b}} (1);
			\draw[gfa] (4) to node[el] {\term{f}} (3);

			\draw[sgtt](3) to[bend right] node[el] {\term{0}} (2);
			\draw[sgtt](3) to[bend left] node[el,anchor=west] {\term{1}} (2);

		\end{tikzpicture}
	}}
	\caption{}
	\label{ec_graph_example_binary_tuple3}
\end{subfigure}
\begin{subfigure}[t]{0.48\textwidth}
	\framebox[\textwidth]{
	\raisebox{0pt}[0.4\textheight][0pt]
	{
		\begin{tikzpicture}
			\node(O) at (0,0){};
			
			\node[gttn] (1)  at (2,2.5)                        {$()$};
			\node[gtn,label=center:\small{a}]  (2)  [above left =0.8cm and 0.6cm of 1]  {\phantom{b}};
			\node[gtn,label=center:\small{b}]  (3)  [above right =0.8cm and 0.6cm of 1] {\phantom{b}};
			\node[gttn] (4)  [above =1.5cm of 2]               {\m{(a,b)}};
			\node[gttn] (5)  [above =1.5cm of 3]               {\m{(b,a)}};
			\node[gtn]  (6)  [above =4cm of 1]               {\m{f(a,b),f(b,a)}};

			\node[gl,align=left] (1l) [below = 0cm of 1] {
				Graph for \s{f(a,b)=f(b,a)}: \\
				\terms{n} = \s{a,b,f(a,b),f(b,a)} \\
				\gfasA{n} = \s{a(),b(),f([a],[b]),f([b],[a])}
			};
						
			\draw[gfa] (2) to node[el] {\term{a}} (1);
			\draw[gfa] (3) to node[el,anchor=west] {\term{b}} (1);
			\draw[gfa] (6) to node[el] {\term{f}} (4);
			\draw[gfa] (6) to node[el,anchor=west] {\term{f}} (5);

			\draw[sgtt] (4) to node[el,anchor=east] {\term{0}} (2);
			\draw[sgtt] (4) to[out=270,in=90] node[el,pos=0.3,anchor=south] {\term{1}} (3);

			\draw[sgtt] (5) to[out=270,in=90] node[el,pos=0.3,anchor=south] {\term{1}} (2);
			\draw[sgtt] (5) to node[el,anchor=west] {\term{0}} (3);

		\end{tikzpicture}
	}}
	\caption{}
	\label{ec_graph_example_binary_tuple4}
\end{subfigure}
\caption{Binary tuples}
\label{ec_graph_example_binary_tuple}
\end{figure}

\noindent
In figures \ref{ec_graph_example_cyclic} we show some graphs with cycles - note that some ECs are infinite.\\
The graph in \ref{ec_graph_example_cyclic1} is obtained by:\\
\lstinline|makeTerm(a,()); makeTerm(f,([a])); assumeEqual([a],[f(a)])|\\
The graph in \ref{ec_graph_example_cyclic2} is obtained by:\\
\lstinline|makeTerm(a,()); makeTerm(b,())|\\
\lstinline|makeTerm(f,([a],[b])); makeTerm(g,([a],[b]))|\\
\lstinline|assumeEqual([f(a,b)],[a]); assumeEqual([g(a,b)],b)|\\
Note that we can share the representation of tuple-ECs ( we use shared tuple-ECs in our implementation).\\
The graph in \ref{ec_graph_example_cyclic3} is obtained by:\\
\lstinline|makeTerm(a,()); makeTerm(b,())|\\
\lstinline|makeTerm(f,([a],[b])); makeTerm(g,([b],[a]))|\\
\lstinline|assumeEqual([f(a,b)],[a]); assumeEqual([g(b,a)],b)|

\begin{figure}[H]
\centering
\begin{subfigure}[t]{0.43\textwidth}
	\framebox[\textwidth]{
	\raisebox{0pt}[0.4\textheight][0pt]
	{
		\begin{tikzpicture}
			\node(O) at (0,0){};
			\node[gttn] (1)  at (2,2.5)                      {$()$};
			\node[gtn]  (2)  [above =1cm of 1]               {\stackC{a}{f(a)}{...}};
			\node[gttn] (3)  [above =1cm of 2]               {\stackB{(a)}{...}};

			\node[gl,align=left] (1l) [below = 0cm of 1] {
				Graph for\s{\term{a=f(a)}}: \\
				\terms{n} = \s{a,f^n(a)} \\
				\gfasA{n} = \s{a(), f([a])}
			};
						
			\draw[gfa] (2) to                            node[el] {\term{a}} (1);
			\draw[gfa] (2.270) to[out=240,in=120,looseness=2] node[el] {\term{f}} (3.90);

			\draw[sgtt](3) to node[el] {\term{0}} (2);

		\end{tikzpicture}
	}}
	\caption{}
	\label{ec_graph_example_cyclic1}
\end{subfigure}
\begin{subfigure}[t]{0.53\textwidth}
	\framebox[\textwidth]{
	\raisebox{0pt}[0.4\textheight][0pt]
	{
		\begin{tikzpicture}
			\node(O) at (0,0){};
			\node[gttn] (1)  at (2,2.5)                       {$()$};
			\node[gtn]  (2)  [above left =1cm and 0.2cm of 1]   {\stackB{a}{...}};
			\node[gtn]  (3)  [above right=1cm and 0.2cm of 1]   {\stackB{b}{...}};
			\node[gttn] (4)  [above      =2.9cm       of 1]   {\stackB{(a,b)}{...}};

			\node[gl,align=left] (1l) [below = 0cm of 1] {
				~Graph for\s{a=f(a,b),b=g(a,b)}: \\
				$\begin{array}{lll}
				\terms{n} & = &
					\left\{
						\begin{array}{l}
							\term{a,b, f(a,b),} \\
							\term{g(a,b),f(f(a,b),a),} \\
							\term{g(f(a,b),b),...} \\
						\end{array}
					\right\} \\
				\gfasA{n} & = & \m{\{a(),b(),f([a],[b]),}\\
				          &   & \m{g([a],[b])\}}
				\end{array}$				
			};
						
			\draw[gfa] (2.270) to[out=270,in=90] node[el] {\term{a}} (1);
			\draw[gfa] (3.270) to[out=270,in=90] node[el,anchor=west] {\term{b}} (1);
			\draw[gfa] (2.270) to[out=225,in=110,looseness=3] node[el] {\term{f}} (4.90);
			\draw[gfa] (3.270) to[out=315,in=70 ,looseness=3] node[el,anchor=west] {\term{g}} (4.90);

			\draw[sgtt] (4) to[out=270,in=90] node[el,anchor=south] {\term{0}} (2);
			\draw[sgtt] (4) to[out=270,in=90] node[el,anchor=south] {\term{1}} (3);

		\end{tikzpicture}
	}}
	\caption{}
	\label{ec_graph_example_cyclic2}
\end{subfigure}
\begin{subfigure}[t]{0.55\textwidth}
	\framebox[\textwidth]{
	\raisebox{0pt}[0.4\textheight][0pt]
	{
		\begin{tikzpicture}
			\node(O) at (0,0){};
			\node[gttn] (1)  at (2,2.5)                       {$()$};
			\node[gtn]  (2)  [above left =1cm and 0.2cm of 1]   {\stackC{a}{f(a,b)}{...}};
			\node[gtn]  (3)  [above right=1cm and 0.2cm of 1]   {\stackC{b}{g(b,a)}{...}};
			\node[gttn] (4)  [above      =1.3cm         of 2]   {$\stackB{(a,b)}{...}$};
			\node[gttn] (5)  [above      =1.3cm         of 3]   {$\stackB{(b,a)}{...}$};

			\draw[gfa] (2)     to[out=270,in=90] node[el] {\term{a}} (1);
			\draw[gfa] (3)     to[out=270,in=90] node[el,anchor=west] {\term{b}} (1);
			\draw[gfa] (2.270) to[out=225,in=135,looseness=1.5] node[el,anchor=east] {\term{f}} (4.90);
			\draw[gfa] (3.270) to[out=315,in=45, looseness=1.5] node[el,anchor=west] {\term{g}} (5.90);

			\draw[sgtt] (4) to node[el,anchor=east] {\term{0}} (2);
			\draw[sgtt] (4) to[out=270,in=90] node[el,pos=0.3,anchor=south] {\term{1}} (3);
			\draw[sgtt] (5) to[out=270,in=90] node[el,pos=0.3,anchor=south] {\term{1}} (2);
			\draw[sgtt] (5) to node[el,anchor=west] {\term{0}} (3);

			\node[gl,align=left] (1l) [below = 0cm of 1] {
				~Graph for\s{a=f(a,b),b=g(b,a)}: \\
				$\begin{array}{lll}
				\terms{n} & = &
					\left\{
						\begin{array}{l}
							\term{a,b, f(a,b),} \\
							\term{g(a,b),f(f(a,b),a),} \\
							\term{g(f(a,b),b),...} \\
						\end{array}
					\right\} \\
				\gfasA{n} & = & \m{\{a(),b(),f([a],[b]),}\\
				          &   & \m{g([b],[a])\}}
				\end{array}$				
			};
						
		\end{tikzpicture}
	}}
	\caption{}
	\label{ec_graph_example_cyclic3}
\end{subfigure}
\caption{Cyclic graphs}
\label{ec_graph_example_cyclic}
\end{figure}

%\newpage
\section{Congruence closure graphs for verification}
In this section we show an algorithm that uses our congruence closure graph data structure to verify a program VC where all clauses are unit ground equality clauses. 
This is not intended as a practical algorithm, but we use it to highlight the challenges posed by the CFG structure and how our data structures are built to answer these challenges. The algorithm mimics, in a sense, the operation of a DPLL prover on a certain class of VCs generated from such a program (including only unit ground equalities and dis-equalities).

\begin{figure}
\begin{lstlisting}
method verify(p: CFG)
	notVerifiedAssertions = $\emptyset$
	s := new Stack[ECGraph]
	g := new ECGraph
	
	traverseInPreOrder(p, entryVisitor, exitVisitor)

method entryVisitor(n : CFGNode)
	s.push(g)
	foreach ($\m{s=t}$$\in$$\clauses{n}$)
		g.assumel(s=t)
		
	if (n.isLeaf)
		foreach ($\m{s\neq t}$$\in$$\clauses{n}$)
			if (g.makeTerm(s)==g.makeTerm(t))
				return
		notVerifiedAssertions.add(n)
	
method exitVisitor
	g := s.pop
	
\end{lstlisting}
\caption{Basic equality verification algorithm - DFS\\
The method \lstinline|traverseInPreOrder(p, inv, outv)| traverses the CFG in pre-order - it is standard but we detail it in \ref{fig_CFG_traversal} in order to prevent ambiguities when applying the method to DAGs.
The method calls the callback inv(n) when first evaluating a CFG-node and outv(n) before back-tracking from the node.\\
The \lstinline|entryVisitor| pushes the current EC-graph on the stack, 
\lstinline|assumes| the axioms from the current CFG-node to the EC-graph, 
and if the CFG-node is a leaf node, it checks whether any dis-equality does not hold - which implies that the assertion at that leaf node holds (as assertions are negated).
}
\label{fig_DPLL_style_verification}
\end{figure}


\begin{figure}
\begin{lstlisting}
method traverseInPreOrder(p, inv, outv)
	ts := new Stack[(CFGNode,Int)]
	n := p.root
	k := 0
	do
		if (k==0) //first visit of the node
			inv(n)
		if (k<|n.successors|)
			ts.push(n,k+1)
			(n,k) := (n.successors[k],0)
		else
			outv(n)
			if (!ts.isEmpty)
				(n,k) := ts.pop
	while (!ts.isEmpty)
\end{lstlisting}
\caption{Pre-order traversal of the CFG\\
This method performs standard pre-order traversal of a CFG, calling a visitor \lstinline|inv| on entry to the node. In addition, the algorithm also calls a visitor \lstinline|outv| when back-tracking from the node.
We detail it here to make be clear about the behaviour in DAG traversal - namely,
that each DAG-node is traversed as many times as there are paths leading to it.
}
\label{fig_CFG_traversal}
\end{figure}

\noindent
For simplicity, we assume dis-equalities occur only in leaf nodes.
The algorithm is presented in figure \ref{fig_DPLL_style_verification}.
The verification algorithm stems almost directly from the definition of the validity of a program - namely, that each assertion (leaf) node must be infeasible on all paths reaching it. The algorithm checks validity by enumerating the paths and, for each path, constructing an EC-graph with all the equalities on the path. It then checks if any of the dis-equalities is inconsistent with the EC-graph.\\
The only optimization in the algorithm is that we keep a stack of EC-graphs so that paths that share a prefix (prefix path in the CFG from the root) do not duplicate the work of constructing the EC-graph for the axioms on the CFG-nodes of the prefix.\\
This algorithm mimics the behaviour of DPLL with lazy CNF conversion for VC formulae of a specific form. 
The form (used by Boogie) encodes nested \textbf{let} expressions that encode the control flow - for each non assertion CFG-node n,
the let expression used is \\
\m{\mathbf{let}~ n_{ok} = Ax_n \Rightarrow \land_{s \in \succs{n}} s_{ok}}. \\
For assertion nodes the expression is\\
\m{\mathbf{let}~ n_{ok} = Ax_n }. \\
\m{Ax_n} is the formula that represents the conjunction of the axioms in \clauses{n}.\\
the \textbf{let} expressions are nested in reverse topological order, and the main formula is \m{root_{ok}}.\\
The formula is valid iff the program is correct.
This encoding is efficient for lazy CNF conversion SMT solvers as it ensures that, at any point during the DPLL run,
only the axioms from one path in the program are available to the prover.\\
Note that SMT solvers often use a slightly different E-graph that supports removing equalities, and hence do not need to maintain a stack of graphs.

\subsection{Complexity}
The complexity of the above algorithm depends first on the complexity of operations on the EC-graph, so we discuss these first.
For a set of equalities on ground terms, the complexity measure is the number of function symbol occurrences in the set.
Different congruence closure algorithms are compared in ~\cite{BachmairTiwari00}. They all result in a data structure that can answer efficiently the question of whether two given terms are equivalent under the conjunction of equalities.
The efficiency of the above operations depends on the specific algorithm. Some algorithms perform congruence and transitive closure eagerly, while others try to minimize the time it takes to add assumptions, and only perform closure on queries relying on caching (e.g. path compression) to reduce the amortized time for the whole algorithm.

In all of the above algorithms the resulting data structure \emph{represents} the set of all terms that were added to it (using \lstinline|makeTerm|) and all of their equivalence class according to the congruence defined by all equalities for which we performed \lstinline|assumEqual| - as in our implementation.

For most implementations, the way to check equality between two terms is to check whether they have the same representative.
The time it takes to map a term t to its representative (our \lstinline|makeTerm(t)| method) depends on the laziness of the algorithm - eager algorithms such as ours answer it in \bigO{\size{t} log(n)} (n is the number of vertices in the graph) as the term needs to be mapped bottom up from constants, and each step up requires a lookup in the equivalent of our \lstinline|superTerms| index.

All of the above algorithms compute the congruence closure only from positive equalities, and then it is possible to check for each dis-equality whether both sides of the dis-equality are in the same equivalence class, which signals a contradiction. 
For a dis-equality \m{s \neq t} this check can be done by adding the terms \m{s,t} to the congruence closure data structure and then checking whether they map to the same equivalence class - so at most \bigO{(\size{s}+\size{t}) \lg(n)}.
It is common to add dis-equalities to a congruence closure graph using edges between the nodes, and a graph is inconsistent if there is an edge between two nodes that are in the same EC. In our EC-graph inconsistency is represented by a dis-equality self edge.\\
We use dis-equality edges in our implementation, but we do not discuss them here as lazy propagation for dis-equalities is is not complete if we just propagate all dis-equality edges - we use the non-unit clause mechanism described in chapter \ref{chapter:gfole} for dis-equalities and add the edges to the EC-graph for easier representation only - hence we do not discuss these edges in this chapter.

\subsubsection*{Complexity for a single set of (dis)equalities}
\textbf{Worst case time:} The best known time complexity for deciding a set (conjunction) of unit ground equalities and dis-equalities is \bigO{n\lg(n)} - several congruence closure decision procedures have been shown sound and complete - for example \cite{DowneySethiTarjan},\cite{NieuwenhuisOliveras03} and (the congruence closure part of) \cite{Shostak84} and \cite{NelsonOppen80}, a survey with comparisons is found in \cite{BachmairTiwari00}. These papers include also complexity analysis.\\
All of the above-mentioned algorithms essentially build a graph with terms as vertices, and perform congruence and transitive closure with differing levels of laziness.
Roughly, the reason for the \bigO{\lg(n)} complexity is that the graph need never have more nodes and edges than there are terms in the original problem, and each time a new equality is inserted into the graph and two nodes are merged, we can remap the super-terms of the term with \emph{less} super-terms, so each term is remapped at most \bigO{\lg_2(n))} times - this is proven in \cite{DowneySethiTarjan}. 
The number of equivalence classes is reduced by at least one for each non-redundant equality (potentially more in the case of congruence closure), so in total at most \m{n} such reductions can happen. An index is used to find the direct super-terms of each term for congruence closure and to update the graph (equivalent to our \lstinline|superTerms| field).\\
The \lstinline|superTerms| index can be implemented as an a-dimensional array for a complexity \bigO{a} for insert, lookup and remap (on merge) where \m{a} is the largest function arity, but space \bigO{n^a} - for a total time complexity of \bigO{n \lg(n)}.
A more space efficient option is a self balancing binary search tree for \m{log(n)} for the above operations but space \bigO{n} - giving an overall complexity of \bigO{n \lg^2(n)}. \\
Another practical solution is to use a hash table for average \bigO{1} operations and \bigO{n} space - giving the optimal average complexity but worst case \bigO{n^2} complexity - see ~\cite{DowneySethiTarjan} for a discussion. We compared the usage of search trees and hash tables in an earlier version of the implementation and hash tables performed consistently and significantly better.\\
Our algorithm is presented in a slightly simplified form, in the actual implementation we use a map \lstinline|superTerms| that maps to super \GFAs{}, rather than super \GTs{}, hence saving the search for affected \GFAs{} (this is more in line with the other algorithms), and we maintain a map from a \GFA{} to its \GT{}. The update of the \lstinline|gfas| field can be postponed until all node merging is done (they are used for e.g. E-matching and superposition) and hence we can get a similar complexity to the common algorithms. 
However, in our setting, the dominant complexity factor is related to the CFG size rather than just the number of terms, as we detail below, and so preventing repeated traversals of the CFG, and preventing several CFG-nodes from performing the same congruence closure operation, is more important to complexity than each individual congruence closure operation.

\textbf{Worst case space:} Equality is described by either linking two vertices with an equality edge or merging the vertices. For transitive closure all the above algorithms do not produce all transitive edges but rather select one representative for each equivalence class and, lazily or eagerly, the path from each vertex to its representative is compressed to one edge, so at most one additional edge is added per vertex.
Space complexity is, hence, \bigO{n}, as we have at most one vertex and one edge per input symbol occurrence - a vertex will have at most as many incoming edges as it has occurrences - if \m{f(a)} and \m{g(a)} occur in the input, the graph will have just one vertex to represent \m{a}, but an incoming edge into \m{a} each for \m{f(a),g(a)}.
As detailed above, unless the array solution is used we get a space complexity of \m{O(n)}.
This linear space property explains, to some degree, the appeal of DPLL based solvers - while the search space is large, 
the size of any candidate model is proportional to the size of the problem (in SMT the CC graph is the representative model for uninterpreted functions).

\subsubsection*{Complexity for a tree-shaped CFG}
We look first at tree-shaped CFGs in order to highlight the effect of CFG shape on verification complexity.

For a CFG we use two complexity measures - n measures the total number of function symbol occurrences in all clauses of all CFG-nodes and e measures the number of CFG edges - as we assume a CFG is of maximal out-degree two, e is proportional to the number of CFG-nodes.

For a CFG with no branches or joins, we can collect all clauses from all CFG-nodes into one set and then the complexity for deciding the validity of the program is as above practically \bigO{n \lg^2(n)}. We assume here that there are no CFG-nodes that are neither a branch nor a join and that have an empty set of clauses.

For a binary-tree-shaped CFG, the maximal number of paths is proportional to e.
If we simply collect all clauses for each path and apply any of the above-mentioned congruence closure graph construction algorithms for a set of clauses, the complexity is \bigO{n^2\ log^2(n)} as we have n paths - this bound is exact as we can have all positive clauses at the root and the only other clauses at the leaves.\\
Using our algorithm from figure \ref{fig_DPLL_style_verification}, the worst case time complexity is still
\bigO{n^2\ log^2(n)} - consider the case where the root node includes the equations \m{c=f^k(a),f^k(b)=d} and each leaf node includes \m{a=b,c \neq d} where \m{k=e} - each leaf node has to perform \m{k} congruence closure operations and hence we get quadratic complexity. 
However, practically, if equations are distributed more or less evenly between CFG-nodes, our algorithm, as it ensures no equation occurrence is evaluated more than once, can be expected to be more efficient.We are not aware of a better complexity bound for this problem. 

In our context, we are interested in two improvements that can improve efficiency in practical cases.
The intuition is that, in many cases, only a small proportion of the axioms are needed in order to prove each assertion.
Instead of each CFG-node simply copying the EC-graph of a predecessor and adding its own axioms, each CFG-node starts with an EC-graph including only its own axioms, and adds axioms from predecessor on-demand - this is the main subject of section 
\ref{section:ugfole:propagation}.\\
The second improvement is that, as our CFGs are constructed from programs in dynamic single assignment (DSA) form, we should be able to prove an assertion even if the EC-graph of each CFG-node only contains terms representing at most three DSA versions of each Boogie program variable - the DSA versions for the node and for its direct predecessors - hence we expect smaller EC-graphs. 
We explore this idea in the chapter \ref{chapter:scoping}. 

\subsubsection*{Complexity for a DAG-shaped CFG}
Even for a loop-free procedure with joins, the complexity of a DFS based algorithm rises to exponential because of join points which cannot be represented as conjunctions.

Our algorithm from figure \ref{fig_DPLL_style_verification} works for a DAG shaped program as well, in fact being a variation of the DPLL procedure:  when backtracking from a node after having explored all its outgoing children, we practically forget that we have explored that node and the next time we reach it (on another path) we again explore all outgoing edges.
The worst case complexity is dominated by the potentially exponential number of paths - for example, consider the program in figure \ref{linear_join_proof} - the program size is proportional to n, but the number of paths is exponential in n.\\
Modern DPLL based solvers have several advantages over our naive procedure - notably:
\begin{itemize}
	\item Clause learning - DPLL(CDCL) can learn a clause that generalizes part of the proof of the infeasibility of one path of exploration, which can help it prune some other paths from being explored
	\item Order of evaluation - DPLL can decide (depending on the encoding of the program VC as a formula and whether we use lazy CNF conversion) to explore the decision tree in a different order than our algorithm does - for example, it could \lstinline|decide| on a literal that occurs at a join node before \lstinline|deciding| the literals that determine which path leads to that join node - in that case the decision tree size might be less than exponential for some programs
\end{itemize}
We cannot expect better than exponential worst case complexity for DAGs as we can encode CNF-SAT to this problem in linear time - each clause with n literals is encoded into an n-branch with the nth literal on the nth direct successor and the branch is immediately joind, and each literal is encoded as described in section \ref{section:preliminaries:semantics}.
The branch-join structures are organized in sequence and the CNF-SAT problem is satisfiable iff the last node (the only leaf node) is reachable - if we have not found a path leading to it in which we have found no contradiction.

\begin{figure}
\begin{lstlisting}
$\m{b_0}$:
	assume $\m{a_0=d}$
	if (*)
		assume $\m{b_1=a_0}$
		assume $\m{a_1=b_1}$
	else
		assume $\m{c_1=a_0}$
		assume $\m{a_1=c_1}$
$\m{j_1}$:

...
$\m{j_{n-1}}$:
	if (*)
		assume $\m{b_n=a_{n-1}}$
		assume $\m{a_n=b_n}$
	else
		assume $\m{c_n=a_{n-1}}$
		assume $\m{a_n=c_n}$
$\m{j_{n}}$:
	assert $\m{a_n=d}$ //negated $\comm{a_n \neq d}$
\end{lstlisting}
\caption{linear join proof\\
The program (from \cite{DPLLJoin}) can be proven in almost linear time using joins,
but is exponential for DPLL and CDCL based provers}
\label{linear_join_proof}
\end{figure}

However, we can easily construct a sequence of programs that each have a Hoare proof linear in the size of the program, but their encoding into SMT would take exponential time for current DPLL-CDCL solvers, the example at \ref{linear_join_proof} is taken from \cite{DPLLJoin}.
This program takes exponential time for current CDCL solvers without provisions for join as the literals \m{a_{i}=a_{i+1}} and \m{a_{i+1}=d} do not exist in the original problem and hence cannot be learned by CDCL (we have tried both Z3 and CVC4 - both timed out after several hours for 50 diamonds, which were verified in less than a second using the algorithm presented later in this chapter). 

\subsection{Joins}
A \newdef{join} for a pair of congruences is the intersection of the congruences and is also a congruence. We use the symbol \newdef{\m{\sqcup}} for the join of two sets of clauses.\\
A congruence can be represented by a finite EC-graph iff it is representable by a finite set of ground equations.
Similarly, a join for two EC-graphs is an EC-graph for the intersection of the congruences represented by the EC-graphs.

For the program in figure \ref{linear_join_proof}, the join at each node \m{j_k} includes the equality \m{a_k=a_{k-1}},
as it is in the join (intersection) of the congruences defined by the sets of equalities in both direct predecessors - written as:\\
\m{a_k=a_{k-1} \in \s{a_{k-1}=b_{k},a_{k}=b_{k}} \sqcup \s{a_{k-1}=c_{k},a_{k}=c_{k}}}.\\
If we modify our verification algorithm to traverse the CFG in topological order (essentially breadth first) rather than in pre-order (essentially depth first), and if we had a way for calculating such joins efficiently, we could prove the above program efficiently as adding the joined equalities to the EC-graph at each join allows a very short proof - essentially, we only need to employ transitive closure for the set of clauses \m{a_0=d, a_1=a_0, ..., a_n=a_{n-1},a_n\neq d}. 
In the rest of this section we show the challenges in the approach that traverses \emph{each CFG-node} in the CFG in topological order and calculates a join at each join node, and uses these joins for verifying the program rather than traversing each \emph{CFG-path} in the CFG. We show in this section that a single forward pass of the CFG cannot calculate all necessary joins, 
and in the next section we show how we overcome this problem. We follow this idea in the rest of this chapter and extend it to richer fragments in later chapters.


\subsubsection*{Properties of joins}
When discussing joins we use the term \emph{joinees} for the two direct predecessor nodes of a join node, and refer to these nodes as \m{p_0} and \m{p_1}  (in the order in which they are introduced in the text unless otherwise noted), and \m{n} for the join node.

We make an important distinction between calculating the actual join of two congruences at a join point in the program (which may not be finitely presentable as a set of equations), 
and between calculating a subset of the equalities in the intersection of congruences (an approximation of the join), which is sufficient to prove all assertions in transitive successors of the join point.
A sufficient join is related to an interpolant between the congruence defined by its transitive predecessors and the congruence and dis-equalities defined by its transitive successors - we discuss interpolants in detail in chapter \ref{chapter:scoping}.

%We use the term \newdef{fragment interpolant} (for a given logical fragment) for an annotation of the CFG with formulae in the fragment, that satisfies the following conditions:\\
%represents a proof (note that we do not limit the vocabulary of each formula) of  a  logical fragment to refer to a set of clauses in thinterpolants within our fragment - that is, a fragment interpolant for a CFG node n for the fragment of unit ground equalities is a conjunction of equalities C s.t. \m{n \models C} (remember that \m{n \models C} means that, for each path P from the root to n, the clauses on the path are sufficient to prove C) and that, on each path P from n to an assertion node \m{na}, \m{C \cup \clauses{P} \models \emptyClause} (remember that \clauses{P} are all the clauses in all nodes on the path P). This definition means that C is provable from the transitive predecessors of n, and is sufficient to prove the transitive successors of n.\\

A \newdef{fragment interpolant} for the whole program for a logical fragment is an annotation $\phi_n$ for each node s.t. $\phi_n$ is in the fragment (in our fragment - a conjunction of unit equalities) and, for each CFG-node n, \m{\clauses{n}  \land \bigwedge\limits_{p \in \preds{n}} \phi_p \models \phi_n}, and, for each assertion node na, \m{\phi_{na}=\false}.  
Note that we do not restrict the vocabulary of \m{\phi_n} - we discuss signature restrictions in chapter \ref{chapter:scoping}, in this chapter we are only interested in the shape of \m{\phi_n}.

Only valid programs have a fragment interpolant, in fact, a fragment interpolant is a form of validity proof for the program.
While the definition of a fragment interpolant is similar to the one for abstract domains in abstract interpretation, our definition is weaker as it does not require a join operator - as we show in the next examples, there are programs from which the join at a node cannot be calculated effectively as a function of the predecessors of the join.

A join (disjunction or intersection) for two congruence relations (and similarly, two EC-graphs) cannot always be represented as the conjunction of a finite set of unit ground equalities (or, equivalently by an EC-graph), or even by an infinite set - for example, the disjunction of the congruences defined by \s{a=b} and \s{a=c} cannot be represented by a set of ground unit equalities. 

\begin{figure}
\begin{lstlisting}
if (*)
	$\m{p_0}:$ 
	assume c=a
else
	$\m{p_1}:$ 
	assume c=b
$\m{n}:$
if (*)
$p_{t}:$
	assume a=b
	$p_{ta}:$
		assert a=c //negated a $\textcolor{gray}\neq$ c
else
	...
\end{lstlisting}
\caption{Non-unit join\\
No set of equations at \m{n} implied by all its predecessors is sufficient to prove the assertion.}
\label{snippet3.5}
\end{figure}

Consider the example in figure \ref{snippet3.5}.
Although each path from the root to the assertion in the program can be proven using only unit (dis)equalities, 
there is no set of ground unit (dis)equalities that holds at \m{n} and is sufficient to prove the assertion.
As shown in ~\cite{GulwaniTiwariNecula04}, even when the disjunction is representable as a set of unit equalities, this set is not necessarily finite. 
In light of this limitation, our objective in calculating a join is only to find small joins, and fall-back to non-unit ground equality clauses in other cases.

\noindent
\textbf{Worst case space complexity:}\\
As stated above, a join is not always finitely representable, and a join as a conjunction of equalities is not always sufficient for proving all assertions. Hence we are interested only in the size of fragment interpolants when they exist - approximations of the join that are sufficient for proving all assertions in transitive successors. We discuss in this section the worst-case minimal size for such a fragment interpolant, if it exists.

~\cite{GulwaniNecula07} shows an exponential lower bound on the size of fragment interpolants - the authors show a program of size n (total function symbols) with only unit equalities (in fact only variable assignments - an even smaller class that does not create cyclic EC-graphs) that ends in a \m{\sqrt{n}} sided join node for which the minimal representation of the equivalence class of a certain variable as a set of equalities is of size \m{\theta(2^{\sqrt{n}})} (even in a representation where all shared sub-expression are represented only once - as in our EC-graph).\\
A rough description of the example is that, at the join point, the equivalence class of a variable includes just one more element,
which is essentially a full binary tree of depth \m{n} of applications of a single binary function \m{f}, 
where at each leaf there is a full binary tree of depth \m{\lg n} of \m{f}. Each such \m{\lg n} depth tree has \m{n} leaves, each of which is either the term \m{0} or \m{1}, but each tree has a different sequence of \m{n} 0s and 1s - so in total there are exponentially many distinct AECs.\\
This is achieved at the join by using only a \m{\sqrt{n}} sized term at each joinee.\\
Designating this program of size n with the variable \m{x} and \s{a,b} standing for \s{0,1} as \m{P(n,x,a,b)}, 
so that at the single leaf node of the program the variable \m{x} is equal to a term of size \m{2^n} with only the constants \m{a,b} and the binary function symbolf f, we can construct the program shown in figure \ref{snippet3.6}.
\begin{figure}
\begin{lstlisting}
$\m{b}$:
	P(n,x,a,b)
$\m{j_0}$:
	P(n,y,c,d)
$\node{j_1}$:
...
$\node{n}$:
	assume a=c
	assume b=d
	assert x=y
\end{lstlisting}
\caption{Exponential sized join proof}
\label{snippet3.6}
\end{figure}

This program can be proven by annontating each CFG-node with a single EC-graph which is the join of the EC-graphs of its predecessors, but the EC-graph at \m{j_1} must include at least two terms of exponential size - once with \s{a,b} at the leaves and once with \s{c,d}, which are only equated at \m{n}.\\
Hence this program can be proven using the hypothetical algorithm with joins, but the minimal size of the EC-graph at the join is \m{\theta(2^{n})}, while DPLL could simply explore each of the \m{\sqrt{n} \times \sqrt{n} = n} paths of the program and produce a proof of size \m{n} for each path - so in total an \bigO{n^2} sized proof.
This means that even for programs within our class we can have exponential sized minimal interpolant.

The complexity of verifying this class of programs (programs including only ground unit (dis)equalities as clauses) is NP-complete, as we have seen there are polynomial encodings to CNF.
As the example above shows there are programs with worst case (single) exponential interpolated proof size as represented using EC-graphs (in our fragment), but we are not aware of a tighter lower or upper bound for the complexity of deciding whether such an interpolated proof exists. Our verification algorithm for general programs limits the size of interpolants, and hence, in programs as above, 
will fall-back to non-unit clauses at joins and rely on resolution and superposition at \node{j_1} in order to derive a contradiction.

\subsubsection*{Calculating a join}
We now discuss how a fragment interpolant can be calculated at the join, which mainly centers around determining which terms need to be represented at the join.

In ~\cite{GulwaniTiwariNecula04}, an \bigO{n^2} algorithm is given for joining two sets of unit equalities, which guarantees that any term that is represented on both sides (that is, for which there is a node in the graph that represents its equivalence class), is also represented at the join, and the join contains the full equivalence class for that term.
The basic reason for this complexity is that any equivalence class at the join must correspond to a pair of equivalence classes, one from each joinee, while not necessarily all such pairs are needed in order to represent all the common terms (that is - terms represented on both sides of the join) at the join.

To illustrate: consider joining two identical congruence closure graphs - the join is exactly the size of each joinee, and the only pairs are two copies of the same equivalence class.
On the other hand, consider joining  \s{a=f^m(a)} and \s{a=f^n(a)} where \m{m} and \m{n} are co-prime - here the join would be \s{a=f^{m n}(a)}, so of a quadratic size representation. 
For both joinees and for the join point, the set of represented terms is \s{f^i(a) \mid i \in \mathbb{N}}.
Here we are using all the pairs where each pair \m{([f^x(a)]_0,[f^y(a))]_1} represents the equivalence class \\
\s{f^{((xn[n^{-1}]_m+ym[m^{-1}]_n) \mathbf{~mod~} mn) + imn}(a) \mid i \in \mathbb{N}}, 
where \m{[n^{-1}]_m} is the multiplicative inverse of \m{n} in the ring \m{Z_m}, by the Chinese remainder theorem.

For any equality \m{s=t}, we can determine whether it holds at a join by adding both s and t to both joinees, and performing a join as above. Adding a term \m{t} to a congruence closure graph is of complexity \m{\size{t}lg \size{t}} and can add at most \size{t} edges to the graph.
For the equality \m{s=t} the time is then proportional to \m{(m+\size{t}+\size{s}) \times (n+\size{t}+\size{s}))} multiplied by some log factor (depending on the way the lookup maps are constructed).

However, it is not immediately clear which terms should be added at which join - consider the example in figure \ref{snippet3.7}.
Here it is easy to see that adding \s{f(a),f(b)} at \m{p_0} is sufficient to prove the assertion.

\begin{figure}
\begin{lstlisting}
if (*)
	$\m{p_0}:$
	assume a=b
else
	$\m{p_1}:$
	assume f(a)=f(b)
$\m{n}:$
if (*)
	assert g(f(a))=g(f(b)) //negated g(f(a)) $\comm{\neq}$ g(f(b))
else
		...
\end{lstlisting}
\caption{Join indirect congruence closure\\
Adding \m{f(a),f(b)} to the EC-graph of \m{p_0} allows us to prove the program with joins.
}
\label{snippet3.7}
\end{figure}

\begin{figure}
\begin{lstlisting}
if (*)
	$\m{p_0}:$
	assume a=b
else
	$\m{p_1}:$
	assume f(a)=f(b)
$\m{n}:$
	assume $\m{a_1}$ = a
	assume $\m{b_1}$ = b
	if (*)
		$p_t:$
		assert $\m{g(f(a_1))=g(f(b_1))}$ //negated $\m{\color{gray}{g(f(a_1))\neq g(f(b_1))}}$
	else
		...
\end{lstlisting}
\caption{Join indirect congruence closure DSA\\
This program includes the variables \m{a,b} in two DSA versions}
\label{snippet3.8}
\end{figure}

On the other hand, consider figure \ref{snippet3.8}, which could be generated as the VC of a program using the DSA form.
In this case, we need to translate the terms \m{g(f(a_1)),g(f(b_1))} through \m{a_1=a,b_1=b} in order to derive the set of terms which would have to be added to the joinees of \m{n} 
(e.g. \s{f(a),f(b)} would suffice) in order to be able to prove the assertion.

\begin{figure}
\begin{lstlisting}
$\node{p_b}:$
if ($\m{c_1}$)
	$\node{p_t}:$
	assume $\m{a=b}$
else
	$\node{p_e}:$
	assume $\m{f(a)=f(b)}$
	assume $\m{g^k(a)=g^k(b)}$
$\node{p_j}:$
...
$\node{p_1}:$
	assume $\m{a_1 = a}$
	assume $\m{b_1 = b}$
$\node{p_2}:$
	assume $\m{a_2 = a_1}$
	assume $\m{b_2 = b_1}$
	...
$\node{p_n}:$
	assume $\m{a_n = a_{n-1}}$
	assume $\m{b_n = b_{n-1}}$
	$\node{p_{na}}:$
	assert $\m{f(a_n)=f(b_n)}$ //negated $\m{\textcolor{gray}{f(a_n) \neq f(b_n)}}$
\end{lstlisting}
\caption{Join congruence closure DSA chain\\
The program variables \m{a,b} have several DSA versions,
and each CFG-node only refers to up to two versions of each variable.
In order to determine which terms need to occur at the join, we need knowledge of the assertion and the entire DSA chain.}
\label{snippet3.23}
\end{figure}

The example in figure \ref{snippet3.23} shows how deeper DSA chains affect the calculation of the set of terms required at a join.
Here we want to propagate \m{f(a_i)=f(b_i)} for each i, but not \m{g(a_i)=g(b_i)}.
We need the information from all the nodes \m{p_1..p_n} in order to determine which terms are needed at \m{p_j}.
If \m{k,n} are of the same order of magnitude, propagating only \s{f(a_i),f(b_i)} at each \m{p_i} would result in an interpolant of size \bigO{k + n}, while an eager interpolant (that tries to include in the join each term that occurs on \emph{either} side, as opposed to those that occur on both sides), that includes \s{g^k(a_i),g^k(b_i)} for each \m{p_i}, is of the size \bigO{k \times n} - practically quadratic vs. linear size interpolant.

\begin{figure}
\begin{lstlisting}
if (*)
	$\m{p_0}:$
	assume $\m{a=b}$
	assume $\m{c=d}$
	assume $\m{f(a,a)=e}$
	assume $\m{f(a,d)=e}$
	assume $\m{f(d,a)=e}$
	assume $\m{f(d,d)=e}$
else
	$\m{p_1}:$
	assume $\m{a=c}$
	assume $\m{b=d}$
	assume $\m{f(c,c)=e}$
	assume $\m{f(c,b)=e}$
	assume $\m{f(b,c)=e}$
	assume $\m{f(b,b)=e}$
$\m{n}:$
if (*)
	$\m{p_t}:$
	assume $\m{a=b=c=d=x}$
	assert $\m{f(x,x)=e}$ //negated $\m{\textcolor{gray}{f(x,x) \neq e}}$
else
	...
\end{lstlisting}
\caption{congruence closure source quadratic}
\label{snippet3.10}
\end{figure}

Even if we know the minimal size of a sufficient interpolant, such a minimal interpolant is not unique - consider, for example, the program in figure \ref{snippet3.10}.
Here, any of the 16 options \\
\s{f(x,y)=e \mid x,y \in \s{a,b,c,d}} would suffice as an interpolant, 
and there is no specific reason to choose any of the interpolants - they are symmetric.

In selecting an interpolant, there are also incremental considerations. 
Our general verification algorithm interleaves the verification steps of different logical fragments.
For the purpose of the discussion in this chapter, this is important because the algorithm might apply a step of the unit ground equality fragment (that is, try and verify the program with the current set of unit equalities), then apply some steps from another fragment which might produce new equalities at some CFG-nodes, and then again try to verify the program with unit equalities. In such a scenario, we want to ensure that all verification steps are performed incrementally, and specifically that join approximations (fragment interpolants) are not recalculated from scratch every time we apply the ground equalities fragment, but rather that the fragment is applied incrementally.

Consider the example in figure \ref{snippet3.10}, where the \lstinline|else| branch has no ground equalities.
Assume we have selected one interplant from the sixteen possible ones and proven our assertion in the \lstinline|then| branch.\\
Now the verification algorithm applies one step of another fragment which produces some new unit clauses at the \lstinline|else| branch, as shown in figure \ref{snippet3.10a}. In the new program, there is exactly one specific minimal interpolant - \m{f(a,d)=e}.
If we have chosen another interplant in the previous stage, we cannot reuse it and have to recalculate the join.

\begin{figure}
\begin{lstlisting}
...
$\m{n}:$
if (*)
	$\m{p_t}:$
	assume $\m{a=b=c=d=x}$
	assert $\m{f(x,x)=e}$ //negated $\m{\textcolor{gray}{f(x,x) \neq e}}$
else
	$\m{p_e}:$
	assert $\m{f(a,d)=e}$ //negated $\m{\textcolor{gray}{f(a,d) \neq e}}$
\end{lstlisting}
\caption{Congruence closure - incremental interpolant\\
This is a modification of the program in figure \ref{snippet3.10}.\\
We assume the dis-equality on the \lstinline|else| branch is added after the join has been calculated.
}
\label{snippet3.10a}
\end{figure}
\noindent

The example in figure \ref{snippet3.22}, adapted from \cite{GulwaniTiwariNecula04}, shows that, for some joins, any set of equalities that represents the congruence at the join is infinite (the congruence is not finitely presented) for finite sets of clauses at the joined nodes.
However, there are several interpolants for the program at the join - including:\\
\m{g(f^2(a))=g(f^2(b)) \land g(f^3(a))=g(f^3(b))} and\\
\m{g(f^6(a))=g(f^6(b))}.


\begin{figure}
\begin{lstlisting}
$\node{p_b}:$
if ($\m{c_1}$)
	$\node{p_0}:$
	assume $\m{a=f(a)}$
	assume $\m{b=f(b)}$
	assume $\m{g(a)=g(b)}$
else
	$\node{p_1}:$
	assume $\m{a=b}$
$\node{n}:$
// Here $\m{\textcolor{gray}{\forall i \cdot g(f^i(a))=g(f^i(b))}}$ holds
if (*)
	$\node{p_{j_t}}:$
	assume $\m{a=f(a)}$
	assume $\m{f^2(b)=c}$
	assume $\m{f^2(c)=c}$
	assert $\m{g(a)=g(c)}$
else
	$\node{p_{j_e}}:$
	assume $\m{f^3(a)=c}$
	assume $\m{f^3(c)=c}$
	assume $\m{b=f(b)}$
	assert $\m{g(c)=g(b)}$
\end{lstlisting}
\caption{Infinite join for two congruences\\
For each i, \m{g(f^i(a))=g(f^i(b))} at the join \m{n} - this congruence is not finitely representable as a set of equalities.\\
}
\label{snippet3.22}
\end{figure}
\noindent

\begin{figure}
\begin{lstlisting}
if (*)
	$\m{p_0}:$
	assume $\m{b=f^m(a)}$
	assume $\m{b=f^m(b)}$
	assume $\m{g(b)=d}$
else
	$\m{p_1}:$
	assume $\m{c=f^n(a)}$
	assume $\m{c=f^n(c)}$
	assume $\m{g(c)=d}$
$\m{n}:$
	//Here $\m{\textcolor{gray}{f^{2mn}(a)=f^{mn}(a)}}$
	//  and $\m{\textcolor{gray}{g(f^{mn}(a))=d}}$
	if (*)
		$p_t:$
		assume $\m{a=f(a)}$
		assert $\m{a = g(d)}$ //negated $\comm{\m{a\neq g(d)}}$
	else
		...
\end{lstlisting}
\caption{congruence closure source quadratic depth\\
m,n are co-prime.\\
The minimal interpolant is of quadratic size.}
\label{snippet3.11}
\end{figure}

In the example in figure \ref{snippet3.11}, the minimal interpolant is of quadratic size (m,n co-prime).
In this case our algorithm will fall back to non-unit guarded equalities (in the fragment of ground superposition), to get at \m{n} (p is the branch condition joined at n):\\
$\m{\lnot p \lor a=f^m(a)}$\\
$\m{\lnot p \lor f^{2m}(a)=f^m(a)}$\\
$\m{\lnot p \lor g(f^{m}(a)) = d}$\\
$\m{      p \lor a=f^n(a)}$\\
$\m{      p \lor f^{2n}(a)=f^n(a)}$\\
$\m{      p \lor g(f^{n}(a)) = d}$\\
Which is an interpolant linear in the size of the input (\bigO{m+n}) - if represented using shared common sub-expressions (as we do in our EC-graph) we only use \bigO{max(\m{m,n})}.\\
At $p_t$, the interpolant is reduced by congruence closure to:\\
$\m{a=f(a)}$\\
$\m{\lnot p \lor a=a}$\\
$\m{\lnot p \lor a=a}$\\
$\m{\lnot p \lor g(a) = d}$\\
$\m{      p \lor a=a}$\\
$\m{      p \lor a=a}$\\
$\m{      p \lor g(a)=d}$\\
Now tautology elimination (\m{taut_{=}}) would leave:\\
$\m{a=f(a)}$\\
$\m{\lnot p \lor g(a)=d}$\\
$\m{      p \lor g(a)=d}$\\
And then reducing resolution (\m{simp_{res}}):\\
$\m{a=f(a)}$\\
$\m{g(a)=d}$\\
Which is sufficient to prove the assertion.

\begin{figure}
\begin{lstlisting}
	if (c1)
		$\m{p_0}:$ 
		assume $\m{a=b}$
	else
		$\m{p_1}:$
		assume $\m{f(a)=a}$
		assume $\m{f(b)=b}$
		assume $\m{g(a)=g(b)}$
$\m{n}:$
	...// Here, for any k, $\m{\textcolor{gray}{g(f^{k}(a))=g(f^{k}(b))}}$ holds
$\m{n_i}:$
	assume $\m{f(a)=a}$ //Here, for any k, $\m{\textcolor{gray}{g(a)=g(f^{k}(b))}}$	
	assert $\m{g(a)=g(b)}$ //negated $\m{\textcolor{gray}{g(a) \neq g(b)}}$
\end{lstlisting}
\caption{Join infinite equivalence class (m,n co-prime)}
\label{snippet3.30}
\end{figure}

The next example, in figure \ref{snippet3.30} (based on \cite{GulwaniTiwariNecula04}), shows an infinite equivalence class \emph{after} the join - the set of provable equations for \m{g(a)} at n is infinite - namely \s{g(a)=g(f^k(b)) \mid k \in \mathbb{N}}. If there is a fragment interpolant then only a finite subset of each such infinite equivalence class is needed to prove the program.

\begin{figure}
\begin{lstlisting}
	if (c1)
		$\m{p_0}:$ 
		assume $\m{a=b}$
	else
		$\m{p_1}:$
		assume $\m{f(a)=a}$
		assume $\m{f^n(b)=b}$
		assume $\m{g(a)=g(b)}$
$\m{n}:$
	...// Here holds, for any k>0, $\m{\textcolor{gray}{g(f^{nk}(a))=g(f^{nk}(b))}}$
$\m{n_i}:$
	assume $\m{f(a)=a}$ //Here, for any k, $\m{\textcolor{gray}{g(a)=g(f^{nk}(b))}}$	
	//Smallest sufficient interpolant $\m{\textcolor{gray}{g(f^{mn}(a))=g(f^{mn}(b))}}$
	...
$\m{p}:$
$\m{n_j}:$
	assume $\m{f^{2m}(b)=f^m(b)}$ 
	assert $\m{g(a)=g(f^{m}(b))}$ //negated $\m{\textcolor{gray}{g(a) \neq g(f^m(b))}}$
\end{lstlisting}
\caption{join infinite equivalence class}
\label{snippet3.30a}
\end{figure}

The example in figure \ref{snippet3.30a} shows that, in a case as in figure \ref{snippet3.30} where an EC at the join is inifinite but an interpolant does exist, the minimal size of any interpolant depends on all equations in successor nodes - hence an algorithm as suggested before that traverses the CFG in topological order and calculates interpolants cannot decide the class of program where interpolants exist.
In order to prove the assertion at \m{n_j}, the smallest equality we would have to include at \m{n} is \m{g(f^{mn}(a))=g(f^{mn}(b))}.
The parameter \m{m} is not known when calculating the interpolant at the node \m{n} if we traverse the CFG in topological order,
and similarly the paramenter \m{n} is not known when traversing the CFG in reverse topological order - this essentially means that abstract interpretation for the domain of finite conjunctions of ground unit equalities cannot define a join operator that decides all programs in this fragment that have an interpolant - the join cannot be determined in one pass of the program.

\begin{figure}
\begin{lstlisting}
	if (c1)
		$\m{p_0}:$ 
		assume $\m{a=b}$
	else
		$\m{p_1}:$
		assume $\m{f(a)=a}$
		assume $\m{f^n(b)=b}$
		assume $\m{g(a)=g(b)}$
$\m{n}:$
	...// Here holds, for any k>0, $\m{\textcolor{gray}{g(f^{nk}(a))=g(f^{nk}(b))}}$
$\m{n_i}:$
	assume $\m{f(a)=a}$ //Here, for any k, $\m{\textcolor{gray}{g(a)=g(f^{nk}(b))}}$	
	...
$\m{p}:$
$\m{n_j}:$
	assume $\m{f^{2m}(b)=f^m(b)}$ 
	assert $\m{g(a)=g(f^{m+1}(b))}$ //negated $\comm{g(a) \neq g(f^{m+1}(b))}$
\end{lstlisting}
\caption{Join infinite equivalence class\\
The assertion holds iff $\exists k,l \cdot k\m{n}=l\m{m}+1$}
\label{snippet3.30b}
\end{figure}

A variant of the problem in \ref{snippet3.30a} is shown in figure \ref{snippet3.30b}.
Here, if $\m{m=n}$ and $\m{n>1}$ then the assertion does not hold, but if m,n are co-prime then the assertion holds, with a minimal interpolant of size greater than mn.

%The join algorithm described in \cite{GulwaniTiwariNecula04} ensures that each equality for which both terms are represented in the graph of both joinees is also represented in the graph of the join. Using this join algorithm we can modify our topological-order algorithm by first applying a pre-processing step that adds all terms that occur in all CFG-nodes to all other CFG-nodes (e.g. we can add the term \m{t} to the set of clauses as the clause \m{t=t}), and then running the algorithm - we are guaranteed that if there is a fragment interpolant it will be found. However, the worst-case complexity is at least exponential (as shown in the example from \cite{GulwaniNecula07}) and potentially up to double exponential as each join might be up to quadratic in the size of the joinees.

\begin{figure}
\begin{lstlisting}
	if (c1)
		$\m{p_0}:$ 
		assume $\m{a_1=a_2}$
		..
		assume $\m{a_1=a_d}$
	else
		$\m{p_1}:$
		assume $\m{f(a_1)=a_1}$
		...
		assume $\m{f(a_d)=a_d}$
		assume $\m{g(a_1,a_1....,a_1)=g(a_1,a_2,...,a_d)}$
$\m{n}:$
//Here, for any k, $\m{\textcolor{gray}{g(f^{k}(a_1),f^{k}(a_1),....,f^{k}(a_1))=g(f^{k}(a_1),f^{k}(a_2),...f^{k}(a_d))}}$
$\m{n_a}:$
	assume $\m{f^{2{m_1}}(a_1)=f^{m_1}(a_1)}$ 
	assume $\m{f^{2{m_2}}(a_2)=f^{m_2}(a_2)}$ 
	....
	assume $\m{f^{2{m_d}}(a_d)=f^{m_d}(a_d)}$ 
	assert $\m{g(f^{m_1}(a_1),f^{m_1}(a_1),...f^{m_1}(a_1))=g(f^{m_1}(a_1),f^{m_2}(a_2),...,f^{m_d}(a_d))}$ 
\end{lstlisting}
\caption{One join super-quadratic interpolant}
\label{snippet3.30c}
\end{figure}

The following example, shown in figure \ref{snippet3.30c}, suggests that, unlike the example in \cite{GulwaniNecula07}, the number of joins is not the only source for super quadratic complexity (remember that their example only allows clauses derived from the DSA form of assignments, and hence cannot create cycles in the EC-graph, which are created by \lstinline|assume| statements):
for any given polynomial degree d, this example requires an interpolant of size at least a polynomial of degree d, as follows:\\
We designate the ith prime number starting at two as \m{p_i}. Given a polynomial degree d, we show a sequence of programs where the nth program is of size linear in n but the minimal interpolant for the nth program is of size polynomial in n.\\
For the nth program in the sequence, we select the integers \m{m_1..m_d} s.t. the program above is in size linear in \m{n} and the minimal interpolant is of minimal size polynomial of degree d in n:\\
We use \m{m_i \triangleq p_i^{\lceil lg_{p_i}{n} \rceil}} - this means that \m{m_i} is the lowest power of \m{p_i} that is greater or equal to \m{n} - hence \m{n \leq m_i < p_in \leq p_dn}. The size of the program (unique sub-terms) is at most \m{4d+2dnp_d}, as each of \m{m_i} is smaller than \m{p_dn}.\\
The only equalities that hold at the join are of the form \\
\m{g(f^{k}(a_1),f^{k}(a_1),....,f^{k}(a_1))=g(f^{k}(a_1),f^{k}(a_2),...f^{k}(a_d))} for some k.\\
Each such k has to satisfy the set of constraints \s{m_i|k \mid 0 < i \leq d}. 
As \m{m_i} are pairwise co-prime, this entails that k is at least \m{\prod\limits_i m_i}, which satisfies \m{\prod\limits_i m_i \geq n^d}. \\
We believe our example suggests that the worst case size complexity for unit interpolants in a CFG might be higher than double exponential, but the exact lower bound for the space complexity of a fragment interpolant for our class of programs remains an open question as far as we are aware. The decision problem of whether a fragment interpolant exists is at least semi-decidable, as we can enumerate all such interpolants and the entailment checks are decidable.

\subsubsection*{Summary}
We have seen that for some programs joins can reduce the verification runtime exponentially, while in other cases a join does not exist as a set of equalities. A fragment interpolant (a sufficient approximation for the join as a set of equalities) for a single binary join for a set of terms is up to quadratic in the number of terms, but the size of the terms needed at each join can be a polynomial of any degree in the input size. A minimal fragment interpolant can also be of quadratic depth and not only size.\\
The minimal representation of the equivalence class of a term at a node after a single binary join can be of size 1 up to infinite, and the size of the representation of a member of the class after n consecutive joins can be exponential.\\
Furthermore, there may not be a single minimal interpolant, and even when there is a minimal interpolant, 
the minimal interpolant for an extension of the program is not necessarily an extension of the minimal interpolant.\\
Most of the examples we have shown in this section are pathological examples that explore the complexity bounds of the fragment interpolant, and not practical examples. In actual VCs of programs, we have found that the choice of which equalities are propagated using fragment interpolants at joins, and which are propagated using non-unit clauses, has a significant effect on the performance of our algorithm. In the next sections we show how our algorithm selects which part of the join is calculated, and which is relegated to richer fragments.
\newpage
\section{Information Propagation}\label{section:ugfole:propagation}
We have seen in the previous section that verification using a DPLL or CDCL based solvers suffers from some inherent problems when joins are concerned, even when all clauses are unit ground clauses. 
While the example program we have seen (figure \ref{linear_join_proof}) is a synthetic example that we do not expect to see as a real program, we have encountered the same problem when quantifiers are involved - namely, 
when a solver explores the search space of a program proof by enumerating program paths, and when quantifier instantiation is involved, 
if the CDCL part of the prover cannot learn a lemma (clause that holds at the join) that involves an instance of a quantified clause, the number of times a quantifier is instantiated can grow exponentially with the number of sequential joins in the program.

We have run the Boogie program in figure \ref{fig_diamond_ROW} for different versions of k using Z3, and counted the number of instantiations of heap axioms (both read-over-write with equal and non-equal indices) - the results are detailed below.\\
We run this program for different values of k with the following results:\\
\begin{tabular}{c|c|c}
  k & ROW= & ROW$\neq$\\
  0 & 3 & 2 \\
  1 & 6 & 8 \\
  2 & 12 & 24 \\
  4 & 88 & 352 \\
  10 & 2868 & 20480\\
  10h & 182 & 172
\end{tabular}\\
The first column shows the number of consecutive branch-join pairs - the meaning of the last row is described below.\\
The second and third columns show the number of instances of the following axioms (heap axioms):\\
ROW=:$\m{\forall h,x,v \cdot h[x:=v][x] = v}$\\
ROW$\neq$:$\m{\forall h,x,y,v \cdot x\neq y \Rightarrow h[y:=v][x] = h[x]}$\\
The last row is a version of the program where k=10 but we have added a copy of the assertion after each join as a hint for the prover - this hint adds a literal to the input that the prover can use in learned lemmas, hence the significant reduction in axiom instances.\\
The results show that heap manipulating programs with many joins can benefit from calculating joins, at least for the prover Z3. We have experienced similar performance with other axioms as well in programs with complicated CFGs.
We have tried the example below also with the SMT solver CVC4. For CVC4 we did not see such an increase in quantifier instantiations for low values of k, but with \m{k=20} the solver timed out after two minutes, while it took less than a second to prove the case of \m{k=10} (similar to the example in figure \ref{linear_join_proof}). The SMTlib input includes no other quantified axioms but the above two. Our verification algorithm presented in this chapter, together with the heap fragment presented in section \ref{section:heaps} solve this problem for \m{k=40} in less than a second.

\begin{figure}
\begin{lstlisting}
	assume r0$\neq$r1
	assume r0$\neq$r2
	assume r1$\neq$r2
	
	heap[r0] := 0;
	heap[r1] := 1;
	heap[r2] := 2;
	
	heap[r1] := 1;
	if (*)
		heap[r1] := 1;
	else
		heap[r2] := 1;
	
	....
	
	heap[r1] := k;
	if (*)
		heap[r1] := k;
	else
		heap[r2] := k;
	assert heap[r0] == 0;

\end{lstlisting}
\caption{Number of read-over-write axiom instantiations with branch-join pairs.}
\label{fig_diamond_ROW}
\end{figure}


\subsection{A verification algorithm}
We have seen in the last section and above that enumerating program paths has some inherent weaknesses when join nodes are present in the program, not all of which are mitigated by current CDCL technology.
In addition, we have seen that for a set of ground equalities, we have a method to select a sub-set that is sufficient to decide whether a given equation is entailed, as shown in the algorithm in figure \ref{fig_lazy_congruence}.

We have also seen in some examples for unit equalities, such as \ref{snippet3.30a}, that in order to calculate an approximation of the join that is sufficient to prove the assertions following the join (a fragment interpolant), we need information both from the transitive successors and transitive predecessors of the join, hence the join cannot be calculated in a single backward or forward pass of the program.

In this section we propose a verification algorithm that aims to avoid the above-mentioned shortcomings.
The algorithm is based on a traversal of the CFG in topological order, but it behaves lazily in propagating information forward in the CFG.
While our algorithm of figure \ref{fig_DPLL_style_verification} which mimics, to a degree, the behavior of DPLL on a certain encoding of VCs, carries the EC-graph of each CFG-node to each of its successors, we propose to propagate this information lazily - initially each CFG-node constructs an EC-graph that contains only its own clauses, and later, when needed, it requests more clauses from its predecessors, with a mechanism similar to the one in figure \ref{fig_lazy_congruence}.

The algorithm is presented for verifying programs with arbitrary sets of clauses, and we show how it is specialized to work with EC-graphs. In this section we discuss the mechanism for propagating clauses.

The outline of the algorithm is shown in figure \ref{verification_algorithm_v2}. 
The algorithm traverses the CFG in topological order and verifies each CFG-node in turn.\\
For each node, the algorithm maintains two sets of clauses, as is common in saturation provers:\\
The first set, \lstinline|todo|, contains clauses that are yet to be processed.\\
The second set, \lstinline|done|, contains clauses that are inter-saturated w.r.t. the logical calculus, 
and also for which all relevant clauses from predecessors have been propagated.\\
The algorithm begins with only the clauses that belong to the node, and gradually propagates clauses from predecessor nodes as they are needed.
For each clause that is selected to be processed, the algorithm first propagates all relevant clauses from predecessors, 
and then performs all inferences with the already saturated clauses.
The process terminates when there are no more non-saturated clauses.

\subsubsection*{The request mechanism}
In figure \ref{clause_import_global} we show the algorithm that imports the relevant clauses for a given clause from all of its transitive predecessors.\\
The algorithm uses the method \lstinline|traverseBF| shown in figure \ref{fig_traverseBF} - 
this method performs a traversal of the CFG starting at n and going backwards and then forward - we present an implementation simply to disambiguate its semantics, the algorithm itself is mostly standard DAG traversal.\\
At each CFG-node in the backwards traversal, the method calls a visitor \\
\lstinline|bVisitor| which returns a set of predecessors relevant for traversal - in our case, each CFG-node may decide not to propagate the request backwards.\\
After backward traversal, the method traverses forward in reverse order, calling the visitor \lstinline|fVisitor| - in our case, this visitor propagates clauses matching the request from direct predecessors.

Our backward visitor \lstinline|importBW| subtracts from a request all cached requests, and if any uncached requests are left it propagates them to its direct predecessor. The cache is updated accordingly.\\
Our forward visitor \lstinline|importFW| propagates all relevant clauses from the direct predecessors and adds them to the current CFG-node.


\subsubsection*{Clause relevance}
Essentially, a clause is relevant for another clause in a calculus if there is an inference in the calculus where both clauses are premises. For a simple example, consider propositional ordered resolution - as resolution is allowed only between clauses with maximal literals with opposing polarity but the same atom, a clause \m{C \lor \underline{A}} (A maximal) is relevant for a clause \m{D \lor \underline{B}} iff \m{A\equiv \lnot B}. \\
For each logical fragment, we have a \lstinline|Request| data structure that encodes the information needed to determine the relevant clauses for a clause (or set of clauses) - for example, for ground ordered resolution, we only need the maximal literal of a clause, hence the request structure includes a set of literals.\\
The relevance criterion for each fragment is implemented in the \lstinline|isRelevant| method referenced in the code - in this section we describe the criterion for EC-graphs and for unit ground superposition - for ordered resolution, for a request \lstinline|r| which is a set of literals and a clause \m{C \lor \underline{A}} (A maximal),  \lstinline|isRelevant| returns  \m{A \in} \lstinline{r}.

\subsubsection*{The request Cache}
The algorithm maintains a cache at each CFG-node, which remembers which requests have already been answered for that node.
When a new request arrives at a node, the cache is subtracted from the request (in order not to re-request previously answered requests from predecessors) and if there are any un-answered requests they are sent to the predecessors. The cache is updated accordingly.
Generally, the request cache structure is identical to a request structure. For ordered resolution, the request cache is a request - a set of literals, the operations \lstinline|add| and \lstinline|subtract| for the cache are simply set union and difference and the \lstinline|isEmpty| operation for the requests simply checks if the set is empty.

The important property of the cache is that a node that has a certain request cached has already imported all relevant clauses (added to the \lstinline|done| set) from all its transitive predecessors.

For ordered resolution, if the request cache at a CFG-node n includes the literal \m{A}, it means that all clauses in all transitive predecessors of n with a maximal literal \m{\lnot A} have been propagated to \lstinline|n.done|.\\
For some fragments, the requests and caches over-approximate the set of relevant clauses (we give an example later in this section) - 
the intuition behind over-approximating the request is that, for most clauses, most CFG-nodes do not have a relevant clause, 
and hence if our cached propagation criterion over-approximates the set of clauses that need to be propagated, it can reduce the number of CFG traversals for later requests. 
In addition, when we apply our algorithm incrementally, a predecessor node can derive a clause that is relevant for a clause already saturated in a previous pass in some successor. The cache allows us to propagate the newly derived clause during the CFG traversal in \lstinline|CFG.verify|, without having to request it again in the successor - we describe this process for EC-graphs in the next section.

\subsubsection*{Clause propagation}
The algorithm propagates clauses (in the method \lstinline|importFW|), by selecting the relevant clauses from the predecessor's \lstinline|done| set, and adding them, relativized as described in \ref{section_path_condition}, to the \lstinline|done| set.
The reason clauses are added to the \lstinline|done| set is to prevent any inference between two imported clauses - such an inference is performed at an earlier node. 
 (we discuss relativization in chapter \ref{chapter:gfole} - for ground unit clauses, we only propagate clauses we can join, and hence, in practice, no relativization takes place - we only mention it here as otherwise the algorithm is unsound). 

\begin{figure}
\begin{lstlisting}
CFG.verify() : Set of unverified assertions
	foreach node n in topological order
		n.verify()
		if (!n.isInfeasible and n.isLeaf)
			result.add(n)

Node.verify()
	todo.enqueue( $\clauses{n}$ )
	while !todo.isEmpty
		c = todo.dequeue
		done.add(c)
		importRelevantClauses(c)
		foreach (d $\in$ inferences(c,done))
			if (d $\notin$ done)
				todo.enqueue(d)
\end{lstlisting}
\caption{Verification algorithm with lazy propagation\\
}
\label{verification_algorithm_v2}
\end{figure}

\begin{figure}
\begin{lstlisting}
Node.importRelevantClauses(c : Clause)
	requestMap := new Map[Node,Request]
	requestMap[this] := makeRequest(c)
		
	traverseBF(this, importBW, importFW)

importBW(n : CFGNode) : Set[CFGNode]
	var r := requestMap[n]
	r.subtract(n.requestCache)
	if (!r.isEmpty && n!=root)
		foreach (p $\in$ $\preds{n}$)
			requestMap[p].add(r)
		return predecessors
	else
		return $\emptyset$

importFW(n : CFGNode)
	r := requestMap[n]
	n.requestCache.add(r)
	foreach (p $\in$ $\preds{n}$)
		foreach (pc $\in$ p.done)
			c := relativize(p,n,pc)
			if (isRelevant(r,c))
				n.done.add(c)
					
\end{lstlisting}
\caption{Basic clause propagation\\
The implementation of \lstinline|traverseBF| is shown in figure \ref{fig_traverseBF}.\\
The clause propagation algorithm first traverses the CFG in reverse topological order, 
starting at the current node. The traversal stops at any CFG-node that already has all relevant clauses cached.\\
After the reverse traversal, the algorithm traverses the CFG forward, propagating relativized clauses 
and updating the cache.
}
\label{clause_import_global}
\end{figure}

\begin{figure}
\begin{lstlisting}
traverseBF(n : CFGNode, bVisitor, fVisitor)
	todoBW = new Set[Node]
	todoBW.add(n)
	todoFW = new Stack[Node]
	while (!todoBW.isEmpty)
		var n := todoBW.removeMax (topological order)
		var ps := bVisitor(n)
		if (ps==$\emptyset$)
			todoFW.push(n)
		else
			todoBW.add(ps)

	while (!todoFW.isEmpty)
		n := todoFW.pop
		fVisitor(n)
			
\end{lstlisting}
\caption{CFG traversal back and forth\\
The implementation is shown only to clarify any ambiguities.\\
The algorithm traverses the CFG backwards from \lstinline|n|, calling \lstinline|bVisitor| on each node traversed.\\
\lstinline|bVisitor| returns the set of predecessors relevant for traversal - if none are returned then traversal does not continue beyond the node (in a DAG a branch node may be relevant for only one successor, in which case it is traversed.)\\
The algorithm then traverses forward from each node where traversal ended, calling \lstinline|fVisitor|.
}
\label{fig_traverseBF}
\end{figure}


In the rest of this section we present the propagation criteria for superposition and for EC-graphs, and compare them.
Our algorithm uses the EC-graph criterion for unit equalities and the superposition criterion for the non-unit and non-ground fragments.

\subsection{Clause propagation criteria}
In this section we compare the propagation criteria for ground unit superposition and for EC-graphs.

\subsubsection*{Superposition based propagation}\label{section:superposition_based_propagation}
We look now at the ground superposition calculus, restricted to unit clauses, shown in figure \ref{gusp_calculus}.

\begin{figure}
$
\begin{array}[c]{llll}
%\vspace{10pt}
\mathrm{res_{=}} &\vcenter{\infer[]{\m{\emptyClause       }                               }{\m{s\neq s}                   }} & 
\parbox[c][1.5cm]{5cm}{}
\\
\mathrm{sup_{=}} &\vcenter{\infer[]{\m{\termRepAt{s}{r}{p} =    t}}{\m{\underline{l}=r} & \m{\underline{s} =    t}}} & 
\parbox[c][1.8cm]{5cm}{
	\m{\sci{1}l = \termAt{s}{p}}\\
	\m{\sci{2}l \succ r}\\
	\m{\sci{3}s \succ t}\\
	\m{\sci{4}s=t \succ l=r}}\\
\mathrm{sup_{\neq}} &\vcenter{\infer[]{\m{\termRepAt{s}{r}{p} \neq t}}{\m{\underline{l}=r} & \m{\underline{s} \neq t}}} & 
\parbox[c][1.8cm]{5cm}{
	\m{\sci{1}l = \termAt{s}{p}}\\
	\m{\sci{2}l \succ r}\\
	\m{\sci{3}s \succ t}}\\
\end{array}
$
\caption{The unit ground superposition calculus \SPU\\
Maximal literals are underlined for clarity}
\label{gusp_calculus}
\end{figure}

It is apparent from the formulation of the ground superposition calculus, that the relevance of a clause for any binary inference depends mostly on its maximal term - similar to ordered resolution where relevance depends on the maximal literal.

For a clause \m{\underline{l}=r} (\m{l \succ r}), the clauses that can participate in an inference with \m{l=r} as the left premise are those in which \m{l} is a sub-term of the maximal term - formally:\\
\m{\underline{s} \bowtie t} where \m{s \succ t}, \m{l \unlhd s},  \m{s \bowtie t \succ l=r} (s is a super-term of \m{l})\\
The clauses that can participate where \m{\underline{l}=r} is the right premise are those in which the maximal term is a sub-term of \m{l} - formally:\\
\m{\underline{s} = t} where \m{s \succ t}, \m{s \unlhd l}, \m{l=r \succ s=t}    (s is a sub-term of \m{l})\\
The clause \m{s \neq t} can only be a right  premise, and then the maximal term of the left premise must be a subterm of \m{s} - formally:\\
\m{l = r} where \m{l \unlhd s}, \m{l \succ r} (includes also \m{s=t})

We can see from the above formulation that for a positive clause \m{l=r} where l is maximal we want all clauses where the maximal term is a sub- or super-term of \m{l}, and for a negative clause \m{s\neq t} (s maximal) we want all positive clauses where the maximal term is a sub-term of s.\\
For example - for the clause \m{\underline{f(b)}=a}, the clauses \m{\underline{f(b)}=b, \underline{g(f(b))}=c,}\\
\m{\underline{b}=a} are all relevant, 
but the clauses \m{\underline{b} \neq a, \underline{f(c)}=f(b),\underline{c}=b, \underline{f(c)}\neq f(b)} are not.

The method \lstinline|isRelevant| for the ground superposition fragment implements the criteria described above.

For ground superposition the requests and cache is implemented as follows:\\
The cache includes two sets of terms, \lstinline|ts$_{\m{lhs}}$| and \lstinline|ts$_{\m{rhs}}$| - the first for terms that occur as maximal positive terms and the second for terms that occur as a sub-term of a maximal term in a clause of any polarity.
If a term t is in \lstinline|ts$_{\m{lhs}}$|, then any clause with a maximal term a (non-strict) super-term of t is propagated.\\
For a term t in \lstinline|ts$_{\m{rhs}}$|, any positive clause with a maximal term that is a (non-strict) sub-term of t is propagated.\\
The \lstinline|subtract| method is simply set difference on each of the two sets.\\
When updating the cache in \lstinline|requestFW|, for a positive clause \m{\underline{l}=r} we add l to \lstinline|ts$_{\m{lhs}}$| and the sub-term closure of \m{l} to \lstinline|ts$_{\m{rhs}}$|. For a clause \m{\underline{s}\neq t} we add the sub-term closure of s to \lstinline|ts$_{\m{rhs}}$|.\\
A clause \m{\underline{s} \bowtie t} is relevant for a request \lstinline|r| (\lstinline|isRelevant(r,C)|) if it is positive and a (non-strict) sub-term of \m{s} is in \lstinline|ts$_{\m{lhs}}$|, or if C is of any polarity and a (non-strict) sub-term s is in \lstinline|ts$_{\m{rhs}}$|.

For example, if we request propagation for the clauses \m{\underline{f(b)}=a} at the CFG-node n, \lstinline|ts$_{\m{lhs}}$| of the cache will include, after propagation, the term \m{f(b)}, and \lstinline|ts$_{\m{rhs}}$| will include \m{b,f(b)} so that if, at a later stage, any CFG-node requests relevant clauses for the clause \m{\underline{b}\neq a}, the request will not be propagated further.\\
The advantage of such a formulation for the cache is that, when we limit the set of terms that can occur at a CFG-node (in all clauses at that node), we can bound the number of times this node has to answer requests. We limit the set of terms both by size bounds and by scoping, as detailed in chapters \ref{chapter:bounds} and \ref{chapter:scoping}.

\subsubsection*{Congruence closure based propagation}
Even for a large set of equalities without any dis-equality,
many superposition derivations can take place, although the set is trivially satisfiable.
More generally, even in a set with dis-equalities, we can find a subset of the equalities that is sufficient to show a refutation if there is one - as we have done in the algorithm in figure \ref{fig_lazy_congruence}.

\bigskip

\noindent
For EC-graph based propagation, assume we have the EC-graph \m{g_n} at node n and the graph \m{g_p} at its direct predecessor p.
Our criterion for propagation from figure \ref{fig_lazy_congruence} is that any equality on a term represented in \m{g_n} be propagated - for each \GT{} \m{u \in g_n} and \GT{} \m{v \in g_p}, the equalities in \eqs{v} are relevant iff \m{\terms{u} \cap \terms{v} \neq \emptyset}.\\
We can send a request that contains all the \GTs{} of \m{g_n}, and return any \GT{} of \m{g_p} that shares a term with any of the requested \GTs{} - \m{g_n} then \lstinline|assumes| \eqs{v} for each such propagated \GT{}. In the next section we show how to decide this criterion efficiently.


\subsubsection*{Comparison of the propagation criteria}
In this sub-section we motivate our choice for using the EC-graph based propagation mechanism rather than superposition for unit ground clauses. The main reason for this choice is that join calculations, and especially incremental join calculations, are more efficient using EC-graphs than ground unit superposition. A second reason is that the representation of a propagation cache is more efficient with EC-graphs as we need to index one object for an entire EC, while for superposition we need to index each term separately, even if two terms were proven equal in the CFG-nodes where they are cached. We demonstrate these two reasons below.

The main difference between the two approaches is that superposition only considers one side of an equation for propagation but must import both sub-terms and super-terms, while the congruence closure (EC-graph) based approach considers both sides of an equation, but only imports sub-terms.

Consider the case where the predecessor clause set is \\
\m{\{\underline{d}=a, \underline{f(b)}=e,\underline{f(c)}=a\}},
and the clause set at n is \m{\{\underline{b}=a,\underline{d} \neq b\}}.\\
Superposition imports \m{\underline{d}=a, \underline{f(b)}=e} while an EC-graph imports \\
\m{d=a, f(c)=a} - only \m{d=a} is actually needed, so both approaches imported (different) useless clauses.

\subsubsection*{Joins}
For CFGs with joins, we cannot simply propagate a clause from the predecessor of a join to the join node, as the clause might not hold in the other joinee. Instead, we can only propagate equalities agreed by both joinees or relativized clauses, which are clauses guarded by the branch condition of the corresponding branch for the join. We have discussed the problem of joining two congruences in the previous section, and we show here an example that compares how EC-graphs and superposition differ in handling joins. The main outcome is that superposition requires more CFG-traversals in order to calculate a join, while the congruence closure based relevance criterion suffices for joins, when there is a join in the unit-ground fragment.

Saturating a set of equalities with superposition essentially establishes a rewrite relation for ground terms at every node which has a unique normal form for each ground term. The relation is only partially represented (by equalities) at the node (similar to the congruence being approximated by an EC-graph), and when a term is requested in the CFG-node, we ensure that the relevant part of the rewrite relation is propagated from predecessors.\\
The main issue we have encountered is joins - the problem is, given two strongly normalizing rewrite relations which agree on an ordering, and given a term t, find the normal form of t at the intersection of the rewrite relations. \\
The intersection of the rewrite relations of two strongly normalizing rewrite systems that agree with the same total well founded ordering and are finitely generated is also strongly normalizing, by the simple argument that the normal form of a term t at the intersection is the minimum of the intersection of the equivalence classes of t in each rewrite relation, where the part of the equivalence class of t at the intersection that is smaller or equal to t is finite because that part of the rewrite relation is also finitely generated and agrees with a well founded ordering).

The critical difference between EC-graphs and superposition is that EC-graphs represent a fully-reduced rewrite system - the set of rewrite rules rewrite each term to its normal form in one step - while superposition constructs a left-reduced rewrite system - each term is the left-hand-side of exactly one rewrite rule, but the right hand side is not necessarily reduced (unit ground superposition constructs the left-reduced rewrite system when we can eliminate clauses using the simplification rule \m{simp_=} from figure \ref{fig_superposition_simp} - for our purpose it is only important that the right hand side is not fully reduced) .\\
For example, the rewrite system \s{\underline{d}=c,\underline{c}=b} is not fully reduced but is left reduced, while the (unique) fully reduced versions is \s{\underline{d}=b, \underline{c}=b}. A non-left-reduced system is, for example \s{\underline{d}=c,\underline{d}=b}, from which superposition (with simplification) derives \s{\underline{d}=c,\underline{c}=b}. With a fully reduced system, we can reach any member of the EC of a term t in two steps (two equations) - one step finds the normal form, and the second step uses any rule with the normal form as a right hand side. When the system is only left reduced, we might need more steps - e.g. for \m{d=c,c=b,b=a,e=a} we need four steps to reach d from e. When looking for the normal form of a term at a join, we have to consider all members of the EC of that term on both joinees and select the minimum of the intersection - hence the number of steps needed to find this minimum is important.\\
Obviously, constructing a fully reduced rewrite system is costlier than a left-reduced one, but, in our experience, it pays off in more efficient join operations.


\begin{figure}
\begin{lstlisting}
$\m{b:}$
if (c1)
	$\m{p_{0}}:$
	assume $\m{\underline{f(b)}=a}$
	assume $\m{\underline{c}=b}$
	assume $\m{\underline{d}=c}$
	assume $\m{\underline{e}=d}$
	$\m{p_{1}}:$
	... //unrelated clauses
else
	$\m{p_{2}}:$
	assume $\m{\underline{f(e)}=a}$
	$\m{p_3}:$
	... //unrelated clauses
$\m{n}:$
assume $\m{\underline{f(e)}=x}$
// Here $\comm{\m{f(e)=a}}$ holds
$\m{na:}$
	assert $\m{\underline{a}=x}$ //negated $\comm{\m{\underline{a} \neq x}}$
\end{lstlisting}
\caption{Propagation condition comparison with joins}
\label{snippet3.14.0}
\end{figure}

\noindent
Consider the example in figure \ref{snippet3.14.0}.\\
For superposition, we can propagate relativized (non-unit) clauses (assuming the joined branch condition \m{C}) as follows:

\bigskip
\noindent
Initially, no CFG-node has any valid inference, and, except for \m{n,n_a}, no clauses to import - we look at the proof process in \m{n}.

\bigskip
\noindent
Initially we request (rhs) \s{f(e),e} and propagate the relativized clauses:\\
\s{\lnot C \lor \underline{e}=d, C \lor \underline{f(e)}=a}

\bigskip
\noindent
We use these to derive:\\
\s{\lnot C \lor \underline{f(d)} = x, C \lor a = x}

\bigskip
\noindent
Next we request (rhs) \s{d,f(d),a} and propagate:\\
\s{\lnot C \lor \underline{d} = c}

\bigskip
\noindent
Which allows us to derive:\\
\s{\lnot C \lor \underline{f(c)} = x}

\bigskip
\noindent
Next we request \s{c,f(c)} and propagate:\\
\s{\lnot C \lor \underline{c}=b}

\bigskip
\noindent
Which allows us to derive:\\
\s{\lnot C \lor \underline{f(b)} = x}

\bigskip
\noindent
Next we request \s{b,f(b)} and propagate:\\
\s{\lnot C \lor \underline{f(b)}=a}

\bigskip
\noindent
Which allows us to derive:\\
\s{\lnot C \lor \underline{a} = x}

\bigskip
\noindent
Finally, at \m{na}:\\
We request \s{a,x} and propagate:\\
\s{\lnot C \lor \underline{a} = x,C \lor \underline{a} = x}\\
We then derive, together with \m{a \neq x}:\\
\s{\lnot C \lor x \neq x,C \lor x \neq x} which allow us to find the refutation.

\bigskip
\noindent
Note that we needed to traverse the CFG several times.\\
Note also that if we apply our simplification inference \m{simp_{res}} from figure \ref{fig_superposition_simp} - \\
$\vcenter{\infer[]{\m{C}                      }{\cancel{\m{C \lor A }} & \cancel{\m{C \lor \lnot A}}}} \parbox[c][1.2cm]{3cm}{}$,\\
we can derive:\\
$\vcenter{\infer[]{\m{\underline{a}=x}                      }{\cancel{\m{C \lor \underline{a} = x}} & \cancel{\m{\lnot C \lor \underline{a} = x}}}} \parbox[c][1.2cm]{3cm}{}$\\
And save some derivation steps. This simplification rule is designed exactly for such cases.

\bigskip
\noindent
For EC-graph, we show in the next section a join algorithm that derive \m{a=f(e)} at the join, 
and can prove the program without any non-unit clauses. 
In addition, the superposition approach traversed the CFG several times, while our approach requires just one traversal.


\bigskip
\noindent
\textbf{Summary:}\\
While both propagation criteria have advantages and disadvantages, and both propagate useless clauses, in our setting we have found the congruence closure based approach to be more suitable in the unit ground equality case.
For larger fragments (non ground and/or non-unit) we will use a hybrid method, where the unit fragment will serve as a base.
In this chapter we only discuss unit ground equalities using EC-graphs from now on, and in chapter \ref{chapter:gfole} we handle joining non-unit clauses.

\newpage
\subsection{Ground unit equality propagation}
In this section we describe the data structure and algorithm used to propagate ground unit equalities.
This algorithm and data structure form the basis of all the other fragments we consider in this thesis.


\bigskip
\noindent
We use the verification algorithm described in figures \ref{verification_algorithm_v2} and \ref{clause_import_global} with the following changes:
\begin{itemize}
\item An EC-graph is used to represent the set \lstinline|done|
\item Requests are represented as sets of \GFAs{} over the EC-graph of the relevant CFG-node
\item The request cache is a set of \GFAs{}, which includes all \GFAs{} in the EC-graph and an additional set of \GFAs{}
\item Each request is translated before being sent to the predecessors, so that the EC-tuple in the \GFA{} is over \GTs{} of the predecessor's EC-graph
\item Instead of propagating relativized clauses from predecessors, each node performs a \emph{meet} of its EC-graph with the join of the EC-graphs of the predecessors, for the requested \GFAs{} (this is just an intuitive description - described in detail later)
\item We maintain a links between each \GT{} in the EC-graph of each CFG-node and the corresponding \GTs{} of EC-graphs of predecessor CFG-nodes. Corresponding \GTs{} are those that share a term. We use these links to translate requests quickly, and also to perform the join and meet
\end{itemize}

We describe first our data-structure and algorithm for sequential nodes (without any join) and in the next section show the needed changes to support joins.

\subsubsection*{Completeness}
Our algorithm aims to propagate equality information in the CFG so that dis-equalities can be refuted and other logical fragments can use the equality information. Equality information should be propagated on-demand, so we have to define how equality information is requested. For EC-graphs, we request equality information by adding terms to the EC-graph. 
In general, we assume each CFG-node n has a set of requests \m{R_n} which are clauses (in our case, unit equalities).

An equality \m{s=t} \newdef{holds} at a CFG-node n (\m{n \models s=t}) iff it holds on all paths leading to n, and it holds on a path P (\m{P \models s=t}) if it is entailed by the clauses on the path (\m{\clauses{P}}).\\
We define soundness and completeness for any algorithm that, given a set of requests \m{R_n} per CFG-node, annotates each CFG-node with a set of clauses \m{\phi_n}  - in our algorithm from figure \ref{verification_algorithm_v2} this is the set \m{done_n}, and for the ground unit equality fragment it is represented using an EC-graph.\\
An equality \m{s=t} is \newdef{proved} at a CFG-node n if \m{s=t \in \phi_n}.\\
An annotation for a CFG-node n is \newdef{sound} if,\\ for each CFG-node n and equality \m{s=t \in R_n}, \m{s=t \in \phi_n \Rightarrow n \models s=t}.\\
An annotation for a CFG-node n is \newdef{complete} if, \\ for each CFG-node n, for each \m{s=t \in R_n}, \m{n \models s=t \Rightarrow s=t \in \phi_n}.\\
For EC-graphs, we use the notation \m{g_n} for the EC-graph of a node \m{n} and \m{[t]_n} for the \GT{} (EC-node in the graph) that represents t in \m{g_n}, if \m{t \in \terms{g_n}}.\\
We use the notation \m{g_n \models s=t} to denote that the equality s=t is entailed by the graph \m{g_n} and that \m{s,t} are represented in \m{g_n} - formally \m{s,t \in \terms{g_n}} and \m{[s]_n = [t]_n}. For a node n, \m{s=t \in \phi_n} is defined as \m{g_n \models s=t}. 

Equality information is requested for EC-graphs by adding the terms to the graph - so, by definition, \m{s=t \in R_n \Rightarrow s,t \in \terms{g_n}}.\\
Soundness for EC-graph annotations is, hence: \m{g_n \models s=t \Rightarrow n \models s=t}.\\
Completeness for EC-graphs is, for each \m{s=t \in R_n}, \m{n \models s=t \Rightarrow g_n \models s=t}.\\
In fact, the completeness definition is too strong for CFGs with joins, we use this definition for CFGs without joins and discuss a weaker condition for DAGs in the next section. \\
A CFG-node with exactly one predecessor is a \newdef{sequential node} while a CFG-node with two or more predecessors is a join node.

Our data-structure and algorithm are incremental in two ways:\\
We can run the algorithm with a certain set of requests \m{R_n}, and guarantee completeness for the resulting annotation (for join-less CFGs, otherwise a weaker guarantee). 
At this stage we can add more requests to \m{R_n} (that is, add more terms to the EC-graphs of some nodes) and the algorithm manipulates our data structure to ensure that the completeness guarantee is re-established.\\
The other option is to add more assumptions (invoke \lstinline|assume(s=t)| for some equations \m{s=t} in some of the CFG-nodes) in which case the algorithm again re-establishes the completeness guarantee (as the relation \m{n \models s=t} might have changed for some n and \m{s=t \in R_n}). Both the above operations model the way other fragments communicate with our fragment.

\subsubsection*{Source links}\label{section:ugfole:sources}
In order to facilitate efficient equality propagation and efficient incremental updates, we maintain a link between each \GT{} (EC-node) in the EC-graph of each CFG-node and the \GTs{} of the EC-graphs of its direct predecessors that share a term with it. These links allow us to translate requests to predecessors and translate responses from predecessors, and also allow us to determine quickly if there is any relevant equality information in any of the transitive predecessors of a CFG-node.\\
We also use these links for incremental updates of CFG-nodes, where a stale link (an edge to a \GT{} that has been merged with another \GT{}) allows us to update the EC-graph of a CFG-node incrementally after the EC-graphs of direct predecessors have been updates (using the \lstinline|mergeMap|).

\noindent
The sources function is part of the state of an EC-graph which is maintained by our algorithm. The function returns, for a CFG-node n, a direct predecessor p and \GT{} \m{u \in g_n} a set of \GTs{}.\\
We use the notation \sources{n}{p}{u} for the source of \m{u \in g_n} in the predecessor p.
When there is no ambiguity, we use \sources{n}{}{u}, \sources{}{p}{u} or \sources{}{}{u}.\\
We extend the sources function to tuples, where the sources of a tuple are the Cartesian product of the tuple of sources.
For example,\\
if \m{\sources{}{}{[a,b]}=\s{[a],[b]}} \\
then \m{\sources{}{}{([a,b],[a,b])}=\s{([a],[a]),([a],[b])([b],[a])([b],[b])}}.

\noindent
The intended invariant of the sources function states that the sources of a \GT{} \m{u \in g_n} are the \GTs{} \m{v \in g_p} s.t. u and v share a term - formally:\\
For a CFG-node n and a direct predecessor p,\\
\m{\forall u \in g_n \cdot}\\
\m{~~~\sources{n}{p}{u} = \s{v \in g_p \mid \terms{u} \cap \terms{v} \neq \emptyset}}



%
%\noindent
%\textbf{Notation}\\
%We write \m{p.P.n} for a CFG-path of length at least two that starts at \m{p} and ends at \m{n} (so \m{n \neq p}).\\

\begin{figure}
\begin{lstlisting}
$\node{n_0}:$
assume $\m{f(a)=g(a)}$
	// $\m{\GFAs{}: a(),f([a]), g([a])}$
	// $\m{\GTs{}:[a],[f(a),g(a)]}$
$\node{n_1}:$
assume $\m{a=b}$
	// $\m{\GFAs{}: a(),b()}$
	// $\m{\GTs{}:[a,b]}$
	// $\m{\sources{}{}{[a,b]_1}=\s{[a]_0}}$
$\node{n_2}:$
assert $\m{f(b)=g(b)}$ //negated $\comm{\m{f(b) \neq g(b)}}$
	// $\m{\GFAs{}: b()}$
	// $\m{\GTs{}:[b]}$
	// $\m{\sources{}{}{[b]_2}=\s{[a,b]_1}}$
\end{lstlisting}
\caption{propagation sources\\
The state before $\m{n_2.}$\lstinline|makeTerm(f([b]))|\\
We list the sources for each EC at each CFG-node.\\
The source invariant holds for the above example.
}
\label{snippet3.16b}
\end{figure}

\noindent
An example for the sources function is shown in figure \ref{snippet3.16b}.

\bigskip
\noindent
We present now a formulation of this invariant that drives the way our algorithm establishes the invariant.
The formulation is local in the sense that, for each \GT{}, detecting a violation of the invariant or fixing a violated invariant involves traversing only a constant number of edges (both \GFA{} and source edges). 
The above formulation does not satisfy locality as calculating the set \terms{u} requires recursive descent in the structure of \m{u} of non-constant depth. In addition, we present the invariant as a conjunction of quantified implications. 
If the invariant does not hold for some assignment to the quantified variables, it is fixed by changing the right-hand-side of the implication. We present several such conjuncts in this section that together form the invariant established by our algorithm and are presented in a way that suggests how the algorithm establishes them.

\begin{figure}[H]
\textbf{The source invariant:}\\
For a CFG-node n and a direct predecessor p.\\
\m{\forall u \in g_n, \fa{f}{t} \in u, \tup{s} \in \sources{}{}{\tup{t}} \cdot}\\
\m{~~~\fa{f}{s} \in \gfas{p} \Rightarrow [\fa{f}{s}] \in \sources{n}{p}{u}}
\end{figure}

\noindent
For example, consider the state in figure \ref{snippet3.16b_graph}. \\
The figure shows the state of three consecutive CFG-nodes - \m{n_0,n_1,n_2}.\\
Source edges are marked in blue. \\
Consider the following assignment to the quantified variables:\\
The CFG-node n is \m{n_1} and the predecessor p is \m{n_0},\\
The \GT{} \m{[f(a)]_1} (assigned to u) contains the \GFA{} \m{f([a,b])_1)} (assigned to \fa{f}{t}) 
where the tuple \m{([a]_0)} (assigned to \tup{s}) satisfies \m{([a]_0) \in \sources{n}{p}{[a,b]_1}} and also 
\m{f([a]_0) \in \gfas{0}} - hence the invariant implies that there is a source edge between the \GTs{} \m{[f(a)]_1} and \m{[f(a)]_0} (\m{[f([a]_0)]_0 \in \sources{n}{p}{[f([a])]_1}}).\\
In our example the invariant holds.
The main loop of the algorithm fixes such local inconsistencies (that arise from node \GT{} merging and other operations).
In our example, if the condition does not hold during the run of our algorithm (that is, there was no source edge from \m{[f(a)]_1} to \m{[f(a)]_0}), our algorithm would establish the invariant locally by adding that edge.

\subsection{Propagation using sources}
We show now show how the sources function is used for information propagation.
In the example in figure \ref{snippet3.16b}, we have not yet added the terms \m{f(b),g(b)} to \m{g_2} and hence we are missing some equality information at \m{n_2} in order to prove the assertion (namely, \m{n_2 \models f(b)=g(b)} - but \m{g_2 \not\models f(b)=g(b)}.

In order to ensure that the information is propagated, we use a local propagation invariant, that works together with the source correctness invariant to ensure that enough information is propagated. 
The local propagation invariant for sequential nodes (with one predecessor) ensures that, for an CFG-node n and a \GT{} \m{u \in g_n}, u 
has all the terms of all its sources - formally:


\begin{figure}[H]
\textbf{The sequential propagation invariant, part 1:}\\
For a sequential CFG-node n and a direct predecessor p.\\
\m{\forall u \in g_n, v \in \sources{}{}{u}, \fa{f}{s} \in v \cdot}\\
\m{~~~\exists \fa{f}{t} \in u \cdot \tup{s} \in \sources{}{}{\tup{t}}}
\end{figure}

\noindent
We use the notation \m{\fa{f}{t} \in u} to denote that the \GFA{} \fa{f}{t} is in the \lstinline|gfas| field of the \GT{} u.\\
The invariant states that each \GFA{} in each source of u has a corresponding \GFA{} in u.
This invariant is maintained by our algorithm and ensures that the set of terms of each \GT{} is a superset of the union of sets of terms of all its sources. 
Note that as our EC-graph is kept congruence closed, this invariant implies that no two \GTs{} share the same source (in a sequential node) - otherwise two \GTs{} that share a source would share a \GFA{}.

In our example in figure \ref{snippet3.16b}, we can see that this invariant does not hold - the EC \m{[b]_2} does not contain the \GFA{} \m{a()}. Our algorithm fixes the invariant when invoking \lstinline|makeTerm(b,())|, so that essentially the equality \m{a=b} is propagated eagerly. We detail in the next section how this invariant is established, for now we just assume that whenever a \GT{} is missing some \GFAs{} according to our invariant, the algorithm adds the missing \GFAs{}. Note that the invariant is phrased in terms of the sources function and refers only to \GTs{} that are a constant distance from each other (in terms of \GFA{} and source edges).


%\noindent
%We now detail the working of our algorithm when we invoke \\
%\lstinline|n$\m{_2}$.makeTerm(f,$\m{[b]}$)|.\\
%We assume the invariant for \m{[b]_2} has been fixed and so \m{\terms{[b]_2} = \s{a,b}}.\\
%In addition, we assume that the \GTs{} \m{[f(b)]_2,[g(b)]_2} have not yet been added to \m{g_2}.
%
%\bigskip
%\noindent
%The first step the algorithm makes is send a request for the set of \GFAs{} \\
%\s{f([a,b]_2)}.\\
%This request is propagated through the sources function to \m{n_1} as \s{f([a,b]_1)} 
%and to \m{n_0} as \s{f([a]_0)}.\\
%The response to the request (a set of \GTs{}) from \m{n_0} is \m{[f(a),g(a)]_0}.\\
%We can find the response easily - for each \GFA{} \fa{f}{s} in the request, 
%we can use the \lstinline|p.superTerms| field to look for e.g. the super-terms of \m{s_0} and look for the \GFA{} \fa{f}{s} (we give some details on how we implement the \lstinline|superTerms| field in the implementation section) - we only need a constant number of lookups to find the right \GFA{}, while if we did not translate the request through the sources function we would have had to match the request with our request \GFAs{} from the constants up - at a complexity that depends on number of all \GFAs{} used to construct the \GT{}.\\
%Our response is the \GT{} \m{[f(a),g(a)]_0}.\\
%\m{n_1} sees a positive response (non-empty set) and adds a new singleton \GT{} to \m{g_1} that contains the \GFA{} \m{f([a]_1)} (from the request) and adds a source edge from the new \GT{} \m{[f(a)]_1} to \m{[f(a),g(a)]_0}.\\
%Now our sequential propagation invariant is broken at \m{n_1}: there is no \GFA{} in \m{[f(a)]_1} that corresponds with the \GFA{} \m{g([a]_0)}.\\
%Our algorithm fixes the invariant locally by adding the \GFA{} - we do this by using the inverse of the source function - \m{[a]_0} has an inverse source \m{[a,b]_1} and hence we add the \GFA{} \m{g([a,b]_1} to \m{[f(a)]_1}.\\
%The \lstinline|superTerms| field of \m{g_1} is updated appropriately.\\
%Next, we perform similar steps at \m{g_2} and end up with the following ECs:\\
%\m{[a,b]_2, [f(a),g(a)]_2}.
%
%\noindent
%A graphic representation of the final state of the system is shown in figure \ref{snippet3.16b_graph}.
%We show source edges using blue dashed arrows.
%We also show the dis-equality encoded as an edge in the graph.
%
%\bigskip
%\noindent
%We have used the sources edges both to translate the request, to translate the response and to propagate equality information.\\
%We have not yet described how our request cache works, how incremental updates are performed and how the propagation invariant is established. 


\begin{figure}
\begin{tikzpicture}
  \node[gttn] (1)              {$$};
	\node[gl]   (1l) [below = 0 of 1] {\m{n_0}};

  \node[gtn,label=center:\scriptsize\m{a}]  (2) [above = 0.5cm of 1] {\phantom{a,b}};

	\node[gtn]  (6)  [above = 2.5cm of 1] {$\stackB{f(a)}{g(a)}$};
	\draw[gfa]  (6) to[out=-100 , in=100] node[el,anchor=east]  {\m{f}} (2);
	\draw[gfa]  (6) to[out= -80 , in= 80] node[el,anchor=west] {\m{g}} (2);

%%%%%%%%%%%%%%%%%%%%%%%%%%%%%%%%%%%%%%%%%%%%%%%%%%%%%%%%%%%%%%
	\node[gttn] (11)  [right = 3cm of 1] {$$};
	\node[gl]   (11l) [below = 0 of 11]   {\m{n_1}};

	\node[gtn]  (12) [above = 0.5cm of 11] {\m{a,b}};

	\node[gtn]  (16)  [above = 2.5cm of 11] { $\stackB{f(a),f(b)}{g(a),g(b)}$};
	\draw[gfa]  (16) to[out=-100 , in=100] node[el]             {\m{f}} (12);
	\draw[gfa]  (16) to[out=- 80 ,in=  80] node[el,anchor=west] {\m{g}} (12);
				
%%%%%%%%%%%%%%%%%%%%%%%%%%%%%%%%%%%%%%%%%%%%%%%%%%%%%%%%%%%%%%

	\node[gttn] (21)  [right = 3.5cm of 11] {$$};
	\node[gl]   (21l) [below = 0 of 21]   {\m{n_2}};

	\node[gtn]  (22) [above = 0.5cm of 21] {\m{a,b}};

	\node[gtn]  (26)  [above = 2.5cm of 21] {$\stackB{f(a),f(b)}{g(a),g(b)}$};
	\draw[gfa]  (26) to[out=-100 ,in= 100] node[el]             {\m{f}} (22);
	\draw[gfa]  (26) to[out=- 80 ,in=  80] node[el,anchor=west] {\m{g}} (22);

%%%%%%%%%%%%%%%%%%%%%%%%%%%%%%%%%%%%%%%%%%%%%%%%%%%%%%%%%%%%%%

	\node(12a) [left = 0.5cm of 12] {};
	\draw[se] ( 12.180) to (   2.0);

	\draw[se] (22) to  (12);

	\draw[se] (16) to  ( 6);
	\draw[se] (26) to  (16);

	\draw[ie] (26) to[loop] node[el,above] {\m{\neq}} (26);

\end{tikzpicture}
\caption{
Source edges\\
We omit tuples as we only use unary functions.\\
Dashed arrows represent dis-equalities.\\
\textcolor{blue}{ Blue dashed arrows} represent source edges.
}
\label{snippet3.16b_graph}
\end{figure}

\bigskip

\noindent
We now demonstrate how equalities are actually propagated by our algorithm.
Consider the example in figure \ref{snippet3.17_graph}.\\
We describe how our algorithm performs \m{n_1.}\lstinline|makeTerm(f,($\m{[a]_1}$))|, ensuring that \m{g_1 \models f(a)=g(a)}.\\
First, the \GFA{} \m{f([a]_0)} is searched in the \lstinline|superTerms| field of \m{n_1}, and not found.\\
As the \GFA{} does not yet exist (the term \m{f(a)} is not represented in \m{g_1}), we create a new \GT{} that includes that \GFA{} and map it in the \lstinline|superTerms| field as a super-term of \m{[a]_0}. We also add a source-edge to the new \GT{}, as mandated by the source invariant.\\
The state is shown in figure \ref{snippet3.17_graph.1}, we mark in red the missing parts of the EC-graph that are needed in order to propagate the equality \m{f(a)=g(a)}.\\
For each source-edge added to a \GT{}, our algorithm looks at each \GFA{} of the new source, and adds the missing ones (in order to establish the propagation invariant).\\
In our case, the \GFA{} \m{f([a]_1)} is present but the \GFA{} \m{g([a]_1)} is missing.\\
We look at the source \GFA{} \m{g([a]_0)} and look at the inverse source of the tuple \m{([a]_0)} - in this case \m{([a]_1)}.\\
We add the \GFA{} \m{g([a]_1)} to our new \GT{} and we are done. The result is shown in figure \ref{snippet3.17_graph.2}.\\
In the next example, we show what happens when the inverse source of the tuple is empty.



\begin{figure}
\begin{subfigure}[t]{0.49\textwidth}
\framebox[\textwidth]{
\raisebox{0pt}[0.2\textheight][0pt]
{
\begin{tikzpicture}
  \node[gttn] (1)              {$$};
	\node[gl]   (1l) [below = 0 of 1] {\m{n_0}};

  \node[gtn,label=center:\scriptsize\m{a}]  (2) [above = 0.5cm of 1] {\phantom{b}};

	\node[gtn]  (6)  [above = 2.5cm of 1] {$\stackB{f(a)}{g(a)}$};
	\draw[gfa]  (6) to[out=-100 , in=100] node[el,anchor=east]  {\m{f}} (2);
	\draw[gfa]  (6) to[out= -80 , in= 80] node[el,anchor=west] {\m{g}} (2);

%%%%%%%%%%%%%%%%%%%%%%%%%%%%%%%%%%%%%%%%%%%%%%%%%%%%%%%%%%%%%%
	\node[gttn] (11)  [right = 3cm of 1] {$$};
	\node[gl]   (11l) [below = 0 of 11]   {\m{n_1}};

	\node[gtn,label=center:\scriptsize\m{a}]  (12) [above = 0.5cm of 11] {\phantom{b}};

%%%%%%%%%%%%%%%%%%%%%%%%%%%%%%%%%%%%%%%%%%%%%%%%%%%%%%%%%%%%%%

	\draw[se] ( 12.180) to (   2.0);
\end{tikzpicture}
}}
\caption{
Before \m{n_1.}\lstinline|makeTerm(f,($\m{[a]_1}$))|
}
\label{snippet3.17_graph}
\end{subfigure}
\begin{subfigure}[t]{0.49\textwidth}
\framebox[\textwidth]{
\raisebox{0pt}[0.2\textheight][0pt]
{
\begin{tikzpicture}
  \node[gttn] (1)              {$$};
	\node[gl]   (1l) [below = 0 of 1] {\m{n_0}};

  \node[gtn,label=center:\scriptsize\m{a}]  (2) [above = 0.5cm of 1] {\phantom{b}};

	\node[gtn]  (6)  [above = 2.5cm of 1] {$\stackB{f(a)}{g(a)}$};
	\draw[gfa]  (6) to[out=-100 , in=100] node[el,anchor=east]  {\m{f}} (2);
	\draw[gfa]  (6) to[out= -80 , in= 80] node[el,anchor=west] {\m{g}} (2);

%%%%%%%%%%%%%%%%%%%%%%%%%%%%%%%%%%%%%%%%%%%%%%%%%%%%%%%%%%%%%%
	\node[gttn] (11)  [right = 3cm of 1] {$$};
	\node[gl]   (11l) [below = 0 of 11]   {\m{n_1}};

	\node[gtn,label=center:\scriptsize\m{a}]  (12) [above = 0.5cm of 11] {\phantom{b}};

	\node[gtn]  (16)  [above = 2.5cm of 11] {\stackB{f(a)}{\textcolor{red}{g(a)}}};
	\draw[gfa]  (16) to[out=-100 , in=100] node[el]             {\m{f}} (12);
	\draw[mgfa]  (16) to[out=- 80 ,in=  80] node[ml,anchor=west] {\m{g}} (12);
				

%%%%%%%%%%%%%%%%%%%%%%%%%%%%%%%%%%%%%%%%%%%%%%%%%%%%%%%%%%%%%%

	\draw[se] ( 12.180) to (   2.0);

	\draw[se] (16.180) to ( 6.0);

\draw[draw=none, use as bounding box] (current bounding box.north west) rectangle (current bounding box.south east);

\end{tikzpicture}
}}
\caption{
After adding \m{f([a]_1)}\\
\m{g([a]_1)} is missing
}
\label{snippet3.17_graph.1}
\end{subfigure}

\begin{subfigure}[t]{0.49\textwidth}
\framebox[\textwidth]{
\raisebox{0pt}[0.2\textheight][0pt]
{
\begin{tikzpicture}
  \node[gttn] (1)              {$$};
	\node[gl]   (1l) [below = 0 of 1] {\m{n_0}};

  \node[gtn,label=center:\scriptsize\m{a}]  (2) [above = 0.5cm of 1] {\phantom{b}};

	\node[gtn]  (6)  [above = 2.5cm of 1] {$\stackB{f(a)}{g(a)}$};
	\draw[gfa]  (6) to[out=-100 , in=100] node[el,anchor=east]  {\m{f}} (2);
	\draw[gfa]  (6) to[out= -80 , in= 80] node[el,anchor=west] {\m{g}} (2);

%%%%%%%%%%%%%%%%%%%%%%%%%%%%%%%%%%%%%%%%%%%%%%%%%%%%%%%%%%%%%%
	\node[gttn] (11)  [right = 3cm of 1] {$$};
	\node[gl]   (11l) [below = 0 of 11]   {\m{n_1}};

	\node[gtn,label=center:\scriptsize\m{a}]  (12) [above = 0.5cm of 11] {\phantom{b}};

	\node[gtn]  (16)  [above = 2.5cm of 11] {\stackB{f(a)}{g(a)}};
	\draw[gfa]  (16) to[out=-100 , in=100] node[el]             {\m{f}} (12);
	\draw[gfa]  (16) to[out=- 80 ,in=  80] node[el,anchor=west] {\m{g}} (12);
				

%%%%%%%%%%%%%%%%%%%%%%%%%%%%%%%%%%%%%%%%%%%%%%%%%%%%%%%%%%%%%%

	\draw[se] ( 12.180) to (   2.0);

	\draw[se] (16.180) to ( 6.0);
\draw[draw=none, use as bounding box] (current bounding box.north west) rectangle (current bounding box.south east);

\end{tikzpicture}
}}
\caption{
After adding \m{g([a]_1)}\\
Invariant is satisfied
}
\label{snippet3.17_graph.2}
\end{subfigure}

\caption{Example for the propagation invariant\\
Missing parts of the graph are in red}
\end{figure}



\bigskip
\noindent
Consider the state in figure \ref{snippet3.18_graph.0}.\\
We add the \GT{} for \m{h(a)} - shown in figure \ref{snippet3.18_graph.1}.\\
We now proceed as before, looking for an inverse source for \m{[f(a)]_0} - none is found.\\
In this case we add a new empty \GT{}, and attach it with a source-edge to \m{[f(a)]_0}.\\
The result is shown in figure \ref{snippet3.18_graph.2}.\\
This \GT{} has no members yet, but it has a source - so the propagation invariant forces us to look for an inverse source for the \GFA{} 
\m{f([a]_0)} with which propagation is complete - shown in figure \ref{snippet3.18_graph.3}.

\bigskip
\noindent
The process of completing \GFAs{} helps us ensure that all relevant equality information is propagated, and only relevant equality information (by the congruence closure propagation criterion - we propagate all equality information for sub-terms). This is the implementation of the algorithm in figure \ref{fig_lazy_congruence}.\\
The algorithm adds at most one \GT{} per predecessor \GT{}.\\
We note here that, in some cases, we may create empty \GTs{} as above and they will never become actual \GTs{} (will never represent any term). This can happen in the case of joins, and also for bounded fragments and scoping - we give an example when discussing joins.
 %- consider the case where the function \m{f} is not in scope at \m{n_1} - we will remain with the empty \GT{} with inverse source \m{[f(a)]_0}. We do not remove these empty \GTs{} as they allow us efficient incremental updates, for example if later \m{n_0} learns (e.g. through quantifier instantiation) that \m{f(a)=b} and b is in scope at \m{n_1}, then we do not need to look at \m{[g(f(a))]_0} again, but rather only at the new source \GFA{} for the empty simply connect 

\bigskip
\noindent
In the above description we have implicitly assumed one important property of source-edges - namely, that each \GT{} can have at most one source - formally:
\begin{figure}[H]
\textbf{The sequential propagation invariant part 2:}\\
For a sequential node n and a predecessor p.\\
\m{\forall u,v \in g_n \cdot}\\
\m{~~~\sources{n}{}{u} \cap \sources{n}{}{v} \neq \emptyset \Rightarrow u=v}
\end{figure}

\noindent
This invariant comes up whenever one of the other invariants forces us to add a source-edge - if that source already has an inverse-source, instead of adding another inverse source we merge the two \GTs{}. 
Note that the first part of the propagation invariant is insufficient here, as the empty \GT{} has no \GFAs{} to operate on. In the above case that would not be a problem, but if the predecessor had a cycle in the EC-graph - e.g. \m{a=h(a)} - we could add an unbounded number of empty \GTs{}, while part two of the propagation invariant (when enforced eagerly) prevents that as we can add at most as many \GTs{} as there are in our predecessor graph.

\begin{figure}
\begin{subfigure}[t]{0.49\textwidth}
\framebox[\textwidth]{
\raisebox{0pt}[0.2\textheight][0pt]
{
\begin{tikzpicture}
  \node[gttn] (1)              {$$};
	\node[gl]   (1l) [below = 0 of 1] {\m{n_0}};

  \node[gtn,label=center:\scriptsize\m{a}]  (2) [above = 0.5cm  of 1] {\phantom{b}};

	\node[gtn]  (4)  [above right = 0.7cm and 0.3cm of 2] {\m{f(a)}};
	\draw[gfa]  (4) to node[el,anchor=west]  {\m{f}} (2);

	\node[gtn]  (6)  [above = 2.5cm of 1] {$\stackB{g(f(a))}{h(a)}$};
	\draw[gfa]  (6) to[] node[el,anchor=west] {\m{g}} (4);
	\draw[gfa]  (6) to[] node[el,anchor=east] {\m{h}} (2);

%%%%%%%%%%%%%%%%%%%%%%%%%%%%%%%%%%%%%%%%%%%%%%%%%%%%%%%%%%%%%%
	\node[gttn] (11)  [right = 3cm of 1] {$$};
	\node[gl]   (11l) [below = 0 of 11]   {\m{n_1}};

	\node[gtn,label=center:\scriptsize\m{a}]  (12) [above = 0.5cm of 11] {\phantom{b}};

%%%%%%%%%%%%%%%%%%%%%%%%%%%%%%%%%%%%%%%%%%%%%%%%%%%%%%%%%%%%%%

	\draw[se] ( 12.180) to (   2.0);

\draw[draw=none, use as bounding box] (current bounding box.north west) rectangle (current bounding box.south east);

%\begin{pgfinterruptboundingbox}
%	\draw[separator] (2.0cm,-0.7cm) to (2.0cm,3.5cm);
%	\draw[separator] (5.5cm,-0.7cm) to (5.5cm,3.5cm);
%\end{pgfinterruptboundingbox}

\end{tikzpicture}
}}
\caption{
Before \m{n_1.}\lstinline|makeTerm(h,($\m{[a]_1}$))|
}
\label{snippet3.18_graph.0}
\end{subfigure}
\begin{subfigure}[t]{0.49\textwidth}
\framebox[\textwidth]{
\raisebox{0pt}[0.2\textheight][0pt]
{
\begin{tikzpicture}
  \node[gttn] (1)              {$$};
	\node[gl]   (1l) [below = 0 of 1] {\m{n_0}};

  \node[gtn,label=center:\scriptsize\m{a}]  (2) [above = 0.5cm  of 1] {\phantom{b}};

	\node[gtn]  (4)  [above right = 0.7cm and 0.3cm of 2] {\m{f(a)}};
	\draw[gfa]  (4) to node[el,anchor=west]  {\m{f}} (2);

	\node[gtn]  (6)  [above = 2.5cm of 1] {$\stackB{g(f(a))}{h(a)}$};
	\draw[gfa]  (6) to[] node[el,anchor=west] {\m{g}} (4);
	\draw[gfa]  (6) to[] node[el,anchor=east] {\m{h}} (2);

%%%%%%%%%%%%%%%%%%%%%%%%%%%%%%%%%%%%%%%%%%%%%%%%%%%%%%%%%%%%%%
	\node[gttn] (11)  [right = 3cm of 1] {$$};
	\node[gl]   (11l) [below = 0 of 11]   {\m{n_1}};

	\node[gtn,label=center:\scriptsize\m{a}]  (12) [above = 0.5cm of 11] {\phantom{b}};

	\node[gtn]  (16)  [above = 2.5cm of 11] {$\stackB{\textcolor{red}{g(f(a))}}{h(a)}$};
%	\draw[gfa]  (16) to[] node[el,anchor=west] {\m{g}} (14);
	\draw[gfa]  (16) to[] node[el,anchor=east] {\m{h}} (12);
				

%%%%%%%%%%%%%%%%%%%%%%%%%%%%%%%%%%%%%%%%%%%%%%%%%%%%%%%%%%%%%%

	\draw[se] ( 12.180) to (   2.0);

	\draw[se] (16.180) to ( 6.0);

\draw[draw=none, use as bounding box] (current bounding box.north west) rectangle (current bounding box.south east);

\end{tikzpicture}
}}
\caption{
After adding \m{h([a]_1)}\\
\m{g(f(a))} is missing
}
\label{snippet3.18_graph.1}
\end{subfigure}

\begin{subfigure}[t]{0.49\textwidth}
\framebox[\textwidth]{
\raisebox{0pt}[0.2\textheight][0pt]
{
\begin{tikzpicture}
  \node[gttn] (1)              {$$};
	\node[gl]   (1l) [below = 0 of 1] {\m{n_0}};

  \node[gtn,label=center:\scriptsize\m{a}]  (2) [above = 0.5cm  of 1] {\phantom{b}};

	\node[gtn]  (4)  [above right = 0.7cm and 0.3cm of 2] {\m{f(a)}};
	\draw[gfa]  (4) to node[el,anchor=west]  {\m{f}} (2);

	\node[gtn]  (6)  [above = 2.5cm of 1] {$\stackB{g(f(a))}{h(a)}$};
	\draw[gfa]  (6) to[] node[el,anchor=west] {\m{g}} (4);
	\draw[gfa]  (6) to[] node[el,anchor=east] {\m{h}} (2);

%%%%%%%%%%%%%%%%%%%%%%%%%%%%%%%%%%%%%%%%%%%%%%%%%%%%%%%%%%%%%%
	\node[gttn] (11)  [right = 3cm of 1] {$$};
	\node[gl]   (11l) [below = 0 of 11]   {\m{n_1}};

	\node[gtn,label=center:\scriptsize\m{a}]  (12) [above = 0.5cm of 11] {\phantom{b}};

	\node[gtn,label=center:\scriptsize\m{\textcolor{red}{f(a)}}]  (14)  [above right = 0.7cm and 0.3cm of 12] {\phantom{f(a)}};
	\draw[mgfa] (14) to node[ml,anchor=west]  {\m{f}} (12);

	\node[gtn]  (16)  [above = 2.5cm of 11] {$\stackB{{g(f(a))}}{h(a)}$};
	\draw[gfa]  (16) to[] node[el,anchor=west] {\m{g}} (14);
	\draw[gfa]  (16) to[] node[el,anchor=east,pos=0.25] {\m{h}} (12);
				

%%%%%%%%%%%%%%%%%%%%%%%%%%%%%%%%%%%%%%%%%%%%%%%%%%%%%%%%%%%%%%

	\draw[se] (12) to ( 2.0);
	\draw[se] (14) to ( 4.0);
	\draw[se] (16) to ( 6.0);
\draw[draw=none, use as bounding box] (current bounding box.north west) rectangle (current bounding box.south east);

\end{tikzpicture}
}}
\caption{
After adding an \\
inverse source for \m{g(f([a]_0))}
}
\label{snippet3.18_graph.2}
\end{subfigure}
\begin{subfigure}[t]{0.49\textwidth}
\framebox[\textwidth]{
\raisebox{0pt}[0.2\textheight][0pt]
{
\begin{tikzpicture}
  \node[gttn] (1)              {$$};
	\node[gl]   (1l) [below = 0 of 1] {\m{n_0}};

  \node[gtn,label=center:\scriptsize\m{a}]  (2) [above = 0.5cm  of 1] {\phantom{b}};

	\node[gtn]  (4)  [above right = 0.7cm and 0.3cm of 2] {\m{f(a)}};
	\draw[gfa]  (4) to node[el,anchor=west]  {\m{f}} (2);

	\node[gtn]  (6)  [above = 2.5cm of 1] {$\stackB{g(f(a))}{h(a)}$};
	\draw[gfa]  (6) to[] node[el,anchor=west] {\m{g}} (4);
	\draw[gfa]  (6) to[] node[el,anchor=east] {\m{h}} (2);

%%%%%%%%%%%%%%%%%%%%%%%%%%%%%%%%%%%%%%%%%%%%%%%%%%%%%%%%%%%%%%
	\node[gttn] (11)  [right = 3cm of 1] {$$};
	\node[gl]   (11l) [below = 0 of 11]   {\m{n_1}};

	\node[gtn,label=center:\scriptsize\m{a}]  (12) [above = 0.5cm of 11] {\phantom{b}};

	\node[gtn,label=center:\scriptsize\m{f(a)}]  (14)  [above right = 0.7cm and 0.3cm of 12] {\phantom{f(a)}};
	\draw[gfa]  (14) to node[el,anchor=west]  {\m{f}} (12);

	\node[gtn]  (16)  [above = 2.5cm of 11] {$\stackB{g(f(a))}{h(a)}$};
	\draw[gfa]  (16) to[] node[el,anchor=west] {\m{g}} (14);
	\draw[gfa]  (16) to[] node[el,anchor=east,pos=0.25] {\m{h}} (12);
				

%%%%%%%%%%%%%%%%%%%%%%%%%%%%%%%%%%%%%%%%%%%%%%%%%%%%%%%%%%%%%%

	\draw[se] (12) to ( 2.0);
	\draw[se] (14) to ( 4.0);
	\draw[se] (16) to ( 6.0);
\draw[draw=none, use as bounding box] (current bounding box.north west) rectangle (current bounding box.south east);

\end{tikzpicture}
}}
\caption{
Final state
}
\label{snippet3.18_graph.3}
\end{subfigure}

\caption{Example for deep propagation}
\end{figure}

















\subsubsection*{The request cache}
In order to reduce the number of traversals of the CFG, we cache at each CFG-node the previous requests, so that no CFG-node propagates the same request twice. The cache for a CFG-node n consists of all the \GFAs{} of \m{g_n} - \gfas{n} - together with another set of \GFAs, \rgfas{n}. We call a member of \rgfas{n} an \RGFA{} (A rejected \GFA{}). For each \GFA{} in the cache $\fa{f}{t} \in \gfas{n} \cup \rgfas{n}$, the members of the EC-tuple \tup{t} are all in \m{g_n} - \m{\tup{t} \in g_n} - this means that we can only cache direct super-terms of terms represented in \m{g_n}. 
The idea is that if a request for a term t has returned an empty response, a request for any super-term of t will also be empty.


\begin{figure}
\begin{lstlisting}
$\node{n_0}:$
assume $\m{b=b}$
	// $\m{\GFAs{}: b()}$
	// $\m{ECs:[b]}$
	// $\m{\rgfas{}: \emptyset}$
$\node{n_1}:$
assume $\m{b=b}$
	// $\m{\GFAs{}: b()}$
	// $\m{ECs:[b]}$
	// $\m{\sources{}{}{[b]_1}=\s{[b]_0}}$
	// $\m{\rgfas{}: \emptyset}$
if (*)
	$\node{n_2}:$
	assert $\m{f(b)=b}$ //negated $\comm{\m{f(b) \neq b}}$
		// $\m{\GFAs{}: \emptyset}$
		// $\m{ECs:\emptyset}$
else
	$\node{n_3}:$
	assert $\m{f(b)=g(b)}$ //negated $\comm{\m{f(b) \neq g(b)}}$
		// $\m{\GFAs{}: b()}$
		// $\m{ECs:[b]}$
		// $\m{\sources{}{}{[b]_2}=\s{[b]_1}}$
\end{lstlisting}
\caption{propagation sources\\
The state before $\m{n_2.}$\lstinline|makeTerm(f([b]))|\\
The local source correctness invariant holds for the above example.
}
\label{snippet3.16c}
\end{figure}

\bigskip
\noindent
Consider the example in figure \ref{snippet3.16c}.\\
The example shows the state before we invoke \lstinline|makeTerm(f,([b]))| on \m{n_2} (and before adding \m{f(b),g(b)} to \m{n_3}).\\
The initial state (excluding \m{n_3}) is shown in figure \ref{snippet3.16c_graph.0}.

\bigskip
\noindent
We now show the operation of \m{n_2.}\lstinline|makeTerm(f([b]))|:\\
As in the previous example, \m{n_2} sends a request for \m{f([b]_2)} which gets translated
down the line to \m{f([b]_0),f([a]_0)}.\\
In this case, \m{n_0} has no information about either of the requested \GFAs{}, and has no predecessors, so it adds the \GFAs{} to the cache by adding an \RGFA{} for each.\\
For \m{n_1} the situation is the same - the predecessor has no information about the request, so we add it to the cache.\\
Finally, \m{n_2} adds the \GFA{} as a singleton \GT{}.\\
The final state, after adding also \m{n_2.}\lstinline|makeTerm(g([b]))|, is shown in figure \ref{snippet3.16c_graph.1}.


\begin{figure}
\begin{subfigure}[t]{0.99\textwidth}
\framebox[\textwidth]{
\begin{tikzpicture}
  \node[gttn] (1)              {$$};
	\node[gl]   (1l) [below = 0 of 1] {\m{n_0}};

  \node[gtn]  (2) [above = 0.5cm of 1] {\m{b}};

%%%%%%%%%%%%%%%%%%%%%%%%%%%%%%%%%%%%%%%%%%%%%%%%%%%%%%%%%%%%%%
	\node[gttn] (11)  [right = 3cm of 1] {$$};
	\node[gl]   (11l) [below = 0 of 11]   {\m{n_1}};

	\node[gtn]  (12) [above = 0.5cm of 11] {\m{b}};
				
%%%%%%%%%%%%%%%%%%%%%%%%%%%%%%%%%%%%%%%%%%%%%%%%%%%%%%%%%%%%%%

	\node[gttn] (21)  [right = 3.5cm of 11] {$$};
	\node[gl]   (21l) [below = 0 of 21]   {\m{n_2}};

	\node[gtn]  (22) [above = 0.5cm of 21] {\m{b}};

%%%%%%%%%%%%%%%%%%%%%%%%%%%%%%%%%%%%%%%%%%%%%%%%%%%%%%%%%%%%%%

	\node(12a) [left = 0.5cm of 12] {};
	\draw[se] ( 12.180) to (   2.0);

	\draw[se] (22) to  (12);

\end{tikzpicture}
}
\caption{The initial state for figure \ref{snippet3.16c}}
\label{snippet3.16c_graph.0}
\end{subfigure}


\begin{subfigure}[t]{0.99\textwidth}
\framebox[\textwidth]{
\begin{tikzpicture}
  \node[gttn] (1)              {$$};
	\node[gl]   (1l) [below = 0 of 1] {\m{n_0}};

  \node[gtn]  (2) [above = 0.5cm of 1] {\m{b}};
%  \node[gtn,label=center:\scriptsize\m{b}]  (3) [above right = 0.57cm and 0.3cm of 1] {\phantom{a,b}};

%	\draw[gfa] (2) to node[el]             {\m{a}} (1.90);
  
%	\node[gttn] (4)  [above = 0.5cm of 2]    {\m{(a)}};

%	\draw[sgtt] (4) to node[el] {0} (2);

	\node[rgtn]  (6)  [above = 1.5cm of 1] { \m{f(b)}};
%	\node[rgtn]  (7)  [above right = 1.5cm and 0.3cm of 1] { \m{g(b)}};
	\draw[rgfa]  (6) to[out=270, in=90] node[rl,anchor=east]  {\m{f}} (2);
%	\draw[rgfa]  (7) to[out=270, in= 90] node[rl,anchor=west] {\m{g}} (2);

%%%%%%%%%%%%%%%%%%%%%%%%%%%%%%%%%%%%%%%%%%%%%%%%%%%%%%%%%%%%%%
	\node[gttn] (11)  [right = 3cm of 1] {$$};
	\node[gl]   (11l) [below = 0 of 11]   {\m{n_1}};

	\node[gtn]  (12) [above = 0.5cm of 11] {\m{b}};

%	\draw[gfa] (12) to[out=-110,in=110] node[el]             {\m{a}} (11.90);
%	\draw[gfa] (12) to[out=- 70,in= 70] node[el,anchor=west] {\m{b}} (11.90);

%	\node[gttn] (14)  [above = 0.5cm of 12]    {\m{(a),(b)}};

%	\draw[sgtt] (14) to node[el] {0} (12);

	\node[rgtn]  (16)  [above = 1.5cm of 11] { \m{f(b)}};
	\draw[rgfa]  (16) to[out=270, in=90] node[rl,anchor=east]  {\m{f}} (12);
				
%%%%%%%%%%%%%%%%%%%%%%%%%%%%%%%%%%%%%%%%%%%%%%%%%%%%%%%%%%%%%%

	\node[gttn] (21)  [right = 3.5cm of 11] {$$};
	\node[gl]   (21l) [below = 0 of 21]   {\m{n_2}};

	\node[gtn]  (22) [above = 0.5cm of 21] {\m{b}};

	\node[gtn]  (26)  [above = 1.5cm of 21] {\m{f(b)}};
	\draw[gfa]  (26) to[out=270 ,in= 90] node[el]             {\m{f}} (22);

%%%%%%%%%%%%%%%%%%%%%%%%%%%%%%%%%%%%%%%%%%%%%%%%%%%%%%%%%%%%%%

	\node(12a) [left = 0.5cm of 12] {};
	\draw[se] ( 12.180) to (   2.0);

	\draw[se] (22) to  (12);

	\draw[re] (16) to  ( 6);
	\draw[re] (26) to  (16);


\end{tikzpicture}
}
\caption{The state after \m{n_2}.\lstinline|makeTerm(f,([b]))| in \ref{snippet3.16c}}
\label{snippet3.16c_graph.1}
\end{subfigure}

\begin{subfigure}[t]{0.99\textwidth}
\framebox[\textwidth]{
\begin{tikzpicture}
  \node[gttn] (1)              {$$};
	\node[gl]   (1l) [below = 0 of 1] {\m{n_0}};

  \node[gtn]  (2) [above = 0.5cm of 1] {\m{b}};
%  \node[gtn,label=center:\scriptsize\m{b}]  (3) [above right = 0.57cm and 0.3cm of 1] {\phantom{a,b}};

%	\draw[gfa] (2) to node[el]             {\m{a}} (1.90);
  
%	\node[gttn] (4)  [above = 0.5cm of 2]    {\m{(a)}};

%	\draw[sgtt] (4) to node[el] {0} (2);

	\node[gtn]  (6)  [above = 1.5cm of 1] { \stackB{f(b)}{g(b)}};
%	\node[gtn]  (7)  [above right = 1.6cm and 0.3cm of 1] { \m{f(b)}};
	\draw[gfa]  (6) to[bend left] node[el,anchor=west]  {\m{g}} (2);
	\draw[gfa]  (6) to[bend right] node[el,anchor=east] {\m{f}} (2);

%%%%%%%%%%%%%%%%%%%%%%%%%%%%%%%%%%%%%%%%%%%%%%%%%%%%%%%%%%%%%%
	\node[gttn] (11)  [right = 3cm of 1] {$$};
	\node[gl]   (11l) [below = 0 of 11]   {\m{n_1}};

	\node[gtn]  (12) [above = 0.5cm of 11] {\m{b}};

	\node[rgtn]  (16)  [above = 1.725cm of 11] { \m{f(b)}};
%	\node[rgtn]  (17)  [above right = 1.5cm and 0.3cm of 1] { \m{g(b)}};
	\draw[rgfa]  (16) to[out=270, in=90] node[rl,anchor=east]  {\m{f}} (12);
%	\draw[rgfa]  (17) to[out=270, in= 90] node[rl,anchor=west] {\m{g}} (12);
				
%%%%%%%%%%%%%%%%%%%%%%%%%%%%%%%%%%%%%%%%%%%%%%%%%%%%%%%%%%%%%%

	\node[gttn] (21)  [right = 3.5cm of 11] {$$};
	\node[gl]   (21l) [below = 0 of 21]   {\m{n_2}};

	\node[gtn]  (22) [above = 0.5cm of 21] {\m{b}};

	\node[gtn]  (26)  [above = 1.725cm of 21] {\m{f(b)}};
	\draw[gfa]  (26) to[out=270 ,in= 90] node[el]             {\m{f}} (22);
%	\draw[gfa]  (26) to[out=- 80 ,in=  80] node[el,anchor=west] {\m{g}} (22);

%%%%%%%%%%%%%%%%%%%%%%%%%%%%%%%%%%%%%%%%%%%%%%%%%%%%%%%%%%%%%%
%	\draw[se] (11) to  ( 1);
%	\draw[se] (21) to  (11);

	\node(12a) [left = 0.5cm of 12] {};
%	\node(7u) [above= 0.1cm of 7.90] {};
%	\node(7r) [right= 0.3cm of 7] {};
%	\draw[me] ( 16.180) to[out=180,in=0] (7r) to[out=180,in=0] (7u) to[out=180,in=0] (   6.0);
	\draw[me] ( 16.180) to (   6.0);

	\draw[se] (12) to  ( 2);
	\draw[se] (22) to  (12);

%	\draw[se] (14.180) to ( 4.0);
%	\draw[re] (16) to  ( 6);
	\draw[re] (26) to  (16);

%	\draw[se] (16) to  ( 6);
%	\draw[se] (26) to  (16);

%	\draw[ie] (26) to[loop] node[el,above] {\m{\neq}} (26);

\draw[draw=none, use as bounding box] (current bounding box.north west) rectangle (current bounding box.south east);

%\begin{pgfinterruptboundingbox}
%	\draw[separator] (2.0cm,-0.7cm) to (2.0cm,3.5cm);
%	\draw[separator] (5.5cm,-0.7cm) to (5.5cm,3.5cm);
%\end{pgfinterruptboundingbox}

\end{tikzpicture}
}
\caption{
The state after \m{n_0}.\lstinline|assumeEqual(f(b),g(b))| in \ref{snippet3.16c}.\\
\textcolor{red} {Red dashed arrows} represent inconsistent sources - \RGFA{} to \GFA{} \\
(only for illustration, not actually included in data structure)
}
\label{snippet3.16c_graph.2}
\end{subfigure}

\begin{subfigure}[t]{0.99\textwidth}
\framebox[\textwidth]{
\begin{tikzpicture}
  \node[gttn] (1)              {$$};
	\node[gl]   (1l) [below = 0 of 1] {\m{n_0}};

  \node[gtn]  (2) [above = 0.5cm of 1] {\m{b}};
%  \node[gtn,label=center:\scriptsize\m{b}]  (3) [above right = 0.57cm and 0.3cm of 1] {\phantom{a,b}};

%	\draw[gfa] (2) to node[el]             {\m{a}} (1.90);
  
%	\node[gttn] (4)  [above = 0.5cm of 2]    {\m{(a)}};

%	\draw[sgtt] (4) to node[el] {0} (2);

	\node[gtn]  (6)  [above = 1.5cm of 1] { \stackB{f(b)}{g(b)}};
%	\node[gtn]  (7)  [above right = 1.6cm and 0.3cm of 1] { \m{f(b)}};
	\draw[gfa]  (6) to[bend left] node[el,anchor=west]  {\m{g}} (2);
	\draw[gfa]  (6) to[bend right] node[el,anchor=east] {\m{f}} (2);

%%%%%%%%%%%%%%%%%%%%%%%%%%%%%%%%%%%%%%%%%%%%%%%%%%%%%%%%%%%%%%
	\node[gttn] (11)  [right = 3cm of 1] {$$};
	\node[gl]   (11l) [below = 0 of 11]   {\m{n_1}};

	\node[gtn]  (12) [above = 0.5cm of 11] {\m{b}};

%	\draw[gfa] (12) to[out=-110,in=110] node[el]             {\m{a}} (11.90);
%	\draw[gfa] (12) to[out=- 70,in= 70] node[el,anchor=west] {\m{b}} (11.90);

%	\node[gttn] (14)  [above = 0.5cm of 12]    {\m{(a),(b)}};

%	\draw[sgtt] (14) to node[el] {0} (12);

	\node[gtn]  (16)  [above = 1.5cm of 11] { \stackB{f(b)}{g(b)}};
%	\node[rgtn]  (17)  [above right = 1.5cm and 0.3cm of 1] { \m{g(b)}};
	\draw[gfa]  (16) to[bend left] node[el,anchor=west]  {\m{g}} (12);
	\draw[gfa]  (16) to[bend right] node[el,anchor=east] {\m{f}} (12);
				
%%%%%%%%%%%%%%%%%%%%%%%%%%%%%%%%%%%%%%%%%%%%%%%%%%%%%%%%%%%%%%

	\node[gttn] (21)  [right = 3.5cm of 11] {$$};
	\node[gl]   (21l) [below = 0 of 21]   {\m{n_2}};

	\node[gtn]  (22) [above = 0.5cm of 21] {\m{b}};

%	\draw[gfa] (22) to[out=-110,in=110] node[el]             {\m{a}} (21.90);
%	\draw[gfa] (22) to[out=- 70,in= 70] node[el,anchor=west] {\m{b}} (21.90);

%	\node[gttn] (24)  [above = 0.5cm of 22]    {\m{(a),(b)}};

%	\draw[sgtt] (24) to node[el] {0} (22);

	\node[gtn]  (26)  [above = 1.725cm of 21] {\m{f(b)}};
	\draw[gfa]  (26) to[out=270 ,in= 90] node[el]             {\m{f}} (22);
%	\draw[gfa]  (26) to[out=- 80 ,in=  80] node[el,anchor=west] {\m{g}} (22);

%%%%%%%%%%%%%%%%%%%%%%%%%%%%%%%%%%%%%%%%%%%%%%%%%%%%%%%%%%%%%%
%	\draw[se] (11) to  ( 1);
%	\draw[se] (21) to  (11);

	\node(12a) [left = 0.5cm of 12] {};
%	\node(7u) [above= 0.1cm of 7.90] {};
%	\node(7r) [right= 0.3cm of 7] {};
%	\draw[me] ( 16.180) to[out=180,in=0] (7r) to[out=180,in=0] (7u) to[out=180,in=0] (   6.0);
	\draw[se] ( 16.180) to (   6.0);

	\draw[se] (12) to  ( 2);
	\draw[se] (22) to  (12);

%	\draw[se] (14.180) to ( 4.0);
%	\draw[re] (16) to  ( 6);
	\draw[me] (26) to  (16);

%	\draw[se] (16) to  ( 6);
%	\draw[se] (26) to  (16);

%	\draw[ie] (26) to[loop] node[el,above] {\m{\neq}} (26);

\draw[draw=none, use as bounding box] (current bounding box.north west) rectangle (current bounding box.south east);

%\begin{pgfinterruptboundingbox}
%	\draw[separator] (2.0cm,-0.7cm) to (2.0cm,3.5cm);
%	\draw[separator] (5.5cm,-0.7cm) to (5.5cm,3.5cm);
%\end{pgfinterruptboundingbox}

\end{tikzpicture}
}
\caption{
The state after \m{n_1}.\lstinline|update|.\\
\m{n_1} is now consistent - but \m{n_2} is not.
}
\label{snippet3.16c_graph.3}
\end{subfigure}


\begin{subfigure}[t]{0.99\textwidth}
\framebox[\textwidth]{
\begin{tikzpicture}
  \node[gttn] (1)              {$$};
	\node[gl]   (1l) [below = 0 of 1] {\m{n_0}};

  \node[gtn]  (2) [above = 0.5cm of 1] {\m{b}};

	\node[gtn]  (6)  [above = 1.5cm of 1] { \stackB{f(b)}{g(b)}};
	\draw[gfa]  (6) to[bend left] node[el,anchor=west]  {\m{g}} (2);
	\draw[gfa]  (6) to[bend right] node[el,anchor=east] {\m{f}} (2);

%%%%%%%%%%%%%%%%%%%%%%%%%%%%%%%%%%%%%%%%%%%%%%%%%%%%%%%%%%%%%%
	\node[gttn] (11)  [right = 3cm of 1] {$$};
	\node[gl]   (11l) [below = 0 of 11]   {\m{n_1}};

	\node[gtn]  (12) [above = 0.5cm of 11] {\m{b}};

	\node[gtn]  (16)  [above = 1.5cm of 11] { \stackB{f(b)}{g(b)}};
	\draw[gfa]  (16) to[bend left] node[el,anchor=west]  {\m{g}} (12);
	\draw[gfa]  (16) to[bend right] node[el,anchor=east] {\m{f}} (12);
				
%%%%%%%%%%%%%%%%%%%%%%%%%%%%%%%%%%%%%%%%%%%%%%%%%%%%%%%%%%%%%%

	\node[gttn] (21)  [right = 3.5cm of 11] {$$};
	\node[gl]   (21l) [below = 0 of 21]   {\m{n_2}};

	\node[gtn]  (22) [above = 0.5cm of 21] {\m{b}};

	\node[gtn]  (26)  [above = 1.5cm of 21] { \stackB{f(b)}{g(b)}};
	\draw[gfa]  (26) to[bend left] node[el,anchor=west]  {\m{g}} (22);
	\draw[gfa]  (26) to[bend right] node[el,anchor=east] {\m{f}} (22);

%%%%%%%%%%%%%%%%%%%%%%%%%%%%%%%%%%%%%%%%%%%%%%%%%%%%%%%%%%%%%%

	\node(12a) [left = 0.5cm of 12] {};
	\draw[se] ( 16.180) to (   6.0);

	\draw[se] (12) to  ( 2);
	\draw[se] (22) to  (12);

	\draw[se] (26) to  (16);

\draw[draw=none, use as bounding box] (current bounding box.north west) rectangle (current bounding box.south east);

\end{tikzpicture}
}
\caption{
The state after \m{n_2}.\lstinline|update|.\\
All nodes are now consistent
}
\label{snippet3.16c_graph.4}
\end{subfigure}

\caption{.\\
\textcolor{gray} {Gray dashed circles} represent \RGFAs{}.\\
\textcolor{gray} {Gray dashed arrows} represent source edges to \RGFAs{} \\
(only for illustration, not actually included in data structure)}
\end{figure}

\bigskip
\noindent
After \m{n_2.}\lstinline|makeTerm(f([b]))|, we also invoke\\
\m{n_2.}\lstinline|makeTerm(g([b]))| adding the corresponding \GFAs{} as in the previous case.\\
Now we are done with \m{n_2}.

\bigskip
\noindent
When \m{n_3} performs \m{n_3.}\lstinline|makeTerm(f([b]))|:\\
The request sent to \m{n_1} is \s{f([b]_3} which is translated to \s{f([b]_1)}.\\
As \m{n_1} has this \RGFA{}, we subtract it from the request and now the request is $\emptyset$.\\
The request is not propagated further and we are done with \m{n_3}.


\bigskip
\noindent
\subsubsection*{Incremental updates}
Assume that we have performed the operations for our last example as described, and completed one pass of \lstinline|CFG.verify()| with the EC-graph fragment.
Now another logical fragment has produced the equality \m{f(b)=g(b)} at \m{n_0}, for example using quantifier instantiation.\\
The state is depicted in figure \ref{snippet3.16c_graph.2}.\\
We can see that now there is information about \m{f(b)} at \m{n_0} that was not propagated to where it is needed at \m{n_2} (and \m{n_3}).
We call such a state inconsistent, and we say that locally, \m{n_1} is inconsistent (shown in red) - on the next pass of verification for our fragment (which again traverses the CFG in topological order), our algorithm invokes an \lstinline|update| on each CFG node before invoking \lstinline|Node.verify|. 
The \lstinline|update| method, when invoked on a CFG-node all of whose predecessors are consistent, ensures that CFG-node is also consistent (and does not break consistency for predecessors, although it might for successors). We do not \lstinline|update| all successors eagerly on every change as it is extremely inefficient - instead, when traversing the CFG in topological order for any fragment, we \lstinline|update| each CFG-node before performing any inferences on it.

\bigskip
\noindent
In order to support an efficient incremental \lstinline|update| method, each CFG-node keeps some history information that allows it to summarize to successors the changes to its EC-graph since the last time the successor was updated.\\
The history is kept mainly in two fields:
\begin{itemize}
	\item The \lstinline|mergeMap| field that was used for merging, including an inverse map
	\item A map that assigns a \lstinline|generation| for each \GT{}. Roughly, each time \lstinline|assumeEqual| or \lstinline|update| are called on a CFG-node, a new generation starts. The idea is that if a \GT{} u exists in the EC-graph at some generation, if u is merged into v then the merged node is of a strictly higher generation. 
	Each CFG-node remembers the last generation of its predecessors for which it is up-to-date, and only considers newer \GTs{} when invoking \lstinline|update|
\end{itemize}

\noindent
In our example, \m{[b]_0} is of generation 1 while \m{[f(b)]_0} is of generation 2. The last predecessor generation of \m{n_1} is 1 (that is, \m{n_1} has all the information of generations up to 1 from predecessors). When \m{n_1} invokes \m{n_1.}\lstinline|update|, it requests a list of changes from \m{n_0} later than generation 1, and receives the list that includes only the \GT{} \m{[f(b)_0]}.
The \lstinline|update| method collects all changes from predecessors.\\
For each new or updated predecessor \GT{}, we check if the corresponding \GT{} is in our cache (using the inverse source-edges) - in our case, for the new \GT{} \m{[f(b)]_0}, the inverse source of the tuple \m{([b]_0)} is \m{([b]_1)}. We check if we have the \GFA{} \m{f([b]_1)} in the cache and we find we have it as an \RGFA{}. Hence we replace the \RGFA{} with a \GT{} that contains the \GFA{} \m{f([b]_1)}, add the source-edge and propagate equality information. The state after \m{n_1.}\lstinline|update| is shown in figure \ref{snippet3.16c_graph.3}.\\
For \m{n_3}, the operation of the \lstinline|update| method is similar, except that here the updated predecessor \GT{} matches a \GT{} rather than an \RGFA{} - hence we simply add the corresponding source-edge and propagate equalities. The final state is shown in figure \ref{snippet3.16c_graph.4}

\bigskip
\noindent
We formalize now two parts (conjuncts) of the local CFG-node invariant that ensures that an \RGFA{} exists in CFG-node only if none of the terms it represents occurs in all its predecessors. The first part is the part that is broken in figure \ref{snippet3.16c_graph.2} (marked in red), where the \RGFA{} \m{[f(b)]_1} should instead be added to a \GT{} as \m{n_0} has the \GT{} \m{[f(b)]_0}.

\begin{figure}[H]
For a CFG node n with one prededessor p.\\
%\textbf{The sequential response invariant:}\\
\m{\forall \tup{t} \in g_n, \tup{s} \in \sources{n}{}{\tup{t}}, f \cdot}\\
\m{~~~\fa{f}{s} \in \gfas{p} \Rightarrow \fa{f}{t} \notin \rgfas{n}}\\
And\\
%\textbf{The sequential request invariant:}\\
\m{\forall \fa{f}{t} \in \gfas{n} \cup \rgfas{n}, \tup{s} \in \sources{n}{}{\tup{t}} \cdot}\\
\m{~~~\fa{f}{s} \in \gfas{p} \cup \rgfas{p}}
\end{figure}



\bigskip
\noindent
The first part of the invariant prevents the situation in figure \ref{snippet3.16c_graph.2} - its violation is fixed by replacing the \RGFA{} \m{[f(b)]_1} with the \GT{} that includes the \GFA{} \m{[f(b)]_1} (the algorithm never removes an \RGFA{} without replacing it with a \GFA{} - the only exception is in a very specific garbage collection process).\\
The second part of the invariant ensures that if some term is in the request cache (\m{\gfas{n} \cup \rgfas{n}}), then it is also in the request cache of the predecessor. 
This invariant is enforced by sending a request for the offending \GFA{}, as we have shown above for invocations of \lstinline|makeTerm|.
The second part of the invariant can also be violated when a new source-edge is added or when two \GTs{} are merged.

\subsection{The equality propagation algorithm}
We have seen now the main points of the algorithm and the reasoning behind them, and now we present the algorithm for sequential nodes in more detail. In the next section we present the changes needed in order to support joins.

\noindent
\textbf{The high-level view}\\
The main part of the algorithm is shared between \lstinline|makeTerm|, \lstinline|assumeEqual| and \lstinline|update|.\\
For each operation, we modify or add some \GTs{} in the EC-graph and add them to a queue of modified \GTs{}.\\
Our main loop maintains two queues - the \lstinline|mergeQ| we have seen for monolithic EC-graphs for \GTs{} that need to be merged and the \lstinline|propQ| (propagation queue) that contains all \GTs{} that have been modified, where modifications include the addition of \GFAs{} and source-edges.
The main loop of the algorithm simply processes both queues and propagates changes from each \GT{} to its neighbours as needed, until the graph is consistent - all conjuncts of our invariant are satisfied. Changes include adding \GFAs{} or sources to \GTs{}, and merging \GTs{}. 

\bigskip
\noindent
For \lstinline|makeTerm| and \lstinline|assumeEqual|, we assume that in the pre-state all parts of the invariant hold, we add information to the graph which might violate the invariant, and then the main loop of the algorithm fixes the broken part.\\
For \lstinline|update|, we assume that all predecessors of the node are consistent, and furthermore we assume that the predecessors have changed only monotonically (adding information) since the last time our node was up-to-date.

\bigskip
\noindent
\lstinline{makeTerm}\\
The algorithm for \lstinline|makeTerm| is presented in figure \ref{EC_makeTerm}, it is an extension of the algorithm for monolithic EC-graphs (which simply adds a singleton \GT{} if it does not already exist).\\
The algorithm first checks if the term already exists (using \lstinline|findGT|), in which case it is simply returned. 
The check is at the cost of a constant number of searches in maps (we simplify here the implementation of the \lstinline|superTerms| field). 
The \lstinline|addGT| method enqueues the newly added \GT{} in \lstinline|propQ|.


\begin{figure}
\begin{lstlisting}
method makeTerm(f:Function, tt : Tuple[GT]) : GT
	t:=addGT(f,tt)
	mainLoop()
	return transitiveMerge(t)
	
method addGT(f:Function, tt: Tuple[GT) : GT
	var t := findGT(f,tt)
	if (t!=null)
		return t
	t := addNewSinlgeton(f,tt)

method findGT(f:Function,tt:Tuple[GT]) : GT
	t := null
	if (tt=())
		if (f$\in$constants.keys)
			t := constants[f]
	else
		t := superTerms[tt[0]].findOrNull(t $\Rightarrow$ f(tt)$\in$t.gfas)
	return t
	
method addNewSingleton(f,tt)
	t := new GT(new GFA(f,tt))
	if (tt=())
		constants[f] := t
	else
		foreach (ti$\in$tt)
			superTerms[ti].add(t)
	rgfas.remove(f(tt))
	propQ.enqueue(t)
\end{lstlisting}
\caption{The algorithm for \lstinline|makeTerm|\\
The method checks if the \GFA{} already exists, 
and otherwise creates a new singleton \GT{} and invokes the main loop
}
\label{EC_makeTerm}
\end{figure}

\bigskip
\noindent
\lstinline|assumeEqual|\\
The algorithm for  \lstinline|assumeEqual| is shown in figure \ref{EC_assumeEqual}, it is an extension of the algorithm for monolithic EC-graphs. The method simply enqueues a merge between the two input \GTs{}.

\begin{figure}
\begin{lstlisting}
method assumeEqual(gt1,gt2:GT)
	if (gt1==gt2)
		return
		
	enqueueMerge(gt1,gt2)
	
	mainLoop()
\end{lstlisting}
\caption{The algorithm for \lstinline|assumeEqual|\\
}
\label{EC_assumeEqual}
\end{figure}

\bigskip

\noindent
\lstinline|update|\\
The algorithm for \lstinline|update| is shown in figure \ref{EC_update}.\\
The algorithm receives a set of new \GTs{} in the predecessor (since the last generation in which we have sampled the predecessor), and a list of predecessor \GTs{} that have new information (that is, have added \GFAs{} since the last sampled generation).\\
The method searches the inverse source of each updated \GT{} and, if found, updates the source edge (using the predecessors \lstinline|mergeMap|) and enqueues the relevant \GT{}.\\
For new \GTs{}, we only check if there is a new predecessor \GT{} that has an inverse source \GFA{} or \RGFA{} in our graph - in which case we add it as a source (converting an \RGFA{} to a singleton \GT{}) and enqueue the corresponding \GT{}.

\begin{figure}
\begin{lstlisting}
method update(updatedGTs : Set[GT], newGTs : Set[GT])
	
	foreach pgt$\in$updatedGTs 
		if sources$^{-1}$.hasKey(pgt)
			gt := sources$^{-1}$[pgt]
			removeSource(gt,pgt)
			addSource(gt,predecessor.transitiveMerge(pgt))
	
	foreach pgt$\in$newGTs
		foreach f(ptt)$\in$pgt.gfas
			if sources$^{-1}$.hasKey(ptt)
				tt := sources$^{-1}$[ptt]
				if rgfas.conatins(f(tt))
					gt := addSingleton(f,tt)
					addSource(gt,pgt)
				else
					gt := findGT(f,tt)
					if (gt!=null)
						addSource(gt,pgt)
				
	mainLoop()
\end{lstlisting}
\caption{The algorithm for \lstinline|update|\\
The inputs are the set of predecessor \GTs{} updated (that is, that have been merged with another \GT{} or have had a \GFA{} added)
and the set of new \GTs{} added to \m{g_p}.
}
\label{EC_update}
\end{figure}



\bigskip
\noindent
\textbf{The main loop:}\\
The code for the main loop is shown in figure \ref{EC_mainLoop}.\\
For each \GT{}, we maintain a list of new \GFAs{} and new or updated source edges.\\
The main loop basically merges \GTs{} until there are none left to merge and then propagates information from each merged or modified \GT{}.
The method \lstinline|mergeOne| merges two \GTs{} as we have seen before in figure \ref{fig_basic_ECGraph_mergeOne}.
The only difference is that each \GFA{} or source-edge added to the merge target is marked as new, and the merge target is added to \lstinline|propQ|.\\
The method \lstinline|propagateOne| propagates all latest changes to a \GT{} to its adjacent \GTs{} - it updates the source-edges 
of the \GT{} and propagates any new source information to super-terms (up). Added source may force us to add new \GFAs{}, which may, in turn, force us to add new sources edges - hence the loop.\\ 
The method also propagates source information for super-terms from new source edges.


\begin{figure}
\begin{lstlisting}
method mainLoop()
	while (!mergeQueue.isEmpty || !propQ.isEmpty)
		while (!mergeQ.isEmpty)
			mergeOne(mergeQ.dequeue)
		while (!propQ.isEmpty)
			propagateOne(propQ.dequeue)
	
method propagateOne(gt:GT)
	if (mergeQ.contains(gt))
		return
		
	while (!gt.newGFAs.isEmpty || !gt.newSources.isEmpty)
		updateForNewGFAs(gt)
		completeDownNewSources(gt)
	
	propagateUpNewSources(gt)
\end{lstlisting}
\caption{The algorithm for \lstinline|mainLoop|\\
The algorithm processes both queues until no work is left.
Merging is done as shown for the monolithic EC-graph.
}
\label{EC_mainLoop}
\end{figure}

The pseudo-code for the incremental update of a \GT{} is shown in figure \ref{EC_propagate_GT}.
\lstinline|updateForNewGFAs| first ensures that all required equality information is propagated to the predecessor (we describe requests below),
and then adds all relevant source-edges, for example, in figure \ref{snippet3.18_graph.1}, the source edge between \m{[h(a)]_1} and \m{[h(a)]_0} is added after the \GFA{} \m{h([a]_1)} is added to the new \GT{}.

The \lstinline|propagateUpNewSources| method looks at all cached requests that are direct super-terms of the \GT{} with an added source, 
and ensure first that all relevant equality information for these super-terms is added to the predecessor. 
Next, we add new source-edges to super-terms, and in the case the super-term is an \RGFA{}, we convert it to a \GT{} - as described in the example in \ref{snippet3.16c_graph.3}.

\begin{figure}
\begin{lstlisting}
method updateForNewGFAs(gt : GT)
	foreach f(tt)$\in$gt.newGFAs //only process new GFAs
		var pgts := predecessor.requestGTs(f,sources[tt])
		addSources(gt,pgts)
	
method completeDownNewSources(gt)
	foreach pgt in gt.newSources //only new sources
		foreach f(ss)$\in$pgt.gfas
			var tt := makeInverseSource(ss)
				addGFA(gt,f,tt)

method propagateUpNewSources(gt)
	foreach pgt in newSources[gt] //only new sources
		foreach f(tt)$\in$superGFAs(gt)$\cup$rgfas[gt]
			var pgts := predecessor.requestGTs(f,sources[tt])
			var gt:=findGT(f,tt)
			if (gt!=null) 
				addSources(gt,pgts)
			else if (!pgts.isEmpty)
				addSingleton(f,tt)
\end{lstlisting}
\caption{The algorithm for propagating changes from a \GT{}\\
New sources may add new \GFAs{} and vice-versa.\\
After all \GFAs{} and sources are added, \\
we propagate source information to super-terms.\\
The methods \lstinline|addSource| and \lstinline|addGFA| are detailed in figure \ref{EC_propagate_helpers}.
}
\label{EC_propagate_GT}
\end{figure}

\begin{figure}
\begin{lstlisting}
method makeInverseSource(pgt:GT) : GT
	var gt:=sources$^{-1}$[pgt]
	if (gt==null)
		gt := new GT()
		addSource(gt,pgt)
	return gt

method addSource(gt:GT,pgt:GT)
	sources[gt].add(pgt)
	sources$^{-1}$[pgt].add(gt)
	propQ.enqueue(gt)
				
method addGFA(gt,f,tt)
	gt2 := findGT(f,tt)
	if (gt2!=null)
		equeueMerge(gt,gt2)
	else
		gfa := new GFA(f,tt)
		gt.gfas.add(gfa)
		foreach (gt2 in tt)
			superTerms[gt2].add(gt)
		propQ.enqueue(gt)
\end{lstlisting}
\caption{Helper functions for the propagation algorithm\\
}
\label{EC_propagate_helpers}
\end{figure}

\noindent
\textbf{Requests:}\\
The algorithm for servicing a request is shown in figure \ref{EC_propagate_request}.\\
The algorithm is basically an extension of the generic algorithm from figure \ref{clause_import_global}.
The algorithm traverses the CFG in reverse topological order starting at n,
where the request (a function symbol and set of tuples of \GTs{}) is filtered through the 
cache at each CFG-node (in \lstinline|requestBW| - any \GFA{} that already exists in \m{g_n} or \rgfas{n} is removed),
and then translated through source-edges to the predecessor.\\
The predecessor is traversed only if the filtered translated request is not empty - this is the key to reducing CFG-traversals.\\
We then traverse the CFG in topological order (\lstinline|requestFW|), starting at each CFG-node that sent no request to its predecessor (or the root), and ending at n.\\
Each CFG-node adds to its cache all the requested \GFAs{} - those for which there are no sources in the predecessor are added to \rgfas{n} (ensuring this request is never again propagated from this CFG-node) and the others are added to \m{g_n} as new \GTs{} using the \lstinline|makeTerm| method (this invocation of \lstinline|makeTerm| will send a new request for the same set of \GFAs{}, but that request is answered immediately as the response is cached in the predecessor - however, it may send further new requests triggered by congruence closure - we discuss this case below).

\begin{figure}
\begin{lstlisting}
method requestGTs(f,tts:Set[Tuple[GT]]) : Set[GT]
	var requestMap := new Map[Set[Tuple[GT]]]
	requestMap[this] := (f,tts)
	
	traverseBF(this,requestBW,requestFW)
	
	return findGTs(f,tts)

method requestBW(n:CFGNode) : Set[CFGNode] 
	(f,tts) := requestMap[n]
	//The request for the predecessor
	var ptts := new Set[Tuple[GT]]
	foreach tt$\in$tts
		t := findGT(f,tt)
		if (t==null) 
			if (f(tt)$\notin$rgfas) 
				ptts.add(sources[tt]) //request not in cache
		
		if (ptts.isEmpty)
			return $\emptyset$ //no request to predecessor
		else  //propagate request to predecessor
			requestMap[predecessor].add((f,ptts))
			return Set(predecessor)
		
method requestFW(n:CFGNode) 
	(f,tts) = requestMap[n]
	foreach tt$\in$tts
		var pgts := predecessor.findGTs(f,sources[tt])
		if (!pgts.isEmpty)
			makeTerm(f,tt)
		else
			rgts.add(f(tt))
\end{lstlisting}
\caption{The algorithm for \lstinline|requestGTs|\\
The method \lstinline|requestGTs(f,tts)| propagates equality information for all terms in f(tt) for tt$\in$tts.\\
}
\label{EC_propagate_request}
\end{figure}


\subsubsection*{An example}
We describe now an example that shows that the main loop of our algorithm may need several up and down propagation in order to ensure consistency. Consider the state in figure \ref{fig_makeTerm_up_down.0}.\\
We now perform \m{n_1.}\lstinline|makeTerm(f,$\m{([a]_1)}$)|.\\
The steps are shown in figures \ref{fig_makeTerm_up_down.0} to \ref{fig_makeTerm_up_down.3} and \ref{fig_makeTerm_up_down.4}, annotated with the method in the algorithm used for each step. We can see why we have a loop in \lstinline|propagateOne| - we may need several steps until one \GT{} is consistent.

In the case where \m{n_0} is a transitive predecessor rather than a direct predecessor (assuming no other equlities on intermediate nodes), the process will be similar, except that each time we draw a new source-edge, we send a request to predecessors that propagates all relevant equality information to the direct predecessor.
In the case of deeper terms, for example, if we replace, in the example above, \m{g(b),g(c)} with 
\m{g(f(b)),g(f(c))} resp. - the process is again similar except that the loop between updating source-edges and completing \GFAs{} happens accross more than one \GT{} and hence more than one call to \lstinline|propagateOne|.


\begin{figure}
\begin{subfigure}[t]{0.99\textwidth}
\framebox[\textwidth]{
\begin{tikzpicture}
	\node[gttn] (1)              {$$};
	\node[gl]   (1l) [below = 0 of 1] {\m{n_0}};

	\node[gtn,label=center:\m{a}]  (2) [above left = 0.5cm and 1.8cm of 1] {\phantom{B}};
	\node[gtn,label=center:\m{b}]  (3) [above left = 0.5cm and 0.4cm of 1] {\phantom{B}};
	\node[gtn,label=center:\m{c}]  (4) [above right= 0.5cm and 0.4cm of 1] {\phantom{B}};
	\node[gtn,label=center:\m{d}]  (5) [above right= 0.5cm and 1.8cm of 1] {\phantom{B}};

	\node[gtn]  (6)  [above right = 1.5cm and 0.3cm of 2] {\stackB{f(a)}{g(b)}};
	\draw[gfa]  (6) to node[el,anchor=east] {\m{f}} (2);
	\draw[gfa]  (6) to node[el,anchor=west] {\m{g}} (3);

	\node[gtn]  (7)  [above left = 1.5cm and 0.3cm of 5] {\stackB{g(c)}{h(d)}};
	\draw[gfa]  (7) to node[el,anchor=east] {\m{g}} (4);
	\draw[gfa]  (7) to node[el,anchor=west] {\m{h}} (5);

%%%%%%%%%%%%%%%%%%%%%%%%%%%%%%%%%%%%%%%%%%%%%%%%%%%%%%%%%%%%%%
	\node[gttn] (11)  [right = 7cm of 1] {$$};
	\node[gl]   (11l) [below = 0 of 11]   {\m{n_1}};

	\node[gtn,label=center:\m{a}]  (12) [above left = 0.5cm and 1.0cm of 11] {\phantom{B,B}};
	\node[gtn,label=center:\m{b,c}]  (13) [above = 0.43cm of 11] {\phantom{B,B}};

 
%%%%%%%%%%%%%%%%%%%%%%%%%%%%%%%%%%%%%%%%%%%%%%%%%%%%%%%%%%%%%%
%	\node (5au) [above = 0.2 of 5a] {};
%	\node (3au) [above = 0.2 of 3a] {};
%	\draw[se] (12a) to (5au) to (3au) to (2a);

	\node (12u) [above = 0.2 of 12] {};
	\node (5u) [above = 0.2 of 5] {};
	\node (4u) [above = 0.2 of 4] {};

	\draw[se] (13) to (12u) to (5u) to (4u) to (3);
	\draw[se] (13) to (12u) to (5u) to (4);

	\node (5r1) [right = 1.0 of 5] {};
	\node (5d1) [below = 0.2 of 5] {};
%	\node (5u1) [above = 0.2 of 5] {};
	\node (4d1) [below = 0.2 of 4] {};
	\node (3d1) [below = 0.2 of 3] {};

	\draw[se] (12) to (5r1) to[out=180,in=0] (5d1) to (4d1) to (3d1) to (2);

\end{tikzpicture}
}\caption{
The state before \m{n_1}.\lstinline{makeTerm(f,(a))}
}
\label{fig_makeTerm_up_down.0}
\end{subfigure}

\begin{subfigure}[t]{0.99\textwidth}
\framebox[\textwidth]{
\begin{tikzpicture}
	\node[gttn] (1)              {$$};
	\node[gl]   (1l) [below = 0 of 1] {\m{n_0}};

	\node[gtn,label=center:\m{a}]  (2) [above left = 0.5cm and 1.8cm of 1] {\phantom{B}};
	\node[gtn,label=center:\m{b}]  (3) [above left = 0.5cm and 0.4cm of 1] {\phantom{B}};
	\node[gtn,label=center:\m{c}]  (4) [above right= 0.5cm and 0.4cm of 1] {\phantom{B}};
	\node[gtn,label=center:\m{d}]  (5) [above right= 0.5cm and 1.8cm of 1] {\phantom{B}};

	\node[gtn]  (6)  [above right = 1.5cm and 0.3cm of 2] {\stackB{f(a)}{g(b)}};
	\draw[gfa]  (6) to node[el,anchor=east] {\m{f}} (2);
	\draw[gfa]  (6) to node[el,anchor=west] {\m{g}} (3);

	\node[gtn]  (7)  [above left = 1.5cm and 0.3cm of 5] {\stackB{g(c)}{h(d)}};
	\draw[gfa]  (7) to node[el,anchor=east] {\m{g}} (4);
	\draw[gfa]  (7) to node[el,anchor=west] {\m{h}} (5);

%%%%%%%%%%%%%%%%%%%%%%%%%%%%%%%%%%%%%%%%%%%%%%%%%%%%%%%%%%%%%%
	\node[gttn] (11)  [right = 7cm of 1] {$$};
	\node[gl]   (11l) [below = 0 of 11]   {\m{n_1}};

	\node[gtn,label=center:\m{a}]  (12) [above left = 0.5cm and 1.0cm of 11] {\phantom{B,B}};
	\node[gtn,label=center:\m{b,c}]  (13) [above = 0.43cm of 11] {\phantom{B,B}};

	\node[gtn]  (16)  [above = 2.13cm of 11] {\stackB{f(a)}{\textcolor{red}{g(b)}}};
	\draw[gfa]  (16) to node[el,anchor=east] {\m{f}} (12);
	\draw[mgfa]  (16) to node[ml,anchor=west] {\m{g}} (13);
 
%%%%%%%%%%%%%%%%%%%%%%%%%%%%%%%%%%%%%%%%%%%%%%%%%%%%%%%%%%%%%%
%	\node (5au) [above = 0.2 of 5a] {};
%	\node (3au) [above = 0.2 of 3a] {};
%	\draw[se] (12a) to (5au) to (3au) to (2a);

	\node (12u) [above = 0.2 of 12] {};
	\node (5u) [above = 0.2 of 5] {};
	\node (4u) [above = 0.2 of 4] {};

	\draw[se] (13) to (12u) to (5u) to (4u) to (3);
	\draw[se] (13) to (12u) to (5u) to (4);

	\node (5r1) [right = 1.0 of 5] {};
	\node (5d1) [below = 0.2 of 5] {};
%	\node (5u1) [above = 0.2 of 5] {};
	\node (4d1) [below = 0.2 of 4] {};
	\node (3d1) [below = 0.2 of 3] {};

	\draw[se] (12) to (5r1) to[out=180,in=0] (5d1) to (4d1) to (3d1) to (2);

	\node (7r) [right = 1.2 of 7.0] {};
	\node (7l) [left = 1.6 of 7.180] {};
	\node (7u) [above = 0.2 of 7] {};
	\draw[se] (16) to (7r) to (7u) to (7l) to (6);

\end{tikzpicture}
}
\caption{
The state after \lstinline|makeSingleton(f,$\m{[a]_1}$)|
}
\label{fig_makeTerm_up_down.1}
\end{subfigure}

\begin{subfigure}[t]{0.99\textwidth}
\framebox[\textwidth]{
\begin{tikzpicture}
	\node[gttn] (1)              {$$};
	\node[gl]   (1l) [below = 0 of 1] {\m{n_0}};

	\node[gtn,label=center:\m{a}]  (2) [above left = 0.5cm and 1.8cm of 1] {\phantom{B}};
	\node[gtn,label=center:\m{b}]  (3) [above left = 0.5cm and 0.4cm of 1] {\phantom{B}};
	\node[gtn,label=center:\m{c}]  (4) [above right= 0.5cm and 0.4cm of 1] {\phantom{B}};
	\node[gtn,label=center:\m{d}]  (5) [above right= 0.5cm and 1.8cm of 1] {\phantom{B}};

	\node[gtn]  (6)  [above right = 1.5cm and 0.3cm of 2] {\stackB{f(a)}{h(b)}};
	\draw[gfa]  (6) to node[el,anchor=east] {\m{f}} (2);
	\draw[gfa]  (6) to node[el,anchor=west] {\m{g}} (3);

	\node[gtn]  (7)  [above left = 1.5cm and 0.3cm of 5] {\stackB{g(c)}{h(d)}};
	\draw[gfa]  (7) to node[el,anchor=east] {\m{g}} (4);
	\draw[gfa]  (7) to node[el,anchor=west] {\m{h}} (5);

%%%%%%%%%%%%%%%%%%%%%%%%%%%%%%%%%%%%%%%%%%%%%%%%%%%%%%%%%%%%%%
	\node[gttn] (11)  [right = 7cm of 1] {$$};
	\node[gl]   (11l) [below = 0 of 11]   {\m{n_1}};

	\node[gtn,label=center:\m{a}]  (12) [above left = 0.5cm and 1.0cm of 11] {\phantom{B,B}};
	\node[gtn,label=center:\m{b,c}]  (13) [above = 0.43cm of 11] {\phantom{B,B}};

	\node[gtn]  (16)  [above = 2.13cm of 11] {\stackB{f(a),g(b)}{g(c)}};
	\draw[gfa]  (16) to node[el,anchor=east] {\m{f}} (12);
	\draw[gfa]  (16) to node[el,anchor=west] {\m{g}} (13);
 
%%%%%%%%%%%%%%%%%%%%%%%%%%%%%%%%%%%%%%%%%%%%%%%%%%%%%%%%%%%%%%
%	\node (5au) [above = 0.2 of 5a] {};
%	\node (3au) [above = 0.2 of 3a] {};
%	\draw[se] (12a) to (5au) to (3au) to (2a);

	\node (12u) [above = 0.2 of 12] {};
	\node (5u) [above = 0.2 of 5] {};
	\node (4u) [above = 0.2 of 4] {};

	\draw[se] (13) to (12u) to (5u) to (4u) to (3);
	\draw[se] (13) to (12u) to (5u) to (4);

	\node (5r1) [right = 1.0 of 5] {};
	\node (5d1) [below = 0.2 of 5] {};
%	\node (5u1) [above = 0.2 of 5] {};
	\node (4d1) [below = 0.2 of 4] {};
	\node (3d1) [below = 0.2 of 3] {};

	\draw[se] (12) to (5r1) to[out=180,in=0] (5d1) to (4d1) to (3d1) to (2);

	\node (7r) [right = 1.2 of 7.0] {};
	\node (7l) [left = 1.6 of 7.180] {};
	\node (7u) [above = 0.2 of 7] {};
	\draw[se] (16) to (7r) to (7u) to (7l) to (6);

	\draw[me] (16) to (7);

\end{tikzpicture}
}
\caption{
The state after \lstinline|completeDownNewSources|
}
\label{fig_makeTerm_up_down.2}
\end{subfigure}

\begin{subfigure}[t]{0.99\textwidth}
\framebox[\textwidth]{
\begin{tikzpicture}
	\node[gttn] (1)              {$$};
	\node[gl]   (1l) [below = 0 of 1] {\m{n_0}};

	\node[gtn,label=center:\m{a}]  (2) [above left = 0.5cm and 1.8cm of 1] {\phantom{B}};
	\node[gtn,label=center:\m{b}]  (3) [above left = 0.5cm and 0.4cm of 1] {\phantom{B}};
	\node[gtn,label=center:\m{c}]  (4) [above right= 0.5cm and 0.4cm of 1] {\phantom{B}};
	\node[gtn,label=center:\m{d}]  (5) [above right= 0.5cm and 1.8cm of 1] {\phantom{B}};

	\node[gtn]  (6)  [above right = 1.5cm and 0.3cm of 2] {\stackB{f(a)}{h(b)}};
	\draw[gfa]  (6) to node[el,anchor=east] {\m{f}} (2);
	\draw[gfa]  (6) to node[el,anchor=west] {\m{g}} (3);

	\node[gtn]  (7)  [above left = 1.5cm and 0.3cm of 5] {\stackB{g(c)}{h(d)}};
	\draw[gfa]  (7) to node[el,anchor=east] {\m{g}} (4);
	\draw[gfa]  (7) to node[el,anchor=west] {\m{h}} (5);

%%%%%%%%%%%%%%%%%%%%%%%%%%%%%%%%%%%%%%%%%%%%%%%%%%%%%%%%%%%%%%
	\node[gttn] (11)  [right = 7cm of 1] {$$};
	\node[gl]   (11l) [below = 0 of 11]   {\m{n_1}};

	\node[gtn,label=center:\m{a}]  (12) [above left = 0.5cm and 1.0cm of 11] {\phantom{B,B}};
	\node[gtn,label=center:\m{b,c}]  (13) [above = 0.43cm of 11] {\phantom{B,B}};
	\node[mgtn,label=center:\m{\textcolor{red}{d}}]  (15) [above right= 0.5cm and 1.0cm of 11] {\phantom{B,B}};

	\node[gtn]  (16)  [above = 2.13cm of 11] {\stackB{f(a),g(b)}{g(c),\textcolor{red}{h(d)}}};
	\draw[gfa]  (16) to node[el,anchor=east] {\m{f}} (12);
	\draw[gfa]  (16) to node[el,anchor=west] {\m{g}} (13);
	\draw[mgfa]  (16) to node[ml,anchor=west] {\m{h}} (15);
 
%%%%%%%%%%%%%%%%%%%%%%%%%%%%%%%%%%%%%%%%%%%%%%%%%%%%%%%%%%%%%%
%	\node (5au) [above = 0.2 of 5a] {};
%	\node (3au) [above = 0.2 of 3a] {};
%	\draw[se] (12a) to (5au) to (3au) to (2a);

	\node (12u) [above = 0.2 of 12] {};
	\node (5u) [above = 0.2 of 5] {};
	\node (4u) [above = 0.2 of 4] {};

	\draw[se] (13) to (12u) to (5u) to (4u) to (3);
	\draw[se] (13) to (12u) to (5u) to (4);

	\node (5r1) [right = 1.0 of 5] {};
	\node (5d1) [below = 0.2 of 5] {};
%	\node (5u1) [above = 0.2 of 5] {};
	\node (4d1) [below = 0.2 of 4] {};
	\node (3d1) [below = 0.2 of 3] {};

	\draw[se] (12) to (5r1) to[out=180,in=0] (5d1) to (4d1) to (3d1) to (2);

	\node (7r) [right = 1.2 of 7.0] {};
	\node (7l) [left = 1.6 of 7.180] {};
	\node (7u) [above = 0.2 of 7] {};
	\draw[se] (16) to (7r) to (7u) to (7l) to (6);

	\draw[se] (16) to (7);

\end{tikzpicture}
}
\caption{
The state after \lstinline|updateForNewGFAs|
}
\label{fig_makeTerm_up_down.3}
\end{subfigure}
\end{figure}

\begin{figure}
\begin{tikzpicture}
	\node[gttn] (1)              {$$};
	\node[gl]   (1l) [below = 0 of 1] {\m{n_0}};

	\node[gtn,label=center:\m{a}]  (2) [above left = 0.5cm and 1.8cm of 1] {\phantom{B}};
	\node[gtn,label=center:\m{b}]  (3) [above left = 0.5cm and 0.4cm of 1] {\phantom{B}};
	\node[gtn,label=center:\m{c}]  (4) [above right= 0.5cm and 0.4cm of 1] {\phantom{B}};
	\node[gtn,label=center:\m{d}]  (5) [above right= 0.5cm and 1.8cm of 1] {\phantom{B}};

	\node[gtn]  (6)  [above right = 1.5cm and 0.3cm of 2] {\stackB{f(a)}{h(b)}};
	\draw[gfa]  (6) to node[el,anchor=east] {\m{f}} (2);
	\draw[gfa]  (6) to node[el,anchor=west] {\m{g}} (3);

	\node[gtn]  (7)  [above left = 1.5cm and 0.3cm of 5] {\stackB{g(c)}{h(d)}};
	\draw[gfa]  (7) to node[el,anchor=east] {\m{g}} (4);
	\draw[gfa]  (7) to node[el,anchor=west] {\m{h}} (5);

%%%%%%%%%%%%%%%%%%%%%%%%%%%%%%%%%%%%%%%%%%%%%%%%%%%%%%%%%%%%%%
	\node[gttn] (11)  [right = 7cm of 1] {$$};
	\node[gl]   (11l) [below = 0 of 11]   {\m{n_1}};

	\node[gtn,label=center:\m{a}]  (12) [above left = 0.5cm and 1.0cm of 11] {\phantom{B,B}};
	\node[gtn,label=center:\m{b,c}]  (13) [above = 0.43cm of 11] {\phantom{B,B}};
	\node[gtn,label=center:\m{d}]  (15) [above right= 0.5cm and 1.0cm of 11] {\phantom{B,B}};

	\node[gtn]  (16)  [above = 2.13cm of 11] {\stackB{f(a),g(b)}{g(c),h(d)}};
	\draw[gfa]  (16) to node[el,anchor=east] {\m{f}} (12);
	\draw[gfa]  (16) to node[el,anchor=west] {\m{g}} (13);
	\draw[gfa]  (16) to node[el,anchor=west] {\m{h}} (15);
 
%%%%%%%%%%%%%%%%%%%%%%%%%%%%%%%%%%%%%%%%%%%%%%%%%%%%%%%%%%%%%%
%	\node (5au) [above = 0.2 of 5a] {};
%	\node (3au) [above = 0.2 of 3a] {};
%	\draw[se] (12a) to (5au) to (3au) to (2a);

	\node (12u) [above = 0.2 of 12] {};
	\node (5u) [above = 0.2 of 5] {};
	\node (4u) [above = 0.2 of 4] {};

	\draw[se] (13) to (12u) to (5u) to (4u) to (3);
	\draw[se] (13) to (12u) to (5u) to (4);

	\node (5r1) [right = 1.0 of 5] {};
	\node (5d1) [below = 0.2 of 5] {};
%	\node (5u1) [above = 0.2 of 5] {};
	\node (4d1) [below = 0.2 of 4] {};
	\node (3d1) [below = 0.2 of 3] {};

	\draw[se] (12) to (5r1) to[out=180,in=0] (5d1) to (4d1) to (3d1) to (2);

	\node (13d) [below = 0.1 of 13] {};
	\node (5dr) [below right = 0.2 and 0.5 of 5] {};
	\draw[se] (15) to (13d) to (5dr) to (5);


	\node (7r) [right = 1.2 of 7.0] {};
	\node (7l) [left = 1.6 of 7.180] {};
	\node (7u) [above = 0.2 of 7] {};
	\draw[se] (16) to (7r) to (7u) to (7l) to (6);

	\draw[se] (16) to (7);

\end{tikzpicture}
\caption{
The final state
}
\label{fig_makeTerm_up_down.4}
\end{figure}





\subsection*{Algorithm properties}
We have sketched an algorithm for the incremental, on-demand propagation of equality information in a CFG without joins.
The condition the algorithm aims to satisfy is that, for a consistent CFG (that is, where all CFG-nodes are consistent w.r.t. the invariants we have shown above), for any CFG-node, any two terms that are represented in the EC-graph of the node are equal in the EC-graph iff it holds at the node that they are equal - \\
\m{\forall n \in \cfg, s,t\in \terms{g_n} \cdot n \models s=t \Leftrightarrow [s]_n=[t]_n}. \\
We have not proved this completeness property formally, our informal argument runs as follows:\\
For a pair of consecutive CFG-nodes, the property holds by strong induction on the maximal depth d of \m{s,t} - the induction hypothesis is that, for all terms smaller than d, both the source edges are correct (that is, there is a source edge iff there is an intersection in the represented terms) and that \m{n \models u=v \Rightarrow [u]_n=[v]_n} holds for all u,v of depth up to d. 
Together with the property that each EC-graph is congruence and transitive closed, and that we merge two \GTs{} that share a source (essentially transitive closure) and together with the invariant conjuncts (ensuring the propagation of all \GFAs{} and the addition of source-edges), we should be able to show that the induction property holds also for depth d.\\
For a CFG of depth more than two (linear or tree-shaped), we need another property, essentially that if a term is represented at some CFG-node and is also represented by some transitive predecessor, then it is represented on all the path between the node and the transitive predecessor. This property is ensured by our request cache - each term represented at a node is either represented in the direct predecessor, one of its sub-terms is not represented, or it is represented by some \RGFA{} in the predecessor. 
Because we disallow an \RGFA{} if it has a \GT{} source, strong induction on both the term depth and the length of the source-edge chain should show that there is a chain of source-edges between each pair of nodes with the same term represented, that are on the same CFG-path.
A formal proof of the above remains as future work.

Another important property is incrementallity - no operation (congruence closure, comparison of function symbols) happens twice for the same inputs, unless one of the participants of the operation has changed. We achieve this property by marking any change in the structure and only operating on modified parts of the structure. We also ensure that equality information is propagated eagerly from predecessors, so that the same congruence closure operation is not performed at a CFG-node and any of its successor. 
We have not performed a formal complexity analysis on the algorithm, and some finer detail of the algorithm are not shown above (e.g. how to find the inverse sources of tuples and how to represent requests that are Cartesian products) can affect the worst case complexity.\\
However, we can see that the total size of each EC-graph is at most the sum of sizes of all the clauses in all its CFG-node's predecessors (no operation in our algorithm adds a \GFA{} that does not come from some input clause) - which suggests a quadratic space complexity and between quadratic and cubic time-complexity (with some logarithmic factor from table lookups).

In the chapters \ref{chapter:scoping} and \ref{chapter:bounds} we show two improvements on the algorithm, namely restricting the set of constants allowed at each CFG-node and restricting the depth of represented term. Both restrictions are enforced at the construction level of the algorithm and so hold at any intermediate state, and allow us to give a strict space bound on the algorithm.
Completeness can be preserved in the first case by generating some non-unit clauses (as we show) and in the second case we can incrementally increase the allowed depth to allow completeness.







\newpage
\section{Joins}\label{section:ugfole:joins}
We have seen, in the previous section, our propagation algorithm for sequential CFG-nodes.
In this section we show the necessary changes to make the algorithm apply for CFG-nodes with two predecessors, where the semantics is that we can only propagate an equality to the join node if it is implied by both predecessors.

As opposed to the sequential case, for joins we cannot guarantee that any equality that holds at a CFG-node on two represented terms also holds in the EC-graph, simply because not all interpolants are representable as a finite conjunction of equalities - as we have seen e.g. in figure \ref{snippet3.5} - the interpolant between \m{\s{c=a} \sqcup \s{c=b}} and \s{a=b \land a\neq c} cannot be represented as a finite conjunction of unit equations. 
Furthermore, we have seen that even in the cases where a unit interpolant exists, the smallest such interpolant may be exponential in the program size, even in the sub-fragment that does not allow cycles in EC-graphs. In light of the above, our algorithm for joins guarantees only a weaker property - we guarantee, for each join, that any pair of terms represented in both joinees and at the join, are equal at the join EC-graph iff they are equal at both joinees. 
The guarantee for source-edge chains is weakened so that, if a term t is represented at a CFG-node n and at some transitive predecessor p, we can only guarantee that \m{[t]_n} is connected by a source-edge chain to \m{[t]_p} if there is a cut in the graph between n and the root that includes p (that is, a set of CFG-nodes that pairwise do not share a path, but that every path from the root to n includes one node in the set) and t is represented on each member of the cut - this simply means that, at any join between p and n, t is represented in both joinees.\\
For equality propagation our aim is to guarantee that if
\m{n \models s=t} for some CFG-node n, and\\ \m{s,t \in \terms{g_n}}, then if there is a cut in the CFG between n and the root where each node \m{p} in the cut satisfies \m{[s]_p=[t]_p} then \m{[s]_n=[t]_n}. A formal proof remains as future work.

\subsection{Equality propagation}
The main difference with joins is that we do not add the \GFAs{} from all sources of a \GT{} to the \GT{}, 
but rather only \GFAs{} that have a corresponding source-\GFA{} in \emph{both} joinees.\\
Consider the example in figure \ref{EC_join_graph_1.0}.\\
We match the \GFAs{} \m{f([b]_0)} and \m{f([a,b]_1)} because they share the same function symbol f, 
and both \GFAs{} are in the sources of the same \GT{}.\\
Compare the situation to the one in figure \ref{EC_join_graph_1.1} - here the function labels f,g do not match, hence we do not add a \GFA{} to the \GT{} \m{[f(a)]_n} - we do have the matching \GFAs{} \m{f([a]_0)} and \m{f([a,b]_1)}, for which we already have a \GFA{} with the inverse source of the tuple - \m{[a]_n}. We can see also how the condition for merging with common sources works - while for the sequential case any two \GTs{} that share a source are merged, for the join case we merge \GTs{} only if they share a source in \emph{all} joinees.\\
Now consider the situation in figure \ref{EC_join_graph_1.1} - here we have a \GT{} in \m{g_n} which represents no terms.
We denote such a \GT{} as an \newdef{infeasible} \GT{}, where a \GT{} is feasible if it represents at least one term.\\
Such infeasible \GTs{} do not arise in the sequential case for consistent EC-graphs, although they do when scoping or depth restrictions are in place.
Our algorithm allows such \GTs{}, but they are only visible to internal methods of the EC-graph. Essentially, they serve as witnesses that the EC-graph is consistent, but they are not visible to users of the EC-graph - that is, the \lstinline|gfas| and \lstinline|superTerms| fields visible to users of the EC-graph do not contain these \GTs{} and any \GFA{} whose tuple is infeasible).\\
For example, if we perform E-matching on the EC-graph of \ref{EC_join_graph_1.1}, matching the (non-ground) term \m{f(x)} with the \GT{} \m{[f(a)]_n}, we get exactly one match (substitution) which is \m{[x \mapsto [a]_n]}. 
These infeasible \GTs{} and \GFAs{} that contains them are also not visible to successors of the EC-node - they cannot be the target of source-edges, although they can be the source of source-edges (in fact they must have a source in each joinee), as shown in the example.


\begin{figure}
\begin{tikzpicture}
  \node (1)  {};
%%%%%%%%%%%%%%%%%%%%%%%%%%%%%%%%%%%%%%%%%%%%%%%%%%%%%%%%%%%%%%
  \node[gttn] (11)  [above right= 2.5cm and 0cm of 1] {};
	\node[gl]   (11l) [below = 0 of 11]   {\m{p_0}};

	\node[gtn,label=center:\m{a}]  (12) [above left  = 0.5cm and 1cm of 11] {\phantom{B}};
	\node[gtn,label=center:\m{b}]  (14) [above right = 0.5cm and 1cm of 11] {\phantom{B}};
	
	\node[gtn,ultra thick]  (18) [above = 2cm of 11] {\stackB{f(a)}{f(b)}};
	\draw[gfa]  (18) to node[el] {\m{f}} (12);
	\draw[gfa,ultra thick]  (18) to node[el,anchor=west] {\m{f}} (14);

%%%%%%%%%%%%%%%%%%%%%%%%%%%%%%%%%%%%%%%%%%%%%%%%%%%%%%%%%%%%%%
	\node[gttn] (21)  [below right= 2.5cm and 0cm of 1] {};
	\node[gl]   (21l) [below = 0 of 21]   {\m{p_1}};

  \node[gtn,label=center:\m{a,b}]  (22) [above  = 0.5cm of 21] {\phantom{B,B}};

  \node[gtn,ultra thick]  (28) [above = 1cm of 22] {\stackB{f(a)}{f(b)}};
	\draw[gfa,ultra thick]  (28) to node[el] {\m{f}} (22);
%%%%%%%%%%%%%%%%%%%%%%%%%%%%%%%%%%%%%%%%%%%%%%%%%%%%%%%%%%%%%%

  \node[gttn] (31)  [right = 6cm of 1] {};
	\node[gl]   (31l) [below = 0 of 31]   {\m{n}};

  \node[gtn,label=center:\m{a}]  (32) [above left  = 0.5cm and 1cm of 31] {\phantom{B}};
  \node[mgtn,label=center:\m{\textcolor{red}{b}}]  (34) [above right = 0.5cm and 1cm of 31] {\phantom{B}};

	\node[gtn,ultra thick]  (38) [above = 2cm of 31] {\stackB{f(a)}{\textcolor{red}{f(b)}}};
	\draw[gfa]  (38) to node[el,anchor=south east] {\m{f}} (32);
	\draw[mgfa,ultra thick,dashed] (38) to node[ml,anchor=south west] {\m{f}} (34);

%%%%%%%%%%%%%%%%%%%%%%%%%%%%%%%%%%%%%%%%%%%%%%%%%%%%%%%%%%%%%%
%	\node (15a) [right = 0.5cm of 15] {};
%	\node (35b) [above left = 0.2cm and 1.5cm of 35] {};
%	\node (35c) [below left = 0.2cm and 1.5cm of 35] {};
%	\node (37b) [above left = 0.2cm and 1.5cm of 37] {};
%	\node (37c) [below left = 0.2cm and 1.5cm of 37] {};
%	\draw[se,ultra thick] (37) to[out=180,in=0] (37b) to[out=180,in=0] (17);
%	\draw[se,ultra thick] (37) to[out=180,in=0] (37c) to[out=180,in=0] (25);

	\node (18a) [right = 0.5cm of 18] {};
	\node (38b) [above left = 0.2cm and 1.5cm of 38] {};
	\node (38c) [below left = 0.2cm and 1.5cm of 38] {};
	\draw[se,ultra thick] (38) to[out=180,in=0] (38b) to[out=180,in=0] (18a) to[out=180,in=0] (18);
	\draw[se,ultra thick] (38) to[out=180,in=0] (38c) to[out=180,in=0] (28);

\end{tikzpicture}

\caption{Example for \GFA{} completeness for joins\\
The join is \m{\s{a=b} \sqcup \s{f(a)=f(b)}}.\\
The matching function edges are shown in bold.}
\label{EC_join_graph_1.0}
\end{figure}

\begin{figure}
\begin{tikzpicture}
  \node (1)  {};
%%%%%%%%%%%%%%%%%%%%%%%%%%%%%%%%%%%%%%%%%%%%%%%%%%%%%%%%%%%%%%
  \node[gttn] (11)  [above right= 2.5cm and 0cm of 1] {};
	\node[gl]   (11l) [below = 0 of 11]   {\m{p_0}};

	\node[gtn,label=center:\m{a}]  (12) [above left  = 0.5cm and 1cm of 11] {\phantom{B}};
	\node[gtn,label=center:\m{b}]  (14) [above right = 0.5cm and 1cm of 11] {\phantom{B}};
	
	\node[gtn,ultra thick]  (18) [above = 2cm of 11] {\stackB{f(a)}{g(b)}};
	\draw[gfa]  (18) to node[el] {\m{f}} (12);
	\draw[gfa,ultra thick]  (18) to node[el,anchor=west] {\m{g}} (14);

%%%%%%%%%%%%%%%%%%%%%%%%%%%%%%%%%%%%%%%%%%%%%%%%%%%%%%%%%%%%%%
	\node[gttn] (21)  [below right= 2.5cm and 0cm of 1] {};
	\node[gl]   (21l) [below = 0 of 21]   {\m{p_1}};

  \node[gtn,label=center:\m{a,b}]  (22) [above  = 0.5cm of 21] {\phantom{B,B}};

  \node[gtn,ultra thick]  (28) [above = 1cm of 22] {\stackB{f(a)}{f(b)}};
	\draw[gfa,ultra thick]  (28) to node[el] {\m{f}} (22);
%%%%%%%%%%%%%%%%%%%%%%%%%%%%%%%%%%%%%%%%%%%%%%%%%%%%%%%%%%%%%%

  \node[gttn] (31)  [right = 6cm of 1] {};
	\node[gl]   (31l) [below = 0 of 31]   {\m{n}};

  \node[gtn,label=center:\m{a}]  (32) [above left  = 0.5cm and 1cm of 31] {\phantom{B}};
%  \node[mgtn,label=center:\m{\textcolor{red}{b}}]  (34) [above right = 1cm and 1cm of 31] {\phantom{B}};

	\node[gtn,ultra thick]  (38) [above = 2cm of 31] {\m{f(a)}};
	\draw[gfa]  (38) to node[el,anchor=south east] {\m{f}} (32);
%	\draw[mgfa,ultra thick,dashed] (38) to node[ml,anchor=south west] {\m{f}} (34);

%%%%%%%%%%%%%%%%%%%%%%%%%%%%%%%%%%%%%%%%%%%%%%%%%%%%%%%%%%%%%%
%	\node (15a) [right = 0.5cm of 15] {};
%	\node (35b) [above left = 0.2cm and 1.5cm of 35] {};
%	\node (35c) [below left = 0.2cm and 1.5cm of 35] {};
%	\node (37b) [above left = 0.2cm and 1.5cm of 37] {};
%	\node (37c) [below left = 0.2cm and 1.5cm of 37] {};
%	\draw[se,ultra thick] (37) to[out=180,in=0] (37b) to[out=180,in=0] (17);
%	\draw[se,ultra thick] (37) to[out=180,in=0] (37c) to[out=180,in=0] (25);

	\node (18a) [right = 0.5cm of 18] {};
	\node (38b) [above left = 0.2cm and 1.5cm of 38] {};
	\node (38c) [below left = 0.2cm and 1.5cm of 38] {};
	\draw[se,ultra thick] (38) to[out=180,in=0] (38b) to[out=180,in=0] (18a) to[out=180,in=0] (18);
	\draw[se,ultra thick] (38) to[out=180,in=0] (38c) to[out=180,in=0] (28);

\end{tikzpicture}

\caption{Example for \GFA{} completeness for joins\\
The join is \m{\s{a=b} \sqcup \m{f(a)=g(b)}}.\\
In this case the highlighted function edges do not match.}
\label{EC_join_graph_1.1}
\end{figure}



\begin{figure}
\begin{tikzpicture}
  \node (1)  {};
%%%%%%%%%%%%%%%%%%%%%%%%%%%%%%%%%%%%%%%%%%%%%%%%%%%%%%%%%%%%%%
  \node[gttn] (11)  [above right= 2.5cm and 0cm of 1] {};
	\node[gl]   (11l) [below = 0 of 11]   {\m{p_0}};

	\node[gtn,label=center:\m{a}]  (12) [above left  = 0.5cm and 1cm of 11] {\phantom{B}};
	\node[gtn,label=center:\m{b}]  (14) [above right = 0.5cm and 1cm of 11] {\phantom{B}};
	
	\node[gtn,ultra thick]  (18) [above = 2cm of 11] {\stackB{f(a)}{f(b)}};
	\draw[gfa]  (18) to node[el] {\m{f}} (12);
	\draw[gfa,ultra thick]  (18) to node[el,anchor=west] {\m{f}} (14);

%%%%%%%%%%%%%%%%%%%%%%%%%%%%%%%%%%%%%%%%%%%%%%%%%%%%%%%%%%%%%%
	\node[gttn] (21)  [below right= 3.0cm and 0cm of 1] {};
	\node[gl]   (21l) [below = 0 of 21]   {\m{p_1}};

  \node[gtn,label=center:\m{a,b}]  (22) [above  = 0.5cm of 21] {\phantom{B,B}};

  \node[gtn,ultra thick]  (28) [above = 1cm of 22] {\stackB{f(a)}{f(b)}};
	\draw[gfa,ultra thick]  (28) to node[el] {\m{f}} (22);
%%%%%%%%%%%%%%%%%%%%%%%%%%%%%%%%%%%%%%%%%%%%%%%%%%%%%%%%%%%%%%

  \node[gttn] (31)  [right = 6cm of 1] {};
	\node[gl]   (31l) [below = 0 of 31]   {\m{n}};

  \node[gtn,label=center:\m{a}]  (32) [above left  = 0.5cm and 1cm of 31] {\phantom{B}};
  \node[gtn,,ultra thick,label=center:\m{}]  (34) [above right = 0.5cm and 1cm of 31] {\phantom{B}};

	\node[gtn,ultra thick]  (38) [above = 2.5cm of 31] {\m{f(a)}};
	\draw[gfa]  (38) to node[el,anchor=south east] {\m{f}} (32);
	\draw[gfa,ultra thick] (38) to node[el,anchor=south west] {\m{f}} (34);

%%%%%%%%%%%%%%%%%%%%%%%%%%%%%%%%%%%%%%%%%%%%%%%%%%%%%%%%%%%%%%
%	\node (15a) [right = 0.5cm of 15] {};
%	\node (35b) [above left = 0.2cm and 1.5cm of 35] {};
%	\node (35c) [below left = 0.2cm and 1.5cm of 35] {};
%	\node (37b) [above left = 0.2cm and 1.5cm of 37] {};
%	\node (37c) [below left = 0.2cm and 1.5cm of 37] {};
%	\draw[se,ultra thick] (37) to[out=180,in=0] (37b) to[out=180,in=0] (17);
%	\draw[se,ultra thick] (37) to[out=180,in=0] (37c) to[out=180,in=0] (25);

	\node (18a) [right = 0.5cm of 18] {};
	\node (38b) [above left = 0.2cm and 1.5cm of 38] {};
	\node (38c) [below left = 0.2cm and 1.5cm of 38] {};
	\draw[se,ultra thick] (38) to[out=180,in=0] (38b) to[out=180,in=0] (18a) to[out=180,in=0] (18);
	\draw[se,ultra thick] (38) to[out=180,in=0] (38c) to[out=180,in=0] (28);

\end{tikzpicture}

\caption{Example for \GFA{} completeness for joins\\
The join is \m{\s{a=b} \sqcup \s{f(a)=f(b)}}.\\
The matching function edges are highlighted.}
\label{EC_join_graph_1.2}
\end{figure}

\bigskip
\noindent
We phrase now the version of the local invariant for joins - showing each conjunct.

The conjunct that defines source edges is unchanged.
The first conjunct defines when we have to add a \GFA{} to a \GT{} based on its sources - when there are \GFAs{} in both sources that agree on the function symbol:
\begin{figure}[H]
%\textbf{The local source correctness invariant:}\\
For a CFG-node n.\\
\m{\forall u \in g_n, v_0 \in \sources{}{0}{u}, \fa{f}{s_0} \in v_0, v_1 \in \sources{}{1}{u}, \fa{f}{s_1} \in v_1 \cdot}\\
\m{~~~\exists \fa{f}{t} \in u \cdot \tup{s_0} \in \sources{}{0}{\tup{t}} \land \tup{s_1} \in \sources{}{1}{\tup{t}}}
\end{figure}

\noindent
The second conjunct defines when we have to merge two \GTs{} at the join - as we have seen, when they share a source in \emph{all} predecessors.

\noindent
Consider the example in figure \ref{snippet3.17a_graph.0}.
No two \GTs{} at the join share both sources, but they all share one source.\\
We now want to invoke \m{n}.\lstinline|assumeEqual([a],[c])| - if we look at the join, we see that\\ \m{(a=b \lor b=c) \land a=c \models a=b=c}.\\
 In figure \ref{snippet3.17a_graph.1} we can see the state after the first merge step of \lstinline|assumeEqual|.
We can see that \m{[a,c]_n} and \m{[b]_n} share sources in both joinees, and hence they are merged.
In our implementation, we mark these \GTs{} for merging already in the 
\lstinline|addSource| method - at the point where a new source is assigned.

\noindent
The invariant conjunct defining this behaviour is:
\begin{figure}[H]
%\textbf{The local source correctness invariant:}\\
For a CFG-node n.\\
\m{\forall u,v \in g_n \cdot}\\
\m{~~~(\forall p \in \preds{n} \cdot \sources{}{p}{u} \cap \sources{}{p}{v} \neq \emptyset) \Rightarrow u=v}
\end{figure}


\begin{figure}
\begin{subfigure}[t]{0.99\textwidth}
\framebox[\textwidth]{
\begin{tikzpicture}
  \node (1)  {};
%%%%%%%%%%%%%%%%%%%%%%%%%%%%%%%%%%%%%%%%%%%%%%%%%%%%%%%%%%%%%%
  \node[gttn] (11)  [above right= 1.5cm and 0cm of 1] {};
	\node[gl]   (11l) [below = 0 of 11]   {\m{p_0}};

  \node[gtn,label=center:\phantom{B,B}]  (12) [above left  = 1cm and 1cm of 11] {\m{a,b}};
  \node[gtn,label=center:\phantom{B,B}]  (14) [above right = 1cm and 1cm of 11] {\m{c}};
	
%%%%%%%%%%%%%%%%%%%%%%%%%%%%%%%%%%%%%%%%%%%%%%%%%%%%%%%%%%%%%%
  \node[gttn] (21)  [below right= 1.5cm and 0cm of 1] {};
	\node[gl]   (21l) [below = 0 of 21]   {\m{p_1}};

  \node[gtn,label=center:\phantom{B,B}]  (22) [above left  = 1.0cm and 1.0cm of 21] {\m{a}};
  \node[gtn,label=center:\phantom{B,B}]  (23) [above right = 1.0cm and 1.0cm of 21] {\m{b,c}};
	
%%%%%%%%%%%%%%%%%%%%%%%%%%%%%%%%%%%%%%%%%%%%%%%%%%%%%%%%%%%%%%

  \node[gttn] (31)  [right = 5cm of 1] {};
	\node[gl]   (31l) [below = 0 of 31]   {\m{n}};

%  \node (31jl)  [above left = -0.2cm and 0cm of 31] {$\sqcup$};

  \node[gtn,label=center:\phantom{B,B}]  (32) [above left  = 1.0cm and 1.2cm of 31] {\m{a}};
  \node[gtn,label=center:\phantom{B,B}]  (33) [above       = 0.9cm           of 31] {\m{b}};
  \node[gtn,label=center:\phantom{B,B}]  (34) [above right = 1.0cm and 1.2cm of 31] {\m{c}};
	

%%%%%%%%%%%%%%%%%%%%%%%%%%%%%%%%%%%%%%%%%%%%%%%%%%%%%%%%%%%%%%
%	\draw[se] (31) to[out=180,in=0] (11);
%	\draw[se] (31) to[out=180,in=0] (21);

	\node (12a) [right = 0.5cm of 12] {};
	\node (23a) [right = 0.5cm of 23] {};
	\draw[se] (32) to[out=180,in=0] (12a) to[out=180,in=0] (12);
	\draw[se] (32) to[out=180,in=0] (22);
	\draw[se] (33) to[out=180,in=0] (12a)  to[out=180,in=0] (12);
	\draw[se] (33) to[out=180,in=0] (23a)  to[out=180,in=0] (23);
	\draw[se] (34) to[out=180,in=0] (14);
	\draw[se] (34) to[out=180,in=0] (23a) to[out=180,in=0] (23);

\end{tikzpicture}
}
\caption{
Join shared sources\\
before \m{n}.\lstinline|assumeEqual([a],[c])|
}
\label{snippet3.17a_graph.0}
\end{subfigure}

\begin{subfigure}[t]{0.99\textwidth}
\framebox[\textwidth]{
\begin{tikzpicture}
  \node (1)  {};
%%%%%%%%%%%%%%%%%%%%%%%%%%%%%%%%%%%%%%%%%%%%%%%%%%%%%%%%%%%%%%
  \node[gttn] (11)  [above right= 1.5cm and 0cm of 1] {};
	\node[gl]   (11l) [below = 0 of 11]   {\m{p_0}};

  \node[gtn,label=center:\phantom{B,B}]  (12) [above left  = 1cm and 1cm of 11] {\m{a,b}};
  \node[gtn,label=center:\phantom{B,B}]  (14) [above right = 1cm and 1cm of 11] {\m{c}};
	
%%%%%%%%%%%%%%%%%%%%%%%%%%%%%%%%%%%%%%%%%%%%%%%%%%%%%%%%%%%%%%
  \node[gttn] (21)  [below right= 1.5cm and 0cm of 1] {};
	\node[gl]   (21l) [below = 0 of 21]   {\m{p_1}};

  \node[gtn,label=center:\phantom{B,B}]  (22) [above left  = 1.0cm and 1.0cm of 21] {\m{a}};
  \node[gtn,label=center:\phantom{B,B}]  (23) [above right = 1.0cm and 1.0cm of 21] {\m{b,c}};
	
%%%%%%%%%%%%%%%%%%%%%%%%%%%%%%%%%%%%%%%%%%%%%%%%%%%%%%%%%%%%%%

  \node[gttn] (31)  [right = 5cm of 1] {};
	\node[gl]   (31l) [below = 0 of 31]   {\m{n}};

%  \node (31jl)  [above left = -0.2cm and 0cm of 31] {$\sqcup$};

  \node[gtn,label=center:\phantom{B,B}]  (32) [above left  = 1.0cm and 0.6cm of 31] {\m{a,c}};
  \node[gtn,label=center:\phantom{B,B}]  (33) [above  right= 0.9cm and 0.6cm of 31] {\m{b}};
	

%%%%%%%%%%%%%%%%%%%%%%%%%%%%%%%%%%%%%%%%%%%%%%%%%%%%%%%%%%%%%%
%	\draw[se] (31) to[out=180,in=0] (11);
%	\draw[se] (31) to[out=180,in=0] (21);

	\node (12a) [right = 0.5cm of 12] {};
	\node (23a) [right = 0.5cm of 23] {};
	\draw[se,ultra thick] (32) to[out=180,in=0] (12a) to[out=180,in=0] (12);
	\draw[se] (32) to[out=180,in=0] (22);
	\draw[se] (32) to[out=180,in=0] (14);
	\draw[se,ultra thick] (32) to[out=180,in=0] (23a) to[out=180,in=0] (23);
	\node (32u) [above left = 0.5cm and 0.5cm of 32] {};
	\draw[se,ultra thick] (33) to[out=180,in=-5] (32u) to[out=175,in=0] (12a)  to[out=180,in=0] (12);
	\draw[se,ultra thick] (33) to[out=180,in=0] (23a)  to[out=180,in=0] (23);

\end{tikzpicture}
}
\caption{
Join shared sources\\
after merging \m{[a],[c]}\\
Common sources are highlighted
}
\label{snippet3.17a_graph.1}
\end{subfigure}

\begin{subfigure}[t]{0.99\textwidth}
\framebox[\textwidth]{
\begin{tikzpicture}
  \node (1)  {};
%%%%%%%%%%%%%%%%%%%%%%%%%%%%%%%%%%%%%%%%%%%%%%%%%%%%%%%%%%%%%%
  \node[gttn] (11)  [above right= 1.5cm and 0cm of 1] {};
	\node[gl]   (11l) [below = 0 of 11]   {\m{p_0}};

  \node[gtn,label=center:\phantom{B,B}]  (12) [above left  = 1cm and 1cm of 11] {\m{a,b}};
  \node[gtn,label=center:\phantom{B,B}]  (14) [above right = 1cm and 1cm of 11] {\m{c}};
	
%%%%%%%%%%%%%%%%%%%%%%%%%%%%%%%%%%%%%%%%%%%%%%%%%%%%%%%%%%%%%%
  \node[gttn] (21)  [below right= 1.5cm and 0cm of 1] {};
	\node[gl]   (21l) [below = 0 of 21]   {\m{p_1}};

  \node[gtn,label=center:\phantom{B,B}]  (22) [above left  = 1.0cm and 1.0cm of 21] {\m{a}};
  \node[gtn,label=center:\phantom{B,B}]  (23) [above right = 1.0cm and 1.0cm of 21] {\m{b,c}};
	
%%%%%%%%%%%%%%%%%%%%%%%%%%%%%%%%%%%%%%%%%%%%%%%%%%%%%%%%%%%%%%

  \node[gttn] (31)  [right = 5cm of 1] {};
	\node[gl]   (31l) [below = 0 of 31]   {\m{n}};

  \node[gtn,label=center:\phantom{B,B}]  (33) [above       = 0.9cm           of 31] {\m{a,b,c}};
	

%%%%%%%%%%%%%%%%%%%%%%%%%%%%%%%%%%%%%%%%%%%%%%%%%%%%%%%%%%%%%%
%	\draw[se] (31) to[out=180,in=0] (11);
%	\draw[se] (31) to[out=180,in=0] (21);

	\node (12a) [right = 0.5cm of 12] {};
	\node (23a) [right = 0.5cm of 23] {};
	\draw[se] (33) to[out=180,in=0] (12a) to[out=180,in=0] (12);
	\draw[se] (33) to[out=180,in=0] (22);
	\draw[se] (33) to[out=180,in=0] (14);
	\draw[se] (33) to[out=180,in=0] (23a) to[out=180,in=0] (23);

\end{tikzpicture}
}
\caption{
Join shared sources\\
Final state
}
\label{snippet3.17a_graph.2}
\end{subfigure}
\end{figure}







\noindent
The third conjunct defines when we have to replace an \RGFA{} with a \GT{} - this condition is crucial for performance and even termination - if we force replacing an \RGFA{} with a \GT{} if \emph{any} of the predecessors represents any term represented by the \RGFA{}, we may have an infinite number of \GTs{} - for example, in the join \m{\s{a=f(a)} \sqcup \s{a=b}}, where some successor of the join includes the equation \m{a=f(a)}, we will have to add a \GT{} for each of \s{f^i(a) \mid i \in \mathbb{N}}, as \m{f(a)=a} does not hold at the join. 
Hence our condition is that an \RGFA{} must be replaced with a \GT{} only if a term is represented in \emph{all} predecessors.\\
The previous rule for merging \GTs{} ensures that in this case the join will not grow larger than the product of the sizes of the joinees (as we have seen in several examples), as each such added \GT{} must have a source at each joinee, and there are at most \m{m \times n} pairs of such sources. This affects the code of \lstinline|update| and \lstinline|propagateUpNewSources|, which are responsible for replacing \RGFAs{} with \GTs{}.\\
Formally, the conjunct is as follows:
\begin{figure}[H]
\m{\forall \tup{t} \in g_n, f \cdot}\\
\m{~~~(\forall p \in \preds{n} \cdot \exists \tup{s} \in \sources{}{p}{\tup{t}} \cdot \fa{f}{s} \in \gfas{p}) \Rightarrow \fa{f}{t} \notin \rgfas{n}}
\end{figure}

The rest of the conjuncts remain the same, defining when an \RGFA{} or \GFA{} must exist in a predecessor and which source-edges must exist for each \GT{}.

\subsection{Strong join}
The join we have defined in the previous section is weak in the sense that it does not guarantee propagating equalities for terms not represented on both sides of the join. In this section we show two such examples, and describe how our algorithm handles them.


\bigskip
\noindent
The first example is show in figure \ref{EC_strong_join.1}.\\
Here, \m{a=b \lor f(a)=f(b) \models f(a)=f(b)}, but our algorihtm for the weak join does not add the \GFA{} \m{f([b]_n)} to \m{[f(a)]_n}, as there is no corresponding \GFA{} for \m{f([a]_0)} - \m{f(a)} is not represented in \m{p_1}.

\bigskip
\noindent
Before presenting our solution, we show another example - in figure \ref{EC_strong_join.2} (the green markings are explained below).\\
In this example, we have invoked \lstinline|assumeEqual(f([a]_n),g([c]_n)| and now we are in the middle of invoking 
\lstinline|assumeEqual(f([b]_n),g([d]_n)| - we have merged the \GTs{} \m{[f(b)]_n} and \m{[g(d)]_n}, but now no invariant forces us to merge \m{[f(a),g(c)]_n} with \m{[f(b),g(d)]_n}.\\
Here \m{(a=b \lor c=d) \land f(a)=g(c) \land f(b)=g(d) \models f(a)=f(b)=g(c)=g(d)},\\
 but our weak join algorithm fails to propagate these equalities as \\
\m{f(a),f(b),g(c),g(d)} are all not represented in the predecessors.

\bigskip
\noindent
We sketch here our solution to this problem without getting into many details, as we do not believe they help clarify the solution.
The approach is to maintain a representation of the congruence \m{n \sqcap p_i} for each \m{i\in\s{0,1}} where $\sqcap$ is a meet (conjunction) - each EC in the meet is called a \newdef{join-EC}, and each join-EC for \m{p_i} is associated with a set of \GTs{} from both \m{g_n} and \m{g_i}. Each \GT{} in \m{g_n} is associated with exactly one join-EC and each \GT{} in \m{g_i} is associated with \emph{at most} one join-EC. Rather than merge \GTs{} in \m{g_n} that share sources in all predecessors, we merge them when they share the join-EC in all predecessors. In addition, we perform congruence and transitive closure on join-ECs.\\
In figure \ref{EC_strong_join.2}, the numbered green lines represent the fact that two \GTs{} share the same join-EC for that predecessor - for example, \m{[f(a),g(c)]_n} shares the same join-EC with \m{[f(b),g(d)]_n} for \m{p_0} because of congruence closure from \m{[a]_n,[b]_n}, that share a join-EC for \m{p_0} as they share a source, and similarly for \m{p_1} because of congruence closure from  \m{[c]_n,[d]_n}.

Looking at the example in figure \ref{EC_strong_join.1}, we modify the rule for adding \GFAs{} from predecessors as follows:\\
Until now, we only allowed adding a \GFA{} if \emph{both} joinees had a \GFA{} in sources with the same function symbol, 
and the tuple of the result had the corresponding source tuple in each. \\
Now we allow adding such a \GFA{} at \m{g_n} if e.g. \m{p_0} has a \GFA{} \m{f(ptt)} and our \GT{} also has \GFA{} (in \m{g_n}) \m{f(tt)}. the resulting tuple will be in the same join-EC for \m{p_0} with \m{ptt} (which may have to be added) and the same join-EC for \m{p_1} with \m{tt}.\\
For our example, this solution is depicted in \ref{EC_strong_join.1a}. We can see that the new \GT{} shares now two sources with the \GFAs{} with the same function symbol - b, which allows us to use the weak join rules to complete this example.\\
The key here is that we only allow adding \GTs{} if they belong to some join-EC for \emph{all} predecessors - otherwise the algorithm might not terminate if one predecessor has a cycle in the EC-graph.

The modifications described above require some more book-keeping state and code to hande join-ECs, but does not increase the worst case space complexity asymptotically and does not break the incrementallity property - that no operation is performed twice. The worst-case space complexity for the weak join is \m{\size{\gfas{0}} \times \size{\gfas{1}} + \size{\gfas{n}}} because each added \GFA{} is associated with a unique pair of sources - one from each predecessor. For the strong join, the formula changes to \\
\m{\size{\gfas{0}} \times \size{\gfas{1}} + \size{\gfas{0}} \times \size{\gfas{n}} + \size{\gfas{1}} \times \size{\gfas{n}} + \size{\gfas{n}}},\\
 according to the number of possible \GFAs{} added.\\
In experiments, we have encountered only a few cases where the strong join discovered equalities not covered by the weak join, but at the same time performance of the strong join was not significantly slower.

\begin{figure}
\begin{tikzpicture}
  \node (1)  {};
%%%%%%%%%%%%%%%%%%%%%%%%%%%%%%%%%%%%%%%%%%%%%%%%%%%%%%%%%%%%%%
  \node[gttn] (11)  [above right= 2.5cm and 0cm of 1] {};
	\node[gl]   (11l) [below = 0 of 11]   {\m{p_0}};

	\node[gtn,label=center:\m{a}]  (12) [above left  = 0.5cm and 1cm of 11] {\phantom{B}};
	\node[gtn,label=center:\m{b}]  (14) [above right = 0.5cm and 1cm of 11] {\phantom{B}};
	
	\node[gtn,ultra thick]  (18) [above = 2cm of 11] {\stackB{f(a)}{f(b)}};
	\draw[gfa]  (18) to node[el] {\m{f}} (12);
	\draw[gfa,ultra thick]  (18) to node[el,anchor=west] {\m{f}} (14);

%%%%%%%%%%%%%%%%%%%%%%%%%%%%%%%%%%%%%%%%%%%%%%%%%%%%%%%%%%%%%%
	\node[gttn] (21)  [below right= 2.5cm and 0cm of 1] {};
	\node[gl]   (21l) [below = 0 of 21]   {\m{p_1}};

  \node[gtn,label=center:\m{a,b}]  (22) [above  = 0.5cm of 21] {\phantom{B,B}};

%  \node[gtn,ultra thick]  (28) [above = 1cm of 22] {\stackB{f(a)}{f(b)}};
%	\draw[gfa,ultra thick]  (28) to node[el] {\m{f}} (22);
%%%%%%%%%%%%%%%%%%%%%%%%%%%%%%%%%%%%%%%%%%%%%%%%%%%%%%%%%%%%%%

  \node[gttn] (31)  [right = 6cm of 1] {};
	\node[gl]   (31l) [below = 0 of 31]   {\m{n}};

  \node[gtn,label=center:\m{a}]  (32) [above left  = 0.5cm and 1cm of 31] {\phantom{B}};

	\node[gtn,ultra thick]  (38) [above = 2cm of 31] {\m{f(a)}};
	\draw[gfa]  (38) to node[el,anchor=south east] {\m{f}} (32);

%%%%%%%%%%%%%%%%%%%%%%%%%%%%%%%%%%%%%%%%%%%%%%%%%%%%%%%%%%%%%%

	\node (12a) [right = 0.5cm of 11] {};
	\node (32b) [above left = 0.2cm and 1.5cm of 32] {};
	\node (32c) [below left = 0.2cm and 1.5cm of 32] {};
	\draw[se] (32) to[out=180,in=0]  (12);
	\draw[se] (32) to[out=180,in=0]  (22);

	\node (18a) [right = 0.5cm of 18] {};
	\node (38b) [above left = 0.2cm and 1.5cm of 38] {};
	\node (38c) [below left = 0.2cm and 1.5cm of 38] {};
	\draw[se] (38) to[out=180,in=0]   (18);
%	\draw[se,ultra thick] (38) to[out=180,in=0] (38c) to[out=180,in=0] (28);

\end{tikzpicture}

\caption{Example for the strong join\\
The join is \m{\s{a=b} \sqcup \m{f(a)=g(b)}}.\\
In this case the highlighted function edges do not match.}
\label{EC_strong_join.1}
\end{figure}



\begin{figure}
\begin{tikzpicture}
  \node (1)  {};
%%%%%%%%%%%%%%%%%%%%%%%%%%%%%%%%%%%%%%%%%%%%%%%%%%%%%%%%%%%%%%
  \node[gttn] (11)  [above right= 1.5cm and 0cm of 1] {};
	\node[gl]   (11l) [below = 0 of 11]   {\m{p_0}};

	\node[gtn]  (12) [above = 1cm of 11] {\m{a,b}};
	
%	\draw[gfa]  (12.270)  to[bend right] node[el,anchor=east]  {\m{a}} (11.90);
%	\draw[gfa]  (12.270)  to[bend left]  node[el,anchor=west]  {\m{b}} (11.90);

%%%%%%%%%%%%%%%%%%%%%%%%%%%%%%%%%%%%%%%%%%%%%%%%%%%%%%%%%%%%%%
	\node[gttn] (21)  [below right= 1.5cm and 0cm of 1] {};
	\node[gl]   (21l) [below = 0 of 21]   {\m{p_1}};

  \node[gtn]  (22) [above  = 1.0cm of 21] {\m{c,d}};
	
%  \draw[gfa] (22.270) to[bend right] node[el,anchor=east] {\m{c}} (21.90);
%  \draw[gfa] (22.270) to[bend left]  node[el,anchor=west] {\m{d}} (21.90);

%%%%%%%%%%%%%%%%%%%%%%%%%%%%%%%%%%%%%%%%%%%%%%%%%%%%%%%%%%%%%%

  \node[gttn] (31)  [below right = 0.4cm and 5cm of 1] {};
	\node[gl]   (31l) [below = 0 of 31]   {\m{n}};

	\node[gtn,label=center:\m{a}]   (32) [above left  = 1cm and 2cm of 31] {\phantom{B}};
	\node[gtn,label=center:\m{b}]   (33) [above left  = 1cm and 0.7cm of 31] {\phantom{B}};
	\node[gtn,label=center:\m{c}]   (34) [above right = 1cm and 0.7cm of 31] {\phantom{B}};
	\node[gtn,label=center:\m{d}]   (35) [above right = 1cm and 2cm of 31] {\phantom{B}};
	
	\node[gtn]   (36) [above = 1cm of 33] {\stackB{f(a)}{g(c)}};
	\draw[gfa]   (36.270) to node[el,anchor=east] {\m{f}} (32.90);
	\draw[gfa]   (36.270) to node[el,anchor=south west,pos=0.7] {\m{g}} (34.90);

	\node[gtn]   (37) [above = 1cm of 34] {\stackB{f(b)}{g(d)}};
	\draw[gfa]   (37.270) to node[el,anchor=south east,pos=0.7] {\m{f}} (33.90);
	\draw[gfa]   (37.270) to node[el,anchor=west] {\m{g}} (35.90);

%%%%%%%%%%%%%%%%%%%%%%%%%%%%%%%%%%%%%%%%%%%%%%%%%%%%%%%%%%%%%%
	\draw[green] (36.0) to node[pl,anchor=north,text=green] {$\m{0,1}$} (37);

	\draw[green] (32) to node[pl,anchor=north,text=green] {$\m{0}$} (33);

	\draw[green] (34) to node[pl,anchor=south,text=green] {$\m{1}$} (35);

	\node (32nw) [above left = 0.2cm and 1.0cm of 32] {};
	\node (33nw) [above left = 0.2cm and 1.0cm of 33] {};
	\draw[se] (32) to (32nw) to (12);
	\draw[se] (33) to (33nw) to (32nw)to (12);

	\node (34sw) [below left = 0.11cm and 1.0cm of 34] {};
	\node (35sw) [below left = 0.09cm and 1.0cm of 35] {};
	\draw[se] (34) to (34sw) to (22);
	\draw[se] (35) to (35sw) to (34sw) to (22);

\end{tikzpicture}
\caption{
Strong join\\
after the first step of $\m{n}$.\lstinline{assume(f(b)=g(d))}
}
\label{EC_strong_join.2}
\end{figure}




\begin{figure}
\begin{tikzpicture}
  \node (1)  {};
%%%%%%%%%%%%%%%%%%%%%%%%%%%%%%%%%%%%%%%%%%%%%%%%%%%%%%%%%%%%%%
  \node[gttn] (11)  [above right= 2.5cm and 0cm of 1] {};
	\node[gl]   (11l) [below = 0 of 11]   {\m{p_0}};

	\node[gtn,label=center:\m{a}]  (12) [above left  = 0.5cm and 1cm of 11] {\phantom{B}};
	\node[gtn,label=center:\m{b}]  (14) [above right = 0.5cm and 1cm of 11] {\phantom{B}};
	
	\node[gtn,ultra thick]  (18) [above = 2cm of 11] {\stackB{f(a)}{f(b)}};
	\draw[gfa]  (18) to node[el] {\m{f}} (12);
	\draw[gfa,ultra thick]  (18) to node[el,anchor=west] {\m{f}} (14);

%%%%%%%%%%%%%%%%%%%%%%%%%%%%%%%%%%%%%%%%%%%%%%%%%%%%%%%%%%%%%%
	\node[gttn] (21)  [below right= 2.5cm and 0cm of 1] {};
	\node[gl]   (21l) [below = 0 of 21]   {\m{p_1}};

  \node[gtn,label=center:\m{a,b}]  (22) [above  = 0.5cm of 21] {\phantom{B,B}};

%  \node[gtn,ultra thick]  (28) [above = 1cm of 22] {\stackB{f(a)}{f(b)}};
%	\draw[gfa,ultra thick]  (28) to node[el] {\m{f}} (22);
%%%%%%%%%%%%%%%%%%%%%%%%%%%%%%%%%%%%%%%%%%%%%%%%%%%%%%%%%%%%%%

  \node[gttn] (31)  [right = 6cm of 1] {};
	\node[gl]   (31l) [below = 0 of 31]   {\m{n}};

  \node[gtn,label=center:\m{a}]  (32) [above left  = 0.5cm and 1cm of 31] {\phantom{B}};
  \node[gtn,label=center:\m{}]  (33) [above right  = 0.5cm and 1cm of 31] {\phantom{B}};

	\node[gtn,ultra thick]  (38) [above = 2cm of 31] {\m{f(a)}};
	\draw[gfa]  (38) to node[el,anchor=south east] {\m{f}} (32);
	\draw[gfa]  (38) to node[el,anchor=south west] {\m{f}} (33);

%%%%%%%%%%%%%%%%%%%%%%%%%%%%%%%%%%%%%%%%%%%%%%%%%%%%%%%%%%%%%%

	\node (12a) [right = 0.5cm of 11] {};
	\node (32b) [above left = 0.2cm and 1.5cm of 32] {};
	\node (32c) [below left = 0.2cm and 1.5cm of 32] {};
	\draw[se] (32) to[out=180,in=0]   (12);
	\draw[se] (32) to[out=180,in=0]  (22);
	\draw[se] (33) to[out=180,in=0]  (14);
	\draw[se] (33) to[out=180,in=0]  (22);

	\node (18a) [right = 0.5cm of 18] {};
	\node (38b) [above left = 0.2cm and 1.5cm of 38] {};
	\node (38c) [below left = 0.2cm and 1.5cm of 38] {};
	\draw[se] (38) to[out=180,in=0]  (18);
%	\draw[se,ultra thick] (38) to[out=180,in=0] (38c) to[out=180,in=0] (28);

	\draw[green] (33) to node[pl,anchor=south,text=green] {$\m{1}$} (32);

\end{tikzpicture}

\caption{Example for the strong join\\
The join is \m{\s{a=b} \sqcup \m{f(a)=g(b)}}.}
\label{EC_strong_join.1a}
\end{figure}


\subsubsection*{Summary}
We have shown how our algorithm supports an incremental, on-demand join for the unit ground equality fragment. 
The algorithm is not changed significantly from the one for sequential nodes, but the guarantee given by the algorithm is weaker.
The space requirement of a join is at most quadratic in the size of the input including all auxiliary data-structures (and infeasible \GTs{}) - we use depth limitations to prevent an exponential blowup for joins in sequence, described in chapter \ref{chapter:bounds}.
In our implementation the scoping and size restrictions are integrated into the entire algorithm including the join.\\
The complexity of our algorithm for each join is, in fact, proportional to the product of the total sizes (number of \GFAs{}) of all the \emph{relevant} ECs at the join - the algorithm never looks at any predecessor \GT{} or \GFA{} that does not represent a sub-term of a member of an EC of one of the requested terms at the join. 
\newpage
\section{Related Work}

%\subsection{Interpolants}
%There has been a lot of work in recent years on the use of interpolants in verification.
%Interpolants for a pair of formulae in first order logic with equality were introduced in ~\cite{Craig57},~\cite{Craig57a}, 
%with an effective and complete method to generate the inerpolants. 
%The algorithm given there can produce non-ground interpolants even when a ground interpolant exists, 
%and similarly non-CNF interpolants when a CNF-interpolant exists.
%
%The conjunction of the clauses we deduce at a program point (including those represented as a congruence closure graph), as presented in this chapter are not exactly an interpolant, as we do not restrict the vocabulary (set of functions and predicates) to be the intersection of the vocabulary of successors and predecessors. We add the vocabulary restriction in chapter \ref{chapter:scoping}.
%
%\subsubsection*{Interpolation in verification}
%Interpolants for propositional logic were used for verification, in the context of inferring an invariant for unbounded SAT based model checking starting with ~\cite{McMillan03}.
%A method to extract interpolants from resolution proofs for the fragment of ground first order logic with equality (and linear inequalities) was introduced in ~\cite{McMillan04}.
%A survey of interpolation techniques (extracting an interpolant from a proof) in ground first order logic with equality is given in ~\cite{BonacinaJohansson2015}, with a comparison in the logical strength of the interpolants produced by different systems.\\
%~\cite{JhalaMcMillan06} introduces the idea of split proofs, also called local proofs in ~\cite{KovacsVoronkov09},
%which are, broadly speaking, the proofs we are aiming at in our proof search - we discuss these in detail in chapter \ref{chapter:scoping}.\\
%In ~\cite{FuchsGoelGrundyKrsticTinelli2012} the authors present an algorithm for interpolating the ground equality fragment by colouring a congruence closure graph of the proof. Their interpolants are always a conjunction of Horn clauses.
%They also present the notion of an interpolation game (section 6) where two sets of formulae communicate only by exchanging Horn-clauses of the common vocabulary in order to generate a refutation.
%This is related to the back-and-forth way in which we request the propagation of additional axioms, the difference is that they derive the strategy for such an exchange from an existing proof of refutation (for the ground (non-unit) fragment), and do not discuss complexity.\\
%The main difference between the above works and our work is that they rely on a refutation proof in order to extract an interpolant,
%while we are actually searching for the proof. 
%However, several of these works try to modify the proof structure in order to extract better (e.g. smaller) interpolants, for example ~\cite{HoderKovacsVoronkov12}. \\
%Superposition based proof systems have been suggested that produce a local proof and hence directly an interpolant - notably ~\cite{McMillan08} and ~\cite{KovacsVoronkov09}. We discuss these in detail in chapter \ref{chapter:scoping}.
%
%~\cite{JhalaMcMillan06} introduces also the idea of a hierarchy of L-restricted interpolants (L being a fragment of the logical language), 
%which are similar to our hierarchy of fragment interpolants. \\
%The main differences are that L-restricted interpolants are only defined for linear program (no branch or join), 
%and that they limit only the language of the interpolants, while we also limit both the information flow between adjacent nodes (the join of the fragment) and the set of deductive rules used (e.g. we can limit the resolution depth/width, while they only limit the shape of the result).
%
%There are several works which extend sequence interpolants to trees and DAGs, mostly based on solving constrained Horn clauses (CHC).\\
%Using the terminology of ~\cite{McMillanRybalchenko2013}, our CFGs describe linear acyclic CHCs, 
%while the tree interpolants mentioned in ~\cite{BlancGuptaKovacsKragl13} and ~\cite{McMillanRybalchenko2013} are acyclic and simple - roughly the difference is that, while in our case the interpolant follows from the disjunction of the predecessors, 
%in tree interpolation the interpolant follows from the conjunction, which is useful for encoding concurrency and recursive function calls.\\
%~\cite{RummerHojjatKuncak13} introduces another incomparable class of CHCs, called disjunctive CHCs, 
%which are acyclic and body-disjoint, which roughly means they can encode directly joins (and also conjunctions) but not branches.
%~\cite{RummerHojjatKuncak15} compares different classes of Horn clauses, along with their interpolation problems.\\
%The above lines of work use a theorem prover to find a refutation and then generate an interpolant from the refutation.
	%
%DAG interpolants are introduced in ~\cite{AlbarghouthiGurfinkelChechik12UFO},
%and further developed in ~\cite{AlbarghouthiGurfinkelChechik12} (combined with abstract interpretation), 
%~\cite{VizelGurfinkel14} (combination with IC3) and others.
%The authors encode the DAG shaped CFG of the unfolding of a program similar to our CFG.\\
%The main differences are:\\
%We encode both the transition relation and set of states of a program location in one set of clauses at a cfg-node,
%while the common encoding is that of set of states at a program point and transition relations on edges.
%The reason for this encoding is that it allows us to apply all the information we have about the set of reachable states at a cfg-node to simplify the encoding of the transition relation. This includes, for example, constant propagation and the elimination of redundant clauses.\\
%As in the above works, the authors' technique relies on a SAT/SMT solver in order to generate interpolants and check the safety of an unfolding.
%As mentioned before, this has the advantage of being able to utilize off the shelf solvers which are often efficient, allowing the solver free choice of the order of evaluation (as opposed to our case and IC3/PDR where an order of evaluation is imposed), and counter examples can be provided to the user. \\
%The disadvantages are that any prover call is potentially of at least exponential run-time (depending on the prover - SAT is exponential, 
%SMT is at least exponential depending on the theories involved, and adding quantifier instantiation depends on the specific prover strategy). A time-out of the prover does not ensure progress (that is, no sub-exponential time-out can ensure progress, even if we mine the prover log for top-level learned clauses), and, in general, there is no guarantee of being even RE in the presence of quantifiers. \\
%We, instead, employ techniques with polynomial bounds for each fragment, but cannot directly use an off the shelf tool and cannot provide counter examples except for trivial cases. We search for proofs of a specific shape (e.g. narrow, local), which can, in some cases, be exponentially longer than those found by CDCL (e.g. ordered resolution is exponentially separated from general resolution). However, on the practical side, in the examples we have analyzed (small software examples, heap using programs), neither the exponential run-time of CDCL nor that of local proofs has been a real problem for proofs of actual programs, rather the main limitation was the handling of quantifiers.\\ 
%Another property of their encoding is a form of incrementality where an interpolant for a previous unrolling (encoding facts learned about the set of states at the given unrolling) is conjoined with the VC sent to the prover.\\
%In our case incrementality is tighter in the sense that we can ensure (under some limitations) that no derivation is performed more than once. \\
%Another issue is that their encoding uses one error state at the end of the CFG, which makes it harder, 
%in the case where a verification has failed but some intermediate lemmas have been learned, to utilize the intermediate results and eliminate error states that have been proven unreachable, and optimize the cfg for the next pass accordingly.\\
%We have not experimented with combining our technique with either abstract interpretation or IC3/PDR as in the above works, which would be interesting future work.
%
%Interpolation from superposition based proofs were explored in ~\cite{McMillan08} and other works and is discussed in chapter \ref{chapter:scoping}.
%
%A method to combine interpolants from different fragments was given in ~\cite{YorshMusuvathi05}.
%

\subsection{Congruence closure}
The earlier papers to discuss congruence closure algorithms are ~\cite{DowneySethiTarjan} (efficient algorithms, complexity bounds and alternative implementations), ~\cite{Shostak84} and ~\cite{NelsonOppen80} (combination with other theories). 

A comprehensive analysis of the join of two EC-graphs is given in ~\cite{GulwaniTiwariNecula04}, including a join algorithm that determines the equality of two terms at a join by adding the terms to the EC-graphs of both joinees. 
The algorithm is eager in the sense that it represents in the join all terms that occur in both joinees, 
and works bottom up from constants as opposed to our algorithm that works top-down and hence avoids considering irrelevant joinee terms.
Their algorithm works for one join and does not help determine which terms are relevant at which CFG-node.
The paper includes several examples, some of which we have repeated here, that show limitations of the congruence closure approach.

In ~\cite{GulwaniNecula07} the problem of global value numbering is discussed, 
which concerns the analysis of programs that may contain loops, but include only assignments (i.e. no assume statements).
For loop-less programs this is a strict subset of our problem, as essentially it means that each EC-graph node has exactly one non-constant gfa, and also no cycles. The paper gives an example loop-less program that would require an exponentially sized EC-graph at one program point in order to prove, which we have mentioned.

In ~\cite{NieuwenhuisOliveras03} the authors give a congruence closure algorithm, mostly similar to previous ones, 
except that they eliminate all functions and replace them by the curry function $\cdot$ of arity two, rewriting all the equations accordingly. They also name sub-terms by constants, for which we simply use the graph nodes. 
We have seen that some operations, especially at join points, are sensitive to function arity, and so this transformation has some attraction, however it has some important disadvantages in our setting:
the first disadvantage is that the transformation is arbitrarily non-symmetric, as we can curryfy an n-ary function to either a left or right chain of n $apply$ instances, a tree of applications of depth $log(n)$ or any other form. 
This makes it harder to keep other fragments predictable to users, and harder to enforce size limitations. In our setting it may also make our algorithm propagate much more equality information than is needed, 
as the original function symbol is not used in filtering most of the terms requested from predecessors (e.g. the \GFA{} f(a,b,c) will be rewritten to $\cdot(f,\cdot(a,\cdot(b,c)))$, and so may propagate $\cdot(b,c)$ even if $\cdot(a,\cdot(b,c))$ will not be propagated, while in our algorithm propagation will only happen if a term with the symbol $f$ occurs).

In ~\cite{ChangLeino2005} the authors describe an abstract interpretation framework that includes the domains of unit ground equalities and heaps (or generally other base domains). A join operation is described on EC-graphs which is works bottom up and does not use information from successor nodes to determine which terms to represent at the join.
The authors note the potential for sharing sub-graphs that are equal on both sides. The authors also note the incompleteness of the weak join, and in general the fact that information will be missing for terms that do not occur on both sides of the join, and suggest adding these terms to the graph on both sides without giving a strategy for selecting which terms to add. The reason is mostly that they apply abstract interpretation as a forward analysis, which is essentially goal insensitive, while our algorithm propagates information across joins as needed by later assertions. The algorithm in the paper is also not incremental, as in abstract interpretation the intermediate states represent an under approximation of the set of states, rather than an over approximation (until widening is applied).
%Hierarchical Superposition / Shostak / Sat mod Sat

\subsection{Information propagation}
Information propagation in a restricted fragment is the main theme of abstract interpretation (AI) (~\cite{CousotCousot77} and many others).
The main difference is that abstract interpretation works on a combination of under-approximation and over-approximation of the set of states in order to verify program properties, and can infer invariants, while we only try to verify a given program by calculating over-approximations, and cannot handle invariants with loops or recursion.
In spite of these differences, there is strong correlation between the join operation of AI and our join operation (although in AI it is usually not incremental - the value at the join node is recalculated), and some of the completeness issues are similar. 
Mainly, AI domains are analyzed as a fixed domain or combination of domains in a fixed order, either forwards or backwards (in program flow sense). 

As mentioned above (\cite{ChangLeino2005}) has developed an AI domain for congruence closure.
Several domains have been suggested for the problem of global value numbering - where only assignments are allowed (essentially, for acyclic programs).  This means in our setting that the only assume statements allowed are an equality \m{v_i = t_{i-1}}, 
where \m{v_i} is the ith DSA version of the variable \m{v} and \m{t_{i-1}} is a term only including constants and variables of the previous DSA version - essentially this means that EC-graphs cannot have loops. This is an early problem coming from compiler optimizations that has an exponential complete solution in \cite{Kildall73} and several other results are compared in \cite{GulwaniTiwariNecula04}.  \cite{Vagvolgyi03b} gives a decision procedure to detect when a join of two sets of equations is finitely generated and gives an algorithm to calculate the join, which stands in accuracy between our strong join and weak join.

For programs with loops, \cite{MullerOlmSeidl04} shows that allowing positive equality assume statements (positive guards) makes the fragment undecidable in the presence of loops, using an encoding of PCP. \cite{GulwaniTiwari07} discuss complexity issues and the relation to unification.

The \textbf{IC3/PDR} (\cite{Bradley12}) is used to infer propositional invariants of programs, essentially by looking at bounded unrollings of the loop and strengthening the invariant for the nth iteration by searching for a pre-image in the transition relation in the n-1 iteration, and if the found pre-image is infeasible, the algorithm tries to strengthen it and propagate it to later iterations.
IC3 has been extended to EPR (\cite{BjornerGurfinkelKorovinLahav2013}, linear arithmetic (e.g. \cite{BjornerGurfinkel2015} - combined with interpolation and polyhedra abstract interpretation) and others.
The technique and its further developments offer several ways to propagate and generalize the clause in order to reduce the number of iterations.
IC3 has proved very effective and has several extensions and generalization strategies, including a combination with interpolation in ~\cite{VizelGurfinkel2014}.\\
Compared to our work, IC3 can infer invariants and each step of IC3 requires a SAT call (or SMT call in some works) in order to find a counter example, and another call in order to calculate the pre-image, the first call only considers one program point and the second two program points, so that the SAT/SMT problem is much smaller than whole program VC.
Compared to our information propagation technique, 
IC3 requests to predecessors are a conjunction of ground literals (a set of models) while we request either a single literal or one ground term. In terms of complexity, IC3 ensures progress for each step, but the worst case cost of each SAT/SMT call is exponential, and even within one unrolling, the number of counter-examples that may be sent as requests can be exponential as well (although it might be that some clause generalization method can prevent this - we are not aware of such a result), and the choice of counter-examples is not easily predictable. As in other SAT/SMT based techniques, it is not immediately clear how to ensure a polynomial run-time for a sub-fragment and how to ensure progress in the presence of quantifiers.\\
A direct comparison when trying to prove an assertion \m{s=t} is that we try to propagate the whole equivalence classes of \m{s,t} and then compare, while IC3 directly asks whether the pre-image of a specific state where \m{s=t} holds. 
For propositional queries we request one literal and the reply is a set of clauses, while IC3 requests a conjunction of literals and the reply is one clause. Hence, IC3 asks more specific questions and propagates much less unneeded information (which is necessary in order for invariants to converge), while we ask less specific questions but can bound the number of answers needed for saturation. It would be interesting to see how we can, in some cases, utilize more directed queries as in IC3, while maintaining polynomial bounds.
In order to handle quantifiers, an efficient representation for infinite counter-examples is needed, we suspect that the representation used in \cite{BaumgartnerPelzerTinelli12} (which extends DPLL to FOL - hence has a representation for partial models with quantifiers) could be a potential, although as above, the problem is that no prover run is guaranteed to even terminate, and no progress is guaranteed for a prover time-out.

Another existing technique is based on modular sat solving (\cite{BaylessValBallHoosHu2013}) - here (in our terminology) each CFG-node gets its own SAT solver, each assertion leaf node searches for a model, which is then communicated to its predecessor (in the paper it is described for a sequence, rather than tree or DAG, CFG), each SAT solver is incremental and can receive new assignments from successors and new lemmas (learned clauses) from predecessors.
This approach is appealing as it is local and potentially each solver could face a much smaller problem than that of a whole VCG solver.
It would be interesting to see how this approach fares when extended to SMT and DAGs (in the paper it is described as a method to implement IC3, so in fact loops are supported, but acyclic sub-CFG are converted to one SAT instance - there is no provision for joins).
However, it faces some of the problems mentioned for IC3 above - each SAT instance could potentially run for exponential time, and a model has to be completely refuted (with an explaining lemma/interpolant) in order to continue verification.

%%\newpage
\section{Complexity}
In this section we present some known results about the complexity of the fragment of unit ground equalities.

We consider programs where the CFG is a DAG with one root and each node has only statements of the form \lstinline{assume t$\bowtie$s}  where \m{t,s} are ground terms, and all leaf nodes are assertion nodes - so our objective is to show that each leaf node is infeasible (if it is).

We consider here several complexity problems:
\begin{itemize}
	\item Flat CFG: Deciding satisfiability for a set of unit (dis)equalities, or equivalently a program with no branches - a linear program
	\item Tree CFG: Deciding the validity of a tree-shaped program cfg where each node has only unit (in)equality assumptions - essentially deciding the satisfiability of the post-condition of each leaf node
	\item DAG CFG: Deciding the validity of a general DAG-shaped program cfg with only unit (dis)equalities - two sub-problems:
	\begin{itemize}
		\item General validity - as in tree-shaped cfgs, deciding the satisfiability of the strongest post-conditions of each leaf node
		\item Fragment validity - deciding the existence of a fragment interpolant for the program, for the fragment of unit ground equalities
	\end{itemize}
\end{itemize}
In this section we only discuss the worst case space complexity of the above satisfiability problems (that is, the worst case size of the smallest proof), and not the incremental complexity of 
finding a partial result (e.g. proving a subset of the assertions), and then extending this partial result using new information (e.g. from other fragments, quantifier instantiation) - we discuss these in the following sections.


\textbf{Input size:} We measure the size of the input for this problem by the number \m{n} of occurrences of function symbols in the input, and the number e of edges in the program cfg - so, for example, a statement \lstinline[mathescape]{assume f(a)=g(b,b)} would have the size 5. \\
Note that for a cfg with a bounded out degree the number of edges is linear in the number of nodes, and each node but the root has at least one incoming edge, so we can use these interchangeably for asymptotic complexity. Our programs have an out degree 2.\\
We can bound the number of cfg edges to be at most linear in the number of function symbol occurrences by preprocessing the 
cfg using semantics preserving transformations (detailed precisely in the appendix) that reduce the number of empty cfg nodes (those without any clauses) - essentially, 
empty sequential nodes are merged with their predecessor,
a diamond join of two empty nodes is reduced to the branch node, 
and sequential joins both of which have one empty side are consolidated to have only one empty node.

\subsection{Linear programs}

\subsection{Tree-shaped CFG}
For a program with no joins (a tree shaped \cfg) the time complexity is still \bigO{n\ log^2(n)} (using search trees), as we can verify a tree-shaped program as follows:
\begin{figure}[H]
\begin{lstlisting}
proc verify(CFG cfg)
	d := empty congruence closure data structure
	n := cfg.root
	e := 0   //next successor to explore
	s := empty stack
	do
		add all (dis)equalities from n to d
		if (n is a leaf) and (d.isConsistent)
			return FAIL(n) //The assertion does not hold
		if (!d.isConsistent) or (e==n.successors.count)
			if (s.isEmpty) //done with all nodes
				return PASS
			else
				(n,d,e) = s.pop //back-track to last branch
		else //explore next successor of a branch
			if (n.successors.count > 1)
				s.push (n,d,e+1) //back-track point
		n := n.successors[e] //select successor e
		e := 0    //reset e for the next branch
	end
\end{lstlisting}
\caption{Basic verification algorithm}
\label{basic_verification_algorithm}
\end{figure}

This algorithm explores the cfg depth first, and stores back-tracking information on a stack at each branch which has not been fully explored.
We have to evaluate only up to n equalities in total and we traverse each edge at most once, so the worst case time complexity is \bigO{n log^2(n)+e} with \m{e} the number of edges. 
The space complexity is at most \bigO{n \times e} (worst case quadratic in the total input size) as we store the data structure at each branch, but some of the congruence closure implementations above support backtracking without asymptotically increasing the space, so the space complexity is reduced to \bigO{n + e}.

As can be seen above, the operations needed on the congruence closure data structure are:
\begin{itemize}
	\item Assume an (in)equality and apply congruence closure
	\item Check for conflict (inconsistency)
	\item Potentially - forget an (in)equality, undoing the relevant implied (congruence and transitive closures) equalities - could replace pushing the whole state on the stack
\end{itemize}

\noindent
\textbf{Efficiency:}\\
A conjunction of positive equalities in our setting is always consistent, so it might be the case that only a fraction of the equalities are needed in order to show that a given inequality is inconsistent. The algorithms above are all eager in that they take all equalities into account and calculate the equivalence classes for all the terms that appear on the path, while it is sufficient to calculate equivalence classes only for terms that appear in inequalities - so it is sufficient, for each assertion,
to only calculate the equivalence classes of each inequality on the path leading to this assertion.
We will show how we exploit this property in the next section.\\
Another possibility for reducing the set of terms considered at each assertion is using scoping - as ground first order logic with equality supports interpolation (although it is not complete for \emph{unit} interpolation), we can significantly reduce the set of terms considered at each node by using the variable scoping inherent in the program (where unit interpolation is supported), in addition to scoping implied by the DSA transformation (that is, at each node at most 2 DSA versions of a variable could be in scope).

\subsection{DAG-shaped CFG}


%%\newpage
\section{Predicate Transformer Semantics} \label{predicate_transformer_semantics_section}
In this section we refine the strongest post-conditions discussed earlier in order to show which part of the verification can be done at the unit level. 

For this discussion we need some extra terminology:\\
As before, we denote the set of clauses known at a node \node{n} as \clauses{n}.
We want to define the subset of \precond{n} and \postcond{n} these which is guaranteed to be unit ground equalities.

Remember the definitions for pre and post conditions:
\begin{figure}[H]
$
\begin{array}{lll}
	\precond{n}  &\triangleq   & \bigwedge\limits_{\substack{\node{p} \in \predsto{n} \\ \clause{c} \in \clauses{p}}} (\rpc{p}{n} \rightarrow \clause{c}) \\
	\postcond{n} & \triangleq  & \lpc{n} \land \precond{n} \land \bigwedge \clauses{n}
\end{array}
$
\caption{pre and post conditions}
\end{figure}

Remember that we can add to \clauses{n} any \clause{C} if $\postcond{n} \vdash \clause{C}$ without losing soundness or completeness.

We assume from now on that $\lpc{n} \in \clauses{n}$.\\
We write the definitions in CNF treating \rpc{p}{n} as a set of literals and using
\m{\lnot \rpc{p}{n}} for \m{\bigvee\limits_{\m{l} \in \rpc{p}{n}} \lnot \m{l}}

The first version of CNF strongest post-conditions is:
\begin{figure}[H]
$
\begin{array}{lll}
	\vspace{10pt}
	\precondZ{n}  & \triangleq & \bigwedge\limits_{\substack{\node{p} \in \predsto{n} \\ \clause{c} \in \clauses{p}}} 
(\clause{c} \lor \lnot \rpc{p}{n} )\\
	\vspace{10pt}
	\postcondZ{n} & \triangleq & \precondZ{n} \land \bigwedge \clauses{n} \\
\end{array}
$
\caption{clause pre and post conditions}
\end{figure}

The clauses $(\clause{c} \lor \lnot \rpc{p}{n} )$ are still not unit clauses and contain path (non equality) literals.
We could remap path literals to constant equality literals as in the transformation from PL to GFOLE but that still leaves us with no unit clauses.

The first step is to reformulate preconditions as a function of only direct predecessors:
\begin{figure}[H]
$
\begin{array}{lll}
	\vspace{10pt}
	\precondI{n}   & \triangleq & \bigwedge\limits_{\substack{ \node{p} \in \preds{n} \\\clause{c} \in \postcondI{p}}}
( \clause{c} \lor \lnot \rpc{p}{n}) \\
	\vspace{10pt}
	\postcondI{n} & \triangleq & \precondI{n} \land \bigwedge \clauses{n}
\end{array}
$
\caption{pre and post conditions 1}
\end{figure}

This is semantically equivalent to the above, the proof relying on the lemma that:\\
$\forall \node{n} \cdot \forall \node{p} \in \preds{n} \cdot \forall \node{p_1} \in \predsto{p} \cdot \rpc{p_1}{n} = \rpc{p}{n} \cup \rpc{p_1}{p}$

For the next version we look at \rpc{p}{n} where \node{p} is a direct predecessor of \node{n}, we have two options:
\begin{itemize}
	\item \node{n} is a sequential node and $\rpc{p}{n}=\emptyset$ which means $\lnot \rpc{p}{n} \Leftrightarrow \false$
	\item \node{n} is a join node that joins the path condition \m{c} and so for a direct predecessor \node{p}, $\rpc{p}{n} \in \s{c,\lnot c}$
\end{itemize}

In order to avoid losing precision because of infeasible predecessors, we define the set of valid predecessors:\\
$\vpredsII{n} \triangleq \s{\node{p} \in \preds{n} \mid  \emptyClause \not\in \postcondII{p}}$

The next step is to remove, syntactically, the redundant \true literals in the clauses - for that we will use the set of \emph{common clauses}
\cprecondII{n} between all of the feasible (not provably infeasible) direct predecessors of a node, which for sequential nodes is just all the clauses from the predecessor:\\
$\cprecondII{n}  \triangleq \bigcap\limits_{\node{p} \in \vpredsII{n}} \postcondII{p}$\\
We define also the complement \uprecondII{n,p}, the clauses unique to a predecessor.

The preconditions of the node will include the empty clause iff it is not the root and all the direct predecessors are provably infeasible.\\
The next iteration of the definitions is as follows:
\begin{figure}[H]
$
\begin{array}{lll}
	\vspace{10pt}
	\vpredsII{n}     & \triangleq & \s{\node{p} \in \preds{n} \mid  \emptyClause \not\in \postcondII{p}} \\
	\vspace{10pt}
	\cprecondII{n}   & \triangleq & \bigcap\limits_{\node{p} \in \vpredsII{n}} \postcondII{p} \\
	\vspace{10pt}
	\uprecondII{n,p} & \triangleq & \postcondII{p} \setminus \cprecondII{n} \\
	\vspace{10pt}
	\precondII{n}    & \triangleq & \bigwedge\cprecondII{n} \wedge \bigwedge\limits_{\substack{\node{p} \in \vpredsII{n}\\\clause{c} \in \uprecondII{n,p}}} 
	(\m{c} \lor \lnot \rpc{p}{n})\\
	\vspace{10pt}
	\postcondII{n} & \triangleq & \precondII{n} \land \bigwedge \clauses{n}
\end{array}
$
\caption{pre and post conditions 2}
\end{figure}

Now a sequential node would get all the unit clauses from the predecessors and join nodes would get all the common unit clauses.
Additionally, there is much less in the way of redundant literals in clauses.
This can serve as a good basis for finding the set of unit clauses that the algorithm will calculate, however, consider the following program:
\begin{figure}[H]
\begin{lstlisting}
$\node{p_b}:$
if ($\m{c_1}$)
	$\node{p_t}:$
	assume a=b
else
	$\node{p_e}:$
	assume b=c
$\node{p_j}:$
assume a=c
	$\node{p_ja}:$
	assert a=b
\end{lstlisting}
\caption{three way join}
\label{snippet3.1}
\end{figure}

Here, \\
$
\begin{array}{lll}
\precondII{p_j}     & = & \s{\lnot \m{c_1} \lor \term{a=b}, \m{c_1} \lor \term{b=c}} \\
\postcondII{p_j}    & = & \s{\term{a=c}, \lnot \m{c_1} \lor \term{a=b}, \m{c_1} \lor \term{b=c}} \\
\postcondII{p_{ja}} & = & \s{\term{a \neq b},\term{a=c}, \lnot \m{c_1} \lor \term{a=b}, \m{c_1} \lor \term{b=c}}
\end{array}
$\\
This is sufficient in order to prove the assertion, but if we restrict \precondII{n} 
to only ground unit equalities that are in both predecessors, we get:\\
$
\begin{array}{lll}
	\restrict{\precondII{p_j}}{u}     & = & \s{} \\
	\restrict{\postcondII{p_j}}{u}    & = & \s{\term{a=c}} \\
	\restrict{\postcondII{p_{ja}}}{u} & = & \s{\term{a \neq b},\term{a=c}}
\end{array}
$\\
Which is insufficient. 
This is the general problem where $\m{(a \sqcup b) \sqcap c}$ is potentially less precise than $\m{(a \sqcap c) \sqcup (b \sqcap c)}$.
This problem is mentioned in ~\cite{GulwaniTiwariNecula04} as the \emph{context sensitive join}.
In order to be able to handle this fragment we want to use the transitive congruence closure of $(\clauses{n} \cup \postcondII{p})$ in the intersection.
We will see more related examples for this later.

Using the above, we define, for a node \node{n} and a direct predecessor \node{p} the \emph{relative clauses} of \node{p} as:\\
$\rClausesIII{p}{n} \triangleq (\CCR{\postcondIII{p} \cup \clauses{n}})$\\
and using that we refine the set of valid predecessors:\\
$\vpredsIII{n}      \triangleq \s{\node{p} \in \preds{n} \mid  \emptyClause \not\in \rClausesIII{p}{n}}$\\
We use \m{\mathbf{CC_R}} because it is the only calculus of the three where the closure is guaranteed to be finite 
(for \m{\mathbf{CC}} the closure is finitely representable as a graph but not a set of equations - discussed later).

The motivation for this formulation comes from programs such as this:
\begin{figure}[H]
\begin{lstlisting}
if (*)
	$\node{p_t}:$
	assume $\m{f(a) \neq f(b)}$
else	
	$\node{p_e}:$
	assume $\m{c=d}$
$\node{p_j}:$
assume $\m{a=b}$
	$\node{p_{ja}}:$
	assert $\m{c=d}$
\end{lstlisting}
\caption{join inequality propagation}
\label{snippet3.2}
\end{figure}

Here $\vpredsII{p_j}=\s{p_e}$ because \\
$\emptyClause \in \CCR{\clauses{p_j} \cup \clauses{p_t}}=\CCR{f(a) \neq f(b),a=b} \\
=\s{a=b,f(a)=f(b),f(a) \neq f(b),\emptyClause}$,\\
 and so we can use the clause \m{c=d} from \node{p_e}, which would not be possible without considering both \clauses{p_j}, \clauses{p_t} and congruence closure.

%\s{s \neq t \mid \exists u,v \cdot \s{\term{s=u,v=t,u \neq v}} \subseteq \m{S} } \cup \\
%\s{ \emptyClause \mid \exists s,t \cdot \s{\term{s=t,s \neq t}} \subseteq \m{S} } \cup \\
%\mathbf{DAI}(\m{S})$

%We will explain the last rule in the following.

\noindent
Now we can define another version of our preconditions:
\begin{figure}[H]
$
\begin{array}{lll}
	\vspace{10pt}
	\rClausesIII{p}{n} & \triangleq & \CCR{\postcondIII{p} \cup \clauses{n}} \\
	\vspace{10pt}
	\vpredsIII{n}      & \triangleq & \s{\node{p} \in \preds{n} \mid  \emptyClause \not\in \rClausesIII{p}{n}} \\
	\vspace{10pt}
	\cprecondIII{n}   & \triangleq & \bigcap\limits_{\node{p} \in \vpredsIII{n}} \rClausesIII{p}{n} \\
	\vspace{10pt}
	\uprecondIII{n,p} & \triangleq & \postcondIII{p} \setminus \cprecondIII{n} \\
	\vspace{10pt}
	\precondIII{n}    & \triangleq & \bigwedge\cprecondIII{n} \wedge \bigwedge\limits_{\substack{\node{p} \in \vpredsIII{n}\\\clause{c} \in \uprecondIII{n,p}}} 
	(\m{c} \lor \lnot \rpc{p}{n})\\
	\vspace{10pt}
	\postcondIII{n} & \triangleq & \CCR{\precondIII{n} \land \bigwedge \clauses{n}}
\end{array}
$
\caption{pre and post conditions 3}
\label{prepost3}
\end{figure}

Now coming back to ~\ref{snippet3.1}:\\
$
\begin{array}{lllll}
\CCR{\postcondIII{p_t} \cup \clauses{p_j}}  & = & \s{a=b,b=c,c=a} \\
\CCR{\postcondIII{p_e} \cup \clauses{p_j}}  & = & \s{a=b,b=c,c=a} \\
\cprecondIII{p_j}                          & = & \s{a=b,b=c,c=a} \\
\precondIII{p_j}                           & = & \s{a=b,b=c,c=a} \\
\postcondIII{p_j}                          & = & \s{a=b,b=c,c=a} \\
\postcondIII{p_{ja}}                       & = & \s{a=b,b=c,c=a,a \neq b,\emptyClause} \\
\end{array}
$\\
This is sufficient in order to prove the assertion, even under the unit restriction.

The space complexity of the pre and post conditions above, referring to \size{\ECs{\mathbf{\postcondIII{n}}}} is, however, in the worst case double exponential in the size of the original program, as we will show in later sections.

Also, as mentioned before, a simple program as this:
\begin{figure}[H]
\begin{lstlisting}
if (*)
	$\node{p_t}:$
	assume $\m{f(a) = f(b)}$
else	
	$\node{p_e}:$
	assume $\m{a=b}$
$\node{p_j}:$
	...
	$\node{p_{ja}}:$
	assert $\m{f(a)=f(b)}$
\end{lstlisting}
\caption{join congruence closure}
\label{snippet3.3}
\end{figure}

Will not verify with the above formulation as (again, showing only unit clauses and not showing reflexive equalities)\\
$
\begin{array}{lllll}
\restrict{\CCR{\postcondIII{p_t} \cup \clauses{p_j}}}{\m{u}}  & = & \s{f(a)=f(b)} \\
\restrict{\CCR{\postcondIII{p_e} \cup \clauses{p_j}}}{\m{u}}  & = & \s{a=b} \\
\restrict{\cprecondIII{p_j}}{\m{u}}                          & = & \s{} \\
\restrict{\precondIII{p_j}}{\m{u}}                           & = & \s{} \\
\restrict{\postcondIII{p_j}}{\m{u}}                          & = & \s{} \\
\restrict{\postcondIII{p_{ja}}}{\m{u}}                       & = & \s{f(a) \neq f(b)} \\
\end{array}
$

The terms \m{f(a)} and \m{f(b)} will not appear at the join as they are not both represented on both sides of the join.

This is obviously highly unsatisfactory - a potentially double exponential set of equations that cannot verify very simple correct programs.
In a sense, the join (and also equality propagation to sequential nodes) is both too eager and too lazy - it propagates equalities regardless of the statements in successor nodes.
As in the propositional case, we want propagation to be goal sensitive - that is, propagate information that is potentially useful in successors. Also as in the propositional case, we have seen that we do not want to propagate, in the unit fragment, all the potentially useful information, as this might too high complexity, or even not be finitely representable.\\
In the next section we discuss complexity issues inherent with the representation of these post conditions in the fragment and th following section will discuss a method to propagate equalities using information both from predecessors and successors.


%%\newpage
\section{Local graph-based invariant}
In this section we present a version of the source and propagation invariant which is local, 
both in the cfg (relates a cfg-node only with its direct predecessors) 
and in each EC-graph (relates an EC-node only with adjacent EC-nodes in the cfg-node and its direct predecessors).\\
This invariant will form the basis for the algorithm that handles the unit ground equality fragment, 
and will help understanding the complexity of information propagation between EC-graphs, 
both standard complexity and incremental complexity.\\
The invariant, being cfg-node and EC-node local, suggests for each broken conjunct an operation to fix it, 
which can be adding a source edge, \GFA{} or rgfa, merging two EC-nodes, or some other operations we will describe in the following.
We will show later that regardless of the order in which these operations are performed, we always end up in the same final state.
This means that the invariant describes a family of algorithms that has two degrees of freedom - 
the first is the order of evaluation, which we will discuss in the implementation chapter, 
and the second is the choice between adding a \GFA{} or an \RGFA{} (weak source invariant, second part), 
which we will discuss in the implementation chapter.\\
We present first the invariant for sequential nodes, and later the weak and strong join.

\subsection{Sequential nodes}

\subsubsection{The sources invariant}
The source invariant, for a sequential node \m{n} with a predecessor \m{p} is as before, simplified for one predecessor:
\begin{figure}[H]
\begin{enumerate}
	\item \m{\forall \fa{f}{t} \in \gfasA{g_n}, \tup{s} \in \sources{n}{p}{\tup{t}} \cdot }\\
		\m{\fa{f}{s} \in \gfasA{g_p} \Rightarrow [\fa{f}{s}]_{g_p} \in \sources{n}{p}{[\fa{f}{t}]_{g_n}}}
	\item \m{\forall \fa{f}{t} \in \gfasA{g_n} \cup \rgfas{n}, \tup{s} \in \sources{n}{p}{\tup{t}} \cdot}\\
		\m{ \fa{f}{s} \in \gfasA{g_p} \cup \rgfas{p}}
	\item \m{\forall \tup{t} \in g_n, \tup{s} \in \sources{n}{p}{\tup{t}},\fa{f}{s} \in \gfasA{g_p} \cdot}\\
		\m{\fa{f}{t} \notin \rgfas{n}} 
\end{enumerate}
\caption{Sequential graph based source invariant}
\label{sequential_weak_source_invariant}
\end{figure}
\subsubsection{The propagation invariant}
Remember that the \textbf{propagation invariant} was phrased as:\\
\m{\forall n \in \cfg,t \in \terms{g_n}, s \in \Ts{\sig} \cdot} \\
\m{s=t \in \sqcup_F(\eqs{g_n},\s{\eqs{g_p}}_{p \in \preds{n}}) \Rightarrow s \in \terms{[t]_{g_n}}}\\
For a sequential node \node{n} with the predecessor \node{p} this simplifies to (with the strongest join):
\begin{figure}[H]
\m{\forall t \in \terms{g_n}, s \in \Ts{\sig} \cdot} \\
\m{(\eqs{g_n} \cup \eqs{g_p}  \models s=t ) \Rightarrow s \in \terms{[t]_{g_n}}}
\caption{Sequential propagation invariant}
\end{figure}
We will show that we can ensure the above (cfg-local but not EC-graph-local) invariant using a combination of the local source invariant above, and the following \emph{local propagation invariant}:
\begin{figure}[H]
\begin{enumerate}
	\item The \textbf{first condition} ensures eager equality propagation:\\
\m{\forall u,v \in g_n \cdot \sources{n}{p}{u} \cap \sources{n}{p}{v} \neq \emptyset \Rightarrow u = v }
	\item The \textbf{second condition} ensures that each term-EC is complete - \emph{gfa completeness}:\\
\m{\forall u \in g_n, v \in \sources{n}{p}{u}, \fa{f}{s} \in v \cdot}\\
\m{\exists \fa{f}{t} \in u \cdot \tup{s} \in \sources{n}{p}{\tup{t}} }
\end{enumerate}
\caption{Sequential graph based propagation invariant}
\end{figure}
The first condition ensures that each predecessor EC node can be the source of at most one EC node at\node{n} - 
we have seen that this was broken in ~\ref{snippet3.16a_graph3} as the EC node \s{f(a),g(b)} in \node{n_1} 
is the source for both nodes \m{f(\s{a,b}),g(\s{a,b})} at \node{n_2}.\\
It is important to emphasize the meaning of sources for tuple ECs:\\
\m{\forall \tup{t} \in g_n,  \cdot \sources{n}{p}{\tup{t}} = \s{\tup{s} \in g_p \mid \bigwedge\limits_i s_i \in \sources{n}{p}{t_i} }}\\
Where also for tuple ECs:\\
\m{\forall \tup{s},\tup{t} \in g_n \cdot \sources{n}{p}{\tup{s}} \cap \sources{n}{p}{\tup{t}} \neq \emptyset \Rightarrow  \tup{s} = \tup{t}}

This invariant is not broken in the above examples, but could be broken if the assertion at \node{n_3} was changed to 
\m{f(b)=f(a)}, as shown in ~\ref{snippet3.16a_graph5}:
\begin{figure}[H]
\begin{tikzpicture}
	\node[gttn] (1)              {$()$};
	\node[gl]   (1l) [below = 0 of 1] {\m{n_1}};

	\node[gtn]  (2) [above left  = 0.6cm and 0.2cm of 1] {\s{a}};
	\node[gtn]  (3) [above right = 0.6cm and 0.2cm of 1] {\s{b}};

	\draw[gfa] (2) to node[el] {\m{a}} (1);
	\draw[gfa] (3) to node[el,anchor=west] {\m{b}} (1);

	\node[gttn] (4)  [above = 0.5cm of 2]    {\m{(a)}};
	\node[gttn] (5)  [above = 0.5cm of 3]    {\m{(b)}};

	\draw[sgtt] (4) to node[el] {0} (2);
	\draw[sgtt] (5) to node[el] {0} (3);

	\node[gtn]  (6)  [above = 2.5cm of 1] {\tiny$\svb{f(a)}{g(b)}$};
	\draw[gfa]  (6) to node[el] {\m{f}} (4);
	\draw[gfa]  (6) to node[el,anchor=west] {\m{g}} (5);

%%%%%%%%%%%%%%%%%%%%%%%%%%%%%%%%%%%%%%%%%%%%%%%%%%%%%%%%%%%%%%
	\node[gttn] (11)  [right = 3cm of 1] {$()$};
	\node[gl]   (11l) [below = 0 of 11]   {\m{n_2}};

	\node[gtn]  (12) [above = 0.5cm of 11] {\s{a,b}};

	\draw[gfa] (12) to[out=-110 ,in=110] node[el] {\m{a}} (11);
	\draw[gfa] (12) to[out=- 70,in= 70] node[el,anchor=west] {\m{b}} (11);

	\node[gttn,ultra thick] (14)  [above = 0.5cm of 12]    {\m{(\s{a,b})}};

	\draw[sgtt] (14) to node[el] {0} (12);

	\node[gtn,ultra thick]  (16)  [above = 2.5cm of 11] {\tiny $\faB{f}{a}{b},\faB{g}{a}{b}$};
	\draw[gfa]              (16) to [out=-100, in= 100] node[el]             {\m{f}} (14);
	\draw[gfa,ultra thick]  (16) to [out=-80 , in=  80] node[el,anchor=west] {\m{g}} (14);
				
%%%%%%%%%%%%%%%%%%%%%%%%%%%%%%%%%%%%%%%%%%%%%%%%%%%%%%%%%%%%%%

	\node[gttn] (21)  [right = 3.5cm of 11] {$()$};
	\node[gl]   (21l) [below = 0 of 21]   {\m{n_3}};

	\node[gtn]  (22) [above = 0.5cm of 21] {\s{a,b}};

	\draw[gfa] (22)  to[out=-110,in=110] node[el]             {\m{a}} (21);
	\draw[gfa] (22)  to[out=- 70,in= 70] node[el,anchor=west] {\m{b}} (21);

	\node[gttn,ultra thick] (24)  [above = 0.5cm of 22]    {\m{(\s{a,b})}};

	\draw[sgtt] (24) to node[el] {0} (22);

	\node[gtn,ultra thick]  (26)  [above = 2.5cm of 21] {\tiny $\faB{f}{a}{b}$};

	\draw[gfa]               (26) to [out=-100, in= 100] node[el]             {\m{f}} (24);
	\draw[mgfa,ultra thick]  (26) to [out=-80 , in=  80] node[ml,anchor=west] {\m{g}} (24);
%%%%%%%%%%%%%%%%%%%%%%%%%%%%%%%%%%%%%%%%%%%%%%%%%%%%%%%%%%%%%%
	\draw[se] (11) to ( 1);
	\draw[se] (21) to  (11);

	\node (12a) [left = 0.5cm of 12] {};
	\node (3c) [above= 0.1cm of 3] {};
	\draw[se] ( 12.180) to[out=180,in=0] (12a.0) to[out=180,in=0] (3c) to[out=180,in=0] (   2.0);
	\draw[se] ( 12.180) to[out=180,in=0] (12a.0) to[out=180,in=0] (   3.0);

	\draw[se] (22) to  (12);

	\node (5c) [above= 0.1cm of 5] {};
	\node (14a) [left = 0.7cm of 14] {};
	\draw[se] (14.180) to[out=180,in=0] (14a.0) to[out=180,in=0] (5c) to[out=180,in=0]( 4.0);
	\draw[se] (14.180) to[out=180,in=0] (14a.0) to[out=180,in=0]( 5.0);
	\draw[se,ultra thick] (24) to (14);

	\draw[se] (16) to  (6);
	\draw[se,ultra thick] (26) to  (16);

\draw[draw=none, use as bounding box] (current bounding box.north west) rectangle (current bounding box.south east);

\begin{pgfinterruptboundingbox}
	\draw[separator] (2.0cm,-0.7cm) to (2.0cm,3.5cm);
	\draw[separator] (5.5cm,-0.7cm) to (5.5cm,3.5cm);
\end{pgfinterruptboundingbox}

\end{tikzpicture}

\caption{
The sources function\\
gfa invariant broken
}
\label{snippet3.16a_graph5}
\end{figure}
\noindent
At \node{n_3} the EC node \m{f(\s{a,b})} is missing the \GFA{} \m{g(\s{a,b})} which is implied by the \GFA{} completeness invariant, 
as shown by the highlighted path.
However, this invariant is not local - consider the following graph (irrelevant \textbf{sources} and \textbf{rgfas} omitted):
\begin{figure}[H]
\begin{tikzpicture}
	\node[gttn,ultra thick] (1)              {$()$};
	\node[gl]   (1l) [below = 0 of 1] {\m{n_1}};

	\node[gtn]              (2) [above left  = 0.6cm and 0.2cm of 1] {\s{a}};
	\node[gtn,ultra thick]  (3) [above right = 0.6cm and 0.2cm of 1] {\s{b}};

	\draw[gfa]             (2) to node[el] {\m{a}} (1);
	\draw[gfa,ultra thick] (3) to node[el,anchor=west] {\m{\mathbf{b}}} (1);

	\node[gttn]             (4)  [above = 0.5cm of 2]    {\m{(a)}};
	\node[gttn,ultra thick] (5)  [above = 0.5cm of 3]    {\m{(b)}};

	\draw[sgtt]             (4) to node[el] {0} (2);
	\draw[sgtt,ultra thick] (5) to node[el] {\textbf{0}} (3);

	\node[gtn,ultra thick]  (6)  [above = 2.5cm of 1] {\tiny$\svb{f(a)}{g(b)}$};
	\draw[gfa]              (6) to node[el]             {\m{f}} (4);
	\draw[gfa,ultra thick]  (6) to node[el,anchor=west] {\m{g}} (5);

%%%%%%%%%%%%%%%%%%%%%%%%%%%%%%%%%%%%%%%%%%%%%%%%%%%%%%%%%%%%%%
	\node[gttn,ultra thick] (11)  [right = 3cm of 1] {$()$};
	\node[gl]   (11l) [below = 0 of 11]   {\m{n_2}};

	\node[gtn]  (12) [above = 0.6cm of 11] {\s{a,c}};
	\node[mgtn] (13) [above right = 0.65cm and 0.6cm of 11] {\s{b}};

	\draw[gfa]  (12) to[bend right] node[el] {\m{a}} (11);
	\draw[gfa]  (12) to[bend left]  node[el] {\m{c}} (11);
	\draw[mgfa] (13) to node[ml,anchor=west] {\m{b}} (11);

	\node[gttn]  (14)  [above = 0.5cm of 12]    {\m{(a,c)}};
	\node[mgttn] (15)  [above = 0.5cm of 13]    {\m{(b)}};

	\draw[sgtt]  (14) to node[el] {0} (12);
	\draw[msgtt] (15) to node[ml] {0} (13);

	\node[gtn,ultra thick]  (16)  [above = 0.76cm of 14] {\tiny$\m{f(\s{a,c})}$};
	\draw[gfa]              (16) to node[el]             {\m{f}} (14);
	\draw[mgfa]  (16) to node[ml,anchor=west] {\m{g}} (15);
 
%%%%%%%%%%%%%%%%%%%%%%%%%%%%%%%%%%%%%%%%%%%%%%%%%%%%%%%%%%%%%%
	\draw[se,ultra thick] (11) to  ( 1);

	\node (12a) [left = 0.5cm of 12] {};
	\node (3c) [above= 0.1cm of 3] {};
	
	\node (5c) [above= 0.1cm of 5] {};
	\node (14a) [left = 0.7cm of 14] {};

	\draw[se,ultra thick] (16) to  (6);

\draw[draw=none, use as bounding box] (current bounding box.north west) rectangle (current bounding box.south east);

\begin{pgfinterruptboundingbox}
	\draw[separator] (2.0cm,-0.7cm) to (2.0cm,3.5cm);
\end{pgfinterruptboundingbox}

\end{tikzpicture}

\caption{
The sources function\\
gfa invariant broken non-local
}
\label{snippet3.16a_graph6}
\end{figure}
In ~\ref{snippet3.16a_graph6} \node{n_2} is missing the whole path of \m{g(b)} in order to be complete, we will look now at how this can be detected.\\
There are principally 3 pre-states and operations that can reach the above state:\\
The \textbf{first} is when the last operation was \lstinline{adding} the term \m{f(a)} at \node{n_2}:
\begin{figure}[H]
\begin{tikzpicture}
  \node[gttn] (1)              {$()$};
	\node[gl]   (1l) [below = 0 of 1] {\m{n_1}};

  \node[gtn]  (2) [above left  = 0.6cm and 0.2cm of 1] {\s{a}};
  \node[gtn]  (3) [above right = 0.6cm and 0.2cm of 1] {\s{b}};
	
	\draw[gfa]  (2) to node[el] {\m{a}} (1);
  \draw[gfa]  (3) to node[el,anchor=west] {\m{\mathbf{b}}} (1);
  
	\node[gttn] (4)  [above = 0.5cm of 2]    {\m{(a)}};
  \node[gttn] (5)  [above = 0.5cm of 3]    {\m{(b)}};

	\draw[sgtt]             (4) to node[el] {0} (2);
	\draw[sgtt] (5) to node[el] {\textbf{0}} (3);

  \node[gtn]  (6)  [above = 2.5cm of 1] {\tiny$\svb{f(a)}{g(b)}$};
	\draw[gfa]  (6) to node[el]             {\m{f}} (4);
  \draw[gfa]  (6) to node[el,anchor=west] {\m{g}} (5);

%%%%%%%%%%%%%%%%%%%%%%%%%%%%%%%%%%%%%%%%%%%%%%%%%%%%%%%%%%%%%%
  \node[gttn] (11)  [right = 3cm of 1] {$()$};
	\node[gl]   (11l) [below = 0 of 11]   {\m{n_2}};

  \node[gtn]  (12) [above = 0.6cm of 11] {\s{a,c}};
	
	\draw[gfa]  (12) to[bend right] node[el] {\m{a}} (11);
	\draw[gfa]  (12) to[bend left]  node[el,anchor=west] {\m{c}} (11);
  
	\node[gttn] (14)  [above = 0.5cm of 12]    {\m{(a,c)}};

	\draw[sgtt] (14) to node[el] {0}          (12);
  
%%%%%%%%%%%%%%%%%%%%%%%%%%%%%%%%%%%%%%%%%%%%%%%%%%%%%%%%%%%%%%
	\draw[se] (11) to  ( 1);

	\node (12a) [left = 0.5cm of 12] {};
	\node  (3c) [above= 0.1cm of 3] {};
	
	\node  (5c) [above= 0.1cm of 5] {};
	\node (14a) [left = 0.7cm of 14] {};

	\draw[se] (14) to (14a) to (5c) to (4);

\draw[draw=none, use as bounding box] (current bounding box.north west) rectangle (current bounding box.south east);

\begin{pgfinterruptboundingbox}
	\draw[separator] (2.0cm,-0.7cm) to (2.0cm,3.5cm);
\end{pgfinterruptboundingbox}

\end{tikzpicture}

\caption{
The sources function\\
gfa invariant broken non-local\\
pre-state for \lstinline|$n_2$.add(f(a))|%\lstinline[mathescape]{$n2$.add(f(a))}
}
\label{snippet3.16a_graph7}
\end{figure}
In ~\ref{snippet3.16a_graph7} there is not (and should not be) a node for \m{b} or \m{(b)} at \node{n_2}.\\
The next step should create the whole chain of \m{g(b)}. 
Instead of \m{g(b)}, this chain could have been arbitrarily deep and large, e.g. \m{g^6(b)} or \m{h(h(a,b),h(b,a))}.\\
To do this incrementally and efficiently we need a way to determine that the nodes
 \m{[b]_{n_1},[(b)]_{n_1}} and the corresponding \GFAs{} have become relevant.\\
In order to determine which nodes have become relevant without traversing the whole of \m{g_{n_1}} (which could have had many other unrelated nodes) we need to walk \emph{down} the graph \m{g_{n_1}},
starting at the new source node \m{[f(a)]_{n_1}}, traversing down each \gfa and stopping at EC-nodes that are already a source to some node in \m{n_2} (or the empty tuple).\\
In our case the traversal would proceed to \m{[g(b)]_{n_1},[(b)]_{n_1},[b]_{n_1},[()]_{n_1}}.\\
We could simply create an EC-node in \m{g_{n_2}} for each EC-node we traverse in \m{g_{n_1}} 
and connect the corresponding \GFA{} edges, but we will show that for several reasons it is preferable
first to mark each node of \m{g_{n_1}} as \emph{relevant} traversing top down as above, 
and then add the corresponding nodes to \m{g_{n_2}} traversing bottom up.\\
One intuitive reason for this is that we will want to use scoping - 
we can only determine if a \GFA{} in \m{g_{n_1}} is in scope in \m{g_{n_2}} (that is, represents at least one term which is in scope in \m{n_2}) by traversing the downward closure of the \GFA{} bottom up. There are several other reasons that we will present in the following, and is a direct extension of our earlier verification algorithm for a subset of the axioms.\\
After the relevant source nodes have been marked as relevant, we would expect the nodes to be added to \m{g_{n_2}} in the following order (with the corresponding \GFA{} edges):\\
\m{[()]_{n_2},[b]_{n_2},[(b)]_{n_2}}.\\
This process of marking source nodes top-down and then adding new nodes bottom-up will be the base for all of our join, meet and other EC graph manipulation algorithms.\\
As can be seen, the inherent time (and space) complexity of this process (assuming that we maintain the relevant lookup tables) is proportional to the size of the difference between the result \m{g_{n_2}} and the pre-state of \m{g_{n_2}}.
Unfortunately, this would not be exactly the case for joins, or when further restrictions, such as scoping and term radius, are enforced. We will, however, draw a complexity bound in all cases which is based on a worst case result size similar to the above.\\
For example, if \m{b} was not in scope at \m{n_2}, we would still need to determine that \m{g(b)} is not in scope there, and hence we would mark the source nodes relevant as above, 
but then determine that \m{[b]_{n_1}} is not \emph{feasible}, 
meaning that it cannot be the source to any node in \m{g_{n_2}}.\\
In order to avoid traversing the same source nodes for other super-terms of \m{b} in \m{g_{n_1}},
we could cache in \m{g_{n_2}} the set of source nodes that we have determined infeasible.\\
The complexity of the process is bounded by the difference in the size of the sub-graph (downward closure of gfa-edges) of the sources of the result and the size of the sub-graph of the sources
before the operation, hence the incremental complexity is proportional approximately to the number of \GFAs{} in the final sub-graph of sources (approximately because \GFAs{} in the sub-graph that were merged could have been traversed twice) the complexity is loglinear in this factor as each node is traversed a constant number of times (2 in the above case), and the log factor comes from searching the lookup tables of e.g. \GFAs{}.\\
The process described above stems directly from the \GFA{} completeness invariant above.\\
\noindent
The \textbf{second} possible pre-state is after we \lstinline{assumed f(a)=g(b)} at \node{n_1}, but before we have updated \node{n_2}:
\begin{figure}[H]
\begin{tikzpicture}
  \node[gttn] (1)              {$()$};
	\node[gl]  (1l) [below = 0 of 1] {\m{n_1}};

  \node[gtn]  (2) [above left  = 0.6cm and 0.2cm of 1] {\s{a}};
  \node[gtn]  (3) [above right = 0.6cm and 0.2cm of 1] {\s{b}};
	
	\draw[gfa]  (2) to node[el] {\m{a}} (1);
  \draw[gfa]  (3) to node[el,anchor=west] {\m{\mathbf{b}}} (1);
  
	\node[gttn] (4) [above = 0.5cm of 2]    {\m{(a)}};
  \node[gttn] (5) [above = 0.5cm of 3]    {\m{(b)}};

	\draw[sgtt] (4) to node[el] {0} (2);
	\draw[sgtt] (5) to node[el] {\textbf{0}} (3);

  \node[gtn]  (6) [above = 2.5cm of 1] {\tiny$\svb{f(a)}{g(b)}$};
	\draw[gfa]  (6) to node[el]             {\m{f}} (4);
  \draw[gfa]  (6) to node[el,anchor=west] {\m{g}} (5);

  \node[hgtn] (6a) [above = 1.5cm of 4] {\tiny$\s{f(\s{a})}$};
%	\draw[hgfa] (6a) to node[hl]          {\m{f}} (4);
%%%%%%%%%%%%%%%%%%%%%%%%%%%%%%%%%%%%%%%%%%%%%%%%%%%%%%%%%%%%%%
  \node[gttn] (11) [right = 3cm of 1] {$()$};
	\node[gl]  (11l) [below = 0 of 11]   {\m{n_2}};

  \node[gtn]  (12) [above = 0.6cm of 11] {\s{a,c}};
	
	\draw[gfa]  (12) to[bend right] node[el,anchor=east] {\m{a}} (11);
	\draw[gfa]  (12) to[bend left]  node[el,anchor=west] {\m{c}} (11);
  
	\node[gttn] (14) [above = 0.5cm of 12]    {\m{(a,c)}};

	\draw[sgtt] (14) to node[el] {0}          (12);

  \node[gtn]  (16) [above = 0.5cm of 14] {\tiny$\m{f\left(\svb{a}{c}\right)}$};
	\draw[gfa]  (16) to node[el]             {\m{f}} (14);
  
%%%%%%%%%%%%%%%%%%%%%%%%%%%%%%%%%%%%%%%%%%%%%%%%%%%%%%%%%%%%%%
	\draw[se] (11) to  ( 1);

	\node (12a) [left = 0.5cm of 12] {};
	\node (3c) [above= 0.1cm of 3] {};
	
	\node (5c) [above= 0.1cm of 5] {};
	\node (14a) [left = 0.7cm of 14] {};

	\draw[se] (14) to (14a) to (5c) to (4);
	\draw[hse] (16) to (6a);
	\draw[he] (6a) to (6);

\draw[draw=none, use as bounding box] (current bounding box.north west) rectangle (current bounding box.south east);

\begin{pgfinterruptboundingbox}
	\draw[separator] (2.0cm,-0.7cm) to (2.0cm,3.5cm);
\end{pgfinterruptboundingbox}

\end{tikzpicture}

\caption{
The sources function\\
gfa invariant broken non-local\\
post-state for \m{n_1.merge(f(a),g(b))}
}
\label{snippet3.16a_graph8}
\end{figure}
In ~\ref{snippet3.16a_graph8} the source edge of the EC-node \m{[f(a)]_{n_2}} has an out-of-date  version of the EC of \m{f(a)} at \m{g_{n_1}}.
This old version points (dotted arrow) to the current version, so when we need to update \m{g_{n_2}}, we need to find all out-of-date sources, map them to the up-to-date versions, and then determine whether existing nodes need merging or new \GFA{} paths need to be added as in the previous case. We maintain enough history information at each EC graph to make this process efficient.

\noindent
The \textbf{third} possible pre-state is before we \lstinline{assume a=c} at \node{n_2}:
\begin{figure}[H]
\begin{tikzpicture}
  \node[gttn] (1)              {$()$};
	\node[gl]  (1l) [below = 0 of 1] {\m{n_1}};

  \node[gtn]  (2) [above left  = 0.6cm and 0.2cm of 1] {\s{a}};
  \node[gtn]  (3) [above right = 0.6cm and 0.2cm of 1] {\s{b}};
	
	\draw[gfa]  (2) to node[el] {\m{a}} (1);
  \draw[gfa]  (3) to node[el,anchor=west] {\m{b}} (1);
  
	\node[gttn] (4) [above = 0.5cm of 2]    {\m{(a)}};
  \node[gttn] (5) [above = 0.5cm of 3]    {\m{(b)}};

	\draw[sgtt] (4) to node[el] {0} (2);
	\draw[sgtt] (5) to node[el] {0} (3);

  \node[gtn]  (6) [above = 2.5cm of 1] {\tiny$\svb{f(a)}{g(b)}$};
	\draw[gfa]  (6) to node[el]             {\m{f}} (4);
  \draw[gfa]  (6) to node[el,anchor=west] {\m{g}} (5);

%%%%%%%%%%%%%%%%%%%%%%%%%%%%%%%%%%%%%%%%%%%%%%%%%%%%%%%%%%%%%%
  \node[gttn] (11) [right = 2.5cm of 1] {$()$};
	\node[gl]  (11l) [below = 0.0cm of 11]   {\m{n_2}};

  \node[gtn]  (12) [above left = 0.6cm and 0.2 of 11] {\s{a}};
  \node[gtn]  (13) [above right= 0.6cm and 0.2 of 11] {\s{c}};
	
	\draw[gfa]  (12) to node[el,anchor=east] {\m{a}} (11);
	\draw[gfa]  (13) to node[el,anchor=west] {\m{c}} (11);
  
	\node[gttn] (15) [above = 0.5cm of 13]    {\m{(c)}};

	\draw[sgtt] (15) to node[el] {0}          (13);

  \node[gtn]  (16) [above = 0.5cm of 15] {\tiny$\s{f(c)}$};
	\draw[gfa]  (16) to node[el]             {\m{f}} (15);
  
%%%%%%%%%%%%%%%%%%%%%%%%%%%%%%%%%%%%%%%%%%%%%%%%%%%%%%%%%%%%%%
%	\draw[se] (11) to  ( 1);

	\node (12a) [left = 0.5cm of 12] {};
	
	\node (3c) [above= 0.1cm of 3] {};

	\draw[se] (12) to (3c) to (2);

\draw[draw=none, use as bounding box] (current bounding box.north west) rectangle (current bounding box.south east);

\begin{pgfinterruptboundingbox}
	\draw[separator] (1.4cm,-0.7cm) to (1.4cm,3.5cm);
\end{pgfinterruptboundingbox}

\end{tikzpicture}

\caption{
The sources function\\
gfa invariant broken non-local\\
pre-state for \m{n_2.assume(a=c)}
}
\label{snippet3.16a_graph9}
\end{figure}
In ~\ref{snippet3.16a_graph9}, at \m{g_{n_2}} we merge the nodes \m{[a]_{n_2}} and \m{[c]_{n_2}}, which propagates the change up until the node \m{[f(c)]_{n_2}} which becomes \m{[f(\s{a,c})]_{n_2}}, and updates the sources for each node on the way according to source and \GFA{} completeness. 
When we reach \m{f(\s{a,c})}, we need to propagate down the \GFA{} \m{[g(b)]_{n_1}} from the sources and complete that path as in the first case.

\noindent
In all of the above cases, the final result is:
\begin{figure}[H]
\begin{tikzpicture}
	\node[gttn] (1)              {$()$};
	\node[gl]   (1l) [below = 0 of 1] {\m{n_1}};

	\node[gtn]  (2) [above left  = 0.6cm and 0.2cm of 1] {\s{a}};
	\node[gtn]  (3) [above right = 0.6cm and 0.2cm of 1] {\s{b}};

	\draw[gfa]  (2) to node[el,anchor=east] {\m{a}} (1);
	\draw[gfa]  (3) to node[el,anchor=west] {\m{b}} (1);

	\node[gttn] (4)  [above = 0.5cm of 2]    {\m{(a)}};
	\node[gttn] (5)  [above = 0.5cm of 3]    {\m{(b)}};

	\draw[sgtt] (4) to node[el] {0} (2);
	\draw[sgtt] (5) to node[el] {0} (3);

	\node[gtn]  (6)  [above = 2.5cm of 1] {\tiny$\svb{f(a)}{g(b)}$};
	\draw[gfa]  (6) to node[el,anchor=east] {\m{f}} (4);
	\draw[gfa]  (6) to node[el,anchor=west] {\m{g}} (5);

%%%%%%%%%%%%%%%%%%%%%%%%%%%%%%%%%%%%%%%%%%%%%%%%%%%%%%%%%%%%%%
	\node[gttn] (11)  [right = 3cm of 1] {$()$};
	\node[gl]   (11l) [below = 0 of 11]   {\m{n_2}};

	\node[gtn]  (12) [above left  = 0.6cm and 0.2cm of 11] {\s{a,c}};
	\node[gtn]  (13) [above right = 0.6cm and 0.6cm of 11] {\s{b}};

	\draw[gfa] (12) to[bend right] node[el,anchor=east] {\m{a}} (11);
	\draw[gfa] (12) to[bend left]  node[el,anchor=west] {\m{c}} (11);
	\draw[gfa] (13) to             node[el,anchor=west] {\m{b}} (11);

	\node[gttn] (14)  [above = 0.5cm of 12] {\m{(\s{a,c})}};
	\node[gttn] (15)  [above = 0.5cm of 13] {\m{(b)}};

	\draw[sgtt] (14) to node[el] {0} (12);
	\draw[sgtt] (15) to node[el] {0} (13);

	\node[gtn]  (16)  [above = 2.5cm of 11] {\tiny$\svb{f(\s{a,c})}{g(b)}$};
	\draw[gfa]  (16) to node[el,anchor=east] {\m{f}} (14);
	\draw[gfa]  (16) to node[el,anchor=west] {\m{g}} (15);
 
%%%%%%%%%%%%%%%%%%%%%%%%%%%%%%%%%%%%%%%%%%%%%%%%%%%%%%%%%%%%%%
	\draw[se] (16) to  (6);

\draw[draw=none, use as bounding box] (current bounding box.north west) rectangle (current bounding box.south east);

\begin{pgfinterruptboundingbox}
	\draw[separator] (1.5cm,-0.7cm) to (1.5cm,3.5cm);
\end{pgfinterruptboundingbox}

\end{tikzpicture}

\caption{
The sources function\\
gfa invariant non-local fixed
}
\label{snippet3.16a_graph10}
\end{figure}
Note that, as opposed to union-find based approaches, equivalent final results are identical.
The three scenarios we have shown could all have been produced during different verifications of the same program, 
if the order of evaluation of the cfg nodes, or the order of evaluation of verification fragments, were different.
As the complexity in generating this result is proportional roughly to the (graph) size of the result, 
we can analyze complexity almost regardless of the order of evaluation.
We will see later that with joins it is harder to keep complexity independent of evaluation order, but we will show a weaker property regarding complexity in different evaluation orders.

\noindent
We will consider one more example for sequential nodes, before formalizing the invariant:
\begin{figure}[H]
\begin{tikzpicture}
	\node[gttn] (1)              {$()$};
	\node[gl]   (1l) [below = 0 of 1] {\m{n_1}};

	\node[gtn]  (2) [above left = 1.0cm and 1.8cm of 1] {\s{a}};
	\node[gtn]  (3) [above left = 1.0cm and 0.4cm of 1] {\s{b}};
	\node[gtn]  (4) [above right= 1.0cm and 0.4cm of 1] {\s{c}};
	\node[gtn]  (5) [above right= 1.0cm and 1.8cm of 1] {\s{d}};

	\draw[gfa]  (2) to node[el,anchor=east] {\m{a}} (1);
	\draw[gfa]  (3) to node[el,anchor=east] {\m{b}} (1);
	\draw[gfa]  (4) to node[el,anchor=west] {\m{c}} (1);
	\draw[gfa]  (5) to node[el,anchor=west] {\m{d}} (1);

	\node[gttn] (2a) [above = 1.0cm of 2]    {\m{(a)}};
	\node[gttn] (3a) [above = 1.0cm of 3]    {\m{(b)}};
	\node[gttn] (4a) [above = 1.0cm of 4]    {\m{(c)}};
	\node[gttn] (5a) [above = 1.0cm of 5]    {\m{(d)}};

	\draw[sgtt] (2a) to node[el] {0} (2);
	\draw[sgtt] (3a) to node[el] {0} (3);
	\draw[sgtt] (4a) to node[el] {0} (4);
	\draw[sgtt] (5a) to node[el] {0} (5);

	\node[gtn]  (6)  [above right = 1.0cm and 0.0cm of 2a] {\tiny$\svb{f(a)}{f(b)}$};
	\draw[gfa]  (6) to node[el,anchor=east] {\m{f}} (2a);
	\draw[gfa]  (6) to node[el,anchor=west] {\m{f}} (3a);

	\node[gtn]  (7)  [above left = 1.0cm and 0.0cm of 5a] {\tiny$\svb{f(c)}{f(d)}$};
	\draw[gfa]  (7) to node[el,anchor=east] {\m{f}} (4a);
	\draw[gfa]  (7) to node[el,anchor=west] {\m{f}} (5a);

%%%%%%%%%%%%%%%%%%%%%%%%%%%%%%%%%%%%%%%%%%%%%%%%%%%%%%%%%%%%%%
	\node[gttn] (11)  [right = 6cm of 1] {$()$};
	\node[gl]   (11l) [below = 0 of 11]   {\m{n_2}};

	\node[gtn]  (12) [above left  = 1.0cm and 0.4cm of 11] {\s{a}};
	\node[gtn]  (13) [above right = 1.0cm and 0.4cm of 11] {\s{b,c}};

	\draw[gfa] (12) to             node[el,anchor=east] {\m{a}} (11);
	\draw[gfa] (13) to[bend right] node[el,anchor=east] {\m{b}} (11);
	\draw[gfa] (13) to[bend left]  node[el,anchor=west] {\m{c}} (11);

	\node[gttn] (12a)  [above = 1.0cm of 12] {\m{(a)}};
	\draw[sgtt] (12a) to node[el] {0} (12);
 
%%%%%%%%%%%%%%%%%%%%%%%%%%%%%%%%%%%%%%%%%%%%%%%%%%%%%%%%%%%%%%
	\node (5au) [above = 0.2 of 5a] {};
	\node (3au) [above = 0.2 of 3a] {};
	\draw[se] (12a) to (5au) to (3au) to (2a);

	\node (12u) [above = 0.2 of 12] {};
	\node (5u) [above = 0.2 of 5] {};
	\node (5d) [below = 0.2 of 5] {};
	\node (4u) [above = 0.2 of 4] {};
	\node (4d) [below = 0.2 of 4] {};

	\draw[se] (13) to (12u) to (5u) to (4u) to (3);
	\draw[se] (13) to (12u) to (5u) to (4);

\draw[draw=none, use as bounding box] (current bounding box.north west) rectangle (current bounding box.south east);

\begin{pgfinterruptboundingbox}
	\draw[separator] (3.5cm,-0.7cm) to (3.5cm,5.5cm);
\end{pgfinterruptboundingbox}

\end{tikzpicture}

\caption{
The sources function\\
multiple up-down propagations\\
before \lstinline|$\m{n_1}$.add(f(a))|
}
\label{graph_sequential_multi_propagation_pre}
\end{figure}
\noindent
In ~\ref{graph_sequential_multi_propagation_pre}, we show the relevant sources before adding the term \m{f(a)} to the \node{n_2}.\\
We now show the post-state, with the propagation paths highlighted:
\begin{figure}[H]
\begin{tikzpicture}
	\node[gttn] (1)              {$()$};
	\node[gl]   (1l) [below = 0 of 1] {\m{n_1}};

	\node[gtn]              (2) [above left = 1.0cm and 1.8cm of 1] {\s{a}};
	\node[gtn,ultra thick]  (3) [above left = 1.0cm and 0.3cm of 1] {\s{b}};
	\node[gtn,ultra thick]  (4) [above right= 1.0cm and 0.3cm of 1] {\s{c}};
	\node[gtn,ultra thick]  (5) [above right= 1.0cm and 1.8cm of 1] {\s{d}};

	\draw[gfa]              (2) to node[el,anchor=east] {\m{a}} (1);
	\draw[gfa]              (3) to node[el,anchor=east] {\m{b}} (1);
	\draw[gfa]              (4) to node[el,anchor=west] {\m{c}} (1);
	\draw[gfa,ultra thick]  (5) to node[el,anchor=west] {\m{d}} (1);

	\node[gttn]             (2a) [above = 1.0cm of 2]    {\m{(a)}};
	\node[gttn,ultra thick] (3a) [above = 1.0cm of 3]    {\m{(b)}};
	\node[gttn,ultra thick] (4a) [above = 1.0cm of 4]    {\m{(c)}};
	\node[gttn,ultra thick] (5a) [above = 1.0cm of 5]    {\m{(d)}};

	\draw[sgtt] (2a) to node[el] {0} (2);
	\draw[sgtt,ultra thick] (3a) to node[el] {0} (3);
	\draw[sgtt,ultra thick] (4a) to node[el] {0} (4);
	\draw[sgtt,ultra thick] (5a) to node[el] {0} (5);

	\node[gtn,ultra thick]  (6)  [above right = 1.0cm and -0.1cm of 2a] {\tiny$\svb{f(a)}{f(b)}$};
	\draw[gfa,ultra thick]  (6) to node[el,anchor=east] {\m{f}} (2a);
	\draw[gfa,ultra thick]  (6) to node[el,anchor=west] {\m{f}} (3a);

	\node[gtn,ultra thick]  (7)  [above left = 1.0cm and -0.1cm of 5a] {\tiny$\svb{f(c)}{f(d)}$};
	\draw[gfa,ultra thick]  (7) to node[el,anchor=east] {\m{f}} (4a);
	\draw[gfa,ultra thick]  (7) to node[el,anchor=west] {\m{f}} (5a);

%%%%%%%%%%%%%%%%%%%%%%%%%%%%%%%%%%%%%%%%%%%%%%%%%%%%%%%%%%%%%%
	\node[gttn] (11)  [right = 6cm of 1] {$()$};
	\node[gl]   (11l) [below = 0 of 11]   {\m{n_2}};

	\node[gtn]              (12) [above left  = 1.0cm and 0.8cm of 11] {\s{a}};
	\node[gtn,ultra thick]  (13) [above       = 0.9cm           of 11] {\s{b,c}};
	\node[gtn]              (14) [above right = 1.0cm and 0.8cm of 11] {\s{d}};

	\draw[gfa] (12) to             node[el,anchor=east] {\m{a}} (11);
	\draw[gfa] (13) to[bend right] node[el,anchor=east] {\m{b}} (11);
	\draw[gfa] (13) to[bend left]  node[el,anchor=west] {\m{c}} (11);
	\draw[gfa] (14) to             node[el,anchor=west] {\m{d}} (11);

	\node[gttn] (12a)  [above = 1.0cm of 12] {\m{(a)}};
	\draw[sgtt] (12a) to node[el] {0} (12);

	\node[gttn] (13a)  [above = 1.0cm of 13] {\m{(\s{b,c})}};
	\draw[sgtt] (13a) to node[el] {0} (13);

	\node[gttn] (14a)  [above = 1.0cm of 14] {\m{(d)}};
	\draw[sgtt] (14a) to node[el] {0} (14);
 
	\node[gtn]  (16)  [above = 1.0cm of 13a] {\tiny$\m{f(\s{a,b,c,d}}$};
	\draw[gfa]  (16) to node[el,anchor=east] {\m{f}} (12a);
	\draw[gfa]  (16) to node[el,anchor=east] {\m{f}} (13a);
	\draw[gfa]  (16) to node[el,anchor=west] {\m{f}} (14a);
%%%%%%%%%%%%%%%%%%%%%%%%%%%%%%%%%%%%%%%%%%%%%%%%%%%%%%%%%%%%%%
	\draw[se,ultra thick] (11) to (1);
	\node (5au) [above = 0.2 of 5a] {};
	\node (3au) [above = 0.2 of 3a] {};
	\draw[se,ultra thick] (12a) to (5au) to (3au) to (2a);

	\node (12u) [above = 0.2 of 12] {};
	\node (5u) [above = 0.2 of 5] {};
	\node (5d) [below = 0.2 of 5] {};
	\node (4u) [above = 0.2 of 4] {};
	\node (4d) [below = 0.2 of 4] {};

	\draw[se,ultra thick] (13) to (12u) to (5u) to (4u) to (3);
	\draw[se,ultra thick] (13) to (12u) to (5u) to (4);
	
	
	\node (a1)  [left  = 0.2cm of  2a] {};
	\node (a2)  [above = 0.2cm of  6 ] {};
	\node (a3)  [right = 0.2cm of  3a] {};
	\node (a4)  [above right = 0.0cm and 0.2cm of 3 ] {};
	\node (a5)  [above = 0.0cm of  4u ] {};
	\node (a6)  [above right = 0.0cm and 0.0cm of 12u ] {};
	\node (a7)  [above right = 0.2cm and 0.2cm of 13.180] {};
	\node (a8)  [above = 0.0cm of a7] {};
	\node (a9)  [above = 0.0cm of a6] {};
	\node (a10) [above = 0.0cm of a5] {};
	\node (a11) [left  = 0.0cm of 4a] {};
	\node (a12) [above = 0.2cm of 7] {};
	\node (a13) [right = 0.2cm of 5a] {};
	\node (a14) [right = 0.2cm of 5] {};
	\node (a15) [right = 0.2cm of 1] {};

	\draw[->,thin,green,out=90,in=-90] 
		(a1) to[in=180] (a2) to[out=0,in=90] (a3) to[out=-90,in=180] (a4) to[out=0,in=180] (a5) to[out=0,in=180] (a6) to[out=0,in=180] (a7) 
		to[out=0,in=0]  (a9) to[out=180,in=0] (a10) to[out=180,in=-90] (a11) to[out=90,in=180] (a12) 
		to[out=0,in=90] (a13) to[out=-90,in=90] (a14) to[out=-90,in=45] (a15);
		
\draw[draw=none, use as bounding box] (current bounding box.north west) rectangle (current bounding box.south east);

\begin{pgfinterruptboundingbox}
	\draw[separator] (4.0cm,-0.7cm) to (4.0cm,5.0cm);
\end{pgfinterruptboundingbox}

\end{tikzpicture}

\caption{
The sources function\\
multiple up-down propagations\\
after \lstinline{add(f(a))}
}
\label{graph_sequential_multi_propagation_post}
\end{figure}
The order of traversal would be as depicted by the green line in ~\ref{graph_sequential_multi_propagation_post}.

We will now state the local invariant, including the set of relevant terms \rtA{n} for the node \m{n}:
\begin{figure}[H]
\begin{enumerate}
	\item 
		\m{\forall u,v \in g_n \cdot }\\
		\m{\sources{n}{p}{u} \cap \sources{n}{p}{v} \neq \emptyset \Rightarrow u = v }
	\item All sources are marked as relevant terms:\\
		\m{\forall t \in g_n, s \in \sources{n}{p}{t} \cdot }\\
		\m{s \in \rtA{n}}
	\item All relevant terms are downward closed:\\
		\m{\forall s \in \rtA{n}, \fa{f}{v} \in s, \cdot \tup{v} \in \rtA{n}}
	\item Relevant terms are constructed bottom up:\\
		\m{\forall s \in \rtA{n}, \fa{f}{v} \in s, \tup{u} \in \sourcesInv{n}{p}{\tup{v}} \cdot}\\
		\m{\fa{f}{u} \in \gfasA{g_n}}
\end{enumerate}
\caption{Sequential graph based local propagation invariant}
\end{figure}


\subsubsection{Incremental complexity}
In the sequential case, the EC-graph based definition of \GFA{} completeness directly hints at an algorithm to ensure it - the first conjunct forces us to merge nodes,
the second and third add relevant source terms, and the fourth adds nodes in \m{g_n}.\\
Each relevant term ends up as a source to an actual node in \m{g_n} and each node in \m{g_n} is
a sub-term-node of a term-node that existed in the pre-state.\\
Each source term is only traversed a constant number of times (at most - marked as relevant, then as source, and potentially becomes not up-to-date), and each of the rules that compare two \GFAs{} need only be evaluated a constant number of times, if, whenever a node performs an \lstinline{update} operation, it receives from the predecessor a list of changes.

\newpage
\subsection{Join completeness criteria}
As opposed to the sequential case, we can define several different join completeness criteria.\\
The spectrum runs between the weakest join:
\begin{figure}[H]
\m{\forall s \in \terms{g_n}, t \in \Ts{\sig} \cdot }\\
\m{((\forall p \in \preds{n} \cdot (s \in \terms{g_p} \land [s]_{g_p} = [t]_{g_p})) \Rightarrow }\\
\m{(t \in \terms{g_n} \land [s]_{g_n}=[t]_{g_n}))}
\caption{Weak join completeness}
\end{figure}

Which is simply an extension of the sequential case, and the complete join:
\begin{figure}[H]
\m{\forall s \in \terms{g_n}, t \in \Ts{\sig} \cdot }\\
\m{(\forall p \in \preds{n} \cdot \eqs{g_p} \cup \eqs{g_n} \models s=t) \Rightarrow}\\
\m{ (t \in \terms{g_n} \land [s]_{g_n}=[t]_{g_n})}
\caption{Strong join completeness}
\end{figure}



%
%Remember that the \GFA{} completeness invariant for the sequential case was:
%\m{\forall u \in g_n, v \in \sources{n}{p}{u}, \fa{f}{s} \in v \cdot}\\
%\m{\exists \fa{f}{t} \in u \cdot \tup{s} \in \sources{n}{p}{\tup{t}} }
%
%Adapting the above to the join case requires, at the very least, that we match \GFAs{} in \emph{all} direct predecessors and only then force the join graph to have a corresponding gfa. Note that in the sequential case we matched the \GFA{} on the function symbol, and then on the source tuple EC. 
%For a given EC term node \m{t \in g_n}, assuming we have a source node \m{s_p \in g_p} (\m{s_p \in \sources{n}{p}{t}}) for each predecessor \m{p},
%and assuming we have a \GFA{} \m{\fa{f}{v_p} \in s_p} for each predecessor (with the same function symbol), 
%then we would need a tuple EC \m{\tup{u} \in g_n} s.t. for each predecessor, \m{\tup{v_p} \in \sources{n}{p}{\tup{u}}},
%if such a \tup{u} exists, then obviously \fa{f}{u} should be in \gfasA{t}, the question is when \emph{should} such a \tup{u} exist?
We want to generalize the graph based local invariant for sequential nodes to binary join nodes, however, even for the weak join , the extension is not direct - consider, for example:\\
\m{ \s{a=f(b)} \sqcup \s{a=g(b)}}\\
Here, if we want to add the term a at the join, we would mark the EC-node for the term b as a relevant node in both joinees,
and hence we would add an EC-node for b to the join, although it is not needed.\\
We will now see how we solve this problem for the weak join, and later add the missing parts for the strong join.

%As another example, consider, on the one hand:\\
%\m{ \s{a=f(b)} \sqcup \s{a=f(c),b=c}}\\
%and on the other hand:\\
%\m{ \s{a=f(b)} \sqcup \s{a=f(c)}}\\
%(we will use \m{p_0,p_1} for the predecessor \cfg nodes)\\
%In the first case, the term EC node for \m{a} at the join node EC graph \m{g_n} would have a source with the \GFA{} \m{f([b]_{g_p})} at each predecessor, and we would expect a term EC node \m{[b]_{g_n}} for \m{b} to exist, 
%and hence the term EC node for \m{a} at \m{g_n} to have the  corresponding \GFA{} - that is, \m{f([b]_{g_n}) \in [a]_{g_n}}.\\
%In the second case, although the sources for \m{[a]_{g_n}} share a \GFA{} with the same function symbol \m{f}, 
%we do not expect to have a node in \m{u \in g_n} s.t. \\
%\m{[b]_{g_{p_0}} \in \sources{n}{p_0}{u}} and \m{[c]_{g_{p_1}} \in \sources{n}{p_1}{u}},
%as such a node does not correspond to any term that is equal in the join
%(unless we have \lstinline{assumed} at \m{n} that \m{b=c}).
%We can observe locally that no such term EC node is expected because there is no \GFA{} with a common function symbol between 
%\m{[b]_{g_{p_1}}} and \m{[c]_{g_{p_2}}}.

\subsection{Weak join}
In ~\ref{snippet3.16b}, we would need to add the terms \m{f(\s{a,b}),g(\s{a,b})} to \m{g_{p_0}} \\
and \m{f(\s{b,c}),g(\s{b,c})} to \m{g_{p_1}}, as shown in figure ~\ref{snippet3.16b_graph1}:
\begin{figure}[H]
\begin{tikzpicture}
	\node[gttn] (1)              {$()$};
	\node[gl]   (1l) [below = 0 of 1] {\m{s}};

	\node[gtn]  (2) [above left  = 1cm and 0.5cm of 1] {\s{a}};
	\node[gtn]  (3) [above right = 1cm and 0.5cm of 1] {\s{c}};

	\draw[gfa] (2) to node[el] {\m{a}} (1);
	\draw[gfa] (3) to node[el,anchor=west] {\m{c}} (1);

	\node[gttn] (4)  [above = 1cm of 2]    {\m{(a)}};
	\node[gttn] (5)  [above = 1cm of 3]    {\m{(c)}};

	\draw[sgtt] (4) to node[el] {0} (2);
	\draw[sgtt] (5) to node[el] {0} (3);

	\node[gtn]  (6)  [above = 1cm of 4] {\tiny$\stackB{f(a)}{g(a)}$};
	\draw[gfa]  (6) to[out=-110,in=110] node[el] {\m{f}} (4);
	\draw[gfa]  (6) to[out=- 70,in= 70] node[el,anchor=west] {\m{g}} (4);

	\node[gtn]  (7)  [above = 1cm of 5] {\tiny$\stackB{f(c)}{g(c)}$};
	\draw[gfa]  (7) to[out=-110,in=110] node[el] {\m{f}} (5);
	\draw[gfa]  (7) to[out=- 70,in= 70] node[el,anchor=west] {\m{g}} (5);

%%%%%%%%%%%%%%%%%%%%%%%%%%%%%%%%%%%%%%%%%%%%%%%%%%%%%%%%%%%%%%
	\node[gttn] (11)  [above right = 2.5cm and 4.5cm of 1] {$()$};
	\node[gl]   (11l) [below = 0 of 11]   {\m{p_0}};

	\node[gtn]  (12) [above = 1cm of 11] {\s{a,b}};

	\draw[gfa] (12) to[out=-110,in=110] node[el] {\m{a}} (11);
	\draw[gfa] (12) to[out=- 70,in= 70] node[el,anchor=west] {\m{b}} (11);

	\node[gttn] (14)  [above = 1cm of 12]    {\m{f(\s{a,b})}};

	\draw[sgtt] (14) to node[el] {0} (12);

	\node[gtn]  (16)  [above = 3.5cm of 11] {\tiny $\faB{f}{a}{b},\faB{g}{a}{b}$};
	\draw[gfa]  (16) to[out=-100, in= 100] node[el] {\m{f}} (14);
	\draw[gfa]  (16) to[out=- 80 ,in=  80] node[el,anchor=west] {\m{g}} (14);
				
%%%%%%%%%%%%%%%%%%%%%%%%%%%%%%%%%%%%%%%%%%%%%%%%%%%%%%%%%%%%%%
	\node[gttn] (21)  [below right = 2.5cm and 4.5cm of 1] {$()$};
	\node[gl]   (21l) [below = 0 of 21]   {\m{p_1}};

	\node[gtn]  (22) [above = 1cm of 21] {\s{b,c}};

	\draw[gfa] (22) to[out=-110,in=110] node[el]             {\m{b}} (21);
	\draw[gfa] (22) to[out=- 70,in= 70] node[el,anchor=west] {\m{c}} (21);

	\node[gttn] (24)  [above = 1cm of 22]    {\m{f(\s{b,c})}};

	\draw[sgtt] (24) to node[el] {0} (22);

	\node[gtn]  (26)  [above = 3.5cm of 21] {\tiny $\faB{f}{b}{c},\faB{g}{b}{c}$};
	\draw[gfa]  (26) to[out=-100,in=100] node[el]             {\m{f}} (24);
	\draw[gfa]  (26) to[out=- 80,in= 80] node[el,anchor=west] {\m{g}} (24);

%%%%%%%%%%%%%%%%%%%%%%%%%%%%%%%%%%%%%%%%%%%%%%%%%%%%%%%%%%%%%%

	\node[gttn] (31)  [right = 9cm of 1] {$()$};
	\node[gl]   (31l) [below = 0 of 31]   {\m{n}};

%	\node (31jl)  [above left = -0.2cm and 0cm of 31] {$\sqcup$};

	\node[gtn] (32) [above = 0.5cm of 31] {\m{b}};

	\draw[gfa] (32) to node[el] {\m{b}} (31);

	\node[gttn] (34)  [above = 0.5cm of 32]    {\m{(b)}};

	\draw[sgtt] (34) to node[el] {0} (32);

	\node[gtn]  (36)  [above = 2.5cm of 31] {\tiny $\stackB{f(b)}{g(b)}$};
	\draw[gfa]  (36) to[out=-100 ,in=100] node[el] {\m{f}} (34);
	\draw[gfa]  (36) to[out=- 80 ,in= 80] node[el,anchor=west] {\m{g}} (34);


%	\node (31jl)  [above left = -0.2cm and 0cm of 31] {$\sqcup$};
%	\node (32jl)  [above left = -0.2cm and 0cm of 32] {$\sqcup$};
%	\node (34jl)  [above left = -0.2cm and 0cm of 34] {$\sqcup$};
%	\node (36jl)  [above left = -0.2cm and 0cm of 36] {$\sqcup$};

%%%%%%%%%%%%%%%%%%%%%%%%%%%%%%%%%%%%%%%%%%%%%%%%%%%%%%%%%%%%%%
	\node (11a) [left = 0.3cm of 11] {};
	\draw[se] (11) to[out=180,in=0] (11a) to[out=180,in=0] (1);
	\node (21a) [left = 0.3cm of 21] {};
	\draw[se] (21) to[out=180,in=0] (21a) to[out=180,in=0] (1);

	\node (3c) [above= 0.1cm of 3] {};
	\node (12a) [left = 0.3cm of 12] {};
	\draw[se] ( 12.180) to[out=180,in=0] (12a.0) to[out=180,in=0] (   2.0);
	%  \draw[se] ( 12.180) to[out=180,in=0] (12a.0) to[out=180,in=0] (3c) to[out=180,in=0] (   2.0);


	\node (5c) [above= 0.1cm of 5] {};
	\node (14a) [left = 0.5cm of 14] {};
	\draw[se] (14.180) to[out=180,in=0] (14a.0) to[out=180,in=0]( 4.0);
	%  \draw[se] (14.180) to[out=180,in=0] (14a.0) to[out=180,in=0] (5c) to[out=180,in=0]( 4.0);
	\draw[se] (16) to[out=180,in=0]  (6);

	\node (3d) [below= 0.1cm of 3] {};
	\node (22a) [left = 0.3cm of 22] {};
	\draw[se] ( 22.180) to[out=180,in=0] (22a.0) to[out=180,in=0] (   3.0);

	\node (5d) [below= 0.1cm of 5] {};
	\node (24a) [left = 0.5cm of 24] {};
	%  \draw[se] (24.180) to[out=180,in=0] (24a.0) to[out=180,in=0] (5d) to[out=180,in=0]( 4.0);
	\draw[se] (24.180) to[out=180,in=0] (24a.0) to[out=180,in=0]( 5.0);
	\draw[se] (26.180) to[out=180,in=0]  (7.0);

	\draw[se] (31) to[out=180,in=0] (11);
	\draw[se] (31) to[out=180,in=0] (21);

	\draw[se] (32) to[out=180,in=0] (12);
	\draw[se] (32) to[out=180,in=0] (22);
	\draw[se] (34) to[out=180,in=0] (14);
	\draw[se] (34) to[out=180,in=0] (24);
	\draw[se] (36) to[out=180,in=0] (16);
	\draw[se] (36) to[out=180,in=0] (26);

	\draw[ie] (36) to[loop above] node[el,below] {\m{\neq}} (36);


\draw[draw=none, use as bounding box] (current bounding box.north west) rectangle (current bounding box.south east);

\begin{pgfinterruptboundingbox}
	\draw[separator] (2.5cm,-3.3cm) to (2.5cm,6.9cm);
	\draw[separator] (6.5cm,-3.3cm) to (6.5cm,6.9cm);
\end{pgfinterruptboundingbox}

\end{tikzpicture}

\caption{
Join sources
}
\label{snippet3.16b_graph1}
\end{figure}


In ~\ref{snippet3.17a_graph} we have a join where some join nodes share a source in one joinee, 
but not in both (irrelevant sources and rgfas are omitted):
\begin{figure}[H]
\begin{tikzpicture}
  \node (1)  {};
%%%%%%%%%%%%%%%%%%%%%%%%%%%%%%%%%%%%%%%%%%%%%%%%%%%%%%%%%%%%%%
  \node[gttn] (11)  [above right= 2.5cm and 0cm of 1] {$()$};
	\node[gl]   (11l) [below = 0 of 11]   {\m{p_0}};

  \node[gtn]  (12) [above left  = 1cm and 1cm of 11] {\s{a,b}};
  \node[gtn]  (14) [above right = 1cm and 1cm of 11] {\s{c}};
	
  \draw[gfa] (12)  to[bend right] node[el]             {\m{a}} (11);
  \draw[gfa] (12)  to[bend left]  node[el,anchor=west] {\m{b}} (11);
  \draw[gfa] (14)  to             node[el]             {\m{c}} (11);

%%%%%%%%%%%%%%%%%%%%%%%%%%%%%%%%%%%%%%%%%%%%%%%%%%%%%%%%%%%%%%
  \node[gttn] (21)  [below right= 2.5cm and 0cm of 1] {$()$};
	\node[gl]   (21l) [below = 0 of 21]   {\m{p_1}};

  \node[gtn]  (22) [above left  = 1.0cm and 1.0cm of 21] {\s{a}};
  \node[gtn]  (23) [above right = 1.0cm and 1.0cm of 21] {\s{b,c}};
	
  \draw[gfa] (22) to             node[el,anchor=east] {\m{a}} (21);
  \draw[gfa] (23) to[bend right] node[el,anchor=west] {\m{b}} (21);
  \draw[gfa] (23) to[bend left]  node[el,anchor=west] {\m{c}} (21);

%%%%%%%%%%%%%%%%%%%%%%%%%%%%%%%%%%%%%%%%%%%%%%%%%%%%%%%%%%%%%%

  \node[gttn] (31)  [right = 6cm of 1] {$()$};
	\node[gl]   (31l) [below = 0 of 31]   {\m{n}};

%  \node (31jl)  [above left = -0.2cm and 0cm of 31] {$\sqcup$};

  \node[gtn]  (32) [above left  = 1.0cm and 1.0cm of 31] {\m{a}};
  \node[gtn]  (33) [above       = 0.9cm           of 31] {\m{b}};
  \node[gtn]  (34) [above right = 1.0cm and 1.0cm of 31] {\m{c}};
	
  \draw[gfa]  (32)  to node[el,anchor=east] {\m{a}} (31);
  \draw[gfa]  (33)  to node[el,anchor=west] {\m{b}} (31);
  \draw[gfa]  (34)  to node[el,anchor=west] {\m{c}} (31);

%%%%%%%%%%%%%%%%%%%%%%%%%%%%%%%%%%%%%%%%%%%%%%%%%%%%%%%%%%%%%%
%	\draw[se] (31) to[out=180,in=0] (11);
%	\draw[se] (31) to[out=180,in=0] (21);

	\node (12a) [right = 0.5cm of 12] {};
	\node (23a) [right = 0.5cm of 23] {};
	\draw[se] (32) to[out=180,in=0] (12a) to[out=180,in=0] (12);
	\draw[se] (32) to[out=180,in=0] (22);
	\draw[se] (33) to[out=180,in=0] (12a)  to[out=180,in=0] (12);
	\draw[se] (33) to[out=180,in=0] (23a)  to[out=180,in=0] (23);
	\draw[se] (34) to[out=180,in=0] (14);
	\draw[se] (34) to[out=180,in=0] (23a) to[out=180,in=0] (23);

\draw[draw=none, use as bounding box] (current bounding box.north west) rectangle (current bounding box.south east);

\begin{pgfinterruptboundingbox}
	\draw[separator] (3.5cm,-3.3cm) to (3.5cm,3.9cm);
\end{pgfinterruptboundingbox}

\end{tikzpicture}

\caption{
Join shared sources\\
before \lstinline{n.assume(a=c)}
}
\label{snippet3.17a_graph}
\end{figure}
Here \m{[b]_n} shares the source \m{[\s{a,b}]_{p_0}} with \m{[a]_n}, and the source \m{[\s{b,c}]_{p_1}} with \m{[c]_n}.

\noindent
Now we show what happens to ~\ref{snippet3.17a_graph} if we try to \lstinline{assume a=c} at the join:
\begin{figure}[H]
\begin{tikzpicture}
  \node (1)  {};
%%%%%%%%%%%%%%%%%%%%%%%%%%%%%%%%%%%%%%%%%%%%%%%%%%%%%%%%%%%%%%
  \node[gttn] (11)  [above right= 2.5cm and 0cm of 1] {$()$};
	\node[gl]   (11l) [below = 0 of 11]   {\m{p_0}};

  \node[gtn]  (12) [above left  = 1cm and 1cm of 11] {\s{a,b}};
  \node[gtn]  (14) [above right = 1cm and 1cm of 11] {\s{c}};
	
  \draw[gfa] (12)  to[bend right] node[el]             {\m{a}} (11);
  \draw[gfa] (12)  to[bend left]  node[el,anchor=west] {\m{b}} (11);
  \draw[gfa] (14)  to             node[el]             {\m{c}} (11);

%%%%%%%%%%%%%%%%%%%%%%%%%%%%%%%%%%%%%%%%%%%%%%%%%%%%%%%%%%%%%%
  \node[gttn] (21)  [below right= 2.5cm and 0cm of 1] {$()$};
	\node[gl]   (21l) [below = 0 of 21]   {\m{p_1}};

  \node[gtn]  (22) [above left  = 1.0cm and 1.0cm of 21] {\s{a}};
  \node[gtn]  (23) [above right = 1.0cm and 1.0cm of 21] {\s{b,c}};
	
  \draw[gfa] (22) to             node[el,anchor=east] {\m{a}} (21);
  \draw[gfa] (23) to[bend right] node[el,anchor=west] {\m{b}} (21);
  \draw[gfa] (23) to[bend left]  node[el,anchor=west] {\m{c}} (21);

%%%%%%%%%%%%%%%%%%%%%%%%%%%%%%%%%%%%%%%%%%%%%%%%%%%%%%%%%%%%%%

  \node[gttn] (31)  [right = 6cm of 1] {$()$};
	\node[gl]   (31l) [below = 0 of 31]   {\m{n}};

%  \node (31jl)  [above left = -0.2cm and 0cm of 31] {$\sqcup$};

  \node[gtn]  (32) [above left  = 1.0cm and 1.0cm of 31] {\s{a,c}};
  \node[gtn]  (33) [above right = 1.0cm and 1.0cm of 31] {\s{b}};
	
  \draw[gfa]  (32)  to[bend right] node[el,anchor=east] {\m{a}} (31);
  \draw[gfa]  (32)  to[bend left]  node[el,anchor=east] {\m{c}} (31);
  \draw[gfa]  (33)  to             node[el,anchor=west] {\m{b}} (31);

%%%%%%%%%%%%%%%%%%%%%%%%%%%%%%%%%%%%%%%%%%%%%%%%%%%%%%%%%%%%%%
%	\draw[se] (31) to[out=180,in=0] (11);
%	\draw[se] (31) to[out=180,in=0] (21);

	\node (12a) [right = 0.5cm of 12] {};
	\node (23a) [right = 0.5cm of 23] {};
	\node (32b) [above left = 0.2cm and 1.0cm of 32] {};
	\node (32c) [below left = 0.2cm and 1.0cm of 32] {};
	\draw[se,ultra thick] (32) to[out=180,in=0] (32b) to[out=180,in=0] (12a) to[out=180,in=0] (12);
	\draw[se] (32) to[out=180,in=0] (32c) to[out=180,in=0] (22);
	\draw[se] (32) to[out=180,in=0] (32b) to[out=180,in=0] (14);
	\draw[se,ultra thick] (32) to[out=180,in=0] (32c) to[out=180,in=0] (23a) to[out=180,in=0] (23);
	\draw[se,ultra thick] (33) to[out=180,in=0] (12a) to[out=180,in=0] (12);
	\draw[se,ultra thick] (33) to[out=180,in=0] (23a) to[out=180,in=0] (23);

\draw[draw=none, use as bounding box] (current bounding box.north west) rectangle (current bounding box.south east);

\begin{pgfinterruptboundingbox}
	\draw[separator] (3.5cm,-3.3cm) to (3.5cm,3.9cm);
\end{pgfinterruptboundingbox}

\end{tikzpicture}

\caption{
Join shared sources\\
after \lstinline{n.assume(a=c)} \\
broken propagation invariant
}
\label{snippet3.17b_graph}
\end{figure}
In ~\ref{snippet3.17b_graph} we have highlighted the shared pair of sources for the nodes \m{[\s{a,c}]_n}, \m{[b]_n}  - 
for joins this means that the first part of the propagation invariant is broken.
The fixed graphs are shown in ~\ref{snippet3.17c_graph}
\begin{figure}[H]
\begin{tikzpicture}
  \node (1)  {};
%%%%%%%%%%%%%%%%%%%%%%%%%%%%%%%%%%%%%%%%%%%%%%%%%%%%%%%%%%%%%%
  \node[gttn] (11)  [above right= 2.5cm and 0cm of 1] {$()$};
	\node[gl]   (11l) [below = 0 of 11]   {\m{p_0}};

  \node[gtn]  (12) [above left  = 1cm and 1cm of 11] {\s{a,b}};
  \node[gtn]  (14) [above right = 1cm and 1cm of 11] {\s{c}};
	
  \draw[gfa] (12)  to[bend right] node[el]             {\m{a}} (11);
  \draw[gfa] (12)  to[bend left]  node[el,anchor=west] {\m{b}} (11);
  \draw[gfa] (14)  to             node[el]             {\m{c}} (11);

%%%%%%%%%%%%%%%%%%%%%%%%%%%%%%%%%%%%%%%%%%%%%%%%%%%%%%%%%%%%%%
  \node[gttn] (21)  [below right= 2.5cm and 0cm of 1] {$()$};
	\node[gl]   (21l) [below = 0 of 21]   {\m{p_1}};

  \node[gtn]  (22) [above left  = 1.0cm and 1.0cm of 21] {\s{a}};
  \node[gtn]  (23) [above right = 1.0cm and 1.0cm of 21] {\s{b,c}};
	
  \draw[gfa] (22) to             node[el,anchor=east] {\m{a}} (21);
  \draw[gfa] (23) to[bend right] node[el,anchor=west] {\m{b}} (21);
  \draw[gfa] (23) to[bend left]  node[el,anchor=west] {\m{c}} (21);

%%%%%%%%%%%%%%%%%%%%%%%%%%%%%%%%%%%%%%%%%%%%%%%%%%%%%%%%%%%%%%

  \node[gttn] (31)  [right = 5cm of 1] {$()$};
	\node[gl]   (31l) [below = 0 of 31]   {\m{n}};

%  \node (31jl)  [above left = -0.2cm and 0cm of 31] {$\sqcup$};

  \node[gtn]  (32) [above= 1.0cmof 31] {\s{a,b,c}};
	
  \draw[gfa]  (32)  to[bend right] node[el,anchor=east] {\m{a}} (31);
  \draw[gfa]  (32)  to             node[el,anchor=west] {\m{b}} (31);
  \draw[gfa]  (32)  to[bend left]  node[el,anchor=west] {\m{c}} (31);

%%%%%%%%%%%%%%%%%%%%%%%%%%%%%%%%%%%%%%%%%%%%%%%%%%%%%%%%%%%%%%
%	\draw[se] (31) to[out=180,in=0] (11);
%	\draw[se] (31) to[out=180,in=0] (21);

	\node (12a) [right = 0.5cm of 12] {};
	\node (23a) [right = 0.5cm of 23] {};
	\node (32b) [above left = 0.2cm and 1.0cm of 32] {};
	\node (32c) [below left = 0.2cm and 1.0cm of 32] {};
	\draw[se] (32) to[out=180,in=0] (32b) to[out=180,in=0] (12a) to[out=180,in=0] (12);
	\draw[se] (32) to[out=180,in=0] (32c) to[out=180,in=0] (22);
	\draw[se] (32) to[out=180,in=0] (32b) to[out=180,in=0] (14);
	\draw[se] (32) to[out=180,in=0] (32c) to[out=180,in=0] (23a) to[out=180,in=0] (23);

\draw[draw=none, use as bounding box] (current bounding box.north west) rectangle (current bounding box.south east);

\begin{pgfinterruptboundingbox}
	\draw[separator] (4.0cm,-3.3cm) to (4.0cm,4.2cm);
\end{pgfinterruptboundingbox}

\end{tikzpicture}

\caption{
Join shared sources
}
\label{snippet3.17c_graph}
\end{figure}

The condition for propagation completeness is that separate nodes cannot share a source in all predecessors - this is a generalization of the rule for sequential nodes. We will soon show that this is insufficient, and formalize the complete version of the condition.

\subsubsection{The sources invariant}
Here we phrase the source invariant for the weak join for a cfg-node n:
\begin{figure}[H]
\begin{enumerate}
\item The \textbf{first part} remains as in the sequential case, ensuring we are not missing sources edges:\\
	\m{\forall p \in \preds{n},\fa{f}{t} \in \gfasA{g_n}, \tup{s} \in \sources{n}{p}{\tup{t}} \cdot }\\
	\m{\fa{f}{s} \in \gfasA{g_p} \Rightarrow [\fa{f}{s}]_{g_p} \in \sources{n}{p}{[\fa{f}{t}]_{g_n}}}
\item The \textbf{second part} also remains as in the sequential case, ensuring transitive propagation of equality information:\\
	\m{\forall p \in \preds{n},\fa{f}{t} \in \gfasA{g_n} \cup \rgfas{n}, \tup{s} \in \sources{n}{p}{\tup{t}} \cdot}\\
	\m{\fa{f}{s} \in \gfasA{g_p} \cup \rgfas{p}}
\item The \textbf{third part} is slightly modified - we cannot have an rgfa only if \emph{all} predecessors have a gfa:\\
	\m{\forall \tup{t} \in g_n,f \in \Fs{\sig} \cdot}\\
	\m{(\forall p \in \preds{n} \cdot \exists \tup{s} \in \sources{n}{p}{\tup{t}}\cdot \fa{f}{s} \in \gfasA{g_p}) \Rightarrow \fa{f}{t} \notin \rgfas{n}} 
\end{enumerate}
\caption{Weak join local graph based source invariant}
\label{weak_join_source_invariant}
\end{figure}
All the conditions are still local, and neither joinee can affect the other.

\subsubsection{The propagation invariant}
Remember the non-local \textbf{propagation invariant}:\\
\m{\forall n \in \cfg,t \in \terms{g_n}, s \in \Ts{\sig} \cdot} \\
\m{s=t \in \sqcup_F(\eqs{g_n},\s{\eqs{g_p}}_{p \in \preds{n}}) \Rightarrow s \in \terms{[t]_{g_n}}}\\
For join nodes we will need some extra state in order to phrase these in terms of graphs, as hinted at before.

\noindent
The following example shows a join with the propagation invariant broken:
\begin{figure}[H]
\begin{tikzpicture}
  \node (1)  {};
%%%%%%%%%%%%%%%%%%%%%%%%%%%%%%%%%%%%%%%%%%%%%%%%%%%%%%%%%%%%%%
  \node[gttn] (11)  [above right= 2.5cm and 0cm of 1] {$()$};
	\node[gl]   (11l) [below = 0 of 11]   {\m{p_0}};

	\node[gtn]  (12) [above left  = 1cm and 1cm of 11] {\s{a}};
	\node[gtn]  (14) [above right = 1cm and 1cm of 11] {\s{b}};
	
	\draw[gfa] (12)  to node[el]  {\m{a}} (11);
	\draw[gfa] (14)  to node[el]  {\m{b}} (11);

	\node[gttn] (15)  [above = 1cm of 12]    {\m{(a)}};
	\node[gttn,ultra thick] (17)  [above = 1cm of 14]    {\m{(b)}};

	\draw[sgtt] (15) to node[el] {0} (12);
	\draw[sgtt] (17) to node[el] {0} (14);

	\node[gtn,ultra thick]  (18) [above = 3cm of 11] {\tiny$\faB{f}{a}{b}$};
	\draw[gfa]  (18) to node[el] {\m{f}} (15);
	\draw[gfa,ultra thick]  (18) to node[el,anchor=west] {\m{f}} (17);

%%%%%%%%%%%%%%%%%%%%%%%%%%%%%%%%%%%%%%%%%%%%%%%%%%%%%%%%%%%%%%
	\node[gttn] (21)  [below right= 3.0cm and 0cm of 1] {$()$};
	\node[gl]   (21l) [below = 0 of 21]   {\m{p_1}};

  \node[gtn]  (22) [above  = 1.0cm of 21] {\s{a,b}};
	
  \draw[gfa] (22) to[bend right] node[el,anchor=east] {\m{a}} (21);
  \draw[gfa] (22) to[bend left]  node[el,anchor=west] {\m{b}} (21);

	\node[gttn,ultra thick] (25)  [above = 1cm of 22]    {\m{(\s{a,b})}};

	\draw[sgtt] (25) to node[el] {0} (22);

  \node[gtn,ultra thick]  (28) [above = 1cm of 25] {\tiny$\faB{f}{a}{b}$};
	\draw[gfa,ultra thick]  (28) to node[el] {\m{f}} (25);
%%%%%%%%%%%%%%%%%%%%%%%%%%%%%%%%%%%%%%%%%%%%%%%%%%%%%%%%%%%%%%

  \node[gttn] (31)  [right = 6cm of 1] {$()$};
	\node[gl]   (31l) [below = 0 of 31]   {\m{n}};

  \node[gtn]  (32) [above left  = 1cm and 1cm of 31] {\s{a}};
  \node[gtn]  (34) [above right = 1cm and 1cm of 31] {\s{b}};

  \draw[gfa]  (32)  to node[el,anchor=east]  {\m{a}} (31);
  \draw[gfa]  (34)  to node[el,anchor=west]  {\m{b}} (31);

	\node[gttn] (35)  [above = 1cm of 32]    {\m{(a)}};
	\node[gttn] (37)  [above = 1cm of 34]    {\m{(b)}};

	\draw[sgtt] (35) to node[el] {0} (32);
	\draw[sgtt] (37) to node[el] {0} (34);

	\node[gtn,ultra thick]  (38) [above right = 1cm and 1cm of 35] {$\m{f(a)}$};
	\draw[gfa]  (38) to node[el,anchor=south east] {\m{f}} (35);
	\draw[mgfa,ultra thick,dashed] (38) to node[ml,anchor=south west] {\m{f}} (37);

%%%%%%%%%%%%%%%%%%%%%%%%%%%%%%%%%%%%%%%%%%%%%%%%%%%%%%%%%%%%%%
	\node (15a) [right = 0.5cm of 15] {};
	\node (35b) [above left = 0.2cm and 1.5cm of 35] {};
	\node (35c) [below left = 0.2cm and 1.5cm of 35] {};
	\node (37b) [above left = 0.2cm and 1.5cm of 37] {};
	\node (37c) [below left = 0.2cm and 1.5cm of 37] {};
	\draw[se,ultra thick] (37) to[out=180,in=0] (37b) to[out=180,in=0] (17);
	\draw[se,ultra thick] (37) to[out=180,in=0] (37c) to[out=180,in=0] (25);

	\node (18a) [right = 0.5cm of 18] {};
	\node (38b) [above left = 0.2cm and 1.5cm of 38] {};
	\node (38c) [below left = 0.2cm and 1.5cm of 38] {};
	\draw[se,ultra thick] (38) to[out=180,in=0] (38b) to[out=180,in=0] (18a) to[out=180,in=0] (18);
	\draw[se,ultra thick] (38) to[out=180,in=0] (38c) to[out=180,in=0] (28);

\draw[draw=none, use as bounding box] (current bounding box.north west) rectangle (current bounding box.south east);

\begin{pgfinterruptboundingbox}
	\draw[separator] (3.5cm,-3.3cm) to (3.5cm,6.5cm);
\end{pgfinterruptboundingbox}

\end{tikzpicture}

\caption{
Weak join \GFA{} completeness - broken
}
\label{snippet3.18a_graph}
\end{figure}
Here we see that if a \GFA{} with the same function symbol exists in the source (here source for \m{f(a)}) all joinees, 
where the \GFA{} tuples in all joinees are the source for a tuple in the join node (here \m{[(b)]_{g_n}}), then we must have that \GFA{} also in the join. Again this is a generalization of the condition for sequential nodes.

\subsubsection{Top down weak join completeness}
Here we discuss a top down approach to ensuring weak join completeness.

\subsubsection*{Source pairs}
The key notion here is that of ordered pairs of EC nodes, one from each predecessor.
We will call these pairs \emph{source-pairs} and will denote them as \gta{s_0}{s_1} for a pair of EC nodes \m{s_0 \in g_{p_0},s_1 \in g_{p_1}}\\
We will also overload the notation for terms (as opposed to EC nodes) in order to reduce notations where there is no ambiguity - so e.g. \gta{b}{c} would mean the ordered pair \m{([b]_{p_0},[c]_{p_1})}.\\
We write \m{\gta{t_0}{t_1} \in \sourcesB{n}{t}} iff, for each i, \\
\m{[t_i]_{g_{p_i}} \in \sources{n}{p_i}{[t]_{g_n}}}.

As we have seen before, each such pair can be the source of at most one EC node at the join, 
but not all such pairs can be the source of \emph{any} node at the join - for example, \gta{f(\s{a,b})}{\s{a,b}} 
would not be the source of any node in \ref{snippet3.18a_graph} - we would then say that this pair is \emph{infeasible} - it does not represent any term at the join. \\
A pair \gta{s_0}{s_1} is obviously feasible if \m{\terms{s_0} \cap \terms{s_1} \neq \emptyset},
but is also feasible if \m{\exists t \in g_n \cdot \gta{s_0}{s_1} \in \sourcesB{n}{t}}.\\
This means that if we were to \lstinline{assume a=b} at the join then \gta{a}{b} would become feasible, 
so that feasibility can change with added equalities, but only in a monotonic manner.\\
Feasibility also satisfies congruence closure, as we will detail formally.\\
We denote the set of feasible source pairs at a join node n as \fsps{n}.\\
We extend the above source-pairs to source-pairs of tuples, where a tuple-source-pair is feasible iff each pair of its elements is feasible - formally:\\
\m{\gta{\tup{s}}{\tup{t}} \in \fsps{n} \triangleq \bigwedge\limits_i \gta{s_i}{t_i} \in \fsps{n}}\\
Now we can define a feasible source-pair as follows:
\begin{figure}[H]
\begin{enumerate}
	\item \m{\forall t \in g_n, \gta{s_0}{s_1} \in \sourcesB{n}{t} \cdot}\\
		\m{\gta{s_0}{s_1} \in \fsps{n}}
	\item \m{\forall t \in \gta{\tup{s_0}}{\tup{s_1}} \in \fsps{n}, \fa{f}{s_0} \in \gtas{p_0}, \fa{f}{s_1} \in \gtas{p_1}, \cdot}\\
		\m{\gta{\fa{f}{s_0}}{\fa{f}{s_1}} \in \fsps{n}}
%\m{\feasible{\gta{s_0}{s_1}} \triangleq }\\
%\m{(\exists t \in g_n \cdot \gta{s_0}{s_1} \in \sourcesB{n}{t})}\\
%\m{ \lor (\exists \fa{f}{u_0} \in s_0,\fa{f}{u_1} \in s_1 \cdot \feasible{\gta{\tup{u_0}}{\tup{u_1}}})}
\end{enumerate}
\caption{Weak join - feasible source pairs}
\end{figure}

That is, a pair is feasible if it is already the source of some term at the join (which covers the above case of \lstinline{assuming a=c} at the \m{n} in ~\ref{snippet3.17b_graph}), or if there is a matching feasible pair of \GFAs{} - by congruence closure.\\
Using this definition we can now phrase \GFA{} completeness for the weak join:
\begin{figure}[H]
\m{\forall t \in g_n, \gta{s_0}{s_1} \in \sourcesB{n}{t}, \fa{f}{v_0} \in s_0, \fa{f}{v_1} \in s_1 \cdot}\\
\m{\gta{\tup{v_0}}{\tup{v_1}} \in \fsps{n} \Rightarrow \exists \fa{f}{u} \in t \cdot \gta{\tup{v_0}}{\tup{v_1}} \in \sourcesB{n}{\tup{u}}}
\caption{Weak join - non-local propagation invariant}
\end{figure}
However, as in the sequential case, we want a more operational bottom-up way of determining which nodes are actually needed, and we do not want to consider all \m{\size{g_{p_0}} \times \size{g_{p_1}}} source-pairs regardless of the terms at \m{g_n} - 
we want the overall complexity to depend on the final result, and so a source-pair that provably cannot contribute to the result should not be considered.\\
We will maintain a set of \emph{potentially relevant} source pairs per join node - denoted by \prgtas{n} for the join node \m{n} (the actual concrete representation in the algorithm will be discussed later).\\
Now we define which source-pairs we consider potentially relevant, similar to relevant terms in the sequential case:
\begin{figure}[H]
\begin{enumerate}
	\item For each term-EC-node at \m{g_n}, each pair of sources should be considered:\\
	\m{\forall t \in g_n, \gta{s_0}{s_1} \in \sourcesB{n}{t} \cdot \gta{s_0}{s_1} \in \prgtas{n}}
	\item For each source-pair with a matching \gfa pair, all source-pairs of the matching tuple must be considered:\\
	\m{\forall \gta{s_0}{s_1} \in \prgtas{n}, \fa{f}{u_0} \in s_0, \fa{f}{u_1} \in s_1 \cdot \gta{\tup{u_0}}{\tup{u_1}} \in \prgtas{n}}
\end{enumerate}
\caption{Weak join - relevant potential source-pairs}
\label{wj_relevant_potential_source_pairs}
\end{figure}
For the strong join we will need a third condition that makes a source pairs relevant.\\
Note that this definition means that checking for feasibility is closed in \prgtas{n} (it is sufficient to check feasibility only on source-pairs in \gtas{n}).\\
A source pair that is both relevant and feasible is still not necessarily the source of node at the join - for example:
\begin{figure}[H]
\begin{tikzpicture}
  \node (1)  {};
%%%%%%%%%%%%%%%%%%%%%%%%%%%%%%%%%%%%%%%%%%%%%%%%%%%%%%%%%%%%%%
  \node[gttn] (11)  [above right= 2.5cm and 0cm of 1] {$()$};
	\node[gl]   (11l) [below = 0 of 11]   {\m{p_0}};

	\node[gtn]              (12) [above left  = 1cm and 1cm of 11] {\s{a}};
	\node[gtn,ultra thick]  (13) [above       =       0.9cm of 11] {\s{b}};
	\node[gtn,ultra thick]  (14) [above right = 1cm and 1cm of 11] {\s{c}};
	
	\draw[gfa]             (12)  to node[el,anchor=east]  {\m{a}} (11);
	\draw[gfa,ultra thick] (13)  to node[el,anchor=west]  {\m{\mathbf{b}}} (11);
	\draw[gfa,ultra thick] (14)  to node[el,anchor=west]  {\m{\mathbf{c}}} (11);

	\node[gttn]             (12a)  [above = 1cm of 12]    {\m{(a)}};
	\node[gttn,ultra thick] (14a)  [above = 1cm of 14]    {\m{(b,c)}};

	\draw[sgtt]             (12a) to node[el,anchor=east] {0} (12);
	\draw[sgtt,ultra thick] (14a) to[out=-90,in=90] node[el,anchor=north] {\textbf{0}} (13);
	\draw[sgtt,ultra thick] (14a) to node[el,anchor=north west] {\textbf{1}} (14);

	\node[gtn,ultra thick]  (15) [above = 3.5cm of 11] {\tiny$\svb{f(a)}{g(b,c)}$};
	\draw[gfa]              (15) to node[el,anchor=south east] {\m{f}} (12a);
	\draw[gfa,ultra thick]  (15) to node[el,anchor=south west] {\m{\mathbf{g}}} (14a);

%%%%%%%%%%%%%%%%%%%%%%%%%%%%%%%%%%%%%%%%%%%%%%%%%%%%%%%%%%%%%%
	\node[gttn] (21)  [below right= 3.0cm and 0cm of 1] {$()$};
	\node[gl]   (21l) [below = 0 of 21]   {\m{p_1}};

  \node[gtn,ultra thick]  (22) [above  = 1.0cm of 21] {\s{a,b}};
	
  \draw[gfa]             (22) to[bend right] node[el,anchor=east] {\m{a}} (21);
  \draw[gfa,ultra thick] (22) to[bend left]  node[el,anchor=west] {\m{b}} (21);

	\node[gttn]             (22a)  [above = 1cm of 22]    {\m{(\s{a,b})}};
	\node[gttn,ultra thick] (23a)  [right = 1cm of 22a]   {\m{(\s{a,b},\s{a,b})}};

	\draw[sgtt,ultra thick] (22a) to node[el,anchor=east] {\textbf{0}} (22);
	\draw[sgtt,ultra thick] (23a) to[out=-90,in=90] node[el,anchor=north west] {\textbf{0,1}} (22);

  \node[gtn,ultra thick] (25) [above = 1cm of 22a] {\tiny$\svb{f(\s{a,b})}{g(\s{a,b},\s{a,b})}$};
	\draw[gfa,ultra thick] (25) to node[el,anchor=west] {\m{\mathbf{f}}} (22a);
	\draw[gfa,ultra thick] (25) to node[el,anchor=west] {\m{\mathbf{g}}} (23a);
%%%%%%%%%%%%%%%%%%%%%%%%%%%%%%%%%%%%%%%%%%%%%%%%%%%%%%%%%%%%%%

  \node[gttn] (31)  [right = 6cm of 1] {$()$};
	\node[gl]   (31l) [below = 0 of 31]   {\m{n}};

  \node[gtn]  (32) [above = 1cm of 31] {\s{a}};

  \draw[gfa]  (32)  to node[el,anchor=east]  {\m{a}} (31);

	\node[gttn] (32a)  [above = 1cm of 32]    {\m{(a)}};

	\draw[sgtt] (32a) to node[el] {0} (32);

	\node[gtn]  (35) [above = 1cm of 32a] {$\m{f(a)}$};
	\draw[gfa]  (35) to node[el,anchor=east] {\m{f}} (32a);

%%%%%%%%%%%%%%%%%%%%%%%%%%%%%%%%%%%%%%%%%%%%%%%%%%%%%%%%%%%%%%
	\node (18a) [right = 0.5cm of 18] {};
	\node (35b) [above left = 0.2cm and 1.5cm of 35] {};
	\node (35c) [below left = 0.2cm and 1.5cm of 35] {};
	\draw[se] (35) to (35b) to (15);
	\draw[se] (35) to (35c) to (25);

\draw[draw=none, use as bounding box] (current bounding box.north west) rectangle (current bounding box.south east);

\begin{pgfinterruptboundingbox}
	\draw[separator] (4.5cm,-3.3cm) to (4.5cm,6.5cm);
\end{pgfinterruptboundingbox}
\end{tikzpicture}
\caption{
Join \GFA{} completeness\\
relevant and feasible source-pairs
}
\label{snippet3.19_graph2}
\end{figure}
Here the source-pair \gta{b}{\s{a,b}} is both feasible and relevant, but should not appear in the join because the source pair \gta{c}{\s{a,b}} it shares in its only relevant super-tuple with is not relevant.\\
This implies that an algorithm that only adds necessary (by fragment completeness) nodes in our setting would have to do either of the following:
\begin{itemize}
	\item Add an EC-node at the join for each feasible relevant source-pair and then prune those that do not reach a pre-existing EC-node,
	similarly, we could avoid marking source-pairs as feasible altogether and simply create an EC-node for each relevant source-pair and trim those that remain infeasible. 
	\item Add a second pass of marking only feasible relevant source-pairs top-down and then adding only EC-nodes (bottom up) for those source-pairs that were marked in the second stage
\end{itemize}
Asymptotically there is no difference between the two approaches, as in both cases all the traversed source-pairs have already been traversed at least once, and also the state that we need to keep after a join operation in order to keep incremental performance bounds do not change.\\
We show here the encoding that would allow us to perform the second option, using \gtas{n} for the set of source-pairs marked in the second phase:
\begin{figure}[H]
\begin{enumerate}
	\item For each term-EC-node at \m{g_n}, each pair of sources that is feasible should be considered:\\
	\m{\forall t \in g_n, \gta{s_0}{s_1} \in \sourcesB{n}{t} \cdot}\\
	\m{\gta{s_0}{s_1} \in \gtas{n}}
	\item Downward closure:\\
	\m{\forall \gta{s_0}{s_1} \in \gtas{n}, \fa{f}{u_0} \in s_0, \fa{f}{u_1} \in s_1 \cdot }\\
	\m{\gta{\tup{u_0}}{\tup{u_1}} \in \fsps{n} \Rightarrow \gta{\tup{u_0}}{\tup{u_1}} \in \gtas{n}}
	\item Bottom up node  addition:\\
	\m{\forall \gta{s_0}{s_1} \in \gtas{n}, \fa{f}{u_0} \in s_0, \fa{f}{u_1} \in s_1, \tup{t} \in \sourcesInvB{n}{\gta{\tup{u_0}}{\tup{u_1}}} \cdot }\\
	\m{\fa{f}{t} \in \gfasA{n}}
\end{enumerate}
\caption{Weak join - relevant source-pairs}
\label{wj_relevant_source_pairs}
\end{figure}
For completeness we give here also the formulation for feasibility restricted to relevant terms:
\begin{figure}[H]
\begin{itemize}
	\item \m{\forall t \in g_n, s_0,s_1, \gta{s_0}{s_1} \in \sourcesB{n}{t} \cdot}\\
		\m{\gta{s_0}{s_1} \in \fsps{n}}
%	\item \m{\forall t \in \gta{\tup{s_0}}{\tup{s_1}} \in \fsps{n}, \fa{f}{s_0} \in \gfas{p_0}, \fa{f}{s_1} \in \gfas{p_1}, \cdot}\\
%		\m{\gta{\fa{f}{s_0}}{\fa{f}{s_1}} \in \prgtas{n} \Rightarrow \gta{\fa{f}{s_0}}{\fa{f}{s_1}} \in \fsps{n}}
\end{itemize}
%\m{\feasible{\gta{s_0}{s_1}} \triangleq }\\
%\m{(\exists t \in g_n \cdot \gta{s_0}{s_1} \in \sourcesB{n}{t})}\\
%\m{ \lor ((\exists \fa{f}{u_0} \in s_0,\fa{f}{u_1} \in s_1 \cdot \feasible{\gta{\tup{u_0}}{\tup{u_1}}}) \land \gta{s_0}{s_1} \in \prgtas{n})}
\caption{Weak join - relevant feasible source pairs}
\label{wj_relevant_feasible_source_pairs}
\end{figure}
Our weak propagation invariant will be the least fixed point of the conjunction of \ref{wj_relevant_potential_source_pairs}, 
\ref{wj_relevant_source_pairs} and \ref{wj_relevant_feasible_source_pairs}.\\
We need a least fixed point for cases such as:\\
\m{\s{a=f(b),b=g(b)} \sqcup \s{a=f(c),c=g(c)}}\\
where adding the term a to the join could mark\gta{b}{c}{}as feasible if we take a non-minimal fixed point.\\
As our algorithm will satisfy the invariant by monotonically adding elements (that is, expanding \prgtas{n},\gtas{n}, \m{g_n} etc),
we will naturally reach the least fixed point (we will show termination, soundness and completeness in the appendix).


%\m{O(\size{g_{p_0}}\times\size{g_{p_1}} + \size{g_n}\times(\size{g_{p_0}} + \size{g_{p_1}}))} 
%simply by counting all possible pairs of both kinds 

%%


Each node in the graph is labeled with a set of pairwise disjoint \newdef{ground function application equivalence class}- a gfa. Each such \GFA{} is of the form $\fa{f}{n}$ where \m{f} is a function symbol in the signature and \tup{n} is a tuple of nodes in the graph. 
Essentially each \GFA{} represents an AEC of the equivalence relation defined by the graph.
Each \GFA{} appears at most once in the graph - this ensures that the graph is a partition of terms.\\
We treat the graph as a set of nodes, and each node as a non-empty set of gfas.
The graph includes also undirected edges between nodes that represent inequality, at most one per pair of nodes.\\
The set of terms represented in the graph and at a node, \GFA{} are defined as the least fixed point of the following: \\
For a gfa:            \m{\fa{g}{s} \in \terms{\fa{f}{m}} \equivdef f \equiv g \land \tup{s} \in \terms{\tup{m}}}\\
For a tuple of nodes: \m{\tup{s} \in \terms{\tup{m}} \equivdef \bigwedge\limits_i s_i \in \terms{m_i}}\\
For a node:  \m{t \in \terms{n} \equivdef \exists \fa{f}{m} \in n \cdot t \in \terms{\fa{f}{m}}}\\
For a graph: \m{t \in \terms{g} \equivdef \exists n \in g \cdot t \in \terms{n}}

We denote by \m{[t]_g} the node in the graph g that represents the term t , when \m{t \in \terms{g}}:\\
\m{[t]_g = n} iff \m{t \in \terms{n} \land n \in g}
\begin{theorem}
The above definition for \m{[t]_g} is well defined if \m{t \in \terms{g}} - that is\\
\m{\forall t \in \terms{g} \cdot \forall m,n \in g \cdot t \in \terms{m} \land t \in \terms{n} \Rightarrow m=n )}\\
\textbf{Proof:}\\
We show this by induction on the depth of t: for a constant term c, of (depth 1), \\
\m{c \in \terms{n}} iff \m{c() \in n}, and similarly for m. As nodes are pairwise disjoint sets of \gfas this means \m{m=n}.\\
For depth \m{k+1} and term \m{t=\fa{f}{s}}, we know from i.h. that \\
\m{\forall m',n' \in g,i \cdot (s_i \in m' \land s_i \in n' \Rightarrow m'=n')}.
If \m{t \in \terms{m} \land t \in \terms{n}} then, by definition there are \tup{m'},\tup{n'} s.t. \m{\fa{f}{m'} \in m \land \fa{f}{n'} \in n} and\\
\m{\bigwedge\limits_i t_i \in \terms{m'_i}} and
\m{\bigwedge\limits_i t_i \in \terms{n'_i}}.\\
By the induction hypothesis this implies that \m{\tup{m'} = \tup{n'}} (tuples are equal iff they are of the same length and equal for all indices).\\
This implies \m{\fa{f}{m'}=\fa{f}{n'}} (\gfas are equal iff the function symbol and tuple are equal).\\
By the pairwise disjointness of nodes we get \m{m=n}.\\
\QED
\end{theorem}

\textbf{Definitions:}\\
\m{g \models s=t} when \m{s,t \in \terms{g}} and \m{[t]_g=[s]_g}, extended in the standard way for tuples.\\
\m{g \models s \neq t} iff \m{s,t \in \terms{g}} and there is an edge in g between \m{[t]_g} and \m{[s]_g}.\\
\m{g \models \emptyClause} iff, for some \m{l,r}, \m{g \models l=r} and \m{g \models l \neq r}.\\
To represent an EC graph as a minimal set of equations we need a representative \gfa for each node, 
and then the set of equalities is exactly equalities between the representatives of \gfas at each node.\\
We choose always the representative of least weight, and among them by an arbitrary order on terms (unrelated to the one for superposition) - formally:\\
\m{\eqs{g} \triangleq \cup \s{\eqs{n} \mid n \in g}}\\
\m{\eqs{n} \triangleq \s{s=t \mid s,t \in gfareps(n) \land t = min_{<_{rep}}(gfareps(n) \setminus \s{s})}}\\
\m{gfareps(n) \triangleq \s{gfarep(\fa{f}{s}) \mid \fa{f}{s} \in n}}\\
\m{gfarep(\fa{f}{s}) \triangleq f(rep(\tup{s})}\\
\m{gfarep(\fa{f}{}) \triangleq \fa{f}{}}\\
\m{rep(n) \triangleq min_{weight,<_{rep}}(gfareps(n))} \\
\m{rep(\tup{n})_i \triangleq rep(n_i)} \\
\m{weight(\fa{f}{t}) \triangleq 1+\Sigma_i weight(t_i)}\\
Where \m{min_{weight,<_{rep}}} is the minimum in lexicographic order, first on weight then on the arbitrary term order\\
This is well defined as the weight ordering implies that a term is always smaller than super-terms.

\bigskip
\noindent
\textbf{Properties:}\\
\textbf{Consistency:} The graph is called consistent iff \m{g \not\models \emptyClause}.\\
\textbf{Sub-term closure:} The graph is by definition sub-term closed, as a term is in the graph only if all direct sub-terms are.\\
\textbf{Congruence closure:} The graph is by definition congruence closed, by the uniqueness of 

\subsection{Operations}
\textbf{Constructor:} The constructor takes a set of axioms and \lstinline{assumes} them.\\
\textbf{Assume:} Assuming an axiom \m{s \bowtie t} means \lstinline{add}ing \m{t,s} to \m{g} and then:\\
For \m{s \neq t} we add an (undirected) edge between \m{g(s)} and \m{g(t)}. \\
For \m{s = t} we proceed as follows:
\begin{lstlisting}
assume( $\m{s=t}$ )
	ns = add(s)
	nt = add(t)
	mergeSet($\s{ns,nt}$)
	
mergeSet(S0 : Set[Node])
	mergeOnce(S0)
	while not all nodes are pairwise disjoint
		S = choose all nodes that share a shared gfa
			mergeOnce(S)
	
mergeOnce(S : set of node)
	n = $\cup$ S 
	add n to nodes
	foreach node $\m{m \in g}$
		replace each $\m{\fa{f}{s} \in m}$
			with $\m{\fa{f}{s}[S \mapsto n]}$
	replace each disequality edge $\m{(m,m')}$
		with $\m{(m,m')[S \mapsto n]}$
	remove S from nodes
\end{lstlisting}
We use the syntax 
\m{n[m \mapsto l]} 
for nodes \m{m,n,l} to define substitution similar to term substitution - formally:\\
\m{n[m \mapsto l] \triangleq \ite{m \equiv n}{l}{n}}\\
The set extension:\\
\m{n[S \mapsto l] \triangleq \ite{n \in S}{l}{n}}\\
And tuples:\\
\m{\tup{s}[S \mapsto l]_i \triangleq s_i[S \mapsto l]}\\
\lstinline{mergeOnce} performs essentially congruence closure, as seen by the loop condition.\\
The method \lstinline|mergeSet($\s{s=t}$)| ensures the following properties, for a pre-state graph \m{g} and a post state graph{g'} where \m{s,t \in \terms{g}}:\\
\m{\terms{g} \subseteq \terms{g'}} - monotonic in terms\\
\m{\forall u,v \in \Ts{\sig{}} \cdot \eqs{g'} \models u \bowtie v \Leftrightarrow \eqs{g} \cup \s{s=t} \models u=v} - this essentially means that no information is added to the graph except for \m{s=t}.\\
It is easy to see that also the operation reduces the number of equivalence classes of \m{\size{\ECs{\eqs{g}}}} by at least one, and can only reduce the number of \GFAs{} - hence the complexity of the graph can only decrease.\\
Also, no terms are removed, only added. \\
We will later sometimes want to keep a mapping between a node in \m{g} and the node it was merged to in \m{g'}, in order to communicate a \emph{changeset} in the graph for incremental updates.\\
\textbf{Add term:} Adding a term \m{t} to the graph is defined recursively as follows:
\begin{lstlisting}
add( $\fa{f}{s}$ : Term ) : Node
	m = add($\tup{s}$) //tuple extension
	if $\m{\fa{f}{m} \in n}$ for some $\m{n \in g}$
		return n
	else
		n = add new node $\s{\fa{f}{m}}$
		return n
\end{lstlisting}
The above code maintains the invariant that nodes are pairwise disjoint.\\
By definition also \lstinline{$\fa{f}{s} \in $add($\fa{f}{s}$)} at the post-state.

\subsection{Satisfiability}
We can use the above graph to check the satisfiability of a set of unit ground (in)equalities (axioms) by \lstinline{assuming} all axioms and checking if the graph is consistent.\\
However, we want to use only a subset of the axioms that are guaranteed to suffice to show inconsistency iff the axioms are inconsistent.\\
We know that any set of only positive equalities is consistent, so we start with all inequalities and gradually add equalities on sub-terms until we get an inconsistency, or there are no more axioms to add:
As we apply the congruence closure and axiom addition until saturation, the set of equalities represented by the graph is saturated with respect to the congruence closure calculus for the axioms in the set \lstinline{s}.
\begin{theorem}
Theorem: The graph \m{g} and set \m{s} constructed for the initial set of axioms \m{Ax} satisfies\\ 
\m{\forall l \in \terms{g} \cdot ((r \in \terms{g} \land g \models l=r) \Leftrightarrow s \models l=r)}

\noindent
\textbf{Monotonicity lemma:} \\
At each step of \lstinline|mergeOnce|, where \m{g} is the graph at the pre-state and \m{g'} is the graph at the post-state,
the following monotonicity property holds:\\
Except for the last statement that removes the merged nodes, it holds that the set of terms represented by a node in \m{g'} is a superset of the set of terms represented by the same node in \m{g} - by simple induction on the depth of a term.\\
For the last statement, each term represented by a deleted node is also represented by the merged node.
Hence, we can immediately deduce that if \m{g \models s=t} then \m{g' \models s=t}.

\noindent
\textbf{Proof of $\Leftarrow$ (completeness of \lstinline|assume|):} \\
We will show that the set of equalities implied by the graph is saturated with respect to transitivity and congruence closure and includes all the equalities in \m{S}. \\
For an equality \m{l=r} s.t. \m{S \models l=r} and \m{l \in \terms{g}} we proceed by strong induction on the depth of the derivation tree in \m{\mathbf{CC_I}} (which is complete for non-tautological consequences), 
where the induction hypothesis is that for depth k, if \m{S \vdash_{\mathbf{CC_I}} s=t} and \m{s \in \terms{g}} then \m{t \in \terms{g}} and \m{g \models s=t} \\
\textbf{Axioms:} Each axiom \m{l=r} in \m{s} holds in \m{g} because each such axiom is \lstinline{assume}d in \m{g}, and by the definition of \lstinline{assume}, after the \lstinline{assume} \m{[l]_g=[r]_g}, which is the definition of \m{g \models l=r},
and \lstinline|assume| is monotonic.\\
\textbf{Transitivity:} If \m{S \vdash_{\mathbf{CC_I}} l=t,t=r} for some t, by the induction hypothesis \m{t \in \terms{g}} and hence by i.h. \m{t \in \terms{g}} and \m{g \models l=t} and again by i.h. \m{r \in \terms{g}} and \m{g \models t=r} and then \m{[l]_g=[t]_g=[r]_g} hence \m{g \models l=r}\\
\textbf{Congruence Closure:} If \m{S \vdash_{\mathbf{CC_I}} \fa{f}{s}=\fa{f}{t}} by congruence closure and \\
\m{\fa{f}{s} \in \terms{g}} then \m{S \vdash_{\mathbf{CC_I}} \tup{s}=\tup{t}} and then by sub-term closure, \m{\tup{s} \in g} and by i.h. 
\m{\tup{t} \in g} and \m{g \models \tup{s}=\tup{t}}.\\
\fa{f}{s} is represented (by definition) by some \GFA{} \fa{f}{m}, which also, as \m{g \models \tup{s}=\tup{t}}, represents \fa{f}{t},
hence \m{g \models s=t}.\\
\QED

\textbf{Proof of $\Rightarrow$ (soundness of \lstinline|assume|):} \\
We show that for one application of \lstinline|assume(s=t)|, with the pre-state graph g and set of axioms A and the post-state graph \m{g'} and set of axioms \\
\m{A' = A \cup \s{s=t}}, if at the pre-state \m{\forall l,r \in \terms{g} \cdot g\models l=r \Rightarrow A \models l=r} then at the post-state \m{\forall l,r \in \terms{g'} \cdot g' \models l=r \Rightarrow A' \models l=r}:\\ 
If \m{l=r \equiv s=t} or \m{g \models l=r} then we are done, 
otherwise we look at intermediate states of the graph on a call to \lstinline|mergeOnce| for the \GFA{} \fa{f}{x} and set of nodes \m{S}:\\
Such an intermediate state graph q does not satisfy our uniqueness property, 
but we can still define \m{q \models l=r} as \m{\exists n \in q \cdot l,r \in \terms{n}} as this is still well defined.\\
The operation of \lstinline|mergeOnce| is obviously monotonic with respect to the set of equalities represented by the intermediate graph, 
and so we inspect the first instance of \lstinline|mergeOnce| where at the pre-state \m{q \not\models l=r} and at the post-state (\m{q'} being the post-state graph), \m{q' \models l=r}.\\
We denote the intermediate graph after adding \m{n} to the set of nodes but before replacing \GFAs{} as \m{q''}.\\
If \m{q'' \models l=r}, it means that there are nodes \m{n_l,n_r \in q} s.t. \\
\m{l \in \terms{n_l}, r \in \terms{n_r}} and \m{n_l \neq n_r} but for some \GFA{} \fa{f}{u}, \m{\fa{f}{u} \in n_l \cap n_r} 
(hence \m{n_l,n_r \in S}, S the set of nodes being merged).\\
As the set \terms{\fa{f}{u}} is non-empty there is some term \m{v \in \terms{\fa{f}{u}}} and so,
\m{q \models l=v,v=r} and hence \m{A' \models l=r} by transitivity.

Otherwise, \m{q'' \not\models l=r} but \m{q' \models l=r}, we look at the state of the graph before and after the replacement operation on each gfa, we now name w the graph before the first \GFA{} replacement that established \m{l=r} and \m{w'} the graph immediately after the replacement, s.t. \m{w \not\models l=r} and \m{w' \models l=r}.\\
The \m{\mathbf{terms}} function is different between \m{w} and \m{w'} as one node has been relabeled, so we use \m{\terms{n}} for \m{w} and \m{\termsp{n}} for \m{w'}. It is easily seen that \m{\terms{n} \subseteq \termsp{n}}.\\
We name the replaced \GFA{} \fa{h}{m} replaced by \fa{h}{m'} at the node \m{n_1}.\\
By definition and w.l.o.g there is a node \m{n_0 \in w} s.t. \m{l,r \in \termsp{n_0}} but \m{r \notin \terms{n_0}}.\\
We now show by strong induction on the maximal depth of l,r that\\
 \m{w' \models l=r \Rightarrow A' \models l=r}:\\
We name r as the term \fa{f}{y}.\\
We name the \GFA{} that represents r in \termsp{n_0} as \fa{f}{v}.\\
r cannot be a constant as all \GFAs{} replaced are of at least depth 2, so \terms{n_0} and \termsp{n_0} agree on constants.\\
Now we look at two cases:

If \m{l \in \terms{n_0}}, we look again at two cases:\\
If \fa{f}{v}=\fa{f}{m'} - that is, \fa{f}{v} replaced \fa{f}{m}, then \m{\terms{\fa{f}{m} \subseteq \termsp{\fa{f}{v}}}}, 
and furthermore \m{\forall i \cdot \terms{m_i} \subseteq \termsp{m_i} \subseteq \termsp{v_i}}.\\
and as \terms{\fa{f}{m}} is not empty, there is some term \m{\fa{f}{z} \in \terms{\fa{f}{m}}} s.t. \m{w \models l=\fa{f}{z}}.\\
By the induction hypothesis we also know that for each i, \m{A' \models z_i = y_i}, as the depth of \m{x_i} is less than that of r.\\
Hence, by congruence closure, \m{A' \models \fa{f}{y}=\fa{f}{z}} and by transitivity \m{A' \models l=r}.\\
If \fa{f}{v} is not a replaced node, a similar argument holds, as \\
\m{\terms{\fa{f}{v}} \subseteq \termsp{\fa{f}{v}}}.

If \m{l \notin \terms{n_0}}, and \m{l=\fa{g}{x}} and is then represented by the \GFA{} \m{\fa{g}{u}}.\\
If \m{\fa{g}{x}=\fa{f}{y}} (that is, \m{f\equiv g} and both l,r are represented by the same \GFA{} at \m{n_0} in \m{w'}), then again by the induction hypothesis and congruence closure we get \m{A' \models l=r}.\\
Otherwise, by monotonicity and non-emptiness of the \terms{} function, there is some \m{l' = \fa{g}{x'}} s.t. \m{\fa{g}{x'} \in \terms{\fa{g}{u}}} and hence \\
\m{l' \in \terms{n_0} \subseteq \termsp{n_0}}, and similarly \m{r' = \fa{f}{y'}} s.t. \m{\fa{f}{y'} \in \terms{\fa{f}{v}}} and \m{r' \in \terms{n_0} \subseteq \termsp{n_0}}. \\
Because \m{l',r' \in \terms{n_0}} we get \m{w \models l'=r'} and hence \m{A' \models l'=r'} by monotonicity. \\
By the induction hypothesis and congruence closure as before, we get \\
\m{A' \models l=l',r=r'} and hence by transitivity \m{A' \models l=r}.\\
\QED
\end{theorem}
%Note that the theorem does not hold for inequalities - for example, from \\
%\m{s = \s{f(a) \neq f(b)}} we will not derive \m{g \models a \neq b} - we will return to this point later.

\begin{theorem}
The graph g constructed for the initial set of axioms \m{Ax} satisfies 
\m{\forall l \in \terms{g},r \in \Ts{\Sigma} \cdot ((r \in \terms{g} \land g \models l=r) \Leftrightarrow Ax \models l=r)}

\noindent
\textbf{Proof:} \\
\textbf{Soundness: ($\Rightarrow$)}\\
For \m{l,r \in g \land  g \models l=r \Rightarrow Ax \models l=r} :\\
This stems directly from the soundness above as \m{s \subseteq Ax} and the logic is monotonic.
%By uniqueness of \gfas the graph is always a partition of the terms represented in it, so it always represents an equivalence relation. 
%An equality \m{g \models l=r} in the graph be generated by:\\
%\lstinline{add} Adding a new \gfa \fa{f}{m} for the term \fa{f}{s} to a singleton node \s{\fa{f}{m}} - so, for some \m{\tup{u},\tup{v}} where \m{\tup{u},\tup{v} \in \tup{m}}, the equality \m{l=\fa{f}{u}=\fa{f}{v}=r} was added - this is justified by congruence closure as it means \m{g \models \tup{u}=\tup{v}}.\\
%\bigskip
%\lstinline{assume}: The invariant of the loop in \lstinline{mergeSet} is that \\
%\m{\forall l,r \in g \land \exists n \cdot l,r \in m \Rightarrow Ax \models l=r}\\
%Initially, this holds as it holds for the input graph.\\
%After \lstinline{mergeOnce(S0)} it holds because \m{\exists s=t \in Ax \mid S0=\s{g(s),g(t)}}, 
%so if \m{g \models l=r} is established on a node merge in \lstinline{mergeOnce(S0)} it means that, at the prestate, \m{l \in \terms{g(s)},r \in \terms{g(t)}} and so \m{g \models l=s,r=t}, and hence \m{Ax models l=s,t=r} and by transitivity \m{Ax \models l=r}.
%If \m{g \models l=r} is established by replacing \gfas after merging \m{g(s),g(t)} to \m{n'}, 
%at the prestate \m{l=\fa{f}{u} \in \terms{m}, r=\fa{h}{v} \in \terms{n}}, \\
%and \m{\tup{u} \in \terms{\tup{mm}}, \\\tup{v} \in \terms{\tup{nn}}} so that \m{\fa{f}{mm} \in m,\fa{f}{nn} \in n}, at the post-state 
%\tup{mm,nn} are replaced by 
%\m{\tup{mm'}==\tup{mm}[\s{g(s),g(t)} \mapsto n']} and \\
%\m{\tup{nn'}==\tup{nn}[\s{g(s),g(t)} \mapsto n']} and \\
%where $\fa{f}{mm'} \equiv \fa{h}{nn}$ implying \m{f = h,mm'=nn'}.
%As \m{n'} is a new node, we get \\
%\m{\bigwedge\limits_i (mm_i==nn_i \lor mm_i,nn_i\in\s{g(s),g(t)})} \\
%Putting it all together we get \m{\bigwedge\limits_i (Ax \models u_i==v_i)} and so \m{Ax \models l=r}.
%\QED

%\noindent
%\textbf{Termination:} \\
%\lstinline{unsat} terminates as the set \m{Axioms \setminus s} is finite and strictly decreasing in each iteration, and the loop terminates when it is empty.\\
%\lstinline{mergeSet} terminates as the set of nodes of the graph strictly decreases with each iteration and is finite.
%\QED

\noindent
\textbf{Completeness:}\\
The completeness proof proceeds as above, the difference is that for an axiom \m{l=r}, if \m{l \in \terms{g}} then the termination condition of the loop ensures that \m{l=r \in S} and hence we can use the result above.\\
%For \m{l,r \in g \land Ax \models l=r \Rightarrow g \models l=r}:\\
%By contradiction, assume that for some \m{l,r \in g}, \m{Ax \models l=r}, but \m{[l]_g \neq [r]_g}.
%As the calculus \m{CC_I} is complete for (non-tautological) \terms{Ax}, assume that there is a derivation \m{D} in the calculus for \m{l=r}. From the shape of derivations we can immediately see that no disequalities take part in such a derivation.\\
%We select a conclusion \m{u=v} of a sub-derivation \m{D'} of \m{D} where \m{u \in g} and \m{g \not\models u=v}, and where \m{D'} is minimal in that sense - the conclusions of all sub-derivations of \m{D'} are either implied by \m{g} or both terms are not in \m{g}.\\
%By case analysis on the derivation at the root of \m{D'}:\\
%\textbf{Axiom:} If \m{u=v \in Ax} and \m{u \in g} then \m{u=v} was \lstinline{assumed} by the termination condition for \lstinline{unsat} and so \m{u=v \in s} and hence \m{[u]_g=[v]_g)} by the completeness above.\\
%\textbf{Transitivity:} If \m{u \in g} and \m{D'} includes proper sub-derivations for \m{u=t,t=v} for some \m{t}, then by the minimality of \m{u=v} we know \m{t \in g} and \m{g \models u=t} and similarly \m{v \in g} and \m{g \models t=v} - hence \m{[u]_g=[t]_g=[v]_g} and so \m{g \models u=v}.\\
%\textbf{Congruence Closure:} If \m{v=\fa{f}{t}} and \m{u=\fa{f}{s} \in g} and \m{D'} includes proper sub-derivations for \m{s_i=t_i} for each i, then by minimality and sub-term closure \m{g \models \tup{s}=\tup{t}} and hence \m{[\tup{s}]_g=[\tup{t}]_g} and so 
%\m{f([\tup{s}]_g)=f([\tup{t}]_g)} and by the pairwise disjointness of the nodes in the graph we know that \m{[u]_g \ni f([\tup{s}]_g)=f([\tup{t}]_g) \in [v]_g} which implies \m{[u]_g=[v]_g}.
\QED
\end{theorem}

\begin{theorem}
The graph \m{g} constructed for the initial set of axioms \m{Ax} satisfies 
\m{Ax \vdash_{\mathbf{CC}} l \neq r \Leftrightarrow g \models l \neq r}

Note that \m{Ax \vdash_{\mathbf{CC}} l \neq r} is not equivalent to \m{Ax \models l \neq r}, for example, from \m{f(a) \neq f(b)} we cannot derive \m{a \neq b}, but \m{\mathbf{CC}} is refutationally complete - that is:\\
\m{Ax \vdash_{\mathbf{CC}} \emptyClause \Leftrightarrow Ax \models \emptyClause }

\noindent
\textbf{Proof:}

\noindent
\textbf{Soundness:}\\
The proof is similar to the proof for equalities except that we do not have congruence closure - the transitivity axiom is the same for both polarities.\\
\QED

\noindent
\textbf{Completeness:}\\
Assuming \m{Ax \vdash_{\mathbf{CC}} l \neq r},  we have a derivation \m{D} for \m{l \neq r} from \m{Ax}.\\
As the only derivation is by transitivity from one equality and one disequality, and as all disequality axioms are included in \m{S}, 
we can show by induction on the derivation depth of \m{D} that \m{g \models l \neq r} using the above theorem for equalities to show that all relevant terms are represented in g and all hence all relevant equalities are implied by g.\\
\QED
\end{theorem}

\begin{theorem}[Refutational Completeness]

The graph \m{g} constructed for the initial set of axioms \m{Ax} satisfies \\
\m{Ax \models \emptyClause \Leftrightarrow g \models \emptyClause}

\noindent
\textbf{Proof:}\\
The empty clause is generated only by resolution, from the above two theorems we know that the premises for this resolution hold in the graph iff they are derivable from the axioms, and the graph is inconsistent iff such premises hold in the graph - hence by the refutational completeness of the congruence closure calculus the graph is inconsistent iff the \m{Ax} are.\\
\QED
\end{theorem}

This process allows us to consider only a subset of the axioms which is guaranteed to be sufficient for showing unsatisfiability -  we would want to apply the same idea for selective propagation of axioms in the program cfg.









































%\newpage
%
%Before discussing how we tackle this problem, we extend the definition of the \textbf{sources} function to tuples:\\
%\m{\sources{n}{p}{\tup{s}}_i \triangleq \sources{n}{p}{s_i}} \\
%sets :\\
%\m{\sources{n}{p}{S} \triangleq \bigcup\limits_{t \in S} \sources{n}{p}{t}} \\
%and paths:\\
%\m{\sourcesB{n}{t} \triangleq \s{t}}\\
%\m{\sourcesB{p.P.n}{t} \triangleq \sourcesB{p.P}{\sourcesB{n}{t}}}\\
%Where \m{n} is a singleton path and \m{p.P.n} is a path from \m{p} to \m{n}.\\
%The last extension is for a (not-essentially direct) predecessor:\\
%\m{\sources{n}{p}{t} \triangleq \bigcup\limits_{P \in \paths{p}{n}} \sourcesB{P}{t}}\\
%For direct predecessors this coincides with the previous definition by construction.

\subsubsection{Source completeness}
As we have seen in ~\ref{snippet3.16b}, if we rely only on EC-graphs for information propagation, we may need to add some terms to some EC-graphs in order to ensure that the necessary equality information can be requested.

As a first attempt at a (non-local) invariant, we might require that all source-chains are complete, by ensuring that on every path between cfg-nodes, for every equality chain along the path, all terms that participate in the equality chain appear in their corresponding cfg-nodes:
\begin{figure}[H]
\m{\forall n \in \cfg,p \in \predsto{n},p.P.q.Q.n \in \paths{p}{n},t \in \terms{g_n},s \in \terms{g_p}}\\
\m{r \in \Ts{\sig{}} \cdot
\clauseseq{p.P.q} \models s=r \land \clauseseq{q.Q.n} \models r=t \Rightarrow r \in \terms{g_q}}
\caption{Equality chain source completeness}
\label{equality_chain_source_completeness}
\end{figure}
However, satisfying this requirement may require us to add an infinite set of terms - for example:
\begin{figure}[H]
\begin{lstlisting}
$\node{s}:$
if ($\m{c_1}$)
	$\node{p_0}:$
	assume $\m{a=f(a)}$
	assume $\m{g(a)=b}$
else
	$\node{p_1}:$
$\node{n}:$
...
$\node{n_1}:$
assume $\m{a=f(a)}$
assert $\m{g(a)=b}$ //negated $\m{\textcolor{gray}{g(a) \neq b}}$
\end{lstlisting}
\caption{equality chain source completeness infinite}
\label{snippet3.16cx}
\end{figure}
Here, for the path \m{p_0.n.n_1}, and the EC-nodes \m{[a]_{n_1}, [a]_{p_0}}, in order to satisfy the above requirement 
we would need to add to \m{g_n} a node for \m{f^k(a)} for any k - hence we would need an infinite graph at \m{n}.\\
However, it is evident that in this case adding only \s{a,g(a)} to \m{n} would allow us to determine that we can propagate \m{\lnot c_1 \lor g(a)=b} to \m{n_1}, as there would be at least one source chain between 
\m{[a]_{p_0}, [a]_{n_1}} and between \\
\m{[g(a)]_{p_0}, [g(a)]_{n_1}}.
We formalize this requirement as follows:
\begin{figure}[H]
\m{\forall n \in \cfg,p \in \predsto{n},P \in \paths{p}{n},t \in \terms{g_n},s \in \terms{g_p} \cdot}\\
\m{\clauseseq{P} \models t=s \Rightarrow [s]_{g_p} \in \sourcesB{P}{[t]_{g_n}}}
\caption{Strong source completeness}
\label{strong_source_completeness}
\end{figure}
This means that if two terms are equivalent on a path, and appear on the graphs on both ends of the path, then each node on the path has a term equivalent to both terms, so that these terms form an equality chain ($\mathbf{sources}$ chain) from \m{[t]_{g_n}} to \m{[s]_{g_p}}.\\
We call this condition \emph{strong source completeness}.\\
However, this condition is also too strong - for example:
\begin{figure}[H]
\begin{lstlisting}
$\node{s}:$
if ($\m{c_1}$)
	$\node{p_0}:$
	assume $\m{f^{2m}(a)=f^m(a)}$
	assume $\m{g(a)=b}$
else
	$\node{p_1}:$
$\node{n}:$
...
$\node{n_1}:$
assume $\m{f^{2n}(a)=f^n(a)}$
assert $\m{g(a)=b}$ //negated $\m{\textcolor{gray}{g(a) \neq b}}$
\end{lstlisting}
\caption{equality chain source completeness quadratic}
\label{snippet3.16d}
\end{figure}
Here, for any k,l, \m{p_0.n.n_1 \models f^k(a)=f^l(a)}, but the minimal witness at \m{n} to \m{[f^m(a)]_{p_0}=[f^n(a)]_{n_1}} is \m{f^{mn}(a)} - that is, it is the minimal (in size, depth and sub-term relation) term \m{t} that when \lstinline{added} to \m{g_n} would satisfy \\
\m{[t]_n \in \sources{n_1}{n}{[f^n(a)]_{n_1}} \land [f^m(a)]_{p_0} \in \sources{n}{p_0}{[t]_n}}.

However, it is evident from this example that, without additional equations at \m{s,p_1,n}, there is no need for the above complete source chain as there are no further equalities provable at \m{n_1} with an interpolant in the fragment.
In this section we are only interested in completeness for the ground unit fragment, and so we will weaken our condition in order to get a more efficient formulation that is still sufficient to prove a weaker notion of source completeness.

%%%%%%%%%%%%%%%%%%%%%%%%%%%%%%%%%%%%%%%%%%%%%%%%%%%%%%%%%%%%%%%%%%%%%%%%%%%%%%%%%%%%%
%%%%%%%%%%%%%%%%%%%%%%%%%%%%%%%%%%%%%%%%%%%%%%%%%%%%%%%%%%%%%%%%%%%%%%%%%%%%%%%%%%%%%
%%%%%%%%%%%%%%%%%%%%%%%%%%%%%%%%%%%%%%%%%%%%%%%%%%%%%%%%%%%%%%%%%%%%%%%%%%%%%%%%%%%%%




\bigskip
\noindent
We 
However, as we have seen in example ~\ref{snippet3.5}, these conditions are too strong.
Instead, we want \emph{interpolation completeness} for the fragment of ground unit equalities:\\
\m{\forall n \in \cfg, s,t \in \terms{g_n} \cdot (n \models_{u} s = t \Rightarrow g_n \models s = t)}\\
This definition means that if there is a fragment interpolant \m{I_{u}} s.t. \m{I_n \models s = t} \\
and \m{s,t \in \terms{g_n}} then \m{g_n \models [s]_{g_n}=[t]_{g_n}}, and similarly for inequalities.

As we represent all information about ground (dis)equalities using \\
EC-graphs, and as we propagate information only between adjacent nodes, 
the node EC graphs form a fragment interpolant for the cfg.\\
We are looking,then, for an interpolant that is sufficient for refuting all assertions, but minimal in the time and space required to construct it incrementally.\\
As we have discussed regarding propagation criteria, we have chosen to propagate eagerly all equalities on sub-terms, 
and hence the interpolation completeness above implies the following condition:
\begin{figure}[H]
\m{\forall n \in \cfg, s \in \terms{g_n},t \in \Ts{\sig} \cdot}\\
\m{(n \models_{u} s = t \Rightarrow t \in [s]_{g_n})}
\caption{strong propagation completeness}
\label{strong_propagation_completeness}
\end{figure}
That is, if the fragment implies that a term in the graph is equal to another term, 
the other term will also be in the graph in the same EC.
Remember also that the graphs are by construction sub-term closed, and hence for any node \node{n}, 
if a term \m{t} is in \m{g_n} then we get essentially the strongest possible interpolant for the sub-term closure of \m{t} (that is, all fragment implied (dis)equalities on the sub-term closure). \\
As we have seen in ~\ref{snippet3.30}, it is not possible to satisfy the above condition for the strong join.

We will implement a weaker version that approximates the above, which might still seem wasteful, but it has the following advantages:
\begin{itemize}
	\item It is predictable for programmers - as opposed to ordering based criteria where it might be harder for users of the tool to predict exactly which equalities should be deduced at each program point (especially with the limitations we will introduce in the next chapter)
	\item It makes the performance of incremental updates easier to predict, as it is much less sensitive to the order in which equations are introduced
	\item It makes re-establishing completeness after local changes to nodes easier, as we can derive locally the actions needed to repair completeness after changes (described at the end of the section)
\end{itemize}
The \textbf{main point} is that  \emph{the only freedom we have in order to define and establish completeness is in choosing which terms appear in the EC graph of which node} - so the propagation algorithm will deal with determining which terms need to be added at which node.

In this section we will give an invariant for cfg nodes, that is local (only refers to the node and its immediate predecessors) that ensures global interpolation completeness. 
Our invariant will describe how information flows along the cfg, in both directions.\\
In terms of the actual performance of verification, we have a second dimension of freedom - the order of evaluation.
We have seen an example of this before in the algorithm in ~\ref{clause_import_global} - there we can either import the relevant clauses from transitive predecessors once per clause, or per each set of clauses, which reduces dramatically the number of cfg traversals. 
We will discuss the various possible orders of evaluation in the implementation chapter, 
and only comment in this chapter how the invariant can limit our choice of order of evaluation.

\subsubsection{Weak source completeness}
Our weaker condition will ensure interpolation completeness and additionally will be local.
Interpolation completeness is as follows:
\begin{figure}[H]
\m{\forall n \in \cfg,t \in \terms{g_n}, s \in \Ts{\sig} \cdot} \\
\m{n \models_{w} s=t \Rightarrow s \in \terms{[t]_{g_n}}}
\caption{Ground unit equality weak completeness}
\label{ground_unit_equality_weak_completeness}
\end{figure}
Meaning that each EC-node in each EC-graph of each cfg-node represents a complete equivalence class of the equivalence relation that contains all equations implied at that cfg-node by any interpolant of the fragment of weak equalities - we will explain the difference in detail in the following.

We will define our invariant locally in two parts, and later show that it implies the above interpolation completeness.

We will require one additional notion in order to define the local weak completeness condition on the \m{\mathbf{sources}} function:\\
We maintain at each \cfg node \node{n} a list of \emph{rejected \gfa} - \rgfas{n}.\\
An \rgfa is a \gfa \fa{f}{s} s.t. \m{\tup{s} \in g_n} but \m{\fa{f}{s} \notin g_n}, it is used to mark the fact that there is no provable (in the fragment) equality on any of the terms of \terms{\fa{f}{s}} at \node{n}.
By definition the sets \gfasA{g_n} and \rgfas{n} are disjoint.

The \emph{weak source invariant} is formally:
\begin{figure}[H]
\begin{enumerate}
	\item \m{\forall n \in \cfg, p \in \preds{n}, \fa{f}{t} \in \gfasA{g_n}, \tup{s} \in \sources{n}{p}{\tup{t}} \cdot }\\
		\m{\fa{f}{s} \in \gfasA{g_p} \Rightarrow [\fa{f}{s}]_{g_p} \in \sources{n}{p}{[\fa{f}{t}]_{g_n}}}
	\item \m{\forall n \in \cfg, p \in \preds{n}, \fa{f}{t} \in \gfasA{g_n} \cup \rgfas{n}, \tup{s} \in \sources{n}{p}{\tup{t}} \cdot}\\
		\m{ \fa{f}{s} \in \gfasA{g_p} \cup \rgfas{p}}
	\item \m{\forall n \in \cfg,\tup{t} \in g_n,f \in \Fs{\sig} \cdot}\\
		\m{(\forall p \in \preds{n} \cdot \exists \tup{s} \in \sources{n}{p}{\tup{t}}\cdot \fa{f}{s} \in \gfasA{g_p}) \Rightarrow \fa{f}{t} \notin \rgfas{n})} 
\end{enumerate}
\caption{Weak source invariant}
\label{weak_source_invariant}
\end{figure}
This invariant is phrased cfg-locally (it only refers to a node and its direct predecessors), but not EC-graph locally.\\
We explain the invariant by examples in the next sub-section, and in detail in the following section.\\
Intuitively this invariant implies that two terms that appear each at the end of a path and are equal on the path will be appear in all nodes on the path \emph{if at every join on the path an equivalent term appears at every joinee}.\\
An \rgfa at a cfg node means essentially that none of the terms represented by the \rgfa appear on all paths leading to the cfg node.
This differs from the strong invariant in that we require a term to appear at the join only if it appears in \emph{all} joinees,
while the strong invariant requires it to appear if it appears in \emph{any} joinee.

\subsubsection{The propagation invariant}
The \emph{propagation invariant}  ensures that all relevant equality information is propagated locally:
\begin{figure}[H]
\m{\forall n \in \cfg,t \in \terms{g_n}, s \in \Ts{\sig} \cdot} \\
\m{s=t \in \sqcup_F(\eqs{g_n},\s{\eqs{g_p}}_{p \in \preds{n}}) \Rightarrow s \in \terms{[t]_{g_n}}}
\caption{Term based propagation invariant}
\label{propagation_invariant_terms}
\end{figure}
Intuitively this means that any equality that is implied by a join, where the term on one side of the equality appears at the join node EC-graph, the other term will also appear at the join and be represented by the same EC-node.\\
Note that this invariant is insufficient without the source invariant, as shown in ~\ref{snippet3.16b}.

Combining the propagation invariant, the weak source invariant and the fact that the empty tuple appears in all EC graphs, we will show that we can ensure \emph{weak source completeness}:
\begin{figure}[H]
\m{\forall n \in \cfg,p \in \predsto{n},P \in \paths{p}{n},t \in \terms{g_n},s \in \terms{g_p} \cdot}\\
\m{n \models_{=} t=s \Rightarrow [s]_{g_p} \in \sourcesB{P}{[t]_{g_n}}}
\caption{Path based weak source completeness}
\label{path_based_weak_source_completeness}
\end{figure}

The propagation invariant above is local in the cfg, but it is not local in the EC-graph of the involved cfg-nodes, 
and it does not suggest an algorithm to establish it.\\
In the next section we will phrase an EC-graph-local versions of the above invariant for the weak fragment which will induce a family of incremental algorithms, and also describe an invariant that ensures propagation as in the strong fragment but only locally.

\subsubsection*{Complexity}
Even for the weakest join variant of our fragment, the worst case space complexity on the number of EC-nodes that need to be added across all EC-graphs of a program is at least exponential, as can be seen at \ref{snippet3.6}, where at \m{j_1} we would need an exponential sized term to connect the source chain \m{[x]_n} to \m{[y]_{j_0}} even under the weak join.
For this reason we will need to add additional limitations to our fragment, namely restricting term depth, in order to ensure complexity bounds - this we do in the next chapter. 
These additions will not alter fundamentally the invariants we present here, but ensure a polynomial complexity bound by defining a bounded version of the fragment. \\
We have encountered some practical cases where the size of the set of rgfas mandated by the invariants above could be significantly larger than the set of \GFAs{} in the corresponding graph (although, by construction, only polynomially so). 
We will discuss in the implementation section different methods for representing this set efficiently.
%\\
%We have not encountered any practical example that exhibits the above-mentioned exponential behaviour, 
%but we have encountered examples of the following kind:
%\begin{figure}[H]
%\begin{lstlisting}
%$\node{s}:$
%if ($\m{c_1}$)
	%$\node{p_0}:$
	%assume $\m{a=b}$
	%assume $\m{f(a,a)=c}$
%else
	%$\node{p_1}:$	
%$\node{n}:$
%if ($\m{c_2}$)
	%$\node{q_0}:$
	%assume $\m{a=b}$
	%assert $\m{f(a,a)=c}$
%else
	%$\node{q_1}:$
%\end{lstlisting}
%\caption{source polynomial explosion}
%\label{snippet3.16g}
%\end{figure}
%Here, at node \m{n}, our invariants enforce adding 

%
%\noindent
%We want to ensure source completeness in a manner which is
%\begin{itemize}
	%\item Efficient: does not increase the overall worst case asymptotic complexity (space and time) of the verification algorithm compared to a minimal solution (minimal set of terms at each \cfg node that ensure source completeness)
	%\item Incremental: the amount of work (time complexity) for repairing completeness after a change should be proportional to the size of the change and the size of the proof of the new equalities (of a Hoare proof of the new equalities from the previously annotated \cfg)
	%\item Local: each step in establishing completeness, after some incremental changes to some of the node graphs, should include information from no more than a node and its immediate predecessors and successors - this allows us more flexibility to be lazy with these updates
%\end{itemize}
%
%In our setting, once source completeness is established, it can be violated by:
%\begin{itemize}
	%\item \lstinline{Adding} a new term to the graph at some node \node{n} which is not connected to an existing term in a transitive predecessor \node{p} (e.g. as a term in a clause learned from another fragment)
	%\item Merging two term EC nodes at some node \node{n} - e.g. if we have an EC for \m{f(a)} and we \lstinline{assume} \m{a=b}, we might now have a new source that equals \m{f(b)} (e.g. as the result of a unit equality learned from another fragment)
%\end{itemize}

\subsubsection{Examples}
We will show now a few examples that demonstrate some of the issues with establishing interpolation completeness, and how our above invariants handle them.

\noindent
We start with a simple example:
\begin{figure}[H]
\begin{lstlisting}
$\node{n_1}:$
	assume $\m{f(a)=g(a)}$
	//rgfas      : $\m{\textcolor{gray}{f(\s{b})_,g(\s{b})}}$ (none with scoping)
$\node{n_2}:$
	assume $\m{a=b}$
	//extra terms: $\m{\textcolor{gray}{f(\s{a,b})_,g(\s{a,b})}}$
$\node{n_3}:$
	assert $\m{f(b)=g(b)}$ //negated $\m{\textcolor{gray}{f(b) \neq g(b)}}$
\end{lstlisting}
\caption{propagation sources},
\label{snippet3.16a}
\end{figure}

In ~\ref{snippet3.16a}, we would need to add \m{f(\s{a,b})_,g(\s{a,b})} to \m{g_{n_2}} in order to ensure completeness.

In the simple case of a sequential node, the propagation invariant means that two nodes should be merged if they share at least one source in the predecessor (although a slightly more involved condition is needed for propagation completeness as we will discuss in the next section).

We show the initial state, with the propagation invariant satisfied but not the source invariant:
\begin{figure}[H]
\begin{tikzpicture}
	\node[gttn] (1)              {$()$};
	\node[gl]   (1l) [below = 0.2cm of 1] {\m{n_1}};

	\node[gtn,ultra thick]  (2) [above left  = 0.5cm and 0.2cm of 1] {\s{a}};
	\node[gtn,ultra thick]  (3) [above right = 0.5cm and 0.2cm of 1] {\s{b}};

	\draw[gfa] (2) to node[el]             {\m{a}} (1.90);
	\draw[gfa] (3) to node[el,anchor=west] {\m{b}} (1.90);

	\node[gttn,ultra thick] (4)  [above = 0.5cm of 2]    {\m{(a)}};
	\node[gttn,ultra thick] (5)  [above = 0.5cm of 3]    {\m{(b)}};

	\draw[sgtt,ultra thick] (4) to node[el] {\textbf{0}} (2);
	\draw[sgtt,ultra thick] (5) to node[el] {\textbf{0}} (3);

	\node[gtn]  (6)  [above = 2.5cm of 1] {\svb{f(a)}{g(b)}};
	\draw[gfa]  (6) to node[el]             {\m{f}} (4);
	\draw[gfa]  (6) to node[el,anchor=west] {\m{g}} (5);

%%%%%%%%%%%%%%%%%%%%%%%%%%%%%%%%%%%%%%%%%%%%%%%%%%%%%%%%%%%%%%
	\node[gttn] (11)  [right = 3cm of 1] {$()$};
	\node[gl,anchor=north]   (11l) [below = 0.2cm of 11]   {\m{n_2}};

	\node[gtn,ultra thick]  (12) [above = 0.5cm of 11] {\s{a,b}};

	\draw[gfa] (12) to[out=-110,in=110] node[el]             {\m{a}} (11.90);
	\draw[gfa] (12) to[out=- 70,in= 70] node[el,anchor=west] {\m{b}} (11.90);

%%%%%%%%%%%%%%%%%%%%%%%%%%%%%%%%%%%%%%%%%%%%%%%%%%%%%%%%%%%%%%
	\node[gttn] (21)  [right = 4cm of 11] {$()$};
	\node[gl]   (21l) [below = 0.2cm of 21]   {\m{n_3}};

	\node[gtn,ultra thick]  (22) [above = 0.5cm of 21] {\s{a,b}};

	\draw[gfa] (22) to[out=-110 ,in=110] node[el]             {\m{a}} (21);
	\draw[gfa] (22) to[out=- 70,in= 70] node[el,anchor=west] {\m{b}} (21);

	\node[gttn,ultra thick] (24)  [above = 0.5cm of 22]    {\m{(\s{a,b})}};

	\draw[sgtt,ultra thick] (24) to node[el] {\textbf{0}} (22);

	\node[gtn]  (26)  [above left  =  0.5cm and 0.2cm of 24] {\faB{f}{a}{b}};
	\node[gtn]  (27)  [above right =  0.5cm and 0.2cm of 24] {\faB{g}{a}{b}};
	\draw[gfa]  (26) to node[el]             {\m{f}} (24);
	\draw[gfa]  (27) to node[el,anchor=west] {\m{g}} (24);

%%%%%%%%%%%%%%%%%%%%%%%%%%%%%%%%%%%%%%%%%%%%%%%%%%%%%%%%%%%%%%
	\draw[se] (11) to  ( 1);
	\draw[se] (21) to  (11);

	\node (12a) [left = 0.3cm of 12] {};
	\node (12b) [left = 0.3cm of 12a] {};
	\node (3a) [above = 0.1cm of 3] {};

	\draw[se,ultra thick] (12.180) to (12a.0) to (12b.0) to[out=170,in=0] (3a.0) to[out=180,in=0] (2.0);
	\draw[se,ultra thick] (12.180) to (12a.0) to (12b.0) to[out=190,in=0] (3.0);
	\draw[se,ultra thick] (22) to  (12);

	\node (24a) [left = 2.9cm of 24] {\color{red}$\times$};
	\node (24b) [left = 1cm of 24a] {\color{red}$\times$};
	\node (24c) [above = 0.2cm of 24b] {\color{red}$\times$};
	\node (5c) [above = 0.5cm of 5] {};
	\draw[me] (24) to (24a);
	\draw[me] (24b) to (5);
	\draw[me] (24c) to (5c) to (4);

	\node (26a) [above = 0.6cm of 24a] {\color{red}$\times$};
	\node (26b) [above = 0.0cm of 26a] {\color{red}$\times$};
	\node (26c) [above = 1cm of 24b] {\color{red}$\times$};
	\node (26u) [above = 0.0cm of 26] {};
	\draw[me] (26.180) to (26a.0);
	\draw[me] (27.180) to (26u) to (26b.0);
	\draw[me] (26c.180) to (6.0);

	\draw[ie] (26) to node[el,below] {\m{\neq}} (27);
	
\draw[draw=none, use as bounding box] (current bounding box.north west) rectangle (current bounding box.south east);

\begin{pgfinterruptboundingbox}
	\draw[separator] (2cm,-0.7cm) to (2cm,3.8cm);
	\draw[separator] (4.5cm,-0.7cm) to (4.5cm,3.8cm);
\end{pgfinterruptboundingbox}
\end{tikzpicture}

\caption{
\scriptsize
The sources function\\
missing rgfas\\
Dashed arrows represent inequalities\\
\color{blue} Blue dashed arrows represent the $\mathbf{sources}$ function\\
\color{gray} Gray dashed arrows represent rejected sources\\
\color{red} Red dashed arrows represent missing or broken sources
}
\label{snippet3.16a_graph1}
\end{figure}
Here we see (in red) the broken source chains, and the reason is that the second part of the source invariant is broken - the node \m{[f(\s{a,b})]_{n_3}} implies that \m{g_{n_2}} should have \m{f([a]_{n_2}),f([b]_{n_2})} either as a \gfa or as an \rgfa (the equality is mandated by the predecessor, not successor).

We now show one failed attempt to satisfy the source invariant by adding an \rgfa:
\begin{figure}[H]
\begin{tikzpicture}
	\node[gttn] (1)              {$()$};
	\node[gl]   (1l) [below = 0 of 1] {\m{n_1}};

	\node[gtn]  (2) [above left  = 0.5cm and 0.2cm of 1] {\s{a}};
	\node[gtn]  (3) [above right = 0.5cm and 0.2cm of 1] {\s{b}};

	\draw[gfa] (2) to node[el]             {\m{a}} (1.90);
	\draw[gfa] (3) to node[el,anchor=west] {\m{b}} (1.90);

	\node[gttn] (4)  [above = 0.5cm of 2]    {\m{(a)}};
	\node[gttn] (5)  [above = 0.5cm of 3]    {\m{(b)}};

	\draw[sgtt] (4) to node[el] {0} (2);
	\draw[sgtt] (5) to node[el] {0} (3);

	\node[gtn]  (6)  [above = 2.5cm of 1] {\svb{f(a)}{g(b)}};
	\draw[gfa]  (6) to node[el]             {\m{f}} (4);
	\draw[gfa]  (6) to node[el,anchor=west] {\m{g}} (5);

%%%%%%%%%%%%%%%%%%%%%%%%%%%%%%%%%%%%%%%%%%%%%%%%%%%%%%%%%%%%%%
	\node[gttn] (11)  [right = 3.8cm of 1] {$()$};
	\node[gl]   (11l) [below = 0 of 11]   {\m{n_2}};

	\node[gtn]  (12) [above = 0.5cm of 11] {\s{a,b}};

	\draw[gfa] (12)     to[out=-110,in=110] node[el]             {\m{a}} (11.90);
	\draw[gfa] (12)     to[out=- 70,in= 70] node[el,anchor=west] {\m{b}} (11.90);

	\node[rgttn] (14)  [above = 0.5cm of 12]    {\m{f(\s{a,b})}};

	\draw[rgtt] (14) to node[rl] {0} (12);

	\node[rgtn]  (16) [above left  = 0.5cm and -0cm of 14] {\faB{f}{a}{b}};
	\node[rgtn]  (17) [above right = 0.5cm and -0cm of 14] {\faB{g}{a}{b}};

	\draw[rgfa]  (16) to node[rl]             {\m{f}} (14);
	\draw[rgfa]  (17) to node[rl,anchor=west] {\m{g}} (14);

%%%%%%%%%%%%%%%%%%%%%%%%%%%%%%%%%%%%%%%%%%%%%%%%%%%%%%%%%%%%%%
	\node[gttn] (21)  [right = 4.5cm of 11] {$()$};
	\node[gl]   (21l) [below = 0 of 21]   {\m{n_3}};

	\node[gtn]  (22) [above = 0.5cm of 21] {\s{a,b}};

	\draw[gfa] (22) to[out=-110,in=110] node[el] {\m{a}} (21.90);
	\draw[gfa] (22) to[out=-70 ,in= 70] node[el,anchor=west] {\m{b}} (21.90);

	\node[gttn] (24)  [above = 0.5cm of 22]    {\m{(\s{a,b})}};

	\draw[sgtt] (24) to node[el] {0} (22);

	\node[gtn]  (26)  [above left  =  0.5cm and -0cm of 24] {\faB{f}{a}{b}};
	\node[gtn]  (27)  [above right =  0.5cm and -0cm of 24] {\faB{g}{a}{b}};
	\draw[gfa]  (26) to node[el] {\m{f}} (24);
	\draw[gfa]  (27) to node[el,anchor=west] {\m{g}} (24);

%%%%%%%%%%%%%%%%%%%%%%%%%%%%%%%%%%%%%%%%%%%%%%%%%%%%%%%%%%%%%%
	\draw[se] (11) to  ( 1);
	\draw[se] (21) to  (11);

	\node (12a) [left = 0.5cm of 12] {};
	\node (12b) [left = 0.5cm of 12a] {};
	\node (3c)  [above = 0.1cm of 3] {};
  \draw[se] (12.180) to[out=180,in=0] ( 12a.0) to[out=180,in=0] (12b.0) to[out=180,in=0] (3c.0) to[out=180,in=0] (2.0);
  \draw[se] (12.180) to[out=180,in=0] ( 12a.0) to[out=180,in=0] (12b.0) to[out=180,in=0] (3.0);
	\draw[se] (22) to  (12);

	\node (14a) [left = 0.5cm of 14] {};
	\node (14b) [left = 0.5cm of 14a] {};
	\node (5c)  [above = 0.1cm of 5] {};
  \draw[me] (14.180) to[out=180,in=0] ( 14a.0) to[out=180,in=0] (14b.0) to[out=180,in=0] (5c.0) to[out=180,in=0] (4.0);
  \draw[me] (14.180) to[out=180,in=0] ( 14a.0) to[out=180,in=0] (14b.0) to[out=180,in=0] (5.0);
	\draw[re] (24) to  (14);

	\node (6a) [right = 0.cm of 6] {};
	\node (16c) [above = 0.cm of 16] {};
  \draw[me] (16.180) to[out=180,in=0] ( 6a.0) to[out=180,in=0] (6.0);
  \draw[me] (17.180) to[out=180,in=0] ( 16c.0) to[out=180,in=0] ( 6a.0) to[out=180,in=0] (6.0);

	\node (17c) [above = 0cm of 17] {};
	\node (17d) [below = 0cm of 17] {};
	\draw[re] (26.180) to[out=180,in=0] (17d.0) to[out=180,in=0] (16.0);
	\node (26a) [above = 0cm of 26] {};
	\draw[re] (27.180) to[out=180,in=0] (26a.0) to[out=180,in=0] (17.0);


	\draw[ie] (26) to node[el,below] {\m{\neq}} (27);

\draw[draw=none, use as bounding box] (current bounding box.north west) rectangle (current bounding box.south east);

\begin{pgfinterruptboundingbox}
	\draw[separator] (2.1cm,-0.7cm) to (2.1cm,3.7cm);
	\draw[separator] (6.5cm,-0.7cm) to (6.5cm,3.7cm);
\end{pgfinterruptboundingbox}

\end{tikzpicture}

\caption{
\tiny
The sources function\\
wrong rgfas
}
\label{snippet3.16a_graph2}
\end{figure}
Here the invariant says that an \rgfa is not allowed if all of its source are \gfas (third part of the weak source invariant) - at \m{n_2} the \rgfa \m{f([\s{a,b}]_{n_2})} has the source \m{[f(\s{f(a),g(b)})]_{n_1}} at \m{n_1}, so we should have a \gfa at \m{n_2} and not an \rgfa.
%In figure ~\ref{snippet3.16a_graph2} the two upper EC-nodes in \node{n_2} share one transitive source, but are not merged - hence the incompleteness. The highlighted source path shows the tuple EC \m{(\s{a,b})} at \m{n_3} 
%whose first term EC has a transitive source \m{\s{a}} at \m{n_1} with the same super-tuple, but the tuples are not connected with the source function (same for the source path to \m{\s{b}}).

We show now the part the propagation invariant plays:
\begin{figure}[H]
\begin{tikzpicture}
  \node[gttn] (1)              {$()$};
	\node[gl]   (1l) [below = 0 of 1] {\m{n_1}};

  \node[gtn]  (2) [above left  = 0.5cm and 0.2cm of 1] {\s{a}};
  \node[gtn]  (3) [above right = 0.5cm and 0.2cm of 1] {\s{b}};
	
	\draw[gfa] (2) to node[el] {\m{a}} (1.90);
  \draw[gfa] (3) to node[el,anchor=west] {\m{b}} (1.90);
  
	\node[gttn] (4)  [above = 0.5cm of 2]    {\m{(a)}};
  \node[gttn] (5)  [above = 0.5cm of 3]    {\m{(b)}};

	\draw[sgtt] (4) to node[el] {0} (2);
	\draw[sgtt] (5) to node[el] {0} (3);

  \node[gtn]  (6)  [above = 2.5cm of 1] {\svb{f(a)}{g(b)}};
	\draw[gfa]  (6) to node[el] {\m{f}} (4);
  \draw[gfa]  (6) to node[el,anchor=west] {\m{g}} (5);

%%%%%%%%%%%%%%%%%%%%%%%%%%%%%%%%%%%%%%%%%%%%%%%%%%%%%%%%%%%%%%
	\node[gttn] (11)  [right = 3.8cm of 1] {$()$};
	\node[gl]   (11l) [below = 0 of 11]   {\m{n_2}};

	\node[gtn]  (12) [above = 0.5cm of 11] {\s{a,b}};
	
	\draw[gfa] (12) to[out=-110 ,in=110] node[el] {\m{a}} (11.90);
	\draw[gfa] (12) to[out=-70,in=70] node[el,anchor=west] {\m{b}} (11.90);

	\node[gttn] (14)  [above = 0.5cm of 12]    {\m{(\s{a,b})}};

	\draw[sgtt] (14) to node[el] {0} (12);

	\node[gtn]  (16) [above left  = 0.5cm and -0cm of 14] {\faB{f}{a}{b}};
	\draw[gfa]  (16) to node[el] {\m{f}} (14);
	\node[gtn]  (17) [above right = 0.5cm and -0cm of 14] {\faB{g}{a}{b}};
	\draw[gfa]  (17) to node[el,anchor=west] {\m{g}} (14);

%%%%%%%%%%%%%%%%%%%%%%%%%%%%%%%%%%%%%%%%%%%%%%%%%%%%%%%%%%%%%%

	\node[gttn] (21)  [right = 4.5cm of 11] {$()$};
	\node[gl]   (21l) [below = 0 of 21]   {\m{n_3}};

	\node[gtn]  (22) [above = 0.5cm of 21] {\s{a,b}};
	
	\draw[gfa] (22) to[out=-110, in=110] node[el] {\m{a}} (21.90);
	\draw[gfa] (22) to[out= -70, in= 70] node[el,anchor=west] {\m{b}} (21.90);

	\node[gttn] (24)  [above = 0.5cm of 22]    {\m{(\s{a,b})}};

	\draw[sgtt] (24) to node[el] {0} (22);

	\node[gtn]  (26)  [above left  =  0.5cm and -0cm of 24] {\faB{f}{a}{b}};
	\node[gtn]  (27)  [above right =  0.5cm and -0cm of 24] {\faB{g}{a}{b}};
	\draw[gfa]  (26) to node[el] {\m{f}} (24);
	\draw[gfa]  (27) to node[el,anchor=west] {\m{g}} (24);

%%%%%%%%%%%%%%%%%%%%%%%%%%%%%%%%%%%%%%%%%%%%%%%%%%%%%%%%%%%%%%
	\draw[se] (11) to  ( 1);
	\draw[se] (21) to  (11);

	\node (12a) [left = 0.5cm of 12] {};
	\node (12b) [left = 0.5cm of 12a] {};
	\node (3c)  [above = 0.1cm of 3] {};
	\draw[se] (12.180) to[out=180,in=0] ( 12a.0) to[out=180,in=0] (12b.0) to[out=180,in=0] (3c.0) to[out=180,in=0] (2.0);
	\draw[se] (12.180) to[out=180,in=0] ( 12a.0) to[out=180,in=0] (12b.0) to[out=180,in=0] (3.0);
	\draw[se] (22) to  (12);

	\node (14a) [left = 0.5cm of 14] {};
	\node (14b) [left = 0.5cm of 14a] {};
	\node (5c)  [above = 0.1cm of 5] {};
	\draw[se] (14.180) to[out=180,in=0] ( 14a.0) to[out=180,in=0] (14b.0) to[out=180,in=0] (5c.0) to[out=180,in=0] (4.0);
	\draw[se] (14.180) to[out=180,in=0] ( 14a.0) to[out=180,in=0] (14b.0) to[out=180,in=0] (5.0);
	\draw[se] (24) to  (14);

	\node (6a) [right = 0.cm of 6] {};
	\node (16c) [above = 0.cm of 16] {};
	\draw[me] (16.180) to[out=180,in=0] ( 6a.0) to[out=180,in=0] (6.0);
	\draw[me] (17.180) to[out=180,in=0] ( 16c.0) to[out=180,in=0] ( 6a.0) to[out=180,in=0] (6.0);

	\node (17c) [above = 0cm of 17] {};
	\node (17d) [below = 0cm of 17] {};
	\draw[se] (26.180) to[out=180,in=0] (17d.0) to[out=180,in=0] (16.0);
	\node (26a) [above = 0cm of 26] {};
	\draw[se] (27.180) to[out=180,in=0] (26a.0) to[out=180,in=0] (17.0);

	\draw[ie] (26) to node[el,below] {\m{\neq}} (27);

\draw[draw=none, use as bounding box] (current bounding box.north west) rectangle (current bounding box.south east);

\begin{pgfinterruptboundingbox}
	\draw[separator] (2.1cm,-0.7cm) to (2.1cm,3.7cm);
	\draw[separator] (6.5cm,-0.7cm) to (6.5cm,3.7cm);
\end{pgfinterruptboundingbox}

\end{tikzpicture}

\caption{
\scriptsize
The sources function\\
propagation invariant broken
}
\label{snippet3.16a_graph3}
\end{figure}
Here the nodes \m{[f(\s{a,b})]_{n_2}} and \m{[g(\s{a,b})]_{n_2}} share a source, and so should be merged.

Here is the desired graph of all EC graphs:
\begin{figure}[H]
\begin{tikzpicture}
  \node[gttn] (1)              {$()$};
	\node[gl]   (1l) [below = 0 of 1] {\m{n_1}};

  \node[gtn]  (2) [above left  = 0.6cm and 0.2cm of 1] {\s{a}};
  \node[gtn]  (3) [above right = 0.6cm and 0.2cm of 1] {\s{b}};

	\draw[gfa] (2) to node[el]             {\m{a}} (1.90);
	\draw[gfa] (3) to node[el,anchor=west] {\m{b}} (1.90);
  
	\node[gttn] (4)  [above = 0.5cm of 2]    {\m{(a)}};
	\node[gttn] (5)  [above = 0.5cm of 3]    {\m{(b)}};

	\draw[sgtt] (4) to node[el] {0} (2);
	\draw[sgtt] (5) to node[el] {0} (3);

	\node[gtn]  (6)  [above = 2.5cm of 1] {\tiny$\svb{f(a)}{g(b)}$};
	\draw[gfa]  (6) to node[el]             {\m{f}} (4);
	\draw[gfa]  (6) to node[el,anchor=west] {\m{g}} (5);

%%%%%%%%%%%%%%%%%%%%%%%%%%%%%%%%%%%%%%%%%%%%%%%%%%%%%%%%%%%%%%
	\node[gttn] (11)  [right = 3cm of 1] {$()$};
	\node[gl]   (11l) [below = 0 of 11]   {\m{n_2}};

	\node[gtn]  (12) [above = 0.5cm of 11] {\s{a,b}};

	\draw[gfa] (12) to[out=-110,in=110] node[el]             {\m{a}} (11.90);
	\draw[gfa] (12) to[out=- 70,in= 70] node[el,anchor=west] {\m{b}} (11.90);

	\node[gttn] (14)  [above = 0.5cm of 12]    {\m{(\s{a,b})}};

	\draw[sgtt] (14) to node[el] {0} (12);

	\node[gtn]  (16)  [above = 2.5cm of 11] {\tiny $\faB{f}{a}{b},\faB{g}{a}{b}$};
	\draw[gfa]  (16) to[out=-100 , in=100] node[el]             {\m{f}} (14);
	\draw[gfa]  (16) to[out=- 80 ,in=  80] node[el,anchor=west] {\m{g}} (14);
				
%%%%%%%%%%%%%%%%%%%%%%%%%%%%%%%%%%%%%%%%%%%%%%%%%%%%%%%%%%%%%%

	\node[gttn] (21)  [right = 3.5cm of 11] {$()$};
	\node[gl]   (21l) [below = 0 of 21]   {\m{n_3}};

	\node[gtn]  (22) [above = 0.5cm of 21] {\s{a,b}};

	\draw[gfa] (22) to[out=-110,in=110] node[el]             {\m{a}} (21.90);
	\draw[gfa] (22) to[out=- 70,in= 70] node[el,anchor=west] {\m{b}} (21.90);

	\node[gttn] (24)  [above = 0.5cm of 22]    {\m{f(\s{a,b})}};

	\draw[sgtt] (24) to node[el] {0} (22);

	\node[gtn]  (26)  [above = 2.5cm of 21] {\tiny $\faB{f}{a}{b},\faB{g}{a}{b}$};
	\draw[gfa]  (26) to[out=-100 ,in= 100] node[el]             {\m{f}} (24);
	\draw[gfa]  (26) to[out=- 80 ,in=  80] node[el,anchor=west] {\m{g}} (24);

%%%%%%%%%%%%%%%%%%%%%%%%%%%%%%%%%%%%%%%%%%%%%%%%%%%%%%%%%%%%%%
	\draw[se] (11) to  ( 1);
	\draw[se] (21) to  (11);

	\node(12a) [left = 0.5cm of 12] {};
	\node(3c) [above= 0.1cm of 3] {};
	\draw[se] ( 12.180) to[out=180,in=0] (12a.0) to[out=180,in=0] (3c) to[out=180,in=0] (   2.0);
	\draw[se] ( 12.180) to[out=180,in=0] (12a.0) to[out=180,in=0] (   3.0);

	\draw[se] (22) to  (12);

	\node(5c)  [above= 0.1cm of 5] {};
	\node(14a) [left = 0.7cm of 14] {};
	\draw[se] (14.180) to[out=180,in=0] (14a.0) to[out=180,in=0] (5c) to[out=180,in=0]( 4.0);
	\draw[se] (14.180) to[out=180,in=0] (14a.0) to[out=180,in=0]( 5.0);
	\draw[se] (24) to  (14);

	\draw[se] (16) to  ( 6);
	\draw[se] (26) to  (16);

	\draw[ie] (26) to[loop] node[el,above] {\m{\neq}} (26);

\draw[draw=none, use as bounding box] (current bounding box.north west) rectangle (current bounding box.south east);

\begin{pgfinterruptboundingbox}
	\draw[separator] (2.0cm,-0.7cm) to (2.0cm,3.5cm);
	\draw[separator] (5.5cm,-0.7cm) to (5.5cm,3.5cm);
\end{pgfinterruptboundingbox}

\end{tikzpicture}
\caption{
The sources function
}
\label{snippet3.16a_graph4}
\end{figure}


In the next example, ensuring the global (non-graph-based) source and propagation invariants can be done in a different way:
\begin{figure}[H]
\begin{lstlisting}
$\node{s}:$
assume $\m{P(f(a))}$
if (*)
	$\node{p_0}:$
	assume $\m{a=b}$
	//extra terms { $\m{\textcolor{gray}{f(\s{a,b}),P(f(\s{a,b}))}}$ }
else
	$\node{p_1}:$
	assume $\m{a=c}$
	//extra terms { $\m{\textcolor{gray}{f(\s{a,c}),P(f(\s{a,c}))}}$ }
$\node{n}:$
//our        extra terms: $\m{\textcolor{gray}{\s{f(\s{a}),P(f(\s{a}))}}}$
//                 rgfas: $\m{\textcolor{gray}{\s{b,c,d}}}$
//sufficient extra terms: $\m{\textcolor{gray}{\s{f(\s{b}),f(\s{c}),P(f(\s{b})),P(f(\s{c}))}}}$
$\node{n_a}:$
assume $\m{a=b=c=d}$
assert $\m{P(f(d))}$ //negated $\m{\textcolor{gray}{\lnot P(f(\s{a,b,c,d}))}}$
\end{lstlisting}
\caption{propagation sources}
\label{snippet3.16f}
\end{figure}
In ~\ref{snippet3.16f}, if we just want to ensure all equalities are propagated, we have two options for extra terms at \node{n}, both of which allow source completeness, but the second option adds more terms. 

The next example shows that we can achieve source completeness globally, without achieving it per path:
\begin{figure}[H]
\begin{lstlisting}
$\node{s}:$
assume $\m{P(f(a))}$
if (*)
	$\node{p_0}:$
	assume $\m{a=b}$
	//extra terms { $\m{\textcolor{gray}{f(\s{a,b}),P(f(\s{a,b}))}}$ }
else
	$\node{p_1}:$
	assume $\m{c=b}$
$\node{n}:$
assume $\m{a=c}$
assert $\m{P(f(b))}$ //negated $\m{\textcolor{gray}{\lnot P(f(b))}}$
\end{lstlisting}
\caption{propagation sources}
\label{snippet3.16e}
\end{figure}

In  ~\ref{snippet3.16e} on the path \m{\sourcesB{s.p_0.n}{[f(\s{a,b,c})]_n}=[\s{f(a)}]_s}, \\
so sources are complete for every pair of cfg-nodes (with the added terms in the comments) as\\
\m{\sources{n}{s}{[f(\s{a,b,c})]_n}=\s{[f(a)]_s}},\\
but not every path is source complete:\\
\m{\sourcesB{s.p_1.n}{[\s{f(\s{a,b,c})}]_n}=\emptyset}.\\
We see that this, along with propagation completeness, is not sufficient, and the assertion will not be proven.\\
This is because the weak source invariant is broken, and cannot be repaired just by adding \rgfas.


\subsubsection{Symmetry}
Consider the following example:
\begin{figure}[H]
\begin{lstlisting}
$\node{s}:$
assume $\m{f(a,d)=g(a,d)}$
assume $\m{f(b,c)=g(b,c)}$
if (*)
	$\node{p_0}:$
	assume $\m{a=b}$
else
	$\node{p_1}:$
	assume $\m{c=d}$
$\node{n}:$
	//Here {$\textcolor{gray}{\m{f(a,c)=g(a,c),f(a,d)=g(a,d)}}$ 
	//      $\textcolor{gray}{\m{f(b,c)=g(b,c),f(b,d)=g(b,d)}}$ }
if (*)
	$\node{n_0}:$
	assume $\m{c=d=y}$
	assume $\m{a=x}$
	assert $\m{f(x,y)=g(x,y)}$ //negated $\m{\textcolor{gray}{f(x,y)\neq g(x,y)}}$
else
	$\node{n_1}:$
	assume $\m{a=b=x}$
	assume $\m{c=y}$
	assert $\m{f(x,y)=g(x,y)}$ //negated $\m{\textcolor{gray}{f(x,y)\neq g(x,y)}}$
\end{lstlisting}
\caption{sources symmetry}
\label{snippet3.16h}
\end{figure}
Here, at the join n, there are four equalities that hold , as noted in the comment.\\
Out of these four equations, either of \s{f(a,c)=g(a,c),f(a,d)=g(a,d)} would suffice to prove the assertion at \m{n_0}, 
and either of\\
 \s{f(a,c)=g(a,c),f(b,c)=g(b,c)} would suffice for \m{n_1}.\\
If we only consider the branch \m{n_0} (e.g. we are at the first pass of verification handling only the unit ground fragment, and the equations at \m{n_1} would be generated only in the next pass using quantifier instantiation), there is no reason to choose either of 
\s{f(a,c)=g(a,c),f(a,d)=g(a,d)} over the other. If we were to orient the problem in some arbitrary fashion in order to choose one of these,
we could end up with \m{f(a,d)=g(a,d)} and then on the second pass we would need to add either one of the set for \m{n_1}.\\
Even in the non-incremental case, where we have only one pass of the unit ground fragment, 
we suspect that minimizing the set of terms needed at such a join point would be at least NP-hard, 
because if we have \m{k} leaves that follow the join point and each would be provable by a different sub-set of the possible terms, we have a variant of the set-cover problem (in order to show this we would have to show that for each set of sets we can find a program suffix of size polynomial in \m{k} for which the given set of sets represents the sets needed to prove the leaves of the suffix).\\
A harder problem would be minimizing the size of the EC-graph \m{g_n}, as each selected communicating term could have a non-constant sized representation, hinting at the minimum-set-cover problem, and yet harder would be a cfg-wide minimization of such terms in all EC-graphs.\\
%While the potential state explosion in a case as the above is polynomial (with the degree being the arity of the function invovled),
%we have found that in practice similar cases actually appear in the VC of programs, as opposed to most other lower bound examples we have seen that are more of a theoretic nature.\\
\textbf{Pragmatics:}\\
We are less concerned with the absolute minimal number of communicating terms as this set is not monotonic (i.e. adding equalities to the program may imply a minimal set that is not a super-set of the minimal set before the addition), and in actual experiments has not been a major complexity factor, but mostly because our experience has been that in most practical cases the main complexity hurdle is actually the case when there are no provable equalities on a given term:\\
The issue is that, for incremental performance, a large part of execution time is taken by leaves deep in the cfg requesting for equalities on a set of terms, and answered negatively, as this operation has the potential complexity of the entire cfg depth (a request from the end of the program to the root that returns a negative answer).
Our invariants allow us to ensure that no cfg-node is requested twice for equations for the same term, and the set of rgfas is used to ensure that - when a node answers a request it always introduces either an rgfa or a \GFA{} for each term requested, so it will never repeat a request. The order of evaluation of nodes and requests, and collecting several requests together have the most significant reducing effect on this complexity, and will be discussed in the implementation chapter.\\
We now discuss our choice of adding all such equalities rather than selecting only one arbitrarily:\\
In our example, if we were to evaluate \m{n_0} before any equality is known at \m{n_1}, 
if we were to choose to add only one of \s{\s{f(a,c),g(a,c)},\s{f(a,d)=g(a,d)}} as an EC-node at n, 
we would still have to process the other in order to ensure completeness, and mark the other equation so that we do not request it again, 
but so that it is useful for answering the request from \m{n_1} when it comes. We have not found a satisfactory way to do this.\\
Another reason for choosing to add all equalities in this case is that any arbitrary choice would be unintuitive for a user debugging verification.\\
We could make \m{n_0} request for these equalities once at a time (e.g. first request for all equalities for \m{f(a,c)}, 
propagate them and use them with the simplification calculus, and only if needed request for \m{f(a,d)} etc).
These singular requests are similar to those performed in the IC3/PDR algorithm, and the problem is that, in practice, the price we pay for traversing the cfg up and down is much larger than the time we save by adding less intermediate terms.\\
Another option we have considered would be to exploit the structure of such sets of \GFAs{} - here the set required by \m{n_0} can be written as (for the function \m{f}) \m{f(\s{a},\s{c,d})} and similarly for \m{n_1} as \m{f(\s{a,b},\s{c})}.\\
The efficient representation problem for such sets is then similar to the problem of representing sets of rgfas, which we discuss in the implementation section.

