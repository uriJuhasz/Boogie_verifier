\chapter{Introduction}
Recent advances in the technology of automated theorem proving (ATP) and abstract interpretation allow the automated verification of increasingly large and complex programs.
Fully automated verification tools can annotate a program with loop and recursion invariants and verify that the annotated program is correct according to a given specification. 

In this thesis we are interested in verifying the correctness of (manually or automatically) annotated programs rather than the inference of invariants - this verification problem is interesting in itself when user-given annotations are available and is commonly a sub-problem for more automated tools that try to verify approximations of the program (such as bounded loop unrolling) and approximations of the annotation (such as loop invariants generated by abstract interpretation or by interpolation on the proof of an unrolled program).

Popular program verification techniques (for annotated programs) include variants of verification condition (VC) generation (VCG) and symbolic execution.

In VCG based verification, an annotated program and a specification are encoded in a mathematical formula that holds iff the program satisfies the specification, as pioneered by the work of Dijkstra with weakest preconditions (\cite{Dijkstra:1975:GCN:360933.360975}). This formula can be fed to an automated theorem prover. VCG allows the use of general first order logic (FOL) theorem provers, whether based on SMT solving (e.g. \cite{DBLP:conf/cav/BarrettCDHJKRT11},\cite{DBLP:conf/tacas/MouraB08}), completion (\cite{HJL99},\cite{DBLP:conf/cade/RiazanovV99}) instantiation (\cite{Korovin2008}), or other (\cite{BaumgartnerPelzerTinelli12}), although SMT solvers are, by far, the most common. 
One of the main disadvantages of VCG is that the translation to FOL loses some explicit information about the program which can be useful for the proof search, such as the control structure of the program and the scope of variables.

Symbolic execution simulates executing the program on each possible path, using symbolic values rather than concrete values for variables. The symbolic execution engine evaluates the feasibility of a program path by using a constraint solver (often a SAT or SMT solver) to check the satisfiability of the branch conditions on a path with the calculated symbolic values - hence, essentially, several small VCs are sent to the solver, each for a specific path in the program. Symbolic execution tools are sensitive to the problem of path explosion - the number of feasible paths in a program can be exponential in the size of the program, even though the reasoning needed for proving two non-disjoint paths can be very similar.

SMT solvers have made significant progress in recent years, and are quite efficient and reliable in solving problems that involve ground FOL combined with other ground theories, notably linear integer and rational arithmetic and arrays. However, when the VC includes quantified formulae, SMT solvers are, in general, not complete, and the performance and even termination of the solver is very sensitive to the input formula, even to parts that are logically unrelated to the proof (as noted e.g. in \cite{LeinoP16}).
Quantified VCs are needed for modeling abstractions such as sets and sequences used in specifications, for encoding some invariants and for modeling used in some verification methodologies, such as permissions or dynamic frames for alias control.

The instability of SMT solvers in the presence of quantifiers makes verifiers based on SMT solvers less accessible even to people that are familiar with program proofs, specification and first order logic - users need understanding of the working of the SMT solver in order to understand why verification fails, especially for a program that previously verified and has been modified only slightly. 
In addition, as FOL is only semi-decidable, any theoretically complete tool will not be bounded in its run-time.
For users, it is sometimes useful to know that if a proof for a property of the program was found using a certain effort of the verifier, and the program has been modified but the proof of the property carries over to the modified program, then the property will be proven if the same effort is spent by the verifier on the modified program, regardless of other properties and unrelated modifications of the program. Classic resource limitations on the prover, such as memory and time, cannot usually satisfy this requirement.
In addition, if some of the properties of a program are easy to prove while others require longer time or do not hold, it is useful to show the user of a verification tool intermediate results, such as proven properties, while still searching for proofs for the harder properties. Such intermediate results can also be used to cooperate between verification and analysis tools. While the verifier can be run for each property separately, this is often highly inefficient as the proofs of different properties of the same program often share a large number of lemmas.

ATPs based on superposition (e.g. \cite{DBLP:conf/cade/RiazanovV99}) handle quantifiers efficiently, 
and, in addition, can be modified to search incrementally for proofs of increasing complexity - for example, the proof depth can be bounded (by blocking inferences of maximal depth). Such ATPs have seen much less use in program verification as superposition is not very efficient in handling ground and propositional formulae, which are often the majority of the VC for program verification. In addition, the extensions suggested for superposition based solvers to support linear arithmetic are not as efficient as those for SMT solvers, and integer arithmetic is often needed for program verification.

Program analysis tools based on abstract interpretation calculate an over-approximation of the set of feasible program states at each program point, by applying an abstraction to the state and transition relation of the program.
If the approximation satisfies the specification for that program point, so does the program.
For a given abstraction, it is sometimes possible for a programmer to predict which properties of a program will be proved, and the approximation at each program point is not affected by modifications to disjoint parts of the program - hence the results of the tool are more stable and predictable to non-expert users. Abstract interpretation can be applied in abstractions of increasing strength, so that the user can expect that properties proven in a certain abstraction will be proven with the same abstraction if the proof is valid for the modified program.
Very few abstract interpreters were suggested for quantified domains (e.g. \cite{DBLP:conf/popl/GulwaniMT08}),
and they often have to lose precision at join points as they are not goal sensitive - the analyzer at a join point cannot predict what precision is needed to prove properties later in the program.

\textbf{Main contribution:}\\
In this thesis we propose a generic verification algorithm that is based on a tighter integration between theorem proving and verification. 
Our algorithm is based on the idea of having a local theorem prover at each program point rather than one global prover, and allowing these provers to exchange information in order to search for a proof for the entire program. Information is exchanged only on-demand between provers rather than eagerly as often in abstract interpretation.
Our algorithm is incremental and applies successively stronger logical fragments in order to prove a program, allowing the report of intermediate results to the user. For a given fragment, the proof of each property is independent from the proofs of other properties, but the proofs can still share lemmas. 

We have instantiated our algorithm for the fragment of ground equalities using a form of congruence closure graphs, and for general FOL using superposition. We have also implemented a hierarchy of bounded fragments that restrict the proof-tree shape in various ways, including the size of terms and proof depth.

We show how our algorithm can take advantage of the scoping inherent in many programming languages, 
so that the vocabulary of each prover is small and local - we show how to preserve completeness under scoping using interpolation for ground FOL.

As we have implemented only very basic support for linear integer arithmetic, our tool cannot prove many VCs on its own. 
However, as intermediate results are usable at any stage, 
we can use the tool as an optimizing pre-processor before running an SMT solver or other tool.

\section{Outline}
In the rest of this chapter we give an overview of the main ideas in this thesis.
In chapter \ref{chapter:preliminaries} we discuss the theoretical background that we assume, our notation and the structure of input programs that we can handle.
In chapter \ref{chapter:ugfole} we present our verification algorithm and instantiate it for unit ground equalities using a form of congruence closure graphs.
In chapter \ref{chapter:gfole} we instantiate our algorithm for ground clauses using ground superposition.
In chapter \ref{chapter:scoping} we show how our algorithm is adapted to search for local proofs, and the use of interpolation to preserve completeness for local proofs in the ground case.
Chapter \ref{chapter:bounds} introduces several restrictions on the shape of proofs that define a hierarchy of decidable logical fragments with predictable complexity, whose limit is the complete fragment of first order logic, thus allowing incremental verification in fragments of increasing strength.
We show how the algorithm developed for ground clauses is extended to quantified clauses in chapter \ref{chapter:quantification}.
In chapter \ref{chapter:implementation} we discuss some implementation issues and present experimental results for an implementation of some of the ideas in this thesis.
We conclude in chapter \ref{chapter:conclusions} and discuss future work.

%\section{Overview}
%Since the first mathematical proofs of computer programs (\cite{FLOYD67},\cite{DBLP:journals/cacm/Hoare69}), many approaches and tools have been developed for automating the proofs and proof-checking of properties of programs.
%
%An automated theorem prover (ATP) receives as input a formula in some logic - often some fragment of first order logic with equality (FOLE) and sometimes with additional theories such as linear integer arithmetic (LIA) - and outputs whether the formula is valid in that logic, and optionally, if it is valid, outputs a proof and otherwise outputs a counter-example (a model for the negation of the formula).
%As first order logic is only semi-decidable, ATPs will, for some inputs, either not terminate, run out of resources (time, memory) or simply return unknown.
%
%In order to verify a program using an ATP, the correctness condition for the program (or for some properties of the program) is encoded into a logical formula and sent to the ATP, often using some variant of weakest preconditions (\cite{Dijkstra:1975:GCN:360933.360975}). The condition sufficient to show that the program is correct is often composed of several properties that have to hold at certain points in the program (e.g. a certain variable must be greater than zero or not null at some point in the program), as was suggested by the original annotation by Floyd. The weakest precondition calculus calculates the weakest condition that, when holds at the initial point of the program, ensures that all properties hold. The VC generated by a verifier may include additional properties that originate in a verification methodology, such as properties modeling aliasing control or concurrency.
%In this setting, a counter-example represents a valid trace of the program in which at least one property does not hold, and a proof represents an actual proof of correctness of the program. An unknown result from the prover gives no information about the program.
 %%(in theory, if a prover logs its proof-search, some lemmas can be mined from the log of a failed proof, but proof-logging might affect the performance of a prover). 
%
%Other tools that infer and prove properties about programs include abstract interpretation based tools (\cite{CousotCousot77}) and other static analysis which analyze the program, annotating each program point with a condition proven to hold at that program point. Such tools often work either with fragments of logic that are decidable or with a decidable subset of a fragment. 
%The annotation calculated by such a tool may be sufficient in proving only some of the properties needed to show program correctness, but it can be injected to another tool, refined using a stronger fragment and used for program debugging.\\
%Many static analyzers work in a fragment where the complexity of the analysis is known, and hence a user can expect consistent analysis results when running the tool on the same program, even if more properties are added or an extension is added to the program. Most static analysis tools handle a logical fragment which is insufficient for proving the whole program, and specifically cannot handle an arbitrary fragment modeled using quantified axioms, which are sometimes handled by VCG - often a new analyzer has to be built for a new fragment.
%
%This work is an attempt at taking the generality of ATPs and the predictability and incrementallity of analysis tools, 
%by allowing a more intimate interaction between the verifier and the prover. Such interaction can allow the prover to take advantage of some of the program properties that are lost or made implicit in the conversion to a VC. 
%The main areas in which we can exploit additional information are detailed in the next sections.
%
%We focus on improving the verification process rather than the inference of invariants, and hence we assume that specification, and particularly loop invariants, are given for the program, and the objective is to verify that the program satisfies the given specification. 
%We assume the program and specification have been encoded into an intermediate language such as Boogie (\cite{BarnettCDJL05}). While such an encoding necessarily loses some information and is at a lower level of abstraction than the original program, this allows us to develop techniques that are applicable to several programming languages and verification methodologies.
%We detail our exact assumptions about the input in chapter \ref{chapter:preliminaries}, here we only mention that we assume that the input CFG is a DAG, and that it is in dynamic single assignment (DSA) form (essentially, each program variable is assigned at most once on each directed path in the CFG).
%
%All of our techniques are incremental (no derivation is repeated, with some exceptions) and allow the extraction of intermediate results at any time, and also allow the addition of additional properties and program extensions at any point. Intermediate results are useful for quick feedback to the user of the verification tool and for interaction with other verification and analysis tools.

\section{Control Flow}
When generating the VC for a given program, some information about the CFG has to be encoded into first order logic.
Mainly, for an assertion anywhere in the code, only information from statements that precede the assertion can be used to prove the assertion, and two statements on parallel branches are treated differently than two statements in sequence.
The weakest precondition (WP) calculus (\cite{Dijkstra:1975:GCN:360933.360975}) is often used as a basis for converting a program and a specification to a logical formula that is valid iff the program satisfies the specification. 
A WP calculus suitable for verification with an intermediate verification language is described in \cite{Leino:2005:EWP:1066417.1710882}, which assumes a program constructed of \lstinline|assert| and \lstinline|assume| statements. As noted in the paper, applying the original formulation of WP to a program with n sequential non-nested branches produces a VC of size that is exponential in n (essentially a case split on each possible path in the CFG), and hence a more efficient formulation is required for many programs.
A common method for encoding a polynomial sized VC (also suggested in \cite{Leino:2005:EWP:1066417.1710882}) 
is to use a nullary predicate per basic block of the original program, which represents either that execution has reached that basic block or that, if execution reaches that basic block, then all assertions in the block and its successors hold. 
In either encoding, some implications are encoded between these nullary predicates which encode the structure of the CFG.
However, in any encoding into a logic that does not support graph structures directly, the explicit graph structure is lost and hence some graph properties that are easily exploitable in an analysis that uses the CFG are lost.
% - for proving each of the assertions \lstinline|e1,e2,e3,e4| a different subset of the statements \lstinline|S1,S2,S3,S4|
To prove that an assertion holds, it must be shown that it holds on any path reaching it from the root.
For each such path, the prover has to show that after executing the statements on the path the assertion holds.
As we only treat passive statements (\lstinline|assert| and \lstinline|assume|) for the DSA form of the program, the order of execution of the statements on the path does not matter for the proof of the program - hence for each path we have to show that the set of assumed or asserted formulae on the path implies the formula in the assertion. We call these sets of formulae the relevant sets for proving the assertion - for example,  in the program in figure \ref{snippet_1.1},
for \lstinline{assert e3}, the set $\{\{$\lstinline{S1,S3}$\}$,$\{$\lstinline{S2,S3}$\}\}$ is the set of all relevant subsets of the program statements relevant for the assertion. Each such subset represents a path reaching the assertion from the root. 


\begin{figure}
\begin{lstlisting}
if (b1)
	S1;
	assert e1;
else
	S2;
	assert e2;
j$_1$:
if (b2)
	S3;
	assert e3;
else
	S4;
	assert e4;
\end{lstlisting}
\caption{Example for the information contained in the CFG}
\label{snippet_1.1}
\end{figure}

\subsection*{Relevance}
The first property that is made implicit in the encoding of the VC is relevance - for example, assume that the above program is encoded into a VC and sent to an SMT solver, and, during the proof search, the SMT solver decides on a literal that encodes the fact that the trace passes through \lstinline|S1| (that is, the \lstinline|then| branch of the first conditional is taken), or, depending on the exact VC encoding, that an assertion does not hold on a path that passes through \lstinline|S1|.
There is nothing preventing the SMT solver from deciding on a literal that occurs only in \lstinline|S2|, although this is not necessary in order to prove any assertion - a model (trace of a failing assertion) that encodes a trace that passes through \lstinline|S1| is fully defined by the interpretation for symbols that occur on the path of the trace - which may not include some of the symbols that occur  \lstinline|S2|.
Some of the simplifications performed by SMT solvers (unit propagation and pure literal elimination) can prevent some of these cases, but for complicated CFGs the proportion of the cases eliminated is limited.
Many of the more successful SMT solvers use incremental or lazy CNF conversion (e.g. \cite{DBLP:conf/cav/BarrettDS02},\cite{DBLP:journals/jacm/DetlefsNS05}) which can prevent much of the interference when the VC is encoded carefully.
The parallel of the above for a superposition based prover is that if \lstinline|S1| and \lstinline|S2| are each encoded into a set of clauses, there is no need to perform any inference between a clause that encodes \lstinline|S1| and a clause that encodes \lstinline|S2|,
as this inference will not participate in the proof that the assertion holds on any path of the program.

\subsection*{Joins}
A second property of the CFG that is less exploitable on a monolithic VC is that of sharing lemmas on joins.
Consider the case where some lemma $\m{C}$ implied separately by \lstinline|S1| and \lstinline|S2| is sufficient, together with the encoding of \lstinline|S3|, to prove that the assertion \lstinline|assert e3| holds. 
For propositional logic, lemmas do not include new literals (although some proofs can be shortened by introducing new literals - e.g. the pigeonhole principle in extended resolution), for ground first order logic with equality (GFOLE - called QF\_UF in the SMT community) and more so for FOLE, the introduction of new literals, even if constructed only from the VC  vocabulary, can sometimes allow a significantly shorter shortest proof.
Abstract interpretation tools search explicitly for such lemmas in a fixed logical fragment. CDCL based SMT solvers (\cite{GRASP}), in general, can learn some of these lemmas (we discuss this issue in more detail later - see also \cite{DPLLJoin}). 
Superposition based provers can also sometimes generate join lemmas.

%\subsection{Directionality}
%\subsection{Goal Directed Proofs}
	%Some resolution based theorem provers, when presented with a set of axioms and a theorem, can search for a proof/refutation of the theorem under the assumption that the axioms are consistent.
	%This has the advantage of reducing the proof search space as it only considers proofs that contain the assertion - when the set of axioms is known to be consistent.
	%In the context of program proofs, we have several sets of axioms:
	%\begin{itemize}
	%\item The mathematical axioms for the relevant domains (integers, booleans, sequences etc), which are often known to be consistent.
	%\item The program specific axioms that come from the type system, data structure definitions and aliasing/specification methodology - these are also usually assumed to be consistent and proven separately.
	%\item The axioms that encode the CFG structure - these are also usually consistent by construction, but can encode unreachable CFG nodes by false branch conditions.
	%\item Axioms that encode the executable program statements - these are also usually consistent, but may conflict with axioms that represent specification.
	%\item Axioms that encode assumptions generated by the specification methodology - e.g. pre-conditions, method call-return post-conditions, loop invariants - when these conflict it usually means a program location is unreachable.
	%\item Assertions - when the rest of the axioms (on execution paths leading to the assertion) are consistent, but the addition of the assertion makes them inconsistent, it means the assertion does not hold.
	%So when trying to refute the negation of an assertion a at program point p, we could usually limit ourselves to either showing that p is unreachable or showing that a does not hold at p.
	%As the second case is usually much more common (a program would usually have many small assertions generated by the specification methodology and programming language semantics as well as explicitly by the programmer, and most would be in reachable CFG nodes).
	%\end{itemize}
	%We can bias our proof search in the direction of assertions: we look for a refutation based at the negation of the said assertion (with some weight on the path condition leading to it) and of course completely ignore any CFG node that cannot reach that assertion.
	%We try to prove each assertion separately, and when an assertion CFG node is proven we can trim the CFG so that all nodes lead to at least one assertion node - and remember the lemmas we found on the way whether the refutation succeeded or not.
	%We only do resolution on atoms and clauses coming (transitively) from the assertion and only do quantifier instantiation on ground terms that come (transitively) from an assertion.

\section{Locality and Scope}
Many current programming languages support scoping for program variables, where a variable can be accessed only in a certain area of the program. For example, a loop counter may be in scope only within the loop body.
An encoding of a program VC most often represents program variables in some form of dynamic single assignment (DSA) form - this form ensures that each program variable is assigned at most once on each program path (usually a program to program transformation replaces each occurrence of a a program variable with some indexed version of that variable to ensure this property. SSA is a specific case of DSA.). Also, a VC for the unrolling of a program (e.g. as in \cite{DBLP:conf/tacas/AlbarghouthiGC12}) often uses some form of DSA. 
Note that the original WP encoding of Dijkstra does not introduce any new symbols, DSA or otherwise, and instead represents, for each post-condition and each path in the program, the final value of each program variable that occurs in the post-condition expressed as an expression over the initial value of the program variables. 
As the number of such paths can be exponential in the program size, additional symbols must be introduced to keep the VC in size polynomial in the input size.

%Consider the example in figure \ref{snippet_1.2}:
%\begin{figure}
%\begin{lstlisting}
%x:=0
%if (*)
	%x := x+1
%if (*)
	%x := x+2
%if (*)
	%x := x+4
%assert x<8
%\end{lstlisting}
%\caption{Example for the use of DSA\\
%The VC for $n$ sequential branches that update a variable is of size proportional to $2^n$ \\
%without DSA and propotional to $n$ with DSA}
%\label{snippet_1.2}
%\end{figure}
%the VC generated by the original WP calculus of Dijkstra (simplifying for associativity and non-deterministic branches) is \\
%$((0+0)+0)+0<8 \land$\\
%$((0+0)+0)+4<8 \land$\\
%$((0+0)+2)+0<8 \land$\\
%$((0+0)+2)+4<8 \land$\\
%$((0+1)+0)+0<8 \land$\\
%$((0+1)+0)+4<8 \land$\\
%$((0+1)+2)+0<8 \land$\\
%$((0+1)+2)+4<8$
%
%\noindent
%The DSA version of the program is shown in figure \ref{snippet_1.2_DSA}
%\begin{figure}
%\begin{lstlisting}
%if (*)
	%x$_1$ := x$_0$+1
%else
	%x$_1$ := x$_0$+0
%if (*)
	%x$_2$ := x$_1$+2
%else
	%x$_2$ := x$_1$+0
%if (*)
	%x$_3$ := x$_2$+4
%else
	%x$_3$ := x$_2$+0
%assert x$_3$<8
%\end{lstlisting}
%\caption{The DSA conversion of \ref{snippet_1.2}}
%\label{snippet_1.2_DSA}
%\end{figure}
%and a possible VC generated using DSA is:\\
%$(\pv{x}_1 = \pv{x}_0+0 \lor \pv{x}_1 = \pv{x}_0+1) \land$\\
%$(\pv{x}_2 = \pv{x}_1+0 \lor \pv{x}_2 = \pv{x}_1+2) \land$\\
%$(\pv{x}_3 = \pv{x}_3+0 \lor \pv{x}_3 = \pv{x}_2+4) \land$\\
%$(\pv{x}_3<8)$

Intuitively, the state of an execution of the program is defined by the current program point and the values of all program variables.
For heap manipulating programs, the heap must be modeled in some way so that the value of the heap at different program points is representable as a FOL term, or we face the same problem of exponential sized WP. 
For example, the Boogie encoding of a program models the heap as an update-able map, for which Boogie emits axioms to the prover - the update-able map behaves as other program variables and hence has several DSA versions.
For assertions that refer to earlier versions of variables - specifically, post-conditions that refer to both the initial and final value of a variable - each earlier version of a variable that is later referenced has to be added to the state.

With this intuition in mind, we can expect that program annotation in the style of Floyd, where each program point is annotated by a formula that describes the set of possible states at that point, will include only the \emph{current} DSA version of each variable.
In terms of FOLE, this means that the only constants that participate in the program annotation at a given program point are the constants that represent the latest DSA versions of each program variable (more care is needed with join points, where we must ensure that each program variable has the same current DSA version on all the predecessors of the join).

We call program annotations that only mention the current DSA version \newdef{scoped annotation}. 
The search space for a scoped annotation depends on the number of variables in the source program, 
while the search space for a non-scoped annotation depends on the number of DSA versions times the number of source program variables - thus looking for only scoped proofs can reduce the proof search-space significantly in some cases.
This reduction is especially significant with some techniques for handling quantifiers, where the number of scoped ground instances of a quantified axiom is much smaller than the number of global ground instances.

Remember that the axioms defining the semantics of statements (whether by Floyd, Hoare or others) always correlate the state before and after the execution of the statement, and hence each axiom instance can relate to more than one program point (generally, assignment and \lstinline|skip| statement axioms refer to two program locations while a binary branch or join axiom refers to three). In addition, each sub-formula of the axiom refers to one specific program location - for example, in the Hoare axiom $\m{Q[x\mapsto v]\{}$\lstinline|x:=v|$\m{\}Q}$, the sub-formula $\m{Q[x\mapsto v]}$ refers to the program point before the statement and $\m{Q}$ refers to the program point after the statement.  
For a scoped proof, this means that we must only consider axiom instances where each sub-formula only contains the DSA versions of variables relevant for the program point it refers to.

The technique we develop in this work can be used to search for a scoped proof, and we mention some logical fragments where this is complete. In other cases we can prioritize the search for a scoped proof over a non-scoped proof or limit the scope in a less strict way while still preserving completeness.


\begin{figure}
\begin{lstlisting}
method m(n : Integer,b:Boolean)
	requires n>0
	
	//new array initialized to all false
	a := new Array[Boolean](n)
	a[0]:=true
	if (b)
		a[0] := false
		j := random(0,n)
		a[j] := true
n$_1$:
	assert $\exists i \cdot 0\leq i < $length(a)$\land$a[$i$]=true
\end{lstlisting}
\caption{Example for the incompleteness of scoped annotation in universal CNF.\\
The \lstinline|random| function is specified as \lstinline|a$\leq$random(a,b)<b|\\
A possible scoped annotation at \lstinline|n$_1$| that is sufficient to prove the assertion is \\
\lstinline|$\exists i \cdot (0\leq i < $length(a)$ \land $a[$i$]=true)|\\
However, this annotation is not in the fragment of universal CNF which is used by many provers, and into which there is a validty preserving conversion from FOLE.\\
A possible non-scoped universal CNF annotation is\\
\indent\lstinline|(b$\Rightarrow ($$0\leq$j$<$length(a) $\land$ a[j]=true)) $\land$ ($\lnot$b $\Rightarrow$ a[0]=true)|\\
There is no scoped annotation for universal CNF
}
\label{snippet_1.4}
\end{figure}

%The program converted to Boogie style is shown in figure \ref{snippet_1.4_Boogie}.
%\begin{figure}
%\begin{lstlisting}
%method m(n : Integer,b:Boolean)
	%assume n>0
	%
	%assume a$\neq$null
	%assume length(a$_0$)$=$n
	%assume $\forall i \cdot (0\leq i < $n$ \Rightarrow $a$_0$[$i$]$=0$
	%if (b){
		%assume $0\leq$j$<$n div 2+1
		%assume a$_1$=a$_0$[j:=5]
	%}else{
		%assume n div 2+1$\leq$k$<$n
		%assume a$_1$=a$_0$[k:=5]
	%}
%n$_1$:
	%assert $\exists i \cdot 0\leq i < $|a|$ \land $a[$i$]=5
	%//negated $\color{gray}{\forall i \cdot 0\leq i < }$|a|$\color{gray}{ \Rightarrow }$a[$\color{gray}{i}$]$\color{gray}{\neq}$ 5
%\end{lstlisting}
%\caption{Example for the incompleteness of scoped annotation - Boogie version\\
%The symbols \lstinline|j,k| are not in scope at \lstinline|n$_1$|\\
%A possible scoped annotation at \lstinline|n$_1$| that is sufficient to prove the assertion is \\
%\lstinline|$\exists i \cdot 0\leq i < $|a|$ \land $a[$i$]=5|\\
%A possible non-scoped annotation is\\
%\indent\lstinline|b$\Rightarrow ($$0\leq$j$<$n div 2+1 $\land$ a$_1$=a$_0$[j:=5]) $\land$|\\
%\indent\lstinline|$\lnot$b $\Rightarrow$ (n div 2+1$\leq$k$<$n$\land$ a$_1$=a$_0$[k:=5])|\\
%There is no annotation for universal clausal FOLE
%}
%\label{snippet_1.4_Boogie}
%\end{figure}
Scoped proofs do not exist for all logical fragments, for example, consider the code in figure \ref{snippet_1.4} - it is easy to see that any scoped annotation at \lstinline|n$_1$| must include an existential quantifier, and hence there is no scoped proof in universal CNF (most calculi used by automated theorem provers do not generate existential conclusions).

We discuss scoped proofs in chapter \ref{chapter:scoping} and also their relation to interpolation. We do not always look for scoped proofs because some logical fragments do not admit a scoped annotation, and sometimes the size of a minimal scoped annotation is significantly larger that of a minimal non-scoped annotation.


\section{Bounded Fragments}
As FOLE is only semi-decidable, and combined with some theories becomes not even recursively enumerable (e.g. linear integer arithmetic with uninterpreted functions and quantifiers as shown in \cite{DBLP:conf/csl/KorovinV07}), 
it is common for program reasoning tools to select decidable fragments of the logic with lower complexity in order to ensure the predictability of the tool. For example, compilers often approximate the possible set of values of variables at each program point using analyses that are guaranteed to terminate quickly, such as constant propagation and definite assignment, for optimization and for reporting warnings and errors to the programmer.

Often, the result of simple and efficient analyses can be used to simplify a program VC, which sometimes shortens verification time.
We take advantage of several such simple analyses and apply them exhaustively after each application of stronger and more expensive fragments - rather than just as a pre-process. 

We verify the program by defining a hierarchy of logical fragments, each of which has a predictable polynomial complexity, and apply these fragments in succession until the program is proven or a user-chosen limit is reached. Our approach differs from refinement methods such as \cite{DBLP:journals/jacm/ClarkeGJLV03} in that the user can select exactly which fragments are applied (rather than depending on a counter-example whose choice is less predictable), 
and so the performance should be more predictable. The intuition is that, while the proof for an entire program VC may be very deep, the actual part of the proof performed at each program point (in e.g. Hoare logic) is often small.
We define the hierarchy of fragments in a way that allows us to search first for small lemmas at program points which combine together to form a proof for the whole program.

%Most theorems proven in mathematics are quantified and are proven from a set of axioms most of which are also quantified.
%For the verification of computer programs, however, a ground VC can be found in several cases.
%The VC generated for a program that contains only assignments, conditionals and ground specifications (assertions, pre- and post-conditions and invariants) is often ground, although quantifiers may be needed in order to model the behavior of arrays and heaps, or to allow under-specification (as abstraction) in method pre- and post-conditions. 
%Hence, it is our experience that a program VC in e.g. CNF form commonly contains a very small proportion of quantified clauses, most which are global axioms.
%
%Theories used in program verification, such as arrays and linear arithmetic, can be axiomatized using quantified axioms, 
%however, efficient decision procedures exist for the ground fragments of several of these theories and the ground fragment is sufficient for proving many programs.
%
%The use of quantified specification in program verification arises in several cases including:
%\begin{itemize}
	%\item Modeling of the data domains of the program or specification, such as sets
	%\item Loop and recursion invariants may require quantifiers even when the rest of the VC is ground
	%\item Modeling non-trivial state of arrays often requires quantification
	%\item Method pre- and post-conditions may use quantifiers in order to abstract away implementation details 
	%\item A verification methodology, such as aliasing control (e.g. using permissions, dynamic frames or ownership), often includes quantified axioms, such as frame rules, and adds certain checks to the VC (e.g. that a memory location specified as not modified is indeed not modified) which require quantification
%\end{itemize}
%
%While decision procedures for several theories have been developed, and active research	exists for other theories, 
%the ability of an intermediate verification language to handle arbitrary quantified axioms greatly increases its generality and the ability to prototype, experiment and use theories for which no efficient decision procedures is implemented.


\textbf{Bounded terms:}
The intuition for limiting the size of terms comes from axiomatizations of recursive abstract predicates used in specification (e.g. for modeling recursive data structures as in \cite{DBLP:conf/ecoop/HeuleKMS13}) and generally from proofs of heap manipulating programs. Often, the recursive definition of a predicate is given as an axiom which, when instantiated, allows to \emph{unfold} the definition of the predicate once - e.g. for a predicate defining the validity of a recursive search tree, instantiating the defining axiom for a node produces the instances of the predicate for the node's children. Such an axiomatization is not complete in FOLE, but is sufficient in many cases. In our experience, proofs for such structures do not often require an arbitrary depth of unfolding of the predicate, and hence limiting instantiation to instances with terms that are not very distant (in terms of number of function applications) from input VC terms, should allow us to look first for proofs that do not look in the heap much deeper than the actual program does. 
We define a measure of term depth that is relative to the set of original VC terms, and also takes into account any equalities derived for the terms. Using this measure, we define a hierarchy of logical fragments where each fragment extends the limit on the relative size of terms.

\textbf{Bounded derivation depth:}
We try to prioritize the search for simpler proofs over the search for complicated proofs, and hence we limit the shape of the proof DAG by classifying inferences according to a cost measure, and limiting the number of inferences of each class in each path in the proof DAG. Inferences that strictly reduce the VC size and are cheap to perform (such as unit propagation), are not limited.
Other more expensive inferences (such as ground superposition) are limited for each fragment by the maximal number allowed on each proof DAG path. The most expensive inferences, such as superposition of non-unit clauses with two non-ground premises where the conclusion has many free variables, are restricted more. Using the maximal number in a proof DAG path is a compositional measure in the sense that it is easy to calculate the measure for each node of a proof DAG from the measure of its immediate children.

We also use a bound on the number of literals allowed in a clause, in order to prevent some cases of combinatorial explosion.
%\section{Heaps}
	Programs using the heap have always posed a challenge for analysis and verification.
	Potential aliasing between two different heap access paths is a generally undecidable problem and for many interesting subsets and over-approximations of the problem computationally very hard.
	As proofs of bigger programs are often intractable and whole program techniques often do not scale, a compositional approach is followed by many tools.
	In order for a modular verifier to be able to reason about method calls without using the method body, some form of aliasing specification must be provided or inferred - so several aliasing specification methodologies have been developed, such as ownership types \cite{DBLP:conf/ecoop/ClarkeNP01},\cite{DBLP:books/sp/Muller02}, separation logic \cite{DBLP:conf/lics/Reynolds02}, permissions \cite{DBLP:conf/sas/Boyland03} and dynamic frames \cite{DBLP:conf/fm/Kassios06},\cite{DBLP:conf/lpar/Leino10}.
Heaps are often represented as a map from a location and sometimes a field identifier to a locations or value and so we can expect to benefit from specialized reasoning procedures about non-extensional maps, which are supported by some provers (in our experience extensionality is not often needed for heaps, as opposed to e.g. sequences and sets).

	In the DSA form of a program VC, every assignment to the heap, and, in general, every entry and return to and from a heap manipulating method or loop body, adds a new DSA version of the whole heap, and specifies which locations in the heap are modified and which are preserved using some form of a frame axiom. For heap writes the classical axioms (\cite{Mccarthy62towardsa}) are often used and some verifiers include a decision procedure for the array theory which can be used for heaps. For other cases, the choice of frame axioms depend on the methodology used.
	For theorem provers, heap axioms pose a challenge as, for programs that manipulate the heap intensively, a long sequence of instances of heap axioms are needed in order to find the possible values of a heap location. 
	Such deep instances are usually avoided for general axioms as they enlarge the proof search space significantly and it is often not clear how deep a tool should look for the instances of one axiom vs. another axiom, but for heap axioms the number of instances is limited by the number of heap operations on any path in the program CFG.
	
		We developed a technique for dedicated handling of the heap which is aware of the interaction between heap manipulation and program flow. The technique allows the instantiation of heap axioms on demand (when a heap-read term is produced by another fragment) and ensures that the necessary axioms are instantiated for each such heap-read. As opposed to techniques used in symbolic execution, 
		handles the first two cases by eagerly applying map read-over-write axioms for heap reads that appear in the program and propagating heap information along the CFG accordingly - for example, a read of a heap location in the post-condition of a method (and hence the last DSA version of the heap) will traverse the DSA chain of the heap for that heap location, using all derived known equalities and dis-equalities on heap-locations and field ids to find the all possible values for that heap location, or the equivalent DSA versions where some property was assumed on that heap location. For framing we detect the framing axioms and apply them when propagating heap information even if the instantiation or superposition logic does not.\\
		Arbitrary quantification on heap locations (and, if needed, arbitrary heap operations) are handled using axioms and superposition.\\
		The technique can be applied at any stage of the proving process and hence is interleaved with other prover steps, as opposed to being applied only as a pre-process.
	
	
	
	
	The verification conditions generated from a program and specification in one of these methodologies includes several elements:
	\begin{itemize}
		\item The direct encoding of original programming language heap access - that is, direct reads and writes of the heap by the program - these can be handled, in general, quite well by a verifier as long as no loops are involved. 
		A problem we have encountered is that determining the value of a heap location at a later stage of a program which has paths with many heap-writes requires many quantifier instantiations of the heap read-over-write axiom. Some SMT solvers limit the instantiation depth of a term (a term is less likely to be used in instantiating a quantified axiom if more instantiations were needed to generate the term)
		\item Assertions about specific heap locations - pre- and post- conditions, loop invariants and method call and return specifications can all introduce assumptions about a specific heap location, without ever assigning it (e.g. in an OOP language a method pre-condition often includes the assertion that \lstinline|this!=null|) - thus some heap locations are accessed without ever being assigned, with only partial information
		\item Assertions that quantify over heap locations - assertions usually quantify over only a portion of the heap (e.g. the set of nodes of one linked list), the exception being assertions or axioms that refer to global properties such as type-correctness, but modular verification methodologies often try to avoid such quantification. References to a portion of the heap require a description of the portion, such as a set of heap locations - for example, an assertion that states that all values in a given integer linked list are positive needs some way to describe the set of locations that belong to the linked list. Some methodologies use a direct encoding of the set of heap locations (such as Dynamic Frames) while other methodologies use recursive predicates and functions to describe a subset of the heap locations.
		For a VC, supporting a directly represented set of heap locations requires handling some set operations, either as a native theory or using axioms. 
		Recursive predicates and functions cannot be axiomatized directly in FOLE as some form of transitive closure is needed. Some verification methodologies infer, or require users to specify, when to instantiate (\emph{unfold}) the axiom defining the recursive predicate, so that the encoding for the ATP is simple. 
		Axioms that describe recursive predicates and functions in FOLE without transitive closure are limited in that there are elements for which they cannot be used to prove that the element does not belong to the set described by the recursive predicate, however these axioms can be used effectively to prove positive properties about heap portions.
	\item Framing - modular verification methodologies need some way to abstract and specify the heap behaviour of method calls - essentially correlating the values of the heap in the pre-state with the one in the post-state.
		The most basic part of this specification is specifying which parts of the heap are not modified by a call, and each verifier has to encode this non-modification in the VC.
		In all the examples we have inspected, including Chalice, Spec\# and Dafny, the frame axiom follows the same template, which is of the form $\forall x,f \cdot \m{P}(x,f) \Rightarrow \m{H_0}(x,f)=\m{H_1}(x,f)$ where $x,f$ is a heap location and field-id and \m{H_0,H_1} are the pre- and post-heaps, respectively (in some cases \m{H_0,H_1} are also quantified in a global axiom).
		\end{itemize}

%\section{Proof direction}

\subsection*{preliminaries}
Several systems have been suggested for proving the correctness of a program relative to a specification.
Hoare's axiomatic semantics ~\cite{DBLP:journals/cacm/Hoare69} is a deductive system for showing program correctness and most current systems are, to some degree, based on it.

The system is parametric in the underlying assertion logic (the logic used to prove assertions on a single program state).
The rule that shows this parametericity is the rule of consequence, which uses implication of the assertion logic in the premise of the rule. As we are not interested in developing a proof tool for the underlying logic (which, in our case means FOL), we will be using a backend proving tool for that logic and refer to it as an oracle for that logic - thus we are a-priori limited only to proofs in which all implications used in the rule of consequence are provable by our oracle.

In this system, as in many other deductive systems, a single program can have several different valid proofs.
As our objective is to find such proofs quickly, we would like to be able to reduce the proof-search-space in order to find proofs faster, while still being able to find a proof for every (relatively) provable program.
(we call a program relatively provable when it has a Hoare proof in which all implications used in the rule of consequence are provable by our oracle).

Given a set of valid proofs for the same program, there are some obvious characteristics by which we could choose which proof we prefer:
\begin{itemize}
	\item Size: measured by the number of nodes in the proof tree and or the total size of terms in the proof
	\item Redundancy: some deduction steps can be eliminated while still leaving the proof valid (the exact definition is more complicated, as extra deduction steps can add unnecessary information (conjuncts) to a rule premise)
	\item Oracle complexity: some assertion-logic theorems take much longer to prove on some oracles - a proof with simpler assertion-logic parts might be preferable (for example, SMT solvers are usually weak on quantifier instantiation, while resolution based solvers without a dedicated integer engine are often weak on arithmetic) - as long as it doesn't take that much longer to find.
\end{itemize}

A search strategy that only covers an "optimal" (in some of the above or other senses) proof might be expected to find such a proof faster if it exists.

For example, consider the following program:
\begin{lstlisting}
	x := y * y + 5
	z := x + 1
	assert z>0
\end{lstlisting}		

we could come up with the following Hoare proof:
\begin{lstlisting}
	{}
	^$\models$^
	{^$y \times y+5=y \times y+5$^}
	x := y*y+5
	{^$x=y \times y+5$^}
	^$\models$^
	{^$x+1=x+1 \land x=y \times y+5$^}
	z := x+1
	{^$z=x+1 \land x=y \times y+5 $^}
	^$\models$^
	{^$z>0$^}
	assert z>0
\end{lstlisting}		

Or this one:
\begin{lstlisting}
	{}
	^$\models$^
	{^$y \times y+5+1>0$^}
	x := y*y+5
	{^$x+1>0$^}
	z := x+1
	{^$z>0$^}
	assert z>0
\end{lstlisting}		

if we were to write the wlp of this program we would get:
$y \times y + 5 + 1>0$ which is in some sense the optimal formula to send to the theorem prover.

However, consider this case:
\begin{lstlisting}
	x := y * y + 5
	if (b)
		x := x+1
	else
		y := y+1
	z := x + 1
	assert z>0
\end{lstlisting}		

The wlp in this case is:
$\lnot b \implies y \times y + 5 + 1>0 \land b \implies y \times y + 5 + 1 + 1>0$

Which is not optimal.

Boogie style (passive program,DSA) we would get: \\
$
\begin{array}{lll}
\mathbf{let}\ J0_{Ok} = x_1+1>0 & &\mathbf{in}\\
\mathbf{let}\ T0_{Ok} = x_1=x_0+1 &\Rightarrow J0_{Ok} & \mathbf{in} \\
\mathbf{let}\ E0_{Ok} = x_1=x_0   \Rightarrow y_1=y_0+1 &\Rightarrow J0_{Ok} &\mathbf{in}\\
\mathbf{let}\ B0_{Ok} = x_0 = y_0 \times y_0 + 5 &\Rightarrow T0_{Ok} \land E0_{Ok} &\mathbf{in}\\
B0_{Ok}
\end{array}
$

%$\mathbf{let}\ J0_{Ok} = x_1+1>0 \ \mathbf{in} $ \\
%$\mathbf{let}\ T0_{Ok} = x_1=x_0+1 \implies J0_{Ok} \ \mathbf{in} $ \\
%$\mathbf{let}\ E0_{Ok} = x_1=x_0   \implies y_1=y_0+1 \implies J0_{Ok} \ \mathbf{in} $ \\
%$\mathbf{let}\ B0_{Ok} = x_0 = y_0 \times y_0 + 5 \implies T0_{Ok} \land E0_{Ok} \ \mathbf{in}$ \\
%$B0_{Ok}$ \\


The second proof is more concise and more goal-directed in the sense that it propagates to each point in the program the minimum information needed there for the proof.

A common search strategy is to lever the generic search strategy in the oracle, by translating the problem from the program logic ( Hoare or equivalent) to the assertion-logic - thus generating the verification condition (VC) for the program.
We assume the specification (user assertions, pre+post conditions of called and verified method, and methodology specific assertions), are all part of the program, encoded as assertions in our intermediate language.

The most common (and first) way of encoding the VC is using some flavour of wp-semantics ~\cite{Dijkstra:1975:GCN:360933.360975}.
A dual method would be to calculate the strongest postcondition of the program.

Another approach that is used to prove (and infer) properties of programs is static analysis, which could be thought of, in broad terms, as calculating a kind of strongest post condition (with a very specific and fixed strategy for guessing loop invariants and for abstracting at join points) over a much smaller fragment of the assertion logic (e.g. linear inequalities with or without disjunction).
These analyses are often focused on a specific domain (e.g. linear arithmetic, arrays, heaps (shape analysis)) and would be strong and fast in proving properties over this domain which is in their fragment, but not much else, and so choosing the fragment and strategy are crucial for a working tool, while a general theorem prover can handle a broader range of problems but often in less depth.
Some theorem provers, especially SMT solvers, incorporate specialized solvers for specific domains and fragments but they rely on the encoding of the problem into a VC to utilize this specialized solver, and hence some information that is explicit in the program (CFG structure, DSA relations between variables, scope) becomes implicit in the VC and so the theorem prover has to apply generic search strategies to try and prove the VC.
Some static analyses also work backwards (similar to wp-semantics).
Some theorem provers can perform a sort of inference using logical interpolation, but a VCG tool usually needs user supplied invariants.

A third common method is symbolic execution ~\cite{Boyer:1975:SFS:390016.808445} - where a program is executed with symbolic values for the inputs, branching the execution on program branches.
A symbolic execution engine collects the information in the program in a manner similar to sp-semantics, except that as it branches execution on branches it does not have to join disjunctive information at join points in the CFG.
A symbolic execution engine can maintain a symbolic version of the program heap at each location, and hence have a dedicated heap domain which can be more efficient than the FOL representation.
The engine can query a theorem prover (or solver for some domain), when the symbolic state affects the program execution: e.g. the  control flow - check if a branch/loop condition is feasible, heap - check if an updated heap location is aliased, assertion - check if an assertion holds.
Symbolic execution can sometimes also be performed backwards.


\subsubsection*{Direction}
The main advantages in analyzing a program forwards:
\begin{itemize}
	\item The transformer (predicate/abstract etc) is often simpler than for backwards analysis - as the actual program is run forward
	\item Partial results are useful - for example, a run of symbolic execution or forward static analysis can prove some of the assertions in a program and also generate some (partial) invariants for certain program locations - these can be used to simplify the problem, run it through another tool or annotate the program.
\end{itemize}

The main advantages in analyzing a program backwards:
\begin{itemize}
	\item The analysis is goal directed - information which can be determined not to be relevant for the proof is not taken into account.
	This can eliminate a large part of the program or state-space or axioms
\end{itemize}


In both cases, using some reasoning tool on intermediate states can be helpful in reducing the search space - for example:
In forward analysis, if we encounter a heap assignment, we can try and determine which other heap locations the assigned location can alias. Without this information, we must assume that any location on the heap could be written and so we would have a larger and more complicated (disjunctive) state.
In both directions, if we can deduce that a certain branch is infeasible, we can disregard that branch.

When generating a VC (either wp or sp), the program flow direction is only implicitly encoded.
If a goal directed theorem prover is used for proving the VC, and the assertions are marked as goals, then at least some of the benefit of backward analysis can be obtained.

Our aim is to try and combine forwards and backwards elements in our tool in order to gain the benefits of both.

For example:
\begin{lstlisting}
function f1()
{
	assume x!=y
	assume x!=z
	x.f = 1
	
	if (c)
		y.f = -1
	else
		z.f = -2
		
	assert x.f>0
}
\end{lstlisting}		

with the DSA form:

\begin{lstlisting}
function f1()
{
	assume x!=y
	assume x!=z
	assume h0[x,f] = 1          //mapRead(h0,x,f) = 1
	
	if (c)
		assume h1 = h0[y.f := -1] //h1 = mapWrite(h0,y,f,-1)
	else
		assume h1 = h0[z.f := -2] //h1 = mapWrite(h0,z,f,-2)
		
	assert h1[x.f]>0            //mapRead(h1,x,f) > 0
}
\end{lstlisting}		

Here we are only interested in the value of \lstinline{x.f}, but we do not know that \expression{x \neq y} and the weakest precondition is (bold conjuncts are goal conjuncts): \\
	$x \neq y$ \\
	$x \neq z$ \\
	$mapRead(h_0,x,f) = 1 $ \\
	$(\lnot c \implies h1 = mapWrite(h_0,z,f,-2) $ \\
	$(      c \implies h1 = mapWrite(h_0,y,f,-1) $ \\
	$\mathbf{\lnot mapRead(h_1,x,f) > 0} $

and with many heap assignments it grows larger and larger, while a very simple forward analysis would generate, after the branch (with the negated assertion):\\
	$x \neq y $ \\
	$x \neq z $ \\
	$h[x,f] = 1 $ \\
	$(\lnot c \implies h[z,f]=-2)$ \\
	$(      c \implies h[y,f]=-1)$ \\
	$\mathbf{\lnot h[x,f] > 0}$

We aim at a more lazy tool, that will first get to the assertion $\mathbf{\lnot h_1[x,f] > 0}$, and then will try to get information about the value of the relevant ground terms (int this case: \{ $h_1$,$x$,$f$,$h_1[x,f]$ \}
The information we get for a specific program location is only information implied by its predecessors - so we get: \\
	for $x$ : \\
	$x \neq y $ \\
	$x \neq z $ \\
	for $h_1[x,f]$ : \\
	we query down the line and get to line 8: \\
	$h_1[x,f] = h_0[y.f := -1][x,f]$ \\
	Now we know (from the map axioms) that $x \neq y$ or $x = y$ are needed - so we query further and get: \\
	$x \neq y $ \\
	And now we have \\
	$h_1[x,f] = h_0[x,f]$ \\
	We query about  \\
	$h_0[x,f]$ \\
	And get \\
	$h_0[x,f]=1$ \\
	And so (congruence closure): \\
	$h_1[x,f]=1$
	
	The same for line 10 we get: \\
	$h1[x,f]=1$
	
	and so we join these to get at line 11: \\
	$h1[x,f]=1$ \\
	And now with \\
	$\lnot h1[x,f]>0$ \\
	we get \\
	$\lnot 1>0$ \\ 
	Which an integer engine could discharge 

\subsubsection*{note}
A note regarding wp vs wlp and partial vs total correctness: as our intermediate representation does not include loops, the only way a program may not terminate (which would mean that some designated set of terminal states is not reachable) is if the CFG is initially not connected to the designated end states or if some combination of \lstinline{assume} statements makes the end states unreachable.
We do not consider non-termination and hence are only interested in partial correctness / wlp (note that a frontend tool could encode a variant check after the loop body (and also recursion variants), hence we are able to handle termination, but are oblivious to it).


