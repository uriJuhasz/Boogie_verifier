\subsection{Ordering for Interpolation}
In this section we describe how we establish an adequate total order on ground terms for interpolation for the whole CFG.

Extending our interpolation calculus to a DAG CFG requires that we have a separating total order on all program constants, so that whenever a constant in scope at a CFG-node is not in scope in the successor, it is separated from the any constant that is in scope in the successor.\\
For a linear program, we can simply assign decreasing limit ordinals to each CFG-node, so that the root node has the maximal limit ordinal and each successor has a lower limit ordinal. We then assign each constant the weight \m{n\omega+1} where n is the limit ordinal index of the last CFG-node where it appears.\\
However, this solution does not extend immediately to tree-shaped CFGs, as different branches may imply different orderings. Consider, for example, figure \ref{snippet4.5.1}.

\begin{figure}
\begin{lstlisting}
$\m{n_1:}$
	assume a$\neq$b
if (*)
	$\m{n_2:}$
		assume f(a)=f(b)
	$\m{n_3:}$
		assert f(a)=c
else
	$\m{n_4:}$
		assume f(a)$\neq$f(b)
	$\m{n_5:}$
		assert f(b)=c
\end{lstlisting}
\caption{Interpolation ordering for trees}
\label{snippet4.5.1}
\end{figure}

The \lstinline|then| branch implies that \m{b\succsep a} while the \lstinline|else|  branch implies that \m{a \succsep b}.
There is no total order with minimal scoping that respects the separating constraint for this program.

Our solution to this problem is to add a further DSA version for every constant at a branch that is in scope in at least one successor of the branch, unless it is in scope at the join of the branch.
For figure \ref{snippet4.5.1} the modified program is shown in figure \ref{snippet4.5.1b}.

\begin{figure}
\begin{lstlisting}
$\m{n_1:}$
	assume a$\neq$b
if (*)
	$\m{n_2.0:}$
		assume a$_1$=a
		assume b$_1$=b
	$\m{n_2:}$
		assume f(a$_1$)=f(b$_1$)
	$\m{n_3:}$
		assert f(a$_1$)=c
else
	$\m{n_4.0:}$
		assume a$_2$=a
		assume b$_2$=b
	$\m{n_4:}$
		assume f(a$_2$)$\neq$f(b$_2$)
	$\m{n_5:}$
		assert f(b$_2$)=c
\end{lstlisting}
\caption{Interpolation ordering for trees - additional DSA versions}
\label{snippet4.5.1b}
\end{figure}

After this transformation, and after ensuring that the scope for each constant is contiguous (by renaming one instance of a constant that appears on two parallel branches but neither at branch nor join point), we can choose any topological order of the nodes and assign limit ordinal indices accordingly.
