\section{Basics}
We use scoping only on constants, as opposed to all function symbols, mostly because program variables usually translate to constants.
We can convert any program VC to be constant based by converting each \fa{f}{t} to \m{apply_{\arity{f}}(c_f.\tup{t})} where, for each arity n, \m{apply_n} is a new function symbol of arity n+1, and for each non-constant function symbol f, \m{c_f} is is a new constant symbol - after the transformation we can apply scoping only to constants with a similar effect to scoping on all functions on the original VC (similar to the transformation in \cite{NieuwenhuisOliveras03}).

\subsection{Notation}
We start with a few definitions:\\
We use the names forward and backward scope to denote the direction in which the scope is calculated - forward scope is intuitively the set of initialized variables and backward scope is intuitively the set of live variables.

\noindent
\textbf{Scopes:}\\
The \textcolor{blue}{local scope} of a CFG-node is the set of constants that appear in clauses at the node:\\
\m{S^0_n \triangleq \s{c \in C_{\Sigma} \mid c \trianglelefteq \clauses{n}}}\\
(\clauses{n} is the initial set of clauses at the CFG-node n).

The \textcolor{blue}{forward scope} of a CFG-node is the set of constants that appear at a CFG-node or its transitive predecessors:

\bigskip
\m{S^F_n \triangleq \bigcup\limits_{p \in \predst{n}} S^0_p ~~(= S^0_n \cup \bigcup\limits_{p \in \preds{n}} S^F_p)}\\
\bigskip

\noindent
Symmetrically,  the \textcolor{blue}{backward scope} of a node are all constants used in transitive successors:

\bigskip
\m{S^B_n \triangleq \bigcup\limits_{s \in \succst{n}} S^0_s ~~(= S^0_n \cup \bigcup\limits_{s \in \succs{n}} S^B_s)}\\
\bigskip

\noindent
And finally the \textcolor{blue}{minimal scope} of a node:

\bigskip
\m{S^{M}_n \triangleq S^F_n \cap S^B_n ~~(\supseteq S^0_n)}\\
\bigskip

\noindent
The definitions above correspond to a program variable being initialized 
(forward scope) and alive (backward scope).
The minimal scope is the set of program locations in which a variable is in scope in most programming languages.
It is easy to see that in our verification algorithm each CFG-node uses at most the forward-scope.
Our objective is a small proof that we can find quickly, and scoping is only a tool for reducing the proof search space, so that we will not always insist on the minimal scope.

\bigskip

\noindent
\textbf{Interpolation:}
\begin{definition}{\textcolor{blue}{language} - \textcolor{blue}{\lang{S}}}

\noindent
For a set of clauses S we define the language of S, \lang{S}, to be the set of constants that appear in S - formally:\\
\m{\lang{S}\triangleq\s{c \mid c() \trianglelefteq S}}.\\
A term, atom, literal, clause or set of clauses are in the language \lang{} iff all the constants that appear in it are in \lang{}, formally:\\
\m{C \in \lang{} \equivdef \forall c \in \consts{} \cdot c() \trianglelefteq C \Rightarrow c \in \mathfrak{L}}.\\
We use \lang{} both for the set of constants and the set of terms, atoms, literals, clauses and sets of clauses in the language,
as long as there is no ambiguity.
\end{definition}

We begin by discussing \textcolor{blue}{binary interpolation} - given two sets of clauses A,B, s.t. \m{A
\cup B \models \emptyClause}, our objective is to find a set of clauses I s.t. \\
\m{A \models I}, \m{I \cup B \models \emptyClause} and \m{I \in \lang{A} \cap \lang{B}}. \\
This problem models the interaction between two consecutive CFG-nodes in order to find a refutation on a path they share. Later we discuss how this extends to the general problem of finding an interpolant for each CFG-node that is sufficient to prove all assertions on all paths that pass through that CFG-node.\\
Ground first order logic with equality admits interpolation (shown e.g. in ~\cite{McMillan04}).

We name the sets of clauses for binary interpolation the top (A) and bottom (B) sets (according to the CFG direction).\\
We use \m{\textcolor{blue}{N^t_0}} for the top set and \m{\textcolor{blue}{N^b_0}} for the bottom set.\\
Theses sets are the initial interpolation problem, but during our interpolation and proof process these sets evolve,
and we use \m{\textcolor{blue}{N^t},\textcolor{blue}{N^b}} when describing the current sets of clauses during an algorithm run.\\
We use \textcolor{blue}{N} for \m{N^t \cup N^b} and similarly \m{\textcolor{blue}{N_0}} for \m{N^t_0 \cup N^b_0}.\\
We use \m{\textcolor{blue}{\langt},\textcolor{blue}{\langb}} for the language of \m{N^t_0,N^b_0} respectively, (top,bottom language), and also:\\
The interface language \textcolor{blue}{\langI} is defined:\\
\m{\langI \triangleq \langt \cap \langb}\\
And the private languages are:\\
\m{\textcolor{blue}{\langtp} \triangleq \langt \setminus \langb}\\
\m{\textcolor{blue}{\langbp} \triangleq \langb \setminus \langt}

\bigskip

\noindent
In our terminology, the binary interpolation problem is:\\
Given \m{N^t_0,N^b_0} find \m{I \in \langI} s.t. \m{N^t_0 \models I} and if \m{N^t_0,N^b_0 \models \emptyClause} then \m{I \cup N^b_0 \models \emptyClause}.


For interpolation in the CFG: we look at any path P in a program CFG s.t. P=Q.p.n.R  (we use . for path concatenation), Q begins at the root, p,n are consecutive CFG-nodes, R ends at an assertion node and Q,R are possibly empty sub-paths.
In our CFG, the assertion at the end of the path holds on the path iff \m{\clauses{P} \models \emptyClause} (that is, the set of all clauses on the path is inconsistent).
Because ground FOL admits interpolation, if \clauses{P} is ground there is a ground interpolant
\m{I \in \lang{\clauses{Q.p}} \cap \lang{\clauses{n.R}}} s.t. \\
\m{\clauses{Q.p} \models I} and \m{\clauses{n.R} \cup I \models \emptyClause}.\\
By definition we can see that \\
\m{\lang{\clauses{Q.p}} \cap \lang{\clauses{n.R}} \subseteq S^{M}_p \cap S^{M}_n}.\\
Moreover, we can see that:\\
\m{S^{M}_n = S^0_n \cup \bigcup\limits_{p \in \preds{n}} (S^M_p \cap S^M_n) \cup \bigcup\limits_{s \in \succs{n}} (S^M_s \cap S^M_n)}
\\
In other words, \m{S^{M}_n} is exactly the minimal language guaranteed to be sufficient to represent the interpolant between the prefix and suffix of \emph{any} path from the root to an assertion through n.
For this reason, we do not consider any scope that is smaller than \m{S^{M}_n} and we would sometimes select a larger scope, either for performance or for the non-ground case. Note that each path through the node might have a different set of possible interpolants, and there may be an exponential number of such paths.

\textbf{Theory symbols:}\\
In some cases, we want some symbols to be global even if they don't occur in all CFG-nodes - for example, 
for linear integer arithmetic, we want the number constants to be in the language of all CFG-nodes.\\
In addition, if we have some background axioms (e.g. from a verification methodology, heap modeling, etc.), we would want all constants that occur in the axioms to be global.
