\chapter{Scoping and Interpolation}\label{chapter:scoping}
We have referred to scoping several times in the previous chapters, and given some examples that hint at the motivation. 
In this chapter we start by defining scoping in our context, and show how we determine the scope of each CFG-node.
We continue by showing an algorithm for generating an interpolant for two sets of unit ground equality clauses - interpolants are closely related to scoping and are needed to achieve verification completeness.
Next we discuss the changes needed to the algorithm to make it applicable to inteprolants for sequences and trees.
We then extend the algorithm to non-unit clauses and DAG-shaped CFGs, generalizing the previous algorithms.

\textbf{Motivation:}
The basic intuition behind using scoping in verification is that of a search for local, small proofs that are composed into a whole-program proof.
Scoping exists in many programming languages which define which program variables are accessible at which program point.
A Hoare annotation of a program usually does not include a program variable name at any point in the program where the variable is not in scope - that is, before its declaration or at a parallel CFG-node (opposite branch) to the declaration.
For a passive program after DSA transformation, scoping can have an even stronger effect, as at most two DSA versions of each program variable can be in scope at the same time. In general, scoping reduces the proof-search-space for the prover at each CFG-node, but in some extreme cases can admit only proofs exponentially longer than the minimal non-local proof.

If we restrict the signature of the clauses that can occur at each CFG-node, 
we lose the completeness guarantee offered by the superposition calculus as even ground superposition is not complete for scoped proofs (local proofs). Several techniques have been developed for deriving interpolants from scoped superposition proofs, and in this chapter we show how we adapt some of these techniques to restore completeness for the ground fragment, and improve the coverage of the non-ground fragment. We also suggest a method to calculate a scope for each CFG-node that is not minimal, but sufficient to preserve completeness.

\section{Basics}
We use scoping only on constants, as opposed to all function symbols, mostly because program variables usually translate to constants.
We can convert any program VC to be constant based by converting each \fa{f}{t} to \m{apply_{\arity{f}}(c_f.\tup{t})} where, for each arity n, \m{apply_n} is a new function symbol of arity n+1, and for each non-constant function symbol f, \m{c_f} is is a new constant symbol - after the transformation we can apply scoping only to constants with a similar effect to scoping on all functions on the original VC (similar to the transformation in \cite{NieuwenhuisOliveras03}).

\subsection{Notation}
We start with a few definitions:\\
We use the names forward and backward scope to denote the direction in which the scope is calculated - forward scope is intuitively the set of initialized variables and backward scope is intuitively the set of live variables.

\noindent
\textbf{Scopes:}\\
The \textcolor{blue}{local scope} of a CFG-node is the set of constants that appear in clauses at the node:\\
\m{S^0_n \triangleq \s{c \in C_{\Sigma} \mid c \trianglelefteq \clauses{n}}}\\
(\clauses{n} is the initial set of clauses at the CFG-node n).

The \textcolor{blue}{forward scope} of a CFG-node is the set of constants that appear at a CFG-node or its transitive predecessors:

\bigskip
\m{S^F_n \triangleq \bigcup\limits_{p \in \predst{n}} S^0_p ~~(= S^0_n \cup \bigcup\limits_{p \in \preds{n}} S^F_p)}\\
\bigskip

\noindent
Symmetrically,  the \textcolor{blue}{backward scope} of a node are all constants used in transitive successors:

\bigskip
\m{S^B_n \triangleq \bigcup\limits_{s \in \succst{n}} S^0_s ~~(= S^0_n \cup \bigcup\limits_{s \in \succs{n}} S^B_s)}\\
\bigskip

\noindent
And finally the \textcolor{blue}{minimal scope} of a node:

\bigskip
\m{S^{M}_n \triangleq S^F_n \cap S^B_n ~~(\supseteq S^0_n)}\\
\bigskip

\noindent
The definitions above correspond to a program variable being initialized 
(forward scope) and alive (backward scope).
The minimal scope is the set of program locations in which a variable is in scope in most programming languages.
It is easy to see that in our verification algorithm each CFG-node uses at most the forward-scope.
Our objective is a small proof that we can find quickly, and scoping is only a tool for reducing the proof search space, so that we will not always insist on the minimal scope.

\bigskip

\noindent
\textbf{Interpolation:}
\begin{definition}{\textcolor{blue}{language} - \textcolor{blue}{\lang{S}}}

\noindent
For a set of clauses S we define the language of S, \lang{S}, to be the set of constants that appear in S - formally:\\
\m{\lang{S}\triangleq\s{c \mid c() \trianglelefteq S}}.\\
A term, atom, literal, clause or set of clauses are in the language \lang{} iff all the constants that appear in it are in \lang{}, formally:\\
\m{C \in \lang{} \equivdef \forall c \in \consts{} \cdot c() \trianglelefteq C \Rightarrow c \in \mathfrak{L}}.\\
We use \lang{} both for the set of constants and the set of terms, atoms, literals, clauses and sets of clauses in the language,
as long as there is no ambiguity.
\end{definition}

We begin by discussing \textcolor{blue}{binary interpolation} - given two sets of clauses A,B, s.t. \m{A
\cup B \models \emptyClause}, our objective is to find a set of clauses I s.t. \\
\m{A \models I}, \m{I \cup B \models \emptyClause} and \m{I \in \lang{A} \cap \lang{B}}. \\
This problem models the interaction between two consecutive CFG-nodes in order to find a refutation on a path they share. Later we discuss how this extends to the general problem of finding an interpolant for each CFG-node that is sufficient to prove all assertions on all paths that pass through that CFG-node.\\
Ground first order logic with equality admits interpolation (shown e.g. in ~\cite{McMillan04}).

We name the sets of clauses for binary interpolation the top (A) and bottom (B) sets (according to the CFG direction).\\
We use \m{\textcolor{blue}{N^t_0}} for the top set and \m{\textcolor{blue}{N^b_0}} for the bottom set.\\
Theses sets are the initial interpolation problem, but during our interpolation and proof process these sets evolve,
and we use \m{\textcolor{blue}{N^t},\textcolor{blue}{N^b}} when describing the current sets of clauses during an algorithm run.\\
We use \textcolor{blue}{N} for \m{N^t \cup N^b} and similarly \m{\textcolor{blue}{N_0}} for \m{N^t_0 \cup N^b_0}.\\
We use \m{\textcolor{blue}{\langt},\textcolor{blue}{\langb}} for the language of \m{N^t_0,N^b_0} respectively, (top,bottom language), and also:\\
The interface language \textcolor{blue}{\langI} is defined:\\
\m{\langI \triangleq \langt \cap \langb}\\
And the private languages are:\\
\m{\textcolor{blue}{\langtp} \triangleq \langt \setminus \langb}\\
\m{\textcolor{blue}{\langbp} \triangleq \langb \setminus \langt}

\bigskip

\noindent
In our terminology, the binary interpolation problem is:\\
Given \m{N^t_0,N^b_0} find \m{I \in \langI} s.t. \m{N^t_0 \models I} and if \m{N^t_0,N^b_0 \models \emptyClause} then \m{I \cup N^b_0 \models \emptyClause}.


For interpolation in the CFG: we look at any path P in a program CFG s.t. P=Q.p.n.R  (we use . for path concatenation), Q begins at the root, p,n are consecutive CFG-nodes, R ends at an assertion node and Q,R are possibly empty sub-paths.
In our CFG, the assertion at the end of the path holds on the path iff \m{\clauses{P} \models \emptyClause} (that is, the set of all clauses on the path is inconsistent).
Because ground FOL admits interpolation, if \clauses{P} is ground there is a ground interpolant
\m{I \in \lang{\clauses{Q.p}} \cap \lang{\clauses{n.R}}} s.t. \\
\m{\clauses{Q.p} \models I} and \m{\clauses{n.R} \cup I \models \emptyClause}.\\
By definition we can see that \\
\m{\lang{\clauses{Q.p}} \cap \lang{\clauses{n.R}} \subseteq S^{M}_p \cap S^{M}_n}.\\
Moreover, we can see that:\\
\m{S^{M}_n = S^0_n \cup \bigcup\limits_{p \in \preds{n}} (S^M_p \cap S^M_n) \cup \bigcup\limits_{s \in \succs{n}} (S^M_s \cap S^M_n)}
\\
In other words, \m{S^{M}_n} is exactly the minimal language guaranteed to be sufficient to represent the interpolant between the prefix and suffix of \emph{any} path from the root to an assertion through n.
For this reason, we do not consider any scope that is smaller than \m{S^{M}_n} and we would sometimes select a larger scope, either for performance or for the non-ground case. Note that each path through the node might have a different set of possible interpolants, and there may be an exponential number of such paths.

\textbf{Theory symbols:}\\
In some cases, we want some symbols to be global even if they don't occur in all CFG-nodes - for example, 
for linear integer arithmetic, we want the number constants to be in the language of all CFG-nodes.\\
In addition, if we have some background axioms (e.g. from a verification methodology, heap modeling, etc.), we would want all constants that occur in the axioms to be global.

\newpage
\section{CFG Node Scope}\label{section:scoping:node_scope}
Intuitively, the set of program locations in which a program variable is in scope is the part of the program where it exists in memory and accessible by the program. We use at least the minimal scope at each node as it is the smallest scope that we can determine syntactically, that is guaranteed to suffice for proving the program in the ground case.

\begin{figure}
\begin{lstlisting}
$\m{n_1:}$
	assume P(c)=T
$\m{n_2:}$
	assume Q(d)=T
$\m{n_3:}$
	assert P(c)=T
\end{lstlisting}
\caption{Minimal scope}
\label{snippet4.2.1}
\end{figure}

For example, consider figure \ref{snippet4.2.1}.
If \m{c \notin S^{M}_{2}} (that is, clauses derived at \m{n_2} cannot have occurrences of c) there is no way to prove the assertion - there is no interpolant (not even with quantifiers) that verifies the assertion which does not include the constant c at \m{I_{n_2}}.\\
In our definition for a fragment interpolant, we have restricted the set of clauses of the interpolant at a node to clauses in the vocabulary in the scope of the node, which we have left as a parameter and until now assumed that this included all constants in the signature. 
In the ground case, it is easy to see that we can use \m{S^F_n} as the scope at node n in our verification algorithm without losing completeness, as the conclusion of each inference rule only contains symbols from the premises, and as clauses are only propagated forward.

\begin{figure}
\begin{lstlisting}
$\m{n_1:}$
	assume c=f(a,e)
	assume d=f(b,e)
$\m{n_2:}$
	...
	//$\m{\textcolor{gray}{S^F_n = \s{a,b,c,d,e}}}$
	//$\m{\textcolor{gray}{S^{M}_n = \s{a,b,c,d}}}$
	//Interpolant for $\m{\textcolor{gray}{S^F_n}}$: $\m{\textcolor{gray}{\s{c=f(a,e),d=f(b,e)}}}$
	//Interpolant for $\m{\textcolor{gray}{S^{M}_n}}$: $\m{\textcolor{gray}{a \neq b \lor c=d}}$
$\m{n_3:}$
	assume a=b
	assert c=d
\end{lstlisting}
\caption{Minimal scope - unit ground fragment}
\label{snippet4.2.2}
\end{figure}

In the ground unit fragment this is not sufficient - for example, consider figure \ref{snippet4.2.2}.
Here, using \m{S^F_n} allows us to prove the program in the ground unit fragment, 
while using \m{S^{M}_n} requires the non-unit fragment.

\begin{figure}
\begin{lstlisting}
$\m{s:}$
	... //Heap axioms
	assume f $\m{\neq}$ g
$\m{n_1:}$
	assume x$\m{\neq}$null
	assume y$\m{\neq}$null
	assume x$\m{\neq}$y
	x.f := 1    //$\m{\textcolor{gray}{h_1=wr(h_0,x,f,1)}}$
$\m{n_2:}$
	x.g := 2    //$\m{\textcolor{gray}{h_2=wr(h_1,x,g,2)}}$
$\m{n_3:}$
	y.f := 3    //$\m{\textcolor{gray}{h_3=wr(h_2,y,f,3)}}$
$\m{n_4:}$
	assert y.g = old(y.g) 
	//negated $\m{\textcolor{gray}{rd(h_3,y,g) \neq rd(h_0,y,g)}}$
	//Here $\m{\textcolor{gray}{S^{M}_{n_4}=\s{y,g,h_0,h_3}}}$
\end{lstlisting}
\caption{Intermediate scope\\
We assume the standard heap axioms without extensionality
}
\label{snippet4.2.4}
\end{figure}

Consider also the example in figure \ref{snippet4.2.4}.
If we restrict the scope at each node to \m{S^{M}_n}, we can still prove the assertion with a low depth bound 
 - that is, without propagating \\
\m{h_3=wr(wr(wr(h_0,x,f,1),x,g,2),y,f,3)}, \\
only \m{rd(h_0,y,g)=rd(h_3,y,g)}.\\
This case is interesting, as it represents a typical combination of heap updates, DSA variables and post-conditions.

In light of the above, we sometimes want to find some compromise scope S s.t. 
\m{S^{M}_n \subseteq S \subseteq S^F_n} and S still allows us to find unit and short proofs.
Here, \m{a \in S^F_n \setminus S^{M}_n}, but we can represent \m{f(a)} using the constant b and \m{b \in S_n}.

For each node n, we are interested in finding a minimal \m{S_n} which satisfies \m{S^{M}_n \subseteq S_n \subseteq S^F_n} but still allows us to avoid resorting to non-unit clauses for completeness (without joins) in the unit fragment (as \m{S^F_n} does).

Another reason that we are interested in finding a scope where each transitive predecessor term is representable is for completeness in the non-ground case. This is discussed in chapter \ref{chapter:quantification} and we only motivate it here briefly.
The standard lifting argument to show completeness for superposition uses, instead of the set of ground clauses, the set of ground instances of all clauses. It has to be proven that each inference between ground instances is an instance of an inference between two clauses in the saturated set. This does not hold in our case (as discussed e.g. in \cite{BaumgartnerWaldmann13}) as some ground instances use symbols that do not exist together in any node-scope.

One solution is to ensure that each term in \langt can be proven equal to a term in \langb. 
We can then replace each mixed-color ground clause by a non-mixed color ground clause that is equivalent under the interpretation
(in \cite{BaumgartnerWaldmann13} a multi-sorted signature is used, and then the above has to hold essentially only for terms of sorts on which \m{N^b} can quantify - this is related to their notion of sufficient completeness).

\begin{figure}
\begin{lstlisting}
$\m{s:}$
	... //Heap axioms
	//precondition
	assume x.f.g=y
	assume f$\neq$g
	assume x.f$\in$P //set of locations
$\m{n_1:}$
	x.f := null //$\m{\textcolor{gray}{h_1=wr(h_0,x,f,null)}}$
$\m{n_2:}$
	assert $\exists$p$\in$P$\cdot$p.g=y
	//negated as
	//$\m{\forall p \in P\cdot rd(h_1,p,g)\neq y}$ 
	// the object pointed to by y is reachable from P
\end{lstlisting}
\caption{Scoping and reachability\\
\lstinline|x.f| at the initial state is unreachable at the assertion,\\
which is equivalent to \m{rd(h_0,x,f)} having no equivalent ground term at \m{n_2}.\\
Without scoping we would use the mixed term \m{rd(h_1,rd(h_0,x,f),g} to instantiate the quantifier.
}
\label{snippet4.2.4.1}
\end{figure}

An interesting example of the above, using a heap theory, is that if a memory locations becomes unreachable at some point in the program, we might not be able to prove some global heap invariants at later points - for example, consider figure \ref{snippet4.2.4.1}.
We only discuss \emph{static syntactic scoping} - scoping determined syntactically as a pre-process using only the original set of clauses at each CFG-node, and not derived equalities. 

\textbf{Algorithm:}
Our algorithm begins by assigning \m{S_n = S^{M}_n} at each node.\\
The main idea is that we can skip adding a constant c from \m{S^F_n} to \m{S_n} if it can be represented as a term over \m{S_n} using the known equalities at n - that is, for some term t over \m{S_n} we have \m{\clauses{n} \models c=t}. We do not actually check entailment but only use positive ground unit equalities syntactically. \\
We traverse the CFG in topological order, at each CFG-node we begin with \m{S_n = S^M_n} and add constants not representable by terms over constants in \m{S_n} until all constants are represented. When there is more than one option we make an arbitrary choice,
but prefer constants whose scope does not end at n.

For a constant c whose minimal scope ends at the CFG-node n, if the unit ground clause \m{c=t} is in \clauses{n} for some ground term t, and if all the constants in t are in scope in successors of n, then any clause \termRepAt{C}{c}{p} can be propagated as \termRepAt{C}{t}{p} and the scope of c can end at n, without losing completeness.


\textbf{Scope effects on size of proof:}
As can be seen in figure \ref{snippet4.2.4}, aggressive scoping can reduce the set of constants at the price of increasing the minimal term depth of a proof. In the example, we were still able to prove the assertion using a low term depth because the heap update axioms could deduce equalities that reduce the depth of the relevant terms.
We have identified some common patterns for VCs where keeping a constant in scope might be beneficial, allowing us to find narrow and shallow proofs while maintaining a small scope per CFG-node.
For each clause appearing in our VC program (\clauses{n}), there are, in general, two possible sources:

Some clauses come from original program statements, especially assignments (heap or local variable), or an \lstinline|assume| statement modeling a branch condition.

The rest of the (non-assertion) clauses come either from specifications (e.g. method pre-conditions, or assumed post-conditions after a call), or from the verification methodology (e.g. the modeling of permissions in permission based verification), these can also sometimes take the form of (DSA transformed) assignments.

We are interested in the first occurrence of a constant along a path - the earliest CFG-node in which it is mentioned.
For most DSA versions of variables, we have identified the following common patterns for initialization at first occurrence:
\begin{itemize}
	\item Assignment of a constant (independent of previous versions) expression - e.g. \lstinline|x:=1|
	\item Assignment of an expression that includes the previous DSA version of some variables (often including itself) 
	- e.g. \lstinline|x:=y| or \lstinline|x:=x+1| or for heaps \lstinline|x.f:=1| (\m{h_1 = wr(h_0,x,f,1)})
	\item The join DSA version of a variable, assigned different values on two different paths
	\item Unassigned variable with some property assumed - several distinct cases:
		\begin{itemize}
			\item The initial DSA version of a variable referenced in the method pre-condition
			\item The initial DSA version of a variable that is modified in a loop, where Boogie would model the loop using a \lstinline|havoc| statement on the variable - this is usually combined with \lstinline|assuming| the loop invariant which often refers to the variables modified in the loop body
			\item The initial DSA version of a variable after a loop - similarly the invariant (and negated loop condition) are \lstinline|assumed|
			\item The initial DSA version of a variable after return from a call in modular verification - usually the post-condition is assumed which might refer to the variable and (especially for heaps) we often have a (possibly quantified) framing assumption that correlates the uninitialized version with the version before the call
			\item A variable used to model non-determinism
		\end{itemize}
\end{itemize}

In the first two cases (variable explicitly assigned one value), we would expect that once the said DSA version goes out of \m{S^{M}_n}scope, we could replace it with the expression used in its initialization, also recursively replaced until all DSA versions are not assigned a definite value. The recursive replacement might increase the depth of the term, 
which would mean that, if we search for a proof with increasing depth as in our example, reasoning will be forced to be more local.



In the third case, we can add path condition qualified non-unit clauses that correlate the join DSA version with the recursively replaced versions of the value assigned at each branch, also recursively, for example, consider the code in figure \ref{snippet4.2.5}.
If we add n such branches in sequence, we would need \m{2^n} clauses, each n literals wide, for the strongest post-condition, if we remove join DSA versions from the scope, and \m{2n} clauses without removing the join DSA versions.
In the above case we can propagate more goal directed clauses also without the join DSA versions, but it is not clear that this can be done in the general case. Examples such as the above motivate us not to remove join DSA versions from the scope.

For all the uninitialized cases, we would also want to keep the uninitialized DSA version of the variable in scope as in this case there is no other way to represent the value of these variables. 
Intuitively, it would make sense that e.g. the return value of a function is a useful value later in the program, and so we should be able to refer to it in clauses.

\begin{figure}[H]
\begin{lstlisting}
$\m{s:}$
	... //Heap axioms
	assume x>0 // $\m{\textcolor{gray}{x_0 > 0}}$
if ($\m{c_0}$)
	x:=x+1 // $\m{\textcolor{gray}{x_1 = x_0+1}}$
else
	x:=x+2 // $\m{\textcolor{gray}{x_1 = x_0+2}}$
$\m{n_0:}$
	// $\m{\textcolor{gray}{\lnot c_0 \lor x_1 = x_0+1}}$
	// $\m{\textcolor{gray}{c_0 \lor x_1 = x_0+2}}$
if ($\m{c_1}$)
	x:=x+3 // $\m{\textcolor{gray}{x_2 = x_1+3}}$
	// $\m{\textcolor{gray}{\lnot c_0 \lor x_1 = x_0+1}}$
	// $\m{\textcolor{gray}{c_0 \lor x_1 = x_0+2}}$
else
	x:=x+5 // $\m{\textcolor{gray}{x_2 = x_1+5}}$
	// $\m{\textcolor{gray}{\lnot c_0 \lor x_1 = x_0+1}}$
	// $\m{\textcolor{gray}{c_0 \lor x_1 = x_0+2}}$
$\m{n_1:}$
	// With $\m{\textcolor{gray}{S_n=\s{x_0,x_2}}}$
	//   $\m{\textcolor{gray}{\lnot c_0 \lor \lnot c_1 \lor x_2 = x_0+4}}$
	//   $\m{\textcolor{gray}{\lnot c_0 \lor c_1 \lor x_2 = x_0+6}}$
	//   $\m{\textcolor{gray}{c_0 \lor \lnot c_1 \lor x_2 = x_0+5}}$
	//   $\m{\textcolor{gray}{c_0 \lor c_1 \lor x_2 = x_0+7}}$
	//   $\m{\textcolor{gray}{x_0>0}}$
	// With $\m{\textcolor{gray}{S_n=\s{x_0,\mathbf{x_1},x_2}}}$
	//   $\m{\textcolor{gray}{\lnot c_0 \lor x_1 = x_0+1}}$
	//   $\m{\textcolor{gray}{c_0 \lor x_1 = x_0+2}}$
	//   $\m{\textcolor{gray}{\lnot c_1 \lor x_2 = x_1+3}}$
	//   $\m{\textcolor{gray}{c_1 \lor x_2 = x_1+4}}$
	//   $\m{\textcolor{gray}{x_0>1}}$
$\m{n_a:}$
	assert x>2
\end{lstlisting}
\caption{Scoping DSA join}
\label{snippet4.2.5}
\end{figure}


\textbf{Implementation:}
We have implemented an (optional) pre-processing step that determines a scope for each CFG-node, based on the above:
Initially, for each node n, we assign \m{S_n = S^{M}_n}. 
We then proceed in topological order and mark a subset \m{K_n} of \m{S^{F}_n} as \emph{key constants} - constants that are determined on some path leading to n (or n itself) not to be representable using other key constants. Each constant in \m{S^F_n} can be represented as a term over \m{K_n}.
In some cases we have to make an arbitrary choice to select a key constant between a set of constants (e.g. \lstinline|assume a=b=c| - one of \s{a,b,c} has to be selected arbitrarily, or \lstinline|a=f(b),b=g(a)| where either of a or b can be selected).
We add each of \m{K_n} to \m{S_n}.
At the end of the process we are assured that any interpolant of the program can be transformed (with the above described recursive substitutions) to an interpolant where the clauses at each node are only over \m{S_n}.


%With regards to joins we have to be careful - for example:
%\begin{figure}[H]
%\begin{lstlisting}
%$\m{s:}$
	%assume a=b
	%assume P(a) $\lor$ Q
	%
	%if (*)
		%$\m{p_0a:}$
		%assume f(a)=c
		%$\m{p_0b:}$
		%//$\m{\textcolor{gray}{S_n=\s{a,c}}}$
	%else
		%$\m{p_1a:}$
		%assume f(b)=c
		%$\m{p_1b:}$
		%//$\m{\textcolor{gray}{S_n=\s{b,c}}}$
	%$\m{n:}$
	%assume d=e
	%$\m{n_a:}$
	%assume $\lnot$Q
	%assert P(
%\end{lstlisting}
%\caption{Scope minimization and joins}
%\label{snippet4.2.3.1}
%\end{figure}




%\newpage
\section{Ground Unit Interpolation}\label{section:scoping:ugfole}
In this section, we demonstrate the basics of our interpolation approach, and demonstrate it on the ground unit fragment (although our interpolants are non-unit ground clauses). We also discuss the inherent complexity of interpolation.
We begin by describing a simple graph-based interpolation procedure for two consecutive CFG-nodes that contain only unit ground clauses, which will form the basis for understanding interpolation for general DAGs and clauses. The treatment here is closely related to the algorithm and interpolation game of ~\cite{FuchsGoelGrundyKrsticTinelli2012}, where the main difference is that they extract an interpolant from a proof (one congruence closure graph) while we search for a proof for two CC graphs whose conjunction is inconsistent - hence we can extend our method for sequences, trees and DAGs.
We follow with a discussion of improvements to the base algorithm and the necessary modifications to support sequence- and tree-shaped CFGs.

\subsection{The problem}
The problem we consider in this section is as follows:\\
We are given two sets of unit clauses (ground equalities and dis-equalities), \m{N^t_0} and \m{N^b_0}.\\
Our objective is to generate a set of ground clauses \\
\m{I \in \langI} s.t. \m{N^t_0 \models I} and that if \m{N^t_0 \cup N^b_0 \models \emptyClause} then \m{I \cup N^b_0 \models \emptyClause}.\\
This is equivalent, in our setting, to having a CFG-node p and its successor n, where \m{N^t_0=\clauses{p},N^b_0=\clauses{n}}.

\textbf{Notation:}\\
We use the following presentation for all of our examples:\\
We use the red letters \m{\textcolor{red}{u,v,w,x,y,z}} for constants in \langtp - exclusive to \m{N^t},\\
the black letters \m{a,b,c,d,e} for constants in the interface language \langI, 
and the blue letters \m{\textcolor{blue}{l,m,n,o,p}} for constants in \langbp.\\
For non-nullary functions we use \m{f,g,h}. \\
Whenever the above symbols are insufficient, we add indices.

\subsection{Basic graph-based algorithm}
Our algorithm is based on the EC-graphs that have been described in chapter \ref{chapter:ugfole}. This algorithm can also be described for standard terms and congruence closure graphs, but the presentation is made easier by using EC-graphs.\\
We represent \m{N^t_0} as the EC-graph \m{g^t} by starting with an empty EC-graph and \lstinline{assuming} all clauses in \m{N^t_0},
and similarly \m{g^b} encodes \m{N^b_0}.\\
We use the definition of the sources function from chapter \ref{chapter:ugfole}, where: \\
\m{\forall s \in g^t, t \in g^b \cdot (s \in \sources{b}{t}{n} \Leftrightarrow \terms{s} \cap \terms{t} \neq \emptyset) }\\
Meaning that the EC-node s in the graph \m{g^t} is a source of the EC-node t in \m{g^b} iff there is a term that is represented by both EC-nodes.\\
We use \sourcesA{n} where \m{g^b,g^t} are clear from the context.\\
We use \m{[s]^t} for the EC-node of the term s in \m{g^t} when \m{s \in \terms{g^t}}, and similarly \m{[t]^b}.\\
We use the source invariant described in chapter \ref{chapter:ugfole}. As we have seen, the source and propagation invariants ensure that for each pair of terms represented in \m{g^b}, if they are equal under the union of equations in \m{g^t,g^b} then they are equal in \m{g^b} - shown formally in figure \ref{fig_propagation_guarantee}.

\begin{figure}
\m{\forall s,t \in g^t \cdot ((\exists u \in \terms{s},v\in \terms{t} \cdot N^t_0 \cup N^b_0 \models u=v) \Rightarrow s = t)}
\caption{Propagation guarantee}
\label{fig_propagation_guarantee}
\end{figure}

%Our algorithm will enforce the source invariant eagerly, but not necessarily the propagation invariant.\\
If we enforce scoping, the above guarantee does not hold - for example, consider the example in figure \ref{example_4.2.1.1_0}.

\begin{figure}
\m{N^t_0=\s{c=f(a,\textcolor{red}{x}),d=f(b,\textcolor{red}{x})}}\\
\m{N^b_0=\s{a=b,c \neq d}}\\
\m{\mathfrak{L}^t = \s{a,b,c,d,\textcolor{red}{x}}}\\
\m{\mathfrak{L}^b = \s{a,b,c,d}}
\caption{Simple unit ground interpolation example.\\
The only interpolation clause is \m{a \neq b \lor c=d}.}
\label{example_4.2.1.1_0}
\end{figure}

The graphs and sources function are depicted in figure \ref{example_4.2.1.1_1}.\\
Without scoping:\\
\m{g^b} \lstinline|add|s the EC-nodes \s{[a],[b],[\textcolor{red}{x}],[c,f([a],\textcolor{red}{x})],[d,f([b],\textcolor{red}{x})]}\\
After \lstinline|assume(a=b)| and enforcing the invariants we get:\\
\s{[a,b],[\textcolor{red}{x}],[c,d,f([a,b],\textcolor{red}{x})]}.\\
However, as \m{\textcolor{red}{x} \notin \langb} in the scoped case, \m{g^b} adds only the EC-nodes\\
\s{[a,b],[c],[d]} in our unit algorithm, and we do not get a refutation.

\textbf{Basic idea:}\\
The idea of our algorithm is simple: the two EC-graphs exchange equalities over pairs of nodes connected by the source function,
each graph performs congruence closure according to the equalities received from the other graph, until one of the graphs is inconsistent. The exchanged equalities are used to form a set of Horn clauses that is an interpolant for the two sets of clauses represented in the graphs.
We use here a small extension to the EC-graphs that we have mentioned briefly before - dis-equality edges.
A dis-equality edge is an edge between two \GTs{} (EC-Graph nodes) and represents a dis-equality between the ECs represented in the two \GTs{}. When two \GTs{} are merged, we take the union of the dis-equality edges of both \GTs{} being merged. An EC-graph is consistent iff it has no dis-equality graph that is a self-loop - an inconsistent EC-graph g is denoted by \m{g \models \emptyClause}.

\noindent
\textbf{Example algorithm run:}\\
We start with the example and then describe the algorithm.\\
The initial state is depicted in figure \ref{example_4.2.1.1_1}.\\
The only possible equality to exchange can be seen in the graph - the only node in \m{g^b} that has more than one source - \m{[a,b]^b}.\\
The algorithm run begins by \m{g^b} sending \m{[a]^t=[b]^t} to \m{g^t} - which \m{g^t} \lstinline{assumes} - the result is shown in figure \ref{example_4.2.1.1_2}.\\
Now the propagation invariant of \m{g^b} is broken, as two nodes, \m{[c]^b,[d]^b} share a source \m{[c,d]^t}. \\
\m{g^t} sends the equality \m{[c]^b=[d]^b} to \m{g^b}, which \lstinline|assumes| the equation - the result is shown in figure \ref{example_4.2.1.1_3}.\\
Now \m{g^b \models \emptyClause} (as it has a dis-equality self loop on the \GT{} \m{[c,d]^b} and we are done.\\
The interpolant we have found is the implication \m{a=b \rightarrow c=d}, which we write as the CNF Horn-clause \m{a\neq b \lor c=d}.

\begin{figure}
\begin{tikzpicture}
	\node[gttn] (1)              {$()$};
	\node[gl]   (1l) [below = 0 of 1] {\m{g^t}};

	\node[gtn]  (1a) [above right = 0.6cm and 0.7cm of 1] {\m{a}};
	\node[gtn]  (1b) [above       = 0.6cm of 1a] {\m{b}};
	\node[gtn]  (1c) [above       = 0.6cm of 1b] {\stackB{c}{f(a,\textcolor{red}{x})}};
	\node[gtn]  (1d) [above       = 0.6cm of 1c] {\stackB{d}{f(b,\textcolor{red}{x})}};

	\node[gtn]  (1x) [above left = 0.6cm and 0.7cm of 1] {\m{\textcolor{red}{x}}};

%%%%%%%%%%%%%%%%%%%%%%%%%%%%%%%%%%%%%%%%%%%%%%%%%%%%%%%%%%%%%%
	\node[gttn] (11)  [right = 6cm of 1] {$()$};
	\node[gl]   (11l) [below = 0 of 11]   {\m{g^b}};

	\node[gtn]  (11ab) [above left = 1.2cm and 0.7cm of 11] {\m{a,b}};
	\node[gtn]  (11c) [above       = 1.2cm of 11ab] {\m{c}};
	\node[gtn]  (11d) [above       = 1.0cm of 11c] {\m{d}};
 
%%%%%%%%%%%%%%%%%%%%%%%%%%%%%%%%%%%%%%%%%%%%%%%%%%%%%%%%%%%%%%
	\draw[se] (11ab) to  ( 1a);
	\draw[se] (11ab) to  ( 1b);
	\draw[se] (11c) to   ( 1c);
	\draw[se] (11d) to   ( 1d);

	\draw[ie] (11c) to  ( 11d);

\draw[draw=none, use as bounding box] (current bounding box.north west) rectangle (current bounding box.south east);

\begin{pgfinterruptboundingbox}
	\draw[separator] (3.0cm,-0.0cm) to (3.0cm,4.8cm);
\end{pgfinterruptboundingbox}

\end{tikzpicture}

\caption{
Simple unit ground interpolation example\\
Initial state.\\
Circles represent EC-nodes - \GTs{}.\\
\textcolor{blue}{Dashed blue edges} represent source edges - edges between \GTs{} of different graphs whose EC shares a term.\\
Dashed black edges represent dis-equality edges.
}
\label{example_4.2.1.1_1}
\end{figure}
\begin{figure}
\begin{tikzpicture}
	\node[gttn] (1)              {$()$};
	\node[gl]   (1l) [below = 0 of 1] {\m{g^t}};

	\node[gtn,ultra thick]  (1ab) [above right = 0.6cm and 0.7cm of 1] {\m{\textbf{a,b}}};
	\node[gtn,ultra thick]  (1cd) [above       = 0.2cm of 1b] {\stackB{c,d}{f(\textbf{[a,b]},\textcolor{red}{x})}};

	\node[gtn]  (1x) [above left = 0.6cm and 0.7cm of 1] {\m{\textcolor{red}{x}}};

%%%%%%%%%%%%%%%%%%%%%%%%%%%%%%%%%%%%%%%%%%%%%%%%%%%%%%%%%%%%%%
	\node[gttn] (11)  [right = 6cm of 1] {$()$};
	\node[gl]   (11l) [below = 0 of 11]   {\m{g^b}};

	\node[gtn]  (11ab)[above left = 0.6cm and 0.7cm of 11] {\m{a,b}};
	\node[gtn]  (11c) [above       = 0.6cm of 11ab] {\m{c}};
	\node[gtn]  (11d) [above       = 1.0cm of 11c] {\m{d}};
 
%%%%%%%%%%%%%%%%%%%%%%%%%%%%%%%%%%%%%%%%%%%%%%%%%%%%%%%%%%%%%%
	\draw[se] (11ab) to  ( 1ab);
	\draw[se] (11c) to   ( 1cd);
	\draw[se] (11d) to   ( 1cd);

	\draw[ie] (11c) to  ( 11d);

\draw[draw=none, use as bounding box] (current bounding box.north west) rectangle (current bounding box.south east);

\begin{pgfinterruptboundingbox}
	\draw[separator] (3.0cm,-0.0cm) to (3.0cm,3.2cm);
\end{pgfinterruptboundingbox}

\end{tikzpicture}

\caption{
Simple unit ground interpolation example\\
After \lstinline|$\m{g^t}$.assume(a=b)|\\
Bold text marks where \m{g^t} performed congruence closure
}
\label{example_4.2.1.1_2}
\end{figure}

\begin{figure}
\begin{tikzpicture}
	\node[gttn] (1)              {$()$};
	\node[gl]   (1l) [below = 0 of 1] {\m{g^t}};

	\node[gtn]  (1ab) [above right = 0.6cm and 0.7cm of 1] {\m{a,b}};
	\node[gtn]  (1cd) [above       = 0.1cm of 1b] {\stackB{c,d}{f([a,b]),\textcolor{red}{x})}};

	\node[gtn]  (1x) [above left = 0.6cm and 0.7cm of 1] {\m{\textcolor{red}{x}}};

%%%%%%%%%%%%%%%%%%%%%%%%%%%%%%%%%%%%%%%%%%%%%%%%%%%%%%%%%%%%%%
	\node[gttn] (11)  [right = 6cm of 1] {$()$};
	\node[gl]   (11l) [below = 0 of 11]   {\m{g^b}};

	\node[gtn]  (11ab)[above left = 0.6cm and 0.7cm of 11] {\m{a,b}};
	\node[gtn]  (11cd)[above       = 1.2cm of 11ab] {\m{\textbf{c,d}}};
 
%%%%%%%%%%%%%%%%%%%%%%%%%%%%%%%%%%%%%%%%%%%%%%%%%%%%%%%%%%%%%%
	\draw[se] (11ab) to  ( 1ab);
	\draw[se] (11cd) to  ( 1cd);

	\draw[ie] (11cd) to[out=45, in=135,looseness=4]  ( 11cd);

\draw[draw=none, use as bounding box] (current bounding box.north west) rectangle (current bounding box.south east);

\begin{pgfinterruptboundingbox}
	\draw[separator] (3.0cm,-0.0cm) to (3.0cm,3.2cm);
\end{pgfinterruptboundingbox}

\end{tikzpicture}

\caption{
Simple unit ground interpolation example\\
After \lstinline|$\m{g^b}$.assume(c=d)|\\
Here, \m{g^b \models \emptyClause}
}
\label{example_4.2.1.1_3}
\end{figure}

\textbf{Algorithm description:}\\
Given the graphs \m{g^b} and \m{g^t}, we utilize the sources function that connects nodes in \m{g^b} with nodes in \m{g^t} that share at least one term. Any node in \m{g^b} or \m{g^t} that has at least one source edge is termed an \textcolor{blue}{interface node}.\\
We assume that, at the initial state, the graphs \m{g^b,g^t} satisfy the source and propagation invariants - specifically, no node in \m{g^t} has more than one source-edge. 

The basic step of the algorithm is composed of two stages - the first stage selects a node \m{t \in g^b} where \m{\size{\sourcesA{t}} >1} and then selects a pair of nodes 
\m{s_0,s_1 \in \sourcesA{t}} s.t. \m{s_0 \neq s_1}, and merges them - that is, applies \\
\lstinline|g$^{\m{t}}$.assumeEqual(s$_0$,s$_1$)|, which performs congruence closure.
Before the second stage we enforce the source invariant in \m{g^b}.

The second stage is similar to the dual of the first stage - after the congruence closure in \m{g^t} and the completion of source, we can get several interface nodes in \m{g^t} that are sources to the same interface node in \m{g^b} - that is,
we have an EC-node \m{s \in g^t} and two EC-nodes \m{t_0,t_1 \in g^b} s.t. \m{s \in \sourcesA{t_0} \cap \sourcesA{t_1}}. 
We now select a subset of such pairs \m{t_0,t_1} and perform \lstinline|g$^{\m{b}}$.assume(t$_0$=t$_1$)| for each.

When there are no more candidates for either the first or the second stage, we are done (and the propagation invariant holds). 
We claim that, once no more steps can be taken, \\ \m{(g^b \models \emptyClause \lor g^t \models \emptyClause) \Leftrightarrow (\eqs{g^t} \cup \eqs{g^b} \models \emptyClause)}.

Our informal argument is quite simple - if we take the resulting graph and merge each pair of nodes connected by a source-edge (one from \m{g^t} and one from \m{g^b}), and then replace each of these nodes in the \GFAs{} in which they appear (in both graphs), 
we get a valid EC-graph without needing to merge any more nodes, as each node is connected with at most one source-edge and each graph is itself congruence and transitively closed. As all dis-equality edges are within a single graph, the combined graph is inconsistent iff one of its constituents is.


\textbf{Extracting interpolants:}\\
The interpolants extracted from our algorithm are always Horn clauses (as noted also in \cite{FuchsGoelGrundyKrsticTinelli2012}).
The interpolant extracted is the conjunction of one Horn-clause-EC per equation transferred from \m{g^t} to \m{g^b}, 
where the head is the equation and the body is the disjunction of the negation of all equalities transferred from \m{g^b} to \m{g^t} until the current stage. In order to get an interpolant set of clauses rather than clause-ECs, we select a representative term for each interface EC-node of \m{g^t,g^n} at each stage, which is an interface term. For example, we can select the minimal (by $\prec$) interface term represented by the EC-node.\\
We discuss the extraction of interpolants in more detail in section \ref{extracting_justification}.

\textbf{Source completeness:}\\
A slightly less obvious point, which is also the reason that GFOLE admits interpolation, is that some new source-edges may have to be drawn after transferring equations, which may connect nodes that were previously not interface nodes, and hence the interpolant may include terms that are not in the initial problem - consider the example in figure \ref{example_4.2.1.2_0}.

\begin{figure}
\m{N^t_0=\s{c=f(a,\textcolor{red}{x}),\textcolor{red}{y}=f(b,\textcolor{red}{x}),d=g(\textcolor{red}{y})}}\\
\m{N^b_0=\s{a=b,d \neq g(c)}}
\caption{Interpolation with new terms\\
The interpolant is \m{a\neq b \lor d=g(c)}.\\
}
\label{example_4.2.1.2_0}
\end{figure}

When we transfer \m{a=b} to \m{g^b}, congruence closure merges \m{[c,f(a,\textcolor{red}{x})]^b} and \m{[\textcolor{red}{y},f(b,\textcolor{red}{x})]^b} into\\ \m{[c,\textcolor{red}{y},f([a,b],\textcolor{red}{x})]^b} and hence \m{[d,g([\textcolor{red}{y}])]^b} is updated to\\
 \m{[d,g([c,\textcolor{red}{y},...])]^b} - the term \m{g(c)} did not appear on the interface originally.\\
Here the term \m{g(c)} already appears in the original \m{N^b} (but not \m{N^t}), but we can easily modify the example to require an entirely new term - hence it is crucial that we use our EC-graph algorithm to ensure that all necessary source-edges are added - show in figure \ref{example_4.2.1.3_0}. The algorithm run is show in figures \ref{example_4.2.1.2_1}-\ref{example_4.2.1.2_5}.


\begin{figure}
\m{N^t_0=\s{c=g(a,\textcolor{red}{x}),d=g(b,\textcolor{red}{x}),e=h(a,\textcolor{red}{x}),\textcolor{red}{y}=h(b,\textcolor{red}{x})=f(\textcolor{red}{y})}}\\
\m{N^b_0=\s{a=b,\textcolor{blue}{m} \neq f(\textcolor{blue}{m}),e=g(c,\textcolor{blue}{l}),\textcolor{blue}{m}=g(d,\textcolor{blue}{l})}}
\caption{Interpolation with new terms\\
The interpolant is \m{a \neq b \lor c=d, a \neq b \lor e=f(e)}.\\
The term \m{f(e)} does not appear in the initial problem at all.
}
\label{example_4.2.1.3_0}
\end{figure}


\begin{figure}
\begin{tikzpicture}
	\node[gttn] (1)              {$()$};
	\node[gl]   (1l) [below = 0 of 1] {\m{g^t}};

	\node[gtn]  (1a) [above right  = 0.6cm and 0.7cm of 1] {\m{a}};
	\node[gtn]  (1b) [above       = 0.6cm of 1a] {\m{b}};
	\node[gtn]  (1c) [above       = 0.6cm of 1b] {\stackB{c}{g(a,\textcolor{red}{x})}};
	\node[gtn]  (1d) [above       = 0.6cm of 1c] {\stackB{d}{g(b,\textcolor{red}{x})}};
	\node[gtn]  (1e) [above       = 0.6cm of 1d] {\stackB{e}{h(a,\textcolor{red}{x})}};

	\node[gtn]  (1x) [above left = 0.6cm and 0.7cm of 1] {\m{\textcolor{red}{x}}};
	\node[gtn]  (1y) [above       = 4.6cm of 1x] {\stackB{\textcolor{red}{y},f(\textcolor{red}{y})}{h(b,\textcolor{red}{x})}};

%%%%%%%%%%%%%%%%%%%%%%%%%%%%%%%%%%%%%%%%%%%%%%%%%%%%%%%%%%%%%%
	\node[gttn] (11)  [right = 6cm of 1] {$()$};
	\node[gl]   (11l) [below = 0 of 11]   {\m{g^b}};

	\node[gtn]  (11ab) [above left = 1.1cm and 0.7cm of 11] {\m{a,b}};
	\node[gtn]  (11c) [above       = 1.2cm of 11ab] {\m{c}};
	\node[gtn]  (11d) [above       = 1.2cm of 11c] {\m{d}};
	\node[gtn]  (11e) [above       = 1.0cm of 11d] {\stackB{e}{g(c,\textcolor{blue}{l})}};

	\node[gtn]  (11l) [above right = 1.2cm and 0.7cm of 11] {\m{\textcolor{blue}{l}}};
	\node[gtn]  (11m) [above       = 4.0cm of 11l] {\stackB{\textcolor{blue}{m}}{g(d,\textcolor{blue}{l})}};
	\node[gtn]  (11fm)[above       = 1.0cm of 11m] {\m{f(\textcolor{blue}{m})}};
 
%%%%%%%%%%%%%%%%%%%%%%%%%%%%%%%%%%%%%%%%%%%%%%%%%%%%%%%%%%%%%%
	\draw[se] (11ab) to  ( 1a);
	\draw[se] (11ab) to  ( 1b);
	\draw[se] (11c) to   ( 1c);
	\draw[se] (11d) to   ( 1d);
	\draw[se] (11e) to   ( 1e);

	\draw[ie] (11m) to   ( 11fm);

\draw[draw=none, use as bounding box] (current bounding box.north west) rectangle (current bounding box.south east);

\begin{pgfinterruptboundingbox}
	\draw[separator] (3.0cm,-0.0cm) to (3.0cm,7.5cm);
\end{pgfinterruptboundingbox}

\end{tikzpicture}

\caption{
Interpolation with new terms\\
Initial state
}
\label{example_4.2.1.2_1}
\end{figure}

\begin{figure}
\begin{tikzpicture}
	\node[gttn] (1)              {$()$};
	\node[gl]   (1l) [below = 0 of 1] {\m{g^t}};

	\node[gtn,ultra thick]  (1ab)[above right  = 0.6cm and 0.7cm of 1] {\m{a,b}};
	\node[gtn,ultra thick]  (1cd) [above = 0.65cm of 1ab]  {\stackB{c,d,}{g(\textbf{[a,b]},\textcolor{red}{x})}};
	
	\node[gtn,ultra thick]  (1ey) [above = 1.0cm of 1cd]  {\stackB{e,\textcolor{red}{y},f([e,\textcolor{red}{y}])}{h(\textbf{[a,b]},\textcolor{red}{x})}};
	\node[gtn]  (1x) [above left = 0.6cm and 0.7cm of 1] {\m{\textcolor{red}{x}}};

%%%%%%%%%%%%%%%%%%%%%%%%%%%%%%%%%%%%%%%%%%%%%%%%%%%%%%%%%%%%%%
	\node[gttn] (11)  [right = 6cm of 1] {$()$};
	\node[gl]   (11l) [below = 0 of 11]   {\m{g^b}};

	\node[gtn]  (11ab) [above left = 0.6cm and 0.7cm of 11] {\m{a,b}};
	\node[gtn]  (11c) [above       = 0.6cm of 11ab] {\m{c}};
	\node[gtn]  (11d) [above       = 0.4cm of 11c] {\m{d}};
	\node[gtn]  (11e) [above       = 1.0cm of 11d] {\stackB{e}{g(c,\textcolor{blue}{l})}};

	\node[gtn]  (11l) [above right = 0.6cm and 0.7cm of 11] {\m{\textcolor{blue}{l}}};
	\node[gtn]  (11m) [above       = 2.7cm of 11l] {\stackB{\textcolor{blue}{m}}{g(d,\textcolor{blue}{l})}};
	\node[gtn]  (11fm)[above       = 1.0cm of 11m] {\m{f(\textcolor{blue}{m})}};
 
%%%%%%%%%%%%%%%%%%%%%%%%%%%%%%%%%%%%%%%%%%%%%%%%%%%%%%%%%%%%%%
	\draw[se] (11ab) to  ( 1ab);
	\draw[se] (11c) to   ( 1cd);
	\draw[se] (11d) to   ( 1cd);
	\draw[se] (11e) to   ( 1ey);

	\draw[ie] (11m) to   ( 11fm);

\draw[draw=none, use as bounding box] (current bounding box.north west) rectangle (current bounding box.south east);

\begin{pgfinterruptboundingbox}
	\draw[separator] (3.0cm,-0.0cm) to (3.0cm,5.5cm);
\end{pgfinterruptboundingbox}

\end{tikzpicture}

\caption{
Interpolation with new terms\\
After \lstinline|$\m{g^b}$.assume(a=b)|
}
\label{example_4.2.1.2_2}
\end{figure}

\begin{figure}
\begin{tikzpicture}
	\node[gttn] (1)              {$()$};
	\node[gl]   (1l) [below = 0 of 1] {\m{g^t}};

	\node[gtn]  (1ab)[above right  = 0.6cm and 0.7cm of 1] {\m{a,b}};
	\node[gtn]  (1cd) [above = 0.7cm of 1ab]  {\stackB{c,d,}{g([a,b],\textcolor{red}{x})}};
	
	\node[gtn]  (1ey) [above = 1.0cm of 1cd]  {\stackB{e,\textcolor{red}{y},f([e,\textcolor{red}{y}])}{h([a,b],\textcolor{red}{x})}};
	\node[gtn]  (1x) [above left = 0.6cm and 0.7cm of 1] {\m{\textcolor{red}{x}}};

%%%%%%%%%%%%%%%%%%%%%%%%%%%%%%%%%%%%%%%%%%%%%%%%%%%%%%%%%%%%%%
	\node[gttn] (11)  [right = 6cm of 1] {$()$};
	\node[gl]   (11l) [below = 0 of 11]   {\m{g^b}};

	\node[gtn]  (11ab) [above left = 0.6cm and 0.7cm of 11] {\m{a,b}};
	\node[gtn,ultra thick]  (11cd) [above       = 0.9cm of 11ab] {\m{c,d}};

	\node[gtn,ultra thick]  (11em)[above       = 1.2cm of 11cd] {\stackB{e,\textcolor{blue}{m}}{g(\textbf{[c,d]},\textcolor{blue}{l})}};

	\node[gtn]  (11fm)[above       = 1.0cm of 11em] {\m{f(\textbf{[e,\textcolor{blue}{m}]})}};

	\node[gtn]  (11l) [above right = 0.6cm and 0.7cm of 11] {\m{\textcolor{blue}{l}}};
 
%%%%%%%%%%%%%%%%%%%%%%%%%%%%%%%%%%%%%%%%%%%%%%%%%%%%%%%%%%%%%%
	\draw[se] (11ab) to  ( 1ab);
	\draw[se] (11cd) to   (1cd);

	\draw[se] (11em) to   ( 1ey);
	\draw[pe] (11fm) to   ( 1ey);

	\draw[ie] (11fm) to   (11em);

\draw[draw=none, use as bounding box] (current bounding box.north west) rectangle (current bounding box.south east);

\begin{pgfinterruptboundingbox}
	\draw[separator] (3.0cm,-0.0cm) to (3.0cm,5.5cm);
\end{pgfinterruptboundingbox}

\end{tikzpicture}

\caption{
Interpolation with new terms\\
After \lstinline|$\m{g^b}$.assume(c=d)|
}
\label{example_4.2.1.2_3}
\end{figure}

\begin{figure}
\begin{tikzpicture}
	\node[gttn] (1)              {$()$};
	\node[gl]   (1l) [below = 0 of 1] {\m{g^t}};

	\node[gtn]  (1ab)[above right  = 0.6cm and 0.7cm of 1] {\m{a,b}};
	\node[gtn]  (1cd) [above = 0.7cm of 1ab]  {\stackB{c,d,}{g([a,b],\textcolor{red}{x})}};
	
	\node[gtn]  (1ey) [above = 1.0cm of 1cd]  {\stackB{e,\textcolor{red}{y},f([e,\textcolor{red}{y}])}{h([a,b],\textcolor{red}{x})}};
	\node[gtn]  (1x) [above left = 0.6cm and 0.7cm of 1] {\m{\textcolor{red}{x}}};

%%%%%%%%%%%%%%%%%%%%%%%%%%%%%%%%%%%%%%%%%%%%%%%%%%%%%%%%%%%%%%
	\node[gttn] (11)  [right = 6cm of 1] {$()$};
	\node[gl]   (11l) [below = 0 of 11]   {\m{g^b}};

	\node[gtn]  (11ab) [above left = 0.6cm and 0.7cm of 11] {\m{a,b}};
	\node[gtn,ultra thick]  (11cd) [above       = 0.9cm of 11ab] {\m{c,d}};

	\node[gtn,ultra thick]  (11em)[above       = 1.2cm of 11cd] {\stackB{e,\textcolor{blue}{m}}{g(\textbf{[c,d]},\textcolor{blue}{l})}};

	\node[gtn]  (11fm)[above       = 1.0cm of 11em] {\m{f(\textbf{[e,\textcolor{blue}{m}]})}};

	\node[gtn]  (11l) [above right = 0.6cm and 0.7cm of 11] {\m{\textcolor{blue}{l}}};
 
%%%%%%%%%%%%%%%%%%%%%%%%%%%%%%%%%%%%%%%%%%%%%%%%%%%%%%%%%%%%%%
	\draw[se] (11ab) to  ( 1ab);
	\draw[se] (11cd) to   (1cd);

	\draw[se] (11em) to   ( 1ey);
	\draw[se,ultra thick] (11fm) to   ( 1ey);

	\draw[ie] (11fm) to   (11em);

\draw[draw=none, use as bounding box] (current bounding box.north west) rectangle (current bounding box.south east);

\begin{pgfinterruptboundingbox}
	\draw[separator] (3.0cm,-0.0cm) to (3.0cm,5.5cm);
\end{pgfinterruptboundingbox}

\end{tikzpicture}

\caption{
Interpolation with new terms\\
After \lstinline|$\m{g^b}$.update|
}
\label{example_4.2.1.2_4}
\end{figure}


\begin{figure}
\begin{tikzpicture}
	\node[gttn] (1)              {$()$};
	\node[gl]   (1l) [below = 0 of 1] {\m{g^t}};

	\node[gtn]  (1ab)[above right  = 0.6cm and 0.7cm of 1] {\m{a,b}};
	\node[gtn]  (1cd) [above = 0.7cm of 1ab]  {\stackB{c,d,}{g([a,b],\textcolor{red}{x})}};
	
	\node[gtn]  (1ey) [above = 1.0cm of 1cd]  {\stackB{e,\textcolor{red}{y},f([e,\textcolor{red}{y}])}{h([a,b],\textcolor{red}{x})}};
	\node[gtn]  (1x) [above left = 0.6cm and 0.7cm of 1] {\m{\textcolor{red}{x}}};

%%%%%%%%%%%%%%%%%%%%%%%%%%%%%%%%%%%%%%%%%%%%%%%%%%%%%%%%%%%%%%
	\node[gttn] (11)  [right = 6cm of 1] {$()$};
	\node[gl]   (11l) [below = 0 of 11]   {\m{g^b}};

	\node[gtn]  (11ab) [above left = 0.6cm and 0.7cm of 11] {\m{a,b}};
	\node[gtn]  (11cd) [above       = 0.9cm of 11ab] {\m{c,d}};

	\node[gtn,ultra thick]  (11em)[above       = 1.3cm of 11cd] {\stackB{e,\textcolor{blue}{m},f(\textbf{[e,\textcolor{blue}{m}]})}{g([c,d],\textcolor{blue}{l})}};

	\node[gtn]  (11l) [above right = 0.6cm and 0.7cm of 11] {\m{\textcolor{blue}{l}}};
 
%%%%%%%%%%%%%%%%%%%%%%%%%%%%%%%%%%%%%%%%%%%%%%%%%%%%%%%%%%%%%%
	\draw[se] (11ab) to  ( 1ab);
	\draw[se] (11cd) to   (1cd);

	\draw[se] (11em) to   ( 1ey);

	\draw[ie] (11em) to[out=45, in=135,looseness=4]  ( 11em);

\draw[draw=none, use as bounding box] (current bounding box.north west) rectangle (current bounding box.south east);

\begin{pgfinterruptboundingbox}
	\draw[separator] (3.0cm,-0.0cm) to (3.0cm,5.5cm);
\end{pgfinterruptboundingbox}

\end{tikzpicture}

\caption{
Interpolation with new terms, final state\\
After \lstinline|$\m{g^b}$.assume(e=f(e))|
The interpolant is \s{a\neq b \lor c=d,a \neq b \lor e = f(e)}\\
f(e) does not appear in the original problem
}
\label{example_4.2.1.2_5}
\end{figure}

We have not specified how the equalities communicated between the two graphs are selected.
For the basic algorithm, any selection will work as long as we continue the process until no equations can be communicated in either direction.
In the next sections, we discuss several improvements of the basic algorithm, 
including not communicating useless equations, interpolation for several \m{g^bs} and generating stronger interpolants.




\subsection{Selecting equations to communicate}
In our verification algorithm, interpolation happens in an incremental environment - that is, after we have calculated the interpolant between \m{N^t} and \m{N^b}, new (dis-)equalities can be added to \m{N^t} and \m{N^b} by other fragments and we need then to find the change in the interpolant.
We are interested in understanding the space of potential interpolants, in order to define an incremental search strategy that generates optimal interpolants 
(in our setting optimal means smaller and usually stronger, 
while interpolants intended for invariants have different requirements - see e.g. the discussion in \cite{DBLP:conf/vmcai/DSilvaKPW10}).

Additionally, in some cases, the Horn clauses extracted are not the strongest possible. 
This happens when not all of the antecedents are needed to prove the consequent, and might depend on the order of communicating equalities.\\
Consider the example in figure \ref{example_4.2.1.3}.

\begin{figure}
\m{N^t=\s{e_1=g(a,\textcolor{red}{x}),e_2=g(b,\textcolor{red}{x}),e_3=h(c,\textcolor{red}{x}),e_4=h(d,\textcolor{red}{x})}}\\
\m{N^b=\s{a=b,c=d,f(e_1,e_3,\textcolor{blue}{l}) \neq f(e_2,e_4,\textcolor{blue}{l})}}
\caption{Example for order dependence in extracted interpolants\\
The interpolant extracted by our algorithm is either \\
\s{a\neq b \lor e_1=e_2,a\neq b \lor c \neq d \lor e_3=e_4} or \\
\s{c\neq d \lor e_3=e_4,a\neq b \lor c \neq d \lor e_1=e_2},\\
depending on the order in which we communicate equalities.\\
The optimal interpolant is:\\
\s{a\neq b \lor e_1=e_2,c\neq d \lor e_3=e_4}
}
\label{example_4.2.1.3}
\end{figure}


\noindent
Here, if we first communicate \m{a=b} the interpolant we get is\\
\s{a\neq b \lor e_1=e_2,a\neq b \lor c \neq d \lor e_3=e_4}\\
While if we first communicate \m{c=d} the interpolant is:\\
\s{c\neq d \lor e_3=e_4,a\neq b \lor c \neq d \lor e_1=e_2}\\
The stronger interpolant is: \\
\s{a\neq b \lor e_1=e_2,c\neq d \lor e_3=e_4}\\
In this case it seems clear that the graph \m{N^t} has two independent parts, but in some other cases it might not be easy to determine the independent parts of \m{N^t} - we discuss an improvement of the basic interpolation algorithm that produces optimal interpolants in this section.
In some cases, there are several incomparable interpolants, each of which is sufficient for a refutation - for example, consider the interpolation problems in figures \ref{example_4.2.1.4} and \ref{example_4.2.1.5} - in both cases we have more than one interpolant and they are all of the same size.

\begin{figure}
\m{N^t=\s{e_1=g(a,\textcolor{red}{x})=h(c,\textcolor{red}{x}),e_2=g(b,\textcolor{red}{x})=h(d,\textcolor{red}{x})}}\\
\m{N^b=\s{a=b,c=d,f(e_1,\textcolor{blue}{l}) \neq f(e_2,\textcolor{blue}{l})}}
\caption{Example for incomparable interpolants\\
The two possible interpolants are:\\
\s{a\neq b \lor e_1=e_2}\\
\s{c\neq d \lor e_1=e_2}\\
Neither is inherently preferable to the other. 
}
\label{example_4.2.1.4}
\end{figure}


\begin{figure}
\m{N^t=\s{e_1=g(a,b,c,\textcolor{red}{x}),e_2=g(b,c,a,\textcolor{red}{x})}}\\
\m{N^b=\s{a=b=c,e_1 \neq e_2}}
\caption{Example for incomparable interpolants - transitivity\\
The three possible interpolants are:\\
\s{a\neq b \lor a \neq c \lor e_1=e_2}\\
\s{a\neq b \lor b \neq c \lor e_1=e_2}\\
\s{a\neq c \lor b \neq c \lor e_1=e_2}\\
}
\label{example_4.2.1.5}
\end{figure}

In the next sections we discuss some modifications of the base interpolation algorithm that allow us to generate better interpolation clauses in an incremental setting, and in a CFG (with one \m{N^t} and several \m{N^bs}).

\subsection{Relevance calculations}
In this section we describe a modification of the base interpolation algorithm presented above, 
which allows us to reduce the number of equations that need to be exchanged, without losing completeness.
We present it here as it is used also in our algorithm for interpolation in the CFG.
The basic idea is to calculate which interface equalities are \emph{relevant} for \m{N^t}, that is, which interface equality communicated from \m{g^b} to \m{g^t} could lead to an interpolation clause. For example, if we look at figure \ref{example_4.2.1.4}, we can see that the interface equalities \\
\s{a=b,c=d}\\
can lead to interpolation clauses, while the interface equalities \\
\m{\{a=c,a=d,a=e_1,a=e_2,b=c,b=d,b=e_1,b=e_2,}\\
\m{c=e_1,c=e_2,d=e_1,d=e_2\}} \\
cannot lead to any interpolation clause, as they do not cause any congruence closure in \m{g^b}.
Hence there is no point in communicating these equalities from \m{g^b} even if they hold in \m{g^b}.

A slightly more elaborate example is given in figure \ref{f.fs.ug.rc.1}, where the only relevant interface equality is a=b, and removing any of the clauses in \m{g^t} makes all interface equalities irrelevant.
\begin{figure}[H]
\m{N^t = \{}\\
\m{\textcolor{red}{y_0}=f(a,\textcolor{red}{x}),\textcolor{red}{y_1}=f(b,\textcolor{red}{x}),}\\
\m{\textcolor{red}{z_0}=g(\textcolor{red}{y_0},\textcolor{red}{x}),\textcolor{red}{z_1}=g(\textcolor{red}{y_1},\textcolor{red}{x}),}\\
\m{c=h(\textcolor{red}{z_0},\textcolor{red}{x}),d=h(\textcolor{red}{z_1},\textcolor{red}{x})\}}\\
\m{N^b = \s{a=b,f(c,\textcolor{blue}{w}) \neq f(d,\textcolor{blue}{w})}}
\caption{Example for relevance dependencies\\
The only possible interpolant is \m{a \neq b \lor c=d}.\\
Removing any of the clauses from \m{N^t} makes \m{a=b} irrelevant for interpolation
}
\label{f.fs.ug.rc.1}
\end{figure}
 
Remember that our interpolants are sets of Horn clauses, which may have a head (positive equality) or no head (all negative equalities).
The basic observation is that there are two ingredients necessary for the generation of an interpolation clause, one for the head and one for the rest of the clause:
\begin{itemize}
	\item A goal: this is a pair of distinct terms that, if shown equal, contribute an interpolation clause- in one of the following ways:
		\begin{itemize}
			\item A direct interface equality - for example, in figure \ref{f.fs.ug.rc.1}, merging \m{[c]^t,[d]^t} produces an interpolation Horn clause with the head c=d, and so the pair \s{c,d} is a goal.
			\item A non-interface dis-equality - for example, if we were to add \m{\textcolor{red}{z_0}\neq\textcolor{red}{z_1}} in figure \ref{f.fs.ug.rc.1} instead of the equations for c,d, merging the pair \s{[z_0]^t,[z_1]^t} would produce a head-less interpolation clause - \m{a\neq b}
			\item Equating a non-interface term with an interface term, if it can contribute to one of the above two goals - for example, in figure \ref{example_4.2.1.3_0}, merging \m{[\textcolor{red}{y}]^t,[e]^t} produces an interpolation clause with the head\\ \m{f(e) = e}
			%\item Equating an interface term with a non-interface term that participates in a dis-equality with another term that is either an interface term or can be equated with one - for example: \\
%\m{N^t = \s{c=f(b,\textcolor{red}{x}),d=g(b,\textcolor{red}{x}),\textcolor{red}{y}=f(a,\textcolor{red}{x}),\textcolor{red}{z}=g(a,\textcolor{red}{x}),\textcolor{red}{y} \neq \textcolor{red}{z}}}\\
		%Where we cannot equate \s{c,d} or \s{y,z}, but we can equate \m{c=y,d=z} and get an interpolation clause with the dis-equa
		\end{itemize}
	\item A path from interface equalities to the goal, using congruence and transitive closure
\end{itemize}

\subsubsection*{The relevance Algorithm}
We use here a set $\textcolor{blue}{\Gamma}$ of interface equalities (equalities on pairs of interface EC-nodes of \m{N^t}) to represent all the equalities that could be communicated from \m{N^b} to \m{N^t} - 
the algorithm is parametric in $\Gamma$, and we later use $\Gamma$ as an over-approximation of the set of interface equalities between \m{N^t} and a set of \m{N^bs} - the algorithm only assumes that $\Gamma$ includes at least all interface equalities of \m{N^b}.

We discuss the relevance calculations that only depends on \m{g^t} and $\Gamma$ - specifically, we assume we are given \m{g^t}, the interface language \m{\mathfrak{L}^I} and $\Gamma$, but not any \m{N^b} (this is needed so that we can calculate the relevance set for several successor CFG-nodes, as described in section \ref{goal_relevance_CFG}).\\
We use now 
%\m{g^t} for the above-mentioned \m{g_n^i} of a given node, and 
\m{I^t} for the set of (forward) interface EC-nodes of \m{g^t} - all EC-nodes in \m{g^t} for which there is a term over \m{\mathfrak{L}^I}.

Our algorithm calculates, given the EC-graph \m{g^t} and a set of equalities $\Gamma$ not yet assumed in \m{g^t}, 
a subset of $\Gamma$ that is guaranteed to be the only literals needed in bodies of the interpolation clauses.

The \textbf{basic idea} is somewhat similar to our join algorithm: we first proceed bottom-up and mark equivalence classes of EC-nodes in \m{g^t} that are equal under $\Gamma$, and then we proceed top-down from each goal (non-interface dis-equality or interface equality) where both sides are in the same equivalence class, and mark all pairs that can contribute to showing that the two goal EC-nodes are equal.

We sketch the algorithm in figure \ref{relevance_algorithm}. The algorithm is described using EC-graphs. The input is an EC-graph 
\lstinline|g| which represents \m{g^t} and a set of equalities $\Gamma$.
The algorithm creates an EC-graph \lstinline|rg| as a sequential successor of \lstinline|g| using the algorithms we have shown in chapter \ref{chapter:ugfole} - so \lstinline|rg| represents $\Gamma \cup$ \eqs{g^t}, and maintains source-edges to \lstinline|g| - edges between EC-nodes of \lstinline|g| and \lstinline|rg| whose ECs shares a term.
We now describe the working of the algorithm.

\begin{figure}
\begin{lstlisting}
method relevance(g : ECGraph, $\Gamma$ : Set[Equality])
		: Set[Pair[GT]]
	var rg := new SequentialECGraph(g)
	//rg is a sequential successor of g, maintaining 
	//the propagation and source invariants
	
	//assume all equations from $\Gamma$ in rg
	foreach (eq $\in$ $\Gamma$)
		rg.assume(eq)
	
	//Add all interface nodes from g to rg
	var irgts := new Set[GT] //interface nodes in rg
	foreach (gt $\in$ $\m{I^t}$)
		irgt.add(rg.makeTerm($\m{\textbf{rep}}$(gt)))
	
	//Calculate $\m{G}$ - we calculate the inverse source of $\m{G}$
	var rG := new Set[Pair[GT]] //unordered pairs
	foreach (rgt in irgts)
		foreach (u,v $\in$ $\m{\mathbf{sources}}$(rgt))
			if (u$\neq$v)
				rG.add((u,v))
	foreach ((u,v) $\in$ $\m{\mathbf{diseqs}}$(g))
		ru := rg.makeTerm($\m{\textbf{rep}}$(u))
		rv := rg.makeTerm($\m{\textbf{rep}}$(v))
		if (ru==rv) //dis-equality refuted under $\Gamma$
			rG.add((u,v))
			
	return relevance(g,rg,rG)
	
method relevance(g,rg : ECGraph, rG : Set[Pair[GT]])
	//Calculate $\m{R}$
	var $\m{R}$ := new Set[Pair[GT]] //unordered pairs
	var todo := new Queue[Pair[GT]](rG)
	while (!todo.isEmpty) //Up to a quadratic number of pairs
		var (u,v) := todo.dequeue
		//Inverse congruence closure
		foreach ($\fa{f}{s} \in$ u) 
			foreach ($\fa{f}{t} \in$ v) 
				foreach (i $\in$ 0..$\arity{f}$)
					if (s$\m{_i}$ $\neq$ t$\m{_i}$)
						if ($\m{R}$.add((s$\m{_i}$,t$\m{_i}$))) todo.enqueue((s$\m{_i}$,t$\m{_i}$))
		
		//Inverse transitive closure
		var m := rg.$\m{\sourcesInv{}{}{u}}$ //also rg.$\comm{\m{=\sourcesInv{}{}{v}}}$
		foreach (w $\in$ $\m{\mathbf{sources}}$(m)$\setminus \{$u,v$\}$)
			if (u$\notin$I or v$\notin$I or w$\notin$I)
				if ($\m{R}$.add((u,w))) todo.enqueue((u,w))
				if ($\m{R}$.add((w,v))) todo.enqueue((w,v))
					
		//Return only interface relevant pairs
		return $\s{(u,v) \in R \mid u,v \in I^t}$
\end{lstlisting}
\caption{Relevance algorithm}
\label{relevance_algorithm}
\end{figure}

We define first the relation \m{\textcolor{blue}{\eqg}} which is the smallest congruence on \m{g^t} that satisfies \m{\Gamma} - formally, it is the least fixed point of the equations in figure \ref{eqg_def} (represented by \lstinline|rg| - specifically, for a pair of EC-node s (u,v) in \lstinline|g|, \m{u \eqg v \equivdef \exists r \in}\lstinline|rg|\m{\cdot u,v \in \sources{}{}{r}}).

\begin{figure}
\begin{enumerate}
	\item \m{\forall s=t \in \Gamma \cdot s \eqg t}
	\item \m{\forall u,v,w \in g^t \cdot ( ( u \eqg v \land v \eqg w ) \Rightarrow u \eqg w)}
	\item \m{\forall \tup{u},\tup{v} \in g^t, \fa{f}{s},\fa{f}{t} \in \gfasA{g^t} \cdot (\tup{u} \eqg \tup{v} \Rightarrow [\fa{f}{s}]^t \eqg [\fa{f}{t}]^t)}
\end{enumerate}
\caption{Definition of \eqg{}\\
These equations are the standard transitive congruence closure of $\Gamma$, 
except that we only allow a congruence closure instance if at least one side of the conclusion is in \m{g^t} }
\label{eqg_def}
\end{figure}

In order to determine all of our goals, we need to calculate first which non-interface nodes could potentially become interface nodes under some interface equalities - we encode this using the set \m{\textcolor{blue}{I^{\Gamma}}} which is the least set that satisfies the equations in figure \ref{Igamma_def}.

\begin{figure}
\begin{enumerate}
	\item \m{I \subseteq I^{\Gamma}}
	\item \m{\forall u \in I^{\Gamma}, v \in g^t \cdot u \eqg v \Rightarrow v \in I^{\Gamma}}
	\item \m{\forall \tup{s} \in I^{\Gamma}, \fa{f}{s} \in \gfasA{g^t} \cdot [\fa{f}{s}]^t \in I^{\Gamma}}
\end{enumerate}
\caption{Definition of \m{I^{\Gamma}}\\
}
\label{Igamma_def}
\end{figure}

The first rule includes all interface nodes, 
the second includes nodes equal in \eqg{} to potential interface nodes, 
and the third includes \GFAs{} for which the tuple is a potential interface node.

In order to define the top-down part we first define the set of goals \textcolor{blue}{G}, which are pairs of EC-nodes - shown in figure \ref{goals_def}.

\begin{figure}
\m{G \triangleq \s{\s{u,v} \mid u\neq v \land u \eqg v \land (u,v \in I^{\Gamma} \lor g^t \models u \neq v )}}
\caption{Definition of G - goal pairs}
\label{goals_def}
\end{figure}


\noindent
Goal pairs are pairs of EC-nodes of \m{g^t} that are equivalent in \eqg and either both are potential interface nodes or they share a dis-equality edge.
Our formulation here is an over-approximation, for example, in:\\
\m{N^t=\s{c=f(a,\textcolor{red}{x}),\textcolor{red}{y}=f(b,\textcolor{red}{x})}}\\
\m{\s{c,\textcolor{red}{y}} \in G} although no clause can be generated, while adding either\\
\s{d=g(a,\textcolor{red}{x}),\textcolor{red}{y}=g(b,\textcolor{red}{x})} or\\
\s{\textcolor{red}{y} \neq f(\textcolor{red}{y})}\\
makes this pair a relevant goal.
We leave refinement of the goal set calculations as future work, the rest of the algorithm is parametric in G.
We only note here that when we interpolate in a CFG (either sequence, tree or DAG) we must add to the set of goals also pairs of EC-nodes that are on the interface with \emph{predecessor} CFG-nodes, in order to ensure complete propagation.

\noindent
The top-down part is defined using the symmetric relation \textcolor{blue}{R} (relevant pairs) between EC-nodes of \m{g^t}, which is the least fixed point of the equations in figure \ref{R_def}.

\begin{figure}
\begin{enumerate}
	\item \m{\forall \s{u,v} \in G \cup R, \fa{f}{s} \in u,\fa{f}{t} \in v \cdot}\\
				\m{((u \neq v \land \tup{s} \eqg \tup{t}) \Rightarrow \s{\tup{s},\tup{t}} \in R)}
	\item \m{\forall \s{u,v} \in G \cup R, w \in g^t \cdot}\\
	\m{((u \neq w \land w \neq v \land (u \notin I \lor v \notin I \lor w \notin I) \land }\\
	\m{u \eqg w \land w \eqg v ) \Rightarrow}\\
	\m{\s{u,w},\s{w,v} \in R)}
%	\item \m{\forall u \in I^R, v \in I^t \cdot (u,v) \in P \cdot (u,v) \in R}
\end{enumerate}
\caption{Definition of R - relevant pairs\\
Where:\\
\m{\s{\tup{s},\tup{t}} \in R \equivdef \forall i \cdot (s_i \not\equiv t_i \Rightarrow \s{s_i,t_i} \in R)}\\
The first rule is inverse congruence closure and the second inverse transitive closure, bounded by \eqg.\\
The relevant equalities are \m{u=v} s.t. \m{\s{u,v} \in R \cap (I^t)^2}.
}
\label{R_def}
\end{figure}


\bigskip

\noindent
\textbf{Examples:}\\
(as we discuss only \m{g^t} we use \m{[.]} instead of \m{[.]^t} for EC-nodes for simplicity)

\noindent
For the example in figure \ref{example_4.2.1.2_0} :\\
\m{g^t = \s{[\textcolor{red}{x}],[\textcolor{red}{y},f(b,\textcolor{red}{x})],[a],[b],[c,f(a,\textcolor{red}{x})],[d,g(\textcolor{red}{y})]}}\\
Which we shorten to:\\
\m{g^t = \s{[\textcolor{red}{x}],[\textcolor{red}{y}],[a],[b],[c],[d]}}\\
\m{I^t = \s{[a],[b],[c],[d]}}\\
We show the example for \m{\Gamma = \s{a=b=c=d}}\\
The graph \lstinline|rg| represents:\\
\m{g^t/\eqg ~=~ \s{\s{[a],[b],[c],[d],[\textcolor{red}{y}]},\s{[\textcolor{red}{x}]}}}\\
(\m{\textcolor{red}{y}} is merged with \m{c} by congruence closure).\\
\m{I^{\Gamma} = \s{[a],[b],[c],[d],[\textcolor{red}{y}]}}\\
(\m{\textcolor{red}{y}} added to the interface as it was merged with the interface node for \m{c}).\\
\lstinline|rg| has one interface node \m{[a]_{rg}}.\\
The set of goal pairs are all pairs of source of merged interface nodes - there are no dis-equalities.\\
\m{G = I^{\Gamma} \times I^{\Gamma} \setminus id}\\
In the loop for calculating R, all of the inverse transitive closure are already in G hence the only added pair is
\m{\s{[a],[b]}} using the inverse congruence closure rule on the \GFAs{} \m{f(a,\textcolor{red}{x}),f(b,\textcolor{red}{x})}
\m{R =  \s{\s{[a],[b]}}} (obtained using rule 1 from the goal \s{c,\textcolor{red}{y}})\\
Relevant equalities:\\
\s{[a]=[b]}

\bigskip

\noindent
For the example in figure \ref{example_4.2.1.4}:\\
\m{g^t = \s{[\textcolor{red}{x}],[a],[b],[c],[d],[e_1],[e_2]}}\\
\m{I^t = \s{[a],[b],[c],[d],[e_1],[e_2]}}\\
\m{\Gamma = (I^t)^2}\\
\m{g^t/\eqg ~=~ \s{I^t,\s{[\textcolor{red}{x}]}}}\\
\m{I^{\Gamma} = I^t}\\
\m{G = I^{\Gamma} \times I^{\Gamma} \setminus id}\\
As in the last example, there are no applications of the inverse transitive rule, only the two applications of the inverse congruence closure rule.\\
\m{R =  \s{\s{[a],[b]},\s{[c],[d]}}} (rule 1 from the goal \s{e_1,e_2})\\
Relevant equalities:\\
\s{[a]=[b],[c]=[d]}

\bigskip

\noindent
Finally, for the example in figure \ref{f.fs.ug.rc.1}:\\
\m{g^t = \s{[\textcolor{red}{x}],[\textcolor{red}{y_0}],[\textcolor{red}{y_1}],[\textcolor{red}{z_0}],[\textcolor{red}{z_1}],[a],[b],[c],[d]}}\\
\m{I^t = \s{[a],[b],[c],[d]}}\\
\m{{\Gamma} = (I^t)^2}\\
\m{g^t/\eqg ~=~ \s{I^t,\s{[\textcolor{red}{y_0}],[\textcolor{red}{y_1}]},\s{[\textcolor{red}{z_0}],[\textcolor{red}{z_1}]}}}\\
\m{I^{\Gamma} = I^t}\\
\m{G = I^{\Gamma} \times I^{\Gamma} \setminus id}\\
Here the last loop (using \lstinline|todo|) finds only one additional pair - \m{([\textcolor{red}{y_0}],[\textcolor{red}{y_1}])} - 
using the inverse of congruence closure. For this pair we apply again the inverse of conruence closure to get \m{([\textcolor{red}{z_0}],[\textcolor{red}{z_1}])} and again to get the pair \m{([a],[b])}.\\
\m{R =  \s{([\textcolor{red}{z_0}],[\textcolor{red}{z_1}]),([\textcolor{red}{y_0}],[\textcolor{red}{y_1}]),([a],[b])}}\\
Relevant equalities:\\
\s{[a]=[b]}

Having calculated the relevant equalities, we only communicate these from \m{g^b} to \m{g^t}, and when there are no relevant equalities left to communicate from \m{g^b} to \m{g^t} and no equalities at all to communicate from \m{g^t} to \m{g^b}, we are done.

\textbf{Incremental updates:}\\
The obvious representation for \eqg{} is a partition (set of disjoint subsets) of \m{g^t}, and for \m{I^{\Gamma}} simply a set.
G can also be represented as a partition of a subset of  \m{g^t}.
R can be represented as a set of pairs, although this set can be quite large in extreme cases - a bounded fragment (in this case, bounding the depth of derivation of \eqg) can mitigate this problem at the price of completeness.

If we enhance our basic algorithm with relevance calculations, we need to update the set of relevant equalities after each \lstinline|assume| in \m{g^t}, and after updates to \m{\Gamma}.
Remember that the interpolation algorithm works more efficiently the less equations there are in $\Gamma$.

Updating after the addition of equations to \m{\Gamma} is easy - each of \eqg,\m{I^{\Gamma}},G and R can only grow.

Removing equations from \m{\Gamma} (e.g. remove \m{s=t} after we have proven the assertion \m{s\neq t}) is more complicated, 
and we might want to allow \m{\Gamma} only to grow. When we use $\Gamma$ as the approximation of a candidate model for successor nodes in the CFG as in section \ref{section:scoping:gfole}, deleting clauses (e.g. by subsumption) may allow us to weaken $\Gamma$ if we can remove equations from $\Gamma$.\\
If we merge two EC-nodes \m{u,v \in g^t} because of an equality derived in some other fragment, 
we should remove the pair \s{u,v} from R and G, however, propagating this change requires some more book-keeping.

 %We can easily see that the only changes come in G and R - when the EC-nodes u,v get merged the pair \s{u,v} is removed from G and R, and potentially more pairs need to be removed. We can perform this calculation incrementally, by remembering, for each pair \m{\s{u,v} \in R}, 
%which pairs have brought it into R - e.g. in figure \ref{example_4.2.1.4}, both the pairs \m{\s{[a],[b]},\s{[c],[d]} \in R} depend on \m{\s{[e_1],[e_2]} \in R} - once we assume \m{a=b}, the nodes \m{[e_1],[e_2]} are merged and then the pair \m{\s{[e_1],[e_2]}} goes out of R, and hence also \m{\s{[c],[d]}}. This leaves R empty and hence we are done.

\subsubsection*{Goal Relevance}\label{goal_relevance_CFG}
For binary interpolation, we can use the same relevance calculations at \m{N^b} as follows:\\
The goals for \m{N^t} are dis-equalities and relevant equalities in \m{N^b}.\\
The goals for \m{N^b} are dis-equalities and relevant equalities in \m{N^t}.\\
As we can see we might need a few rounds of communications between \m{N^t} and \m{N^b} in order to establish the exact set of goals and relevant equalities.\\
The enhanced binary interpolation algorithm with relevance alternates between communicating relevant equalities and actual equations.

\subsection{Scoping in the CFG}
For \textbf{binary interpolation}, the algorithm given above always generates an interpolant when there is one, and can operate incrementally. 
The exact interpolant extracted can be tuned (e.g. congruence closure on interface terms can be performed by \m{N^t} or by \m{N^b}).\\
For \textbf{sequence interpolants} (e.g. an unrolled loop or scoped verification of straight line code), the above algorithm will extend easily.

Each CFG-node n maintains an EC-graph \m{g_n}, which is updated as in the binary case with both its predecessor and successor.
We can communicate equalities in any order - that is, at each stage select one CFG-node and perform one stage of the above algorithm with its predecessor.

Once all consecutive pairs of CFG-nodes have been saturated - meaning that no interface node has more than one source-edge, we are done.
If source completeness is enforced we claim that the algorithm is complete for verification and the generation of sequence interpolants.

Source completeness in the CFG means that, for any pair of CFG-nodes \m{n_1,n_2}, on the path P, for any pair of EC-nodes \m{s \in g_1,t \in g_2}, if there are terms \\
\m{u \in s, v \in t} s.t. \m{\eqs{P} \models u=v} then \m{s \in \sources{2}{1}{t}} - meaning the two EC-nodes are connected by a source chain. We have seen that without joins we can achieve source completeness easily.

The argument for completeness is as for the binary case (we give a proof for a more general case later) - 
we can merge the EC-graphs of all CFG-nodes by merging each pair of EC-nodes connected by a source edge. Source completeness and the fact that each EC-node has at most one source-edge in each direction (termination condition for the binary algorithm), ensure that the resultant merged EC-graph is congruence and transitive closed - hence it is a model (as an equivalence relations) for the clauses on the path, unless one of the EC-graphs is in conflict.

For \textbf{tree interpolants} (an interpolant for a tree-shaped CFG - different from \cite{BlancGuptaKovacsKragl13} which is discussed later),
it is no longer sound to \lstinline|assume| at the EC-graph of a CFG-node an equation from a successor, as it may not hold in another successor.
Here we are essentially calculating a separate sequence interpolant for each path in the CFG, where each CFG-node calculates the interpolant between its prefix (path from the root) and, separately, \emph{each} suffix (path) to an assertion.
The simplest way to use our algorithm for sequence interpolants is to maintain, at each CFG-node, a separate EC-graph per assertion reachable from it.
We can then communicate equations only between EC-graphs of the same assertion, and the interpolant at each EC-node is the union (conjunction) of the interpolation clauses for all assertions. While this is sound and complete, this is highly inefficient, as we will have a quadratic number of EC-graphs, and we are not sharing the common work between different paths. This solution also does not extend well to the DAG case, where the number of prefixes and suffixes can be exponential.

\textbf{An example:}
Consider the example in figure \ref{example_scoping_DAG_interpolation.0}.
The interpolation clause from \m{n_0} is \m{a\neq b \lor c=d}.
We cannot simply \lstinline|assume(a=b)| at \m{g_0} because it is not sound - it will allow us to prove the assertion at \m{n_3} which does not hold.


\begin{figure}
\begin{lstlisting}
$\m{n_0}$:
c := f(a$_0$,$\m{\textcolor{red}{x}}$)
d := f(b$_0$,$\m{\textcolor{red}{x}}$)

$\m{n_1}$:
//Here $\m{\textcolor{red}{x}}$ is out of scope
a$_1$:=a$_0$
b$_1$:=b$_0$
if (a$_1$==b$_1$)
	//Here a$_{\comm{0}}$,b$_{\comm{0}}$ are out of scope
	$\m{n_2}$:
	assume a$_1$==b$_1$
	assert c==d
else	
	$\m{n_3}$:
	assume a$_1$!=b$_1$
	assert c==d //does not hold
\end{lstlisting}
\caption{Example for interpolation in the CFG}
\label{example_scoping_DAG_interpolation.0}
\end{figure}

Our solution is as follows: we maintain two EC-graphs per CFG-node, the first is \m{g_n} as in chapter \ref{chapter:ugfole} which is \emph{sound} in the sense that any equality that holds in \m{g_n} holds at the CFG-node. The second EC-graph, \m{g_n^i}, includes all the equations implied by \emph{all} suffixes, in this sense it is \emph{complete} (any equation on in-scope terms that holds on any path through n holds in \m{g_n^i}). 
The graph \m{g_n^i} is an over-approximation of the set of EC-graphs we would maintain at n in the naive implementation.

We use these two graphs as follows:\\
\m{g_n} is maintained as in the unit-ground algorithm of chapter \ref{chapter:ugfole}.\\
\m{g_n^i} maintains source-edges to \m{g_n} and, independently, to \m{g_s^i} for each direct successor s of n. \m{g_n^i} enforces the source and propagation invariant with \m{g_n} so that all equations that hold in \m{g_n} hold also in \m{g_n^i}, and all terms in \m{g_n} are \lstinline|added| to \m{g_n^i}.

Equations are communicated backwards between \m{g_n^i} and \m{g_p^i} as in the sequence case, except that no forward propagation of equations is performed. 
Instead of the forward propagation of equalities, we extract the interpolation clauses from \m{g_n^i} as described above,
represent them as clause-ECs over \m{g_n}, and propagate them as described in section \ref{section:scoping:gfole}, with a slight modification:\\
The idea is that we can rely on the completeness of superposition with selection (e.g. \cite{BachmairGanzingerSuperposition}) - 
essentially, whenever there is a negative literal (dis-equality) in a clause, we can \emph{select} a negative literal to be the only literal that participates in derivations, even if there are larger (positive or negative) literals. As our interpolation clauses are Horn-clauses (and as a propagation in a tree-shaped CFG does not use branch conditions), each non-unit clause has at least one negative literal.
Unit clauses are propagated using EC-graphs as in the unit-ground algorithm.
We propagate non-unit clauses based on the maximal term of the selected negative literal.
In our example, \m{g_0^i} encodes the equalities \s{a_0=b_0,f(a_0,\textcolor{red}{x})=c,f(b_0,\textcolor{red}{x})=d} while \m{g_0} encodes only
\s{f(a_0,\textcolor{red}{x})=c,f(b_0,\textcolor{red}{x})=d} - the added equality \m{a_0=b_0} is in \m{g_0^i} although it does not hold in all transitive successor assertions.\\
The only generated interpolation clause is \m{a\neq \underline{b} \lor c=d} - propagated only for a request for \m{b} (left-hand-side) - which only comes from \m{n_2}.\\
When propagated using our EC-graphs, it arrives at \m{n_2} as \m{[a_1] \neq [a_1] \lor [c]=[d]}
Inspecting the superposition calculus, we can see that all superposition derivations will have a unit left-hand-side and hence they are performed automatically when a clause is propagated, as the clauses are represented as clause-ECs over EC-nodes from \m{g_n}.
Equality factoring is not relevant as no clause has more than one positive literal, and hence we are left with equality resolution.
For our example the above is sufficient to prove the assertion. 
However, we do not cover all cases of equality resolution - we have to perform a new step - we perform equality resolution on the selected negative literal of a clause \emph{modulo \m{g_n^i}}, when that literal is not an interface literal with the successor.
This means that a clause \m{C \lor \underline{s\neq t}} (\m{s \neq t} selected) at a node n where \m{s,t \notin \langI}, 
is represented as \m{[C]_n \lor \underline{[s]_n \neq [t]_n}}, we then check if \m{[s]_n^i = [t]_n^i} (that is, \m{g_n^i \models s=t}).
If this is the case, we extract a set of \emph{justifications} for the equality - each possible body D for an interpolation clause with the head \m{s = t}, and for each such D we add the clause \m{[C]_n \lor [D]_n} to the set of clause-ECs at n.



\begin{figure}
\m{N^t=\s{d=f(a,\textcolor{red}{x}),e=f(b,\textcolor{red}{x}),f(c,\textcolor{red}{x})}}\\
\m{N^b_0=\s{a=b,d \neq e}}
\m{N^b_1=\s{a=c,g(d,\textcolor{blue}{m}) \neq g(d,\textcolor{blue}{m})}}
\caption{Example for justification clauses\\
The justifications for \m{d=e} are:\\
\s{a=b,a=c}\\
The interpolation clauses are:\\
\m{a\neq b \lor d=e}
\m{a\neq c \lor d=e}
}
\label{example_4.2.2.1}
\end{figure}

\noindent
For example, consider the example in figure \ref{example_4.2.2.1}.
Here we see two possible justification clauses, and hence two interpolation clauses.

We also perform standard equality resolution as a simplification - eagerly replacing \m{[C]_n \lor [s]_n \neq [s]_n} with \m{[C]_n}.
As in the non-unit case, if we derive a unit clause at n it is \lstinline|assumed| in the graph \m{g_n} (and propagated to successors and to \m{g_n^i} as before).

We claim that this is a complete algorithm for the ground unit fragment on a tree-shaped CFG - 
the reason is that we over-approximate the derivations of the superposition calculus with selection on each path in the CFG,
where unit derivations are handled by the EC-graph, As explained above, superposition is also handled by propagation and the EC-graph,
and equality resolution is handled by over-approximating the equational theory on the path, and adding the relevant assumptions to the conclusion. 
We do not give a proof here as we prove a more general result in section \ref{section:scoping:gfole}.
In this section we concentrate on the approximation encoded in \m{g_n^i} and the extraction of assumptions (Horn clause bodies) from \m{g_n^i}.

For \textbf{DAG interpolants}, we use a similar algorithm as for trees, 
except that now we have branch-conditions in clauses and hence cannot rely solely on unit superposition - we describe the solution in section \ref{section:scoping:gfole}.

%\subsection{Goal sensitivity}
%In our setting, we do not want to generate all possible interpolation clauses for the EC-graph of an CFG-node, but rather only those that might be relevant for a refutation in the CFG. \\
%We have used the relation \eqg as an approximation for the equalities in successor nodes. \\
%For the most accurate results, we can use our DAG-based algorithm above as follows:\\
%When the verification algorithm has saturated a CFG-node n (and all its transitive predecessors), 
%we want to generate all relevant interpolation clauses for all paths from n to an assertion.\\
%For each such path Q, we can collect all equalities along the path in one EC-graph \m{g_Q} (so that they are transitive and congruence closed), 
%and use \m{g_Q} as our P - we use \m{g_Q} as \m{g^b} (connecting source edges as per the source invariant) and then \m{P = \s{(u,v) \in (g_n)^2 \mid \exists t \in g_Q \cdot \s{u,v} \subseteq \sourcesA{t}}}.\\
%However, we can have an exponential number of such paths, and furthermore the graph \m{g_Q} will be non-modular in the above formulation.\\
%Hence we over-approximate P as follows: each CFG-node communicates to its direct predecessors the set of all equalities implied on the interface (as in the basic algorithm). These equalities are translated through the sources function to predecessors and at branch points we take the union of the sets from both branch sides (which is an over-approximation). This propagation proceeds in reverse topological order as normal requests until we reach dead-ends (the root, or no equalities implied on direct predecessors).\\
%We then proceed as in requests in topological order from the dead-end CFG-nodes, applying our DAG-based interpolation algorithm in order to generate the interpolation clauses which are propagated down the CFG. If our source function is complete, this algorithm is complete, 
%as P will over-approximate the equalities on all paths to assertions. As we have seen, however, in joins the source function may not be complete.\\
%The process might allow us to deduce new equalities at successor nodes which will require several passes of the above CFG traversal until saturation.\\
%If we maintain the DAG of EC-graphs at each CFG-node, the algorithm is incremental as, whenever an equality is added to P we can use the existing results and only add the necessary DAG-nodes and possibly interpolation clauses. The advantage is that we do not repeat work that was done in one path in another path.
\textbf{An example:}\\
We show an example of our abstraction of the equality theory of suffixes - in figures \ref{example_4.2.1.8_program} and \ref{example_4.2.1.9_program}.
We can see that our over-approximation can sometimes produces undesirable interpolation clauses, which would not be produced by per-path interpolation - this is the price of our approximation, but the advantage is that we use only one additional EC-graph per CFG-node.

\begin{figure}
\begin{lstlisting}
assume $\m{e_1=g(a,\textcolor{red}{x})}$
assume $\m{e_2=g(b,\textcolor{red}{x})}$
assume $\m{e_3=h(c_1,\textcolor{red}{x})=h(c_2,\textcolor{red}{x})}$
assume $\m{e_4=h(d_1,\textcolor{red}{x})=h(d_2,\textcolor{red}{x})}$
....

if (*)
	assume $\m{c_1=d_1}$
else
	assume $\m{c_2=d_2}$
assume $\m{a = b}$
assert $\m{f(e_1,e_3,\textcolor{blue}{l}) = f(e_2,e_4,\textcolor{blue}{l})}$
\end{lstlisting}
\caption{Example for CFG approximations\\
Here, for the root node, we can use \s{c_1=d_1,c_2=d_2,a=b} as the approximation for the equalities on all paths,
rather than apply the algorithm twice, once for \s{c_1=d_1,a=b} and once for \s{c_2=d_2,a=b}
}
\label{example_4.2.1.8_program}
\end{figure}

\begin{figure}
\begin{lstlisting}
assume $\m{e_1=g(a,\textcolor{red}{x})}$
assume $\m{e_2=g(b,\textcolor{red}{x})}$
assume $\m{e_3=h(c_1,c_2,\textcolor{red}{x})}$
assume $\m{e_4=h(d_1,d_2,\textcolor{red}{x})}$
...
if (*)
	assume $\m{c_1 = d_1}$
else
	assume $\m{c_2 = d_2}$
assume $\m{a = b}$
assert $\m{f(e_1,e_3,\textcolor{blue}{l}) = f(e_2,e_4,\textcolor{blue}{l})}$
\end{lstlisting}
\caption{Example for CFG over-approximation\\
Here, for the root node, if we use \s{c_1=d_1,c_2=d_2,a=b} \\
we produce the interpolation clause \m{c_1 \neq c_1 \lor c_2 \neq d_2 \lor e_3=e_4},\\
which is useless as the body holds on any path
}
\label{example_4.2.1.9_program}
\end{figure}


\subsubsection*{Extracting justifications for equalities}\label{extracting_justification}
We show here a simple algorithm for calculating the set of justifications for each equality on a pair of EC-nodes from \m{g_n} that are merged in \m{g_n^i}. 
Extracting a proof from congruence closure graphs has been done efficiently in \cite{DBLP:conf/rta/NieuwenhuisO05}, and also implemented for binary interpolation in \cite{FuchsGoelGrundyKrsticTinelli2012} (section 4.5, definition 4.2).\\
Our requirement is slightly different - we need to get the set of \emph{all} possible premises for a proof that are in $\Gamma$.\\
For example, for the set of equations \\
\m{c=f(a_1,\textcolor{red}{x})=f(b_1,\textcolor{red}{x}),d=f(a_2,\textcolor{red}{x})=f(b_2,\textcolor{red}{x})} where\\
 $\Gamma = \s{a_1=b_1=a_2=b_2=c=d}$,\\
 for the pair \s{c,d}, we must produce the sets \\
\m{\s{a_1\neq a_2},\s{a_1\neq b_2},\s{b_1\neq a_2},\s{b_1\neq b_2}}. \\
But for $\Gamma = \s{a_1=a_2,c=d}$,\\
 for the pair \s{c,d}, we only need the set \\
\m{\s{a_1 \neq a_2}}. 

Our algorithm accepts as input the graphs \m{g_n} and \m{g_n^i}, the set of interface equalities $\Gamma$, and the pair of EC-nodes \m{s,t \in g_n} s.t. \\
\m{\sourcesInv{g_n^i}{g_n}{s} = \sourcesInv{g_n^i}{g_n}{t}}.\\
We know that \m{\eqs{g_n} \cup \Gamma \models \eqs{g_n^i} \models \eqs{g_n}} and hence the algorithm must return, at least, the set\\
\s{D \subseteq \Gamma \mid D \cup \eqs{g_n} \models s=t \land \forall D' \subset D \cdot D \cup \eqs{g_n} \not\models s=t}.\\
The soundness of each set D returned is expressed by \\
\m{D \cup \eqs{g_n} \models s=t} and \m{D \subseteq (\eqs{g_n^i} \setminus \eqs{g_n})}.\\
Our algorithm simply collects all possible justifications recursively until saturation (needed as \m{g_n^i} might have cycles).\\
As the notation for the inverse source function is somewhat cumbersome, we use the notation \m{g_n^i \models u=v} when \m{u,v \in g_n} to denote that both EC-nodes u,v map to the same EC-node in \m{g_n^i}, we denote by \m{u'} the EC-node in \m{g_n^i} to which u maps and so 
\m{u'=v'} is equivalent to \m{g_n^i \models u=v}.

We define the set P(s,t) of interface premises for a pair of EC-nodes \m{s,t \in g_n} s.t. \m{s'=t'} using a set of equations, where P is the least fixed point of the equations. The equations are described in figure \ref{PR_def}. In order to calculate the set, any algorithm that constructs a least fixed point of equations works. We show a simple algorithm in figure \ref{PR_algorithm}. The algorithm implements the fixed point calculation as defined in figure \ref{PR_def}, from the bottom (equalities in $\Gamma$) up. 
Each set of pairs S represents the clause \m{\bigvee \s{s\neq t \mid \s{s,t} \in S}}. 
It is sufficient to include justifications from \m{P(s,t)} that are not subsumed in \m{P(s,t)} - this can be added in the construction algorithm or as a post-step. Also, note that we cache the values of \m{P} we have calculated, so the algorithm is incremental.


\begin{figure}
\begin{lstlisting}
var m : Map[Pair[GT],Set[Set[Pair[GT]]]

method P(s0,t0:GT) : Set[Clause]
	//All pairs are unordered
	var todo := new Queue[Pair[GT]]
	
	var RP := relevance($\m{g_n}$,$\m{g_n^i}$,{s0,t0})
	todo.enqueue(RP)
	
	while (!todo.isEmpty)
		({s,t}) := todo.dequeue
		var r : Set[Set[Pair[GT]]] := $\emptyset$

		if (s=t)
			r.add($\emptyset$)
		else if s=t$\in$$\Gamma$
			r.add({s,t})
		else
			x := $\m{\sourcesInv{}{}{s}}$ //also $\comm{=\sourcesInv{}{}{\m{t}}}$
			//transitivity
			foreach (v $\in$ $\sources{}{}{x} \setminus ${s,t})
				result.add(m[{s,v}] $\biguplus$ m[{v,t}])
			//congruence closure
			foreach ($\fa{f}{w} \in $ x.gfas)
				foreach ($\tup{u},\tup{v} \in \sources{}{}{\tup{w}}$)
					if ($\tup{u}\neq \tup{v}$)
						result.add($\m{\biguplus\limits_i}$ m[{u$_\m{i}$,v$_\m{i}$}])
		//Check if set has changed
		if (result$\setminus$m[{s,t}] $\neq$ $\emptyset$)
			enqueueSuperTerms(todo,{s,t})
		m[{s,t}].add(result)

	return m[{s0,t0}] $\setminus$ {s0,t0}
	
	method enqueueSupers({s,t}:Pair[GT])
		var x := $\m{\sourcesInv{}{}{s}}$
		foreach (u,v $\in$ $\sources{}{}{x}$)
			if ({u,v} $\in$ RP)
				todo.enqueue({u,v})
		foreach (y $\in$ rg.superTerms[x])
			foreach (u,v $\in$ $\sources{}{}{y}$)
				if ({u,v} $\in$ RP)
					todo.enqueue({u,v})
\end{lstlisting}
\caption{Justification extraction algorithm\\
The algorithm finds the fixed point for the equations in figure \ref{PR_def}.\\
The algorithm first calculates the subset of $\Gamma$ that is relevant, using the algorithm from figure \ref{relevance_algorithm}.\\
The algorithm works bottom-up, starting at interface equalities in $\Gamma$.\\
The map \lstinline|m| is an under-approximation of the function P, constructed as a fixed point.\\
The method \lstinline |enqueueSuperTerms| ensures changes in \lstinline|m| are propagated, until saturation.
}
\label{PR_algorithm}
\end{figure}


\begin{figure}
\m{P(s,t) \triangleq PR(s,t) \setminus \s{s,t}}

\bigskip

\m{\forall s \cdot \emptyset \in P(s,s)}

\bigskip

\m{\forall s,t \cdot \s{s=t} \in \Gamma \Rightarrow \s{s\neq t} \in P(s,t)}

\bigskip
Transitivity\\
\m{\forall s,t \cdot \forall u \in \sources{}{}{s'} \setminus \s{s,t} \cdot PR(s,u) \uplus PR(u,t) \subseteq P(s,t)}

\bigskip

Congruence closure\\
\m{\forall s,t \cdot \forall \fa{f}{w} \in s', \tup{u},\tup{v} \in \sources{}{}{\tup{w}} \cdot \tup{u} \neq \tup{v} \Rightarrow \biguplus\limits_i PR(u_i,v_i) \subseteq P(s,t)}


\bigskip

\noindent
Where:\\
\m{U \uplus V \triangleq \s{S \cup T \mid S \in U \land T \in V}}

\caption{Justification extraction from \m{g_n^i}\\
A justification is a set of dis-equalities on I.\\
PR(s,t) is a partial function, only defined if \m{s' = t'}\\
The set PR(s,t) is the least fixed point of the above equations.\\
The building blocks of justifications are negations of equalities in $\Gamma$.\\
The rules mimic transitivity, reflexivity and congruence closure, collecting justifications for equality.
}
\label{PR_def}
\end{figure}


\textbf{Complexity:} In section \ref{appendix:gamma_approximation} of the appendix we give an alternative algorithm for approximating $\Gamma$.
We use the alternative algorithm to calculate the space complexity bounds for interpolants - we show that the set calculated by our algorithm in \ref{PR_def} can be of exponential size.
The alternative algorithm is more complicated and we do not consider it practical in our setting, but it can be used to generate bounded interpolants - where we bound the maximal number of literals of an interpolant (losing completeness), and hence ensure a complexity bound for an incomplete algorithm.
%In the next sub-section we give an improved algorithm that can be used with a bounded fragment - for example, a bound on clause width means that we should not look for any justification longer than the allowed clause width minus one. For the simple algorithm above, 
%we can bound clause width simply by calculating bottom up and discarding each set that is too large, but this is not very efficient.

\textbf{Goal relevance:}
Extending the idea of goal relevance to sequence interpolants seems straightforward - the communication of relevant equalities proceeds as in the binary case, but now we have to communicate also transitive goals - consider the  example in figure \ref{example_4.2.2.3}.

\begin{figure}
\m{N_1 = \s{f(a_2,\textcolor{red}{x_1})=c_2,f(b_2,\textcolor{red}{x_1})=d_2}}\\
\m{N_2 = \s{c_3=c_2,d_3=d_2,a_3=a_2,b_3=b_2}}\\
\m{N_3 = \s{a_3=b_3,f(c_3,\textcolor{blue}{m}) \neq f(d_3,\textcolor{blue}{m})}}
\caption{Example for transitive propagation of goals\\
The goal \s{c_3,d_3} is propagated as \m{c_2,d_2} to \m{N_1}.}
\label{example_4.2.2.3}
\end{figure}

\noindent
Here, \s{[c_3]_3,[d_3]_3} is a goal at \m{N_3}, which is communicated to \m{N_2} as 
\s{[c_2,c_3]_2,[d_2,d_3]_2} (through source edges), 
which must be communicated to \m{N_1} as \s{[c_2,f(a_1,\textcolor{red}{x_1})]_1,[d_2,f(b_2,\textcolor{red}{x_1})]_1} in order for \m{N_1} to generate the interpolation clause-EC \m{[a_2\neq b_2 \lor c_2=d_2]_1}.\\
Note that even in the case of a non-scoped ground superposition proof, we can view dis-equality literals as goals, 
as the only derivation where a literal of the larger premise is not in the conclusion is equality resolution (equality factoring replaces a literal rather than removing it). We can view the \emph{life cycle} of each dis-equality literal in the premises where, at each proof-tree node whose sub-tree includes the premise, the dis-equality is either copied to the conclusion (if it is not maximal), rewritten using some maximal equality in another clause (negative superposition) or eliminated by equality resolution (dis-equalities that originate from equality factoring start their life cycle at the conclusion of that inference). \\
The important property is that, for each negative superposition inference in the proof-tree, the proof sub-tree that derives the rewriting clause (the left premise of negative superposition) is independent of the proof-tree deriving the clause with the maximal dis-equality.

Hence, if we use an approximation similar to our \m{g_n^i} which includes all maximal equalities in a set of ground clauses, 
we can restrict negative superposition to only clauses (right premises) where the maximal dis-equality is false under \m{g_n^i}. 
This can be extended further (assuming superposition with selection) by only allowing a clause to participate in a derivation if \emph{all} of its negative literals are false under \m{g_n^i} (some care needs to be taken regarding equality factoring).

While the above restriction might not be very effective for general superposition, in our case, we can use the idea to block clause propagation, not just clause derivation - 
if the maximal dis-equality in a clause is not false under \m{g_n^i}, there is no need to propagate it, as it cannot participate in a refutation in any successor node.

Note that in the non-ground case, the life-cycle of a dis-equality includes unification as well, but some ideas from basic superposition (e.g. \cite{DBLP:conf/esop/NieuwenhuisR92}) might help us reduce the number of unifications that need to be considered.



\subsection{Related work}
\textbf{Scoping:}\\
Lexical scoping in the form we are using was introduced in ALGOL 60 \\
(~\cite{DBLP:journals/cacm/BackusBGKMPRSVWWW60}).

\subsubsection*{Unit ground equality (UGFOLE) interpolation}
Interpolation for FOL was first introduced in \cite{Craig57a} with a constructive proof (for non-clausal, non-ground FOL) in \cite{lyndon1959}.
The use of interpolation for verification was introduced by McMillan in 
\cite{DBLP:conf/cav/McMillan03}, where interpolants for propositional logic are extracted from SAT refutations and used to refine an invariant. There has been since a lot of work on the generation and use of interpolants.
Earlier work concentrated on propositional interpolants, mostly extracted from a SAT solver proof for UNSAT on a bounded instance of a model.
Later work extended interpolation to GFOLE, linear arithmetic  (\cite{McMillan05}) and some other theories.
Most work has concentrated on extracting interpolants from SMT proofs.
We deal with the non-unit fragment in the section \ref{section:scoping:gfole}. Universally quantified clauses do not admit interpolation - there are pairs of sets of clauses whose union is inconsistent, but for which there is no universal CNF interpolant - we discuss this issue in \ref{chapter:quantification}.

In \cite{McMillan05} a calculus is given for proofs in UGFOLE which generates an interpolant from a refutation essentially by tracking the assumptions from (in our terminology) \m{N^b} in equalities derived in \m{N^t} (the system also generates interpolants for the combination of unit GFOLE and linear rational arithmetic with inequalities). The calculus used to generate proofs is similar to our CC variants and is less efficient than graph or rewriting based methods.

\cite{BonacinaJohansson2015} gives a survey on ground interpolation, for propositional logic, equality logic and theories.

In \cite{FuchsGoelGrundyKrsticTinelli2012}, the authors present a graph based interpolation method for UGFOLE. 
The algorithm uses a congruence closure (CC) graph that keeps enough information to recover a proof for each equality derived (as in \cite{DBLP:conf/rta/NieuwenhuisO05}). 
The main difference from our algorithm is that they use a single CC graph (the proof) and extract an interpolant from it, while we extract an interpolant from two (or more) separate graphs.

In their algorithm, each node represents one term and is colored as in our formulae (red, black or blue),
Equality edges are colored only red or blue (a pre-process rewrite step ensures the graph is colorable),
Equality edges stemming from axioms are coloured according to the source of the axioms.
Equality edges derived from transitive or congruence closure are coloured according to the colour of the two nodes they connect,
for interface equalities either color may be used.
The algorithm collects \m{N^b} premises (blue edges and paths) and constructs the interpolation clauses.
The authors also suggest a generalization which is an interpolation game (for general fragments, not just UGFOLE).
Two players start each with a set of formulae and, at each stage, exchange formulae, where each formula communicated from one player is implied by that player's set of formulae, and added to the other's set, and is in the common vocabulary. 
The interpolants are constructed as implications where the right side is a formula communicated forward and the left side is the conjunction of all formulae communicated backwards until that point. It is easy to see that our basic algorithm is an instance of this game. 

The extension of this game to sequence-interpolants seems simple, but the extension to trees or DAGs is not immediately obvious.
The interpolation game is a general framework for generating interpolants, 
the authors present one instance (in addition to unit ground equalities) that uses colorable proofs, which we discuss in the next section.
While our algorithm is similar to, and inspired by, theirs, the aim is different - we generate the proof and the interpolant together, rather than transform the proof in order to derive an interpolant. Our algorithm can be used for generating intrpolants from one set of equations to a set of sets of equations, which is not clear how to do with their algorithm without generating a graph per set of equations. We also add the mechanism for communicating relevance, and interpolation for trees and DAGs.

Other approaches include the encoding of equality to propositional logic and then propositional invariant generation (\cite{DBLP:conf/fmcad/KroeningW07}) and color based approaches, which we discuss in section \ref{section:scoping:ugfole}.

\subsubsection*{Interpolant classes}
In \cite{JhalaMcMillan06} the authors introduce sequence interpolants, and the idea of searching for a proof and interpolant in logical fragments of increasing strength incrementally, this is done in the context of progressively unrolling a program. Sequence interpolants have since seen much research, mostly in the context of propositional logic (e.g. \cite{DBLP:conf/fmcad/VizelG09}).

\cite{DBLP:conf/vstte/RummerHK13} gives a classification of different classes of Horn clauses, where our DAG-CFG fits in the class of recursion free linear Horn-Clauses,
 as does \cite{AlbarghouthiGurfinkelChechik12}. Our tree-shaped CFG is encoded as linear head-disjoint recursion free Horn clauses in their classification. The tree interpolants of \cite{BlancGuptaKovacsKragl13} are different as they represent a tree of meets rather than branches and joins, and is used for modeling concurrency and recursion.
\cite{AlbarghouthiGurfinkelChechik12} extracts a DAG interpolant from a proof of the unreachability of an error state in an unfolding of the program.
This interpolant is used, together with abstract interpretation, to strengthen the loop invariants in the program. 
If insufficient the loop is unrolled further. While our DAG-interpolants are similar, we extract them while searching for a proof while they extract the interpolant from an SMT proof.

\subsubsection*{Model based approaches}
Most approaches described above produce a global proof (mostly using SAT or SMT) and extract an interpolant from the proof.
Other systems use resolution based coloured local proofs and we discuss these in section \ref{section:scoping:gfole}.
Another successful approach is that of IC3/PDR (\cite{DBLP:conf/vmcai/Bradley11}, \cite{DBLP:conf/fmcad/EenMB11}).
In this approach (roughly) a program is unrolled and a failing trace is searched locally \emph{backwards} - 
each unrolled instance checks for satisfiability vs. a candidate invariant for the pre-state, if a counter example (CEX) is found it is propagated to a source in the predecessor instance (through the transition relation) and then satisfiability is checked there - if a satisfiable path is found to the initial state, the program does not verify, otherwise a clause is derived that "blocks" the CEX (a clause implied by the current candidate invariant and transition relation) and this clause is propagated forward (through the transition relation) in order to strengthen the invariant of successors. In addition checks are done on candidate invariants to see if any of them is an actual invariant for the whole program, in which case the program is proven.

Our approach is similar in that the proof is local and produced incrementally, and also in that information flows in both directions.
The main difference is that in IC3 the information that flows backwards is full models that are guaranteed, if found feasible, 
to lead to an assertion failure, while in our system each request (or backward communicated equation) represents a set of models, 
where the guarantee is that this set of models includes a model for all suffix paths to assertions (in IC3 we usually have one assertion at the end). 
The complexity difference is that IC3 may request an exponential number of models (if clause generalization does not manage to block them), and finding each potential model can take, in the worst case, exponential time (a SAT-solver call), while each such model is guaranteed to be a real model of a suffix. 
Furthermore, IC3 is formulated for sequence interpolants, while in a DAG, if we want to only propagate models that lead to assertion failures,
we would have to propagate a model on each suffix path - again a potential exponential factor.
The total number of requests and responses in our system can also be exponential (as we have seen some problems have only exponential clausal interpolants), but we do not have the exponential factor from branches and joins, as we over-approximate the set of suffix models.
We pay the price of over-approximating the set of suffix models by generating useless interpolation clauses.
Our algorithm is parametric in $\Gamma$ - if we were to use a $\Gamma$ that is exactly the set of models of suffixes, 
the complexity would be similar to IC3 - it would be interesting future work to find an abstraction for $\Gamma$ less coarse than ours,
but not as precise as per-path models - this might also be useful for IC3 if we could request the refutation of a set of models rather than one by one.
%
%A related variation on IC3 is modular SAT solving (~\cite{DBLP:conf/fmcad/BaylessVBHH13}), 
%where\\ (roughly) each unwind instance of the program gets its own SAT solver, and the proof and interpolation proceeds by the last (error state) solver deciding an unassigned variable, performing unit propagation, and if no conflict is reached it propagates the (temporary) decision to its predecessor, which performs the same steps. If a conflict is reached at some instance, a clause is generated that describes the conflict, and is propagated forward, undoing the relevant decisions. 
%While our algorithm does not make temporary decisions (each CFG-node only has clauses that hold at that program location), the simple version of $\Gamma$ is maintained as an EC-graph that cannot un-assume an assumed equality - 
%for a DAG shaped CFG, if we have managed to prove an assertion (or eliminate a CFG-node in general), we do not need to keep equalities that only hold on paths to that assertion, and so a more accurate formulation of \m{g_n^i} would be able to remove no-longer relevant equations. 
%Furthermore, the non-unit fragment that we discuss in the next section, $\Gamma$ will be an approximation of a candidate model for a set of CNF clauses, where, if a clause is e.g. subsumed, the candidate model might lose an equation - hence allowing \m{g_n^i} to undo equations might improve efficiency. We leave such an improvement as future work. 
%

\newpage
\section{Ground Clause Interpolation}\label{section:scoping:gfole}
The verification algorithm presented in section \ref{section:gfole_basic} for ground programs is complete based on the completeness of ground superposition. However, once we enforce scoping, completeness is lost, as we have seen in examples even for the ground unit fragment.

In this section we present our algorithm for ground interpolation in the CFG, and show it is sound and complete using a modification of the model generation proof.
For non unit clauses, we use the ideas from the unit algorithm - approximating the equational theory of suffixes.
In this section we only work with terms and clauses and not with EC-nodes, clause-ECs or EC-graphs.
We discuss binary interpolation, but only use $\Gamma$ rather than \m{N^b}, 
and hence we show completeness for any \m{N^b} which $\Gamma$ approximates.

The main idea is to apply the superposition calculus to \m{N^t} with, instead of standard unification,
 unification modulo an equality theory that approximates a model for \m{N^b}.

The standard completeness proof for superposition (e.g. \cite{BachmairGanzingerSuperposition}) shows how to build a model for a set of clauses saturated w.r.t. superposition that does not contain the empty clause, where the model is the smallest congruence that includes the maximal equalities of non-redundant clauses from the set. 

We define a set of equalities $\Gamma$ over \langI{} that is constructed from maximal equalities of clauses in \m{N^b} and hence over-approximate a model for \m{N^b}, and we define a congruence relation \eqg{} that satisfies all equations in $\Gamma$.
Our calculus saturates \m{N^t} w.r.t. superposition with unification modulo \eqg{}.
Whenever unification in an inference uses information from \eqg{}, we add the assumptions used, as a set of negated equations, to the conclusion, in order to preserve soundness, as \m{=_{\Gamma}} is an over-approximation.
This process implies, in fact, two-way communications between \m{N^t} and \m{N^b}, as the maximal equality of a clause in \m{N^b} can add an equality to $\Gamma$ which can enable a derivation in \m{N^t} which in turn produces a clause in \m{N^b}.

We show completeness using the model from the standard proof for \m{N^b}, and extending it with a construction similar to the standard model for \m{N^t} that is normalized w.r.t. \m{N^b}.
We use a partial ordering \m{\succ_i} instead of \m{\succ} in \m{N^t} which approximates any ordering on interface terms: 
\m{s \succ_i t} only if, for any \m{N^b}, the normal form of s (in a candidate model for \m{N^b}) is greater than the normal form of t.
The algorithm in this section can be seen as an extension of (the ground part of) both \cite{McMillan08} and \cite{BaumgartnerWaldmann13}, 
adjusted to account for the two-way communication we have described for the unit ground fragment. It is also related to \cite{KovacsVoronkov09}.
We discuss the differences in the related work section.

\subsection{Basics}
In this section we discuss the basic ideas of our algorithm for superposition based ground interpolation.\\
The basics of the algorithm are:
\begin{itemize}
	\item Superposition in \m{N^t} is done with unification modulo an equality theory \m{=_{\Gamma}} that over-approximates a model of \m{N^b}.\\
	Communication from \m{N^b} to \m{N^t} proceeds through refinement of the approximation \m{=_{\Gamma}} while communication from \m{N^t} to \m{N^b} is done through derived interpolation clauses over \langI{}.
	\item Any clause derived under an equality assumption from \m{=_{\Gamma}} is qualified with the assumption.
	\item For superposition in \m{N^t}, we use a separating partial order \m{\prec_i} that approximates the total order by assuming that any pair of terms over \langI{} is unordered. This is done by selecting a limit ordinal \m{l^t} and ensuring that the tkbo weight for each symbol in \langtp{} is larger than \m{l^t} and for each symbol in \langb{} it is smaller. We then perform ordering comparisons on an ordinal truncated at \m{l^t}.
\end{itemize}

\noindent
We begin with an example - shown in figure \ref{example.3.1.1.3.1}
\begin{figure}[H]
\m{N^t = \s{c=f(a,\textcolor{red}{x}),d=f(b,\textcolor{red}{x})}}\\
\m{N^b = \s{a=\textcolor{blue}{l},b=\textcolor{blue}{l},c \neq d}}
\caption{
The ordering is:\\
\m{\textcolor{red}{x} \succsep d \succ c \succ b \succ a \succsep \textcolor{blue}{l}}}
\label{example.3.1.1.3.1}
\end{figure}


\noindent
Here superposition would produces the following refutation:

\bigskip

\noindent
\infer[]
	{\emptyClause}
	{
			{\infer[]
				{\m{c=d}}
				{
					\infer[]
						{\m{\underline{f(\textcolor{blue}{l},\textcolor{red}{u})}=c}}
						{\m{\underline{a}=\textcolor{blue}{l}} & \m{c=f(\underline{a},\textcolor{red}{x})}}
					& 
					\infer[]
						{\m{\underline{f(\textcolor{blue}{l},\textcolor{red}{u})}=d}}
						{\m{\underline{b}=\textcolor{blue}{l}} & \m{d=f(\underline{b},\textcolor{red}{x})}}
				}
			}
		& 
			\m{c \neq d}
	}
	
\bigskip

\noindent
This refutation is not a local proof, as we have mixed-color clauses, literals and terms.
Our algorithm first communicates from \m{N^b} to \m{N^t} the approximation of the model \m{=_{\Gamma}}, 
specifically \m{a=_{\Gamma} b} derived from \s{\underline{a}=\textcolor{blue}{l},\underline{b}=\textcolor{blue}{l}}.
Next, \m{N^t} performs some superposition steps using unification modulo \m{=_{\Gamma}} to generate an interpolation clause which is communicated to \m{N^b}.
\m{N^b} uses the interpolation clause to derive the empty clause.
We use the notation \m{a=b \vdash_{\Gamma} f(a,\textcolor{red}{x})=f(b,\textcolor{red}{x})} to denote that the terms 
\m{f(a,\textcolor{red}{x}),f(b,\textcolor{red}{x})}, which are in the sub-term closure of maximal terms of \m{N^t}, 
are equal under the equality theory \m{=_{\Gamma}}, using the assumption \m{a=b}.
In order to construct $\Gamma$ we use a rough over-approximation of the model for \m{N^b} by considering any two maximal terms in clauses from \m{N^b} that appear in maximal positive literals to be equal - the set of such maximal terms we name M.
The process is summarized below:

\bigskip

\noindent
At \m{N^b}:

\bigskip

\noindent
\infer[]
{\m{a=b \vdash_{\Gamma} f(a,\textcolor{red}{x})=f(b,\textcolor{red}{x})}}
{\infer[]
	{\m{a=b \vdash_{\Gamma} a=b}}
	{
		\infer[]
		{\m{a=b \in \Gamma}}
		{
			\infer[]
			{\m{a \in M}}
			{\m{\underline{a}=\textcolor{blue}{l}}}
			& 
			\infer[]
			{\m{b \in M}}
			{\m{\underline{b}=\textcolor{blue}{l}}}
		}
	}
}
\bigskip

\noindent
At \m{N^t}:

\bigskip

\noindent
\infer[\m{a=b \vdash_{\Gamma} f(a,\textcolor{red}{x})=f(b,\textcolor{red}{x})}]
{\m{b \neq a \lor \underline{d}=c}}
{
	\m{\underline{f(a,\textcolor{red}{x})}=c}
		&
	\m{\underline{f(b,\textcolor{red}{x})}=d}
}
\bigskip

\noindent
The interpolation clause is now \m{b \neq a \lor \underline{d}=c}

\bigskip

\noindent
Back at \m{N^b}:

\bigskip

\noindent
\infer[]
	{\emptyClause}
	{
			\infer[]
				{\m{\textcolor{blue}{l} \neq \textcolor{blue}{l}}}
				{
					\m{\underline{a} = \textcolor{blue}{l}}
					& 
					{\infer[]
						{\m{\underline{a} \neq \textcolor{blue}{l}}}
						{
							\m{\underline{b} = \textcolor{blue}{l}}
							& 
							\infer[]
							{\m{\underline{b} \neq a}}
							{
								{\m{b \neq a \lor \underline{d}=c}}
									&
								\m{\underline{d} \neq c}
							}
						}
					}
				}
	}

\bigskip


\subsubsection*{Notation}
We use two \textcolor{blue}{limit terms} \m{\textcolor{blue}{l^t}} and \m{\textcolor{blue}{l^b}} to describe the minimal top term and minimal interface term, respectively. In practice, we allow \m{N^t} to include some terms that are smaller than \m{l^b}, for example  theory constants (e.g. numbers) which are always at the bottom of the ordering, and e.g. the first DSA version of a method parameter used in both pre- and post-conditions. This addition does not affect soundness or completeness, and we leave it out of the formalization for simplicity.\\
For the ordering $\succ$, we use $\succeq$ for $\succ \cup =$. \\
We use \m{\textcolor{blue}{\curlyeqsucc}} between terms to mean that any sub-term of the left term is greater than the right term - specifically, we write \m{l^t \succ t \curlyeqsucc l^b} to mean that \m{t \in \langI}.

\subsection{Approximation of the bottom model}
We describe now our interpolation calculus.\\
We use our basic approximation \m{=_{\Gamma}} for the model of \m{N^b} and the encoding of assumptions in derivations, and later discuss enhancements to both these parts.\\
We define first the set of maximal terms of clauses in a set of clauses S as \m{\textcolor{blue}{\maxTerms{S}}} - formally:
\begin{definition}{Maximal terms}

\noindent
The set of maximal terms w.r.t. $\succ$\\
\m{\maxTerm{C} \triangleq max_{\succ} \s{t \mid t \trianglelefteq C}}\\
\m{\maxTerms{S} \triangleq \s{\maxTerm{C} \mid C \in S}}

\label{def_maxTerms}
\end{definition}

We use a partial order $\prec_{\m{i}}$ to approximate $\prec$ as described below, we define the set of maximal terms w.r.t. this ordering, and maximal positive terms, as follows:
\begin{definition}{Maximal terms w.r.t. $\succ_{\m{i}}$}

\noindent
The set of maximal terms w.r.t. $\succ_{\m{i}}$\\
\m{\maxTermi{C} \triangleq \s{t \trianglelefteq C \mid \forall s \trianglelefteq C \cdot s \not\succ_{\m{i}} t}}\\
\m{\maxTermsi{S} \triangleq \s{\maxTermi{C} \mid C \in S}}\\
\m{\maxTermip{C} \triangleq \s{t \in \maxTermi{C} \mid \exists s \cdot t=s \in C \land \forall A \in C \cdot A \not\succ_{\m{i}} t=s}}\\
\m{\maxTermsip{S} \triangleq \bigcup\limits_{C \in S} \maxTermip{C}}

\label{def_maxTermsi}
\end{definition}

\noindent
The calculus to calculate $\Gamma$ is described in figure \ref{SP_P}.

\begin{figure}
\[
\begin{array}{llll}
	\vcenter{
		\infer[]
		{\m{l=r \in \Gamma}}
		{
			\m{C \lor \underline{l=r}}
		}
	}&
			\begin{array}{ll}
				\m{l^t \succ l,r \curlyeqsucc l^b}\\
				\m{l=r \succ C}
			\end{array}\\	
	
	\vspace{1cm}
	\\
	\vcenter{
		\infer[]
		{\m{l \in M}}
		{
			\m{C \lor \underline{l}=r} &
		}
	}&
	\begin{array}{ll}
		\m{l^t \succ l \curlyeqsucc l^b}\\
		\m{r \not\curlyeqsucc l^b}\\
		\m{l \succ r}\\
		\m{l=r \succ C}
	\end{array}\\	
	
	\vspace{1cm}
	\\
\vcenter{
	\infer[]
	{\m{s = t \in \Gamma}}
	{
		\m{l \in M} & 
		\m{r \in M}
	}
}	&
	\begin{array}{ll}
		\m{s,t \preceq l^t}\\
		\m{l \trianglelefteq s \trianglelefteq \maxTermsip{N^t}}\\
		\m{r \trianglelefteq t \trianglelefteq \maxTermsi{N^t}}
	\end{array}\\	
\end{array}
\]
\caption{
The calculation of the simple over-approximation $\Gamma$ of the model for \m{N^b}\\
The first rule communicates a maximal interface equality directly.\\
The second rule defines M as the maximal terms of clauses in \m{N^b} that are in \langI{} but for which the normal form in the candidate model is not representable in \langI - this corresponds exactly to our requests.\\
The third rule equates any two super terms of requested terms that are sub-terms of maximal terms (one positive) - an over-approximation.
}
\label{SP_P}
\end{figure}

\noindent
The approximation of the model of \m{N^b} abstracts all terms smaller than \m{l^b} to be equal.\\
The relation \m{\textcolor{blue}{\eqg}} is defined as the smallest congruence that satisfies all of the equalities in $\Gamma$.

\noindent
The third rule defining $\Gamma$ states that each pair of interface terms with each at least one sub-term that has a normal form that is not in the interface (that is, includes a term over \langbp) is considered potentially equal.\\
There are two reasons for this rough rule:\\
The first reason is the way we handle joins - as we have seen in chapter \ref{chapter:gfole}, 
when we have to use a non-unit fall-back at joins with EC-graphs, we get a clause where the maximal literal is an equation on a sub-term of our maximal term, hence for the over-approximation to be complete we need to consider all super-terms of each requested term.\\
The second reason is that we do not communicate any equality where the larger term includes non-interface terms - for example, consider the interpolation problem in figure \ref{example.4.1.3.2.1}:
\begin{figure}[H]
\m{N^t = \s{C,D,\lnot A \lor \underline{g(c,\textcolor{red}{y})} = \textcolor{red}{x}, \lnot B \lor \underline{g(f(b),\textcolor{red}{y})}\neq \textcolor{red}{x}}}\\
\m{N^b = \s{A,B,\lnot C \lor \underline{c}=f(h(a,\textcolor{blue}{m})),\lnot C \lor \underline{h(a,\textcolor{blue}{m})}=\textcolor{blue}{l},\lnot D \lor \underline{b}=\textcolor{blue}{l}}}
\caption{interpolation over-approximation example\\
The ordering satisfies \m{c \succ  b \succ f(h(a,\textcolor{blue}{m})) \succ h(a,\textcolor{blue}{m}) \succ \textcolor{blue}{l}}
}
\label{example.4.1.3.2.1}
\end{figure}

Here, the normal form of both c and f(b) in the model for \m{N^b} is \m{f(\textcolor{blue}{l})}, but the only terms communicated (as M) are \s{c,b}.
The inference we get is:

\bigskip

\noindent
\[
\begin{array}{llll}
	\vcenter{
		\infer[
			\inidasg{c\neq f(b)}{g(c,\textcolor{red}{y})}{g(f(b),\textcolor{red}{y})}
		]
		{\m{ c \neq f(b) \lor \lnot A \lor \lnot B \lor \textcolor{red}{x} \neq \textcolor{red}{x}}}
		{
			\m{\lnot A \lor \underline{g(c,\textcolor{red}{y})} = \textcolor{red}{x}} &
			\m{ \underline{g(f(b),\textcolor{red}{y})}\neq \textcolor{red}{x} \lor \lnot B}
		}
	} 
\\	
\end{array}
\]

\noindent
The interpolant is \s{C,D,\lnot A \lor \lnot B \lor f(b) \neq c}

Note that this encoding of $\Gamma$ is more coarse than the one we had for unit equalities.
We could replace it with our encoding of \m{g_n^i} from the previous section but include all maximal equalities, not just unit equalities. We use the simpler definition above for simplicity.

\textbf{Encoding assumptions:}\\
In order to encode the assumptions from \eqg{} we use the \newdef{interface top disagreement set \idasg{s}{t}} for two terms s,t where \m{s \eqg t}. This set is the justification set we had in section \ref{extracting_justification}, simplified.

For any s,t s.t. \m{s \eqg t}, \idasg{s}{t} is an interface clause (disjunction of dis-\\
equalities) that satisfies \\
\m{N^b \models \idasg{s}{t} \lor N^b \models s=t}\\
This is, in a sense, a negation of an interpolant, as \m{\idasg{s}{t} \lor s=t} is a tautology.\\
The formal definition is found in figure \ref{idasg_def}.\\
We select recursively the top interface-terms that are not identical and add their negation to the clause - for example \\
\m{\idasg{g(f(a),\textcolor{red}{y})}{g(f(b),\textcolor{red}{y})} = f(a) \neq f(b)} \\
rather than \m{a \neq b}.\\
However, note that this is the weakest such interpolant, but not the most concise - for example:\\
\m{\idasg{g(a,f(a),\textcolor{red}{y})}{g(b,f(b),\textcolor{red}{y})} = f(a) \neq f(b) \lor a \neq b} \\
Which is equivalent to, but less concise than \\
\m{a \neq b}\\
And also \\
\m{\idasg{g(a,b,c,\textcolor{red}{y})}{g(c,a,b,\textcolor{red}{y})} = a \neq c \lor b \neq a \lor b \neq c} \\
Where a more concise option is \\
\m{a \neq b \lor b \neq c}.\\
A simple method to get optimal assumptions would be to work with equality constrained clauses (e.g. as in \cite{DBLP:conf/cade/NieuwenhuisR92}) where the equality constraint will include all our assumptions. 
We can represent the equality constraint as an EC-graph and extract justifications only when the clause is in \langI.
This simplifies also the implementation for more precise versions of $\Gamma$ described later. 
We leave this as future work.

\begin{figure}
\m{\idasg{s}{t} \triangleq \bigvee \idasgs{s}{t}}

\bigskip

\noindent
\m{\idasgs{s=\fa{f}{u}}{t=\fa{g}{v}} \triangleq
	\begin{cases}
		\emptyset                                            & \m{s \equiv t}\\
		\s{s \neq t}                                         & \m{s,t \prec{l^t}}\\
		\bigcup\limits_{i} \idasgs{u_i}{v_i}                 & \m{f \equiv g}
	\end{cases}
}
\caption{Top interface disagreement set definition\\
\idasg{s}{t} is a partial function, only defined if \m{s \eqg t}
}
\label{idasg_def}
\end{figure}

\subsection{Interpolation ordering}
We use a partial ordering \m{\textcolor{blue}{\prec_i}} for our interpolation calculus, which is an approximation of the total order $\prec$:\\
The order is essentially a truncated tkbo (transfinite Knuth-Bendix ordering described in section \ref{section:preliminaties:ordering} )- \textcolor{blue}{ttkbo} - the definition is as per tkbo except:
\begin{itemize}
	\item Each interface term has a tkbo weight of 0
	\item Each interface term is unordered with any other interface term
\end{itemize}

\noindent
For example:\\
\m{N^b = \s{A \lor \underline{a}=\textcolor{blue}{l},A \lor \underline{f(b)}=\textcolor{blue}{m}}}\\
Assuming that \m{\textcolor{red}{y} \succ \textcolor{red}{x}} but \m{weight(\textcolor{red}{y})=weight(\textcolor{red}{x})}:\\
\m{g(a,\textcolor{red}{x}),g(f(a),\textcolor{red}{x}),g(f(b),\textcolor{red}{x})} are unordered w.r.t. \m{\prec_i}.\\
\m{g(\textcolor{red}{y},a) \succ_i g(\textcolor{red}{x},f(b)) \succ_i g(f(b),\textcolor{red}{y}) \succ_i g(b,\textcolor{red}{x})}.\\
Note that, although \m{N^t} knows that b is its own normal form, it cannot order the normal forms of \m{b,f(b)}, 
while \m{N^b} can.

\noindent
The following properties of the ordering are easy to see:
\begin{lemma}
\m{\succ_i \subseteq \succ}\\
\m{s \succ_i t \Rightarrow s \succ t}\\
\m{\forall s,t \succeq l^t \cdot s \vartriangleright t \Rightarrow s \succ_i t}
\label{lemma_succ_i}
\end{lemma}

\noindent
For \m{N^b} we use the standard ordering $\prec$.\\
When we are interpolating in the CFG, each EC-node will have one pair of \m{l^t,l^b} for all direct predecessors and one for all direct successors,
and will apply our interpolation calculus \m{SP_I} with the pair for the successors.

\noindent
We could use the abstraction $\Gamma$ also for $\succ_{\m{i}}$, so that we could order more terms (e.g. interface terms that do not have any sub-term as a maximal term in \m{N^b}). This would mean that the ordering might change when a new clause is derived at \m{N^b}.\\
The problem is that the ordering can be inverted  - for example, if \m{b \succ_i a} because neither b nor a are maximal terms in \m{N^b},
but once \m{b = \textcolor{blue}{l}} is derived at \m{N^b}, we cannot anymore order them, and so some more inferences might become valid.
On the one hand it is likely that many interface terms never become maximal terms, and so we can prevent more derivations from taking place, on the other hand, the overhead of incrementally updating the ordering might be expensive. We leave this as future work.

\subsection{Interpolation superposition calculus}
The modified superposition calculus we use for interpolation (for binary interpolation, for \m{N^t}) is shown in figure \ref{SP_I}.

\begin{figure}
When each premise P satisfies \m{P \succeq l^t}:

\[
\begin{array}{llll}

	\vspace{1cm}
	
	\m{sup^i_{=}} &
	\vcenter{
		\infer[]
		{\m{B \lor C \lor D \lor \termRepAt{s}{r}{p}=t}}
		{
			\m{C \lor \underline{l}=r } 
				&
			\m{\underline{s} = t \lor D}
		}
	}&
			\begin{array}{ll}
				\m{\sci{1} l \not\preceq_i r, \sci{2} l=r \not\preceq_i C}\\
				\m{\sci{3} s \not\preceq_i t, \sci{4} s=t \not\preceq_i D}\\
				\m{\sci{5} s=t \not\preceq_i l=r}\\
				\m{\sci{6} \mathbf{l,s \succeq l^t}}\\
				\m{\sci{7} \mathbf{B = \idasg{l}{\termAt{\mathbf{s}}{\mathbf{p}}}}}
			\end{array}\\	
	
	\vspace{1cm}
	
	
	\m{sup^i_{\neq}}&
	\vcenter{
		\infer[]
		{\m{B \lor C \lor D \lor \termRepAt{s}{r}{p}\neq t}}
		{
			\m{C \lor \underline{l}=r} 
				&
			\m{\underline{s} \neq t \lor D}
		}
	} &

			\begin{array}{ll}
				\m{\sci{1} l \not\preceq_i r, \sci{2} l=r \not\preceq_i C}\\
				\m{\sci{3} s \not\preceq_i t, \sci{4} s\neq t \succeq D}\\
				\m{\sci{5} s \neq t \not\prec_i l=r}\\
				\m{\sci{6} \mathbf{l,s \succeq l^t}}\\
				\m{\sci{7} \mathbf{B = \idasg{l}{\termAt{\mathbf{s}}{\mathbf{p}}}}}
			\end{array}	\\
	
	\vspace{1cm}

	\m{res^i_{=}}&
	\vcenter{
		\infer[]
		{\m{B \lor C}}
		{
			\m{C \lor \underline{s \neq t}}
		}
	} &

			\begin{array}{ll}
				\m{\sci{1} s \neq t \succeq C}\\
				\m{\sci{2} \mathbf{s \succeq l^t}}\\
				\m{\sci{3} \mathbf{B = \idasg{s}{t}}}
			\end{array}
		\\

	\vspace{1cm}
	\m{fact^i_{=}}&
	\vcenter{
		\infer[]
		{\m{B \lor C \lor r \neq t \lor s = r}}
		{
			\m{C \lor l = r \lor \underline{s = t}}
		}
	} &
			\begin{array}{ll}
				\m{\sci{1} s \not\preceq_i t, \sci{2} s=t \not\preceq_i C \lor l=r }\\
				\m{\sci{3} l \not\preceq_i r}\\
				\m{\sci{4} \mathbf{s,l \succeq l^t}}\\
				\m{\sci{5} \mathbf{B = \idasg{s}{l}}}
			\end{array}	\\

\end{array}
\]

When, for each premise P, \m{P \prec l^t}, we use the standard calculus \SPG{}.\\
There are no binary inferences between premises \m{C,D} s.t. \m{C \succ l^t \succ D}.

\caption{
The interpolation superposition calculus \m{SP_I}
}
\label{SP_I}
\end{figure}

The calculus \m{SP_I} is a straightforward extension of our unit ground interpolation algorithm to non-unit clauses.
The main differences from the standard calculus \SPG{} are:
\begin{itemize}
	\item We use \m{\not\prec_i} instead of \m{\succ}, as \m{\prec_i} is a partial order.
	\item Instead of the standard (ground) unification \m{l = \termAt{s}{p}} we use \m{l \eqg \termAt{s}{p}}\\
		We write it as \m{B = \idasg{l}{\termAt{s}{p}}} and use B in the conclusion
	\item We add (disjunct) the top interface disagreement set of the unification to the conclusion of each rule 
	\item The calculus applies only to clauses in \m{N^t} - we use SP for clauses in \m{N^b}
\end{itemize}


\begin{definition}{\textbf{Colored clause sets}}

\noindent
A set of clauses N is \textcolor{blue}{colored} (w.r.t. \m{l^t,l^b,\prec}) iff \\
\m{\forall C \in N \cdot (C \curlyeqsucc l^b \lor C \prec l^t)}.\\
We repeat the definitions from before for the two parts of N:\\
\m{N^t \triangleq \s{C \in N \mid C \curlyeqsucc l^b}}\\
\m{N^b \triangleq \s{C \in N \mid C \prec l^t}}\\
As \m{ l^t \succ l^b}, \m{N^t \cap N^b = \emptyset}.
\label{coloured_set}
\end{definition}

\noindent
The calculus \m{SP_I} preserves the property of being colored:
\begin{lemma}{Preservation of the coloured property by \m{SP_I}}

\noindent
Our calculus has the property that:\\
\m{\forall N \cdot N=N^t \cup N^b \Rightarrow SP_I(N)=(SP_I(N))^t \cup (SP_I(N))^b}\\
Which means that if we begin with a colored set of clauses, the result of any derivation in \m{SP_I} will also be colored.
\end{lemma}



\textbf{Completeness:}
The completeness proof for our ground interpolation calculus is given in section \ref{appendix:interpolation:completeness} of the appendix.
The proof follows the common model generation method for completeness proofs, where the model is divided to a model for \m{N^b} and a a model for \m{N^t}.

%
%%%%%%%%%%%%%%%%%%%%%%%%%%%%%%%%%%%%%%%%%%%%%%%%%%%%%%%%%%%%%%%%%%%%%%
%\subsection{Redundancy elimination}
%Redundancy elimination is an important part of the strength of superposition - many generated clauses can be eliminated, reducing the search space of the prover.
%In this section we examine the interaction of simplification inferences with our scoped proof and interpolation algorithm.\\
%Consider the following binary interpolation problem:\\
%\m{N^t = \{A,\lnot B \lor \textcolor{red}{x}=g(a,\underline{\textcolor{red}{z}}),\lnot B \lor \textcolor{red}{y}=g(b,\underline{\textcolor{red}{z}}),}\\
%\m{\lnot B \lor f(\underline{\textcolor{red}{x}},c)=\textcolor{red}{w}, \lnot B \lor f(\underline{\textcolor{red}{y}},d) \neq \textcolor{red}{w}\}}\\
%\m{N^b = \s{B,\textcolor{blue}{M},\lnot A \lor \underline{a}=\textcolor{blue}{l},\lnot A \lor \underline{b}=\textcolor{blue}{l},\lnot \textcolor{blue}{M} \lor \underline{c}=\textcolor{blue}{m},\lnot \textcolor{blue}{M} \lor \underline{d}=\textcolor{blue}{m}}}\\
%With \\
%\m{\textcolor{red}{z} \succ \textcolor{red}{y} \succ \textcolor{red}{x} \succ \textcolor{red}{w} \succsep d \succ c \succ b \succ a \succ B \succ A \succsep \textcolor{blue}{m} \succ \textcolor{blue}{l} \succ \textcolor{blue}{M} }\\
%The interpolation clauses are \\
%\s{A,B, \lnot C \lor \lnot D \lor a \neq b \lor c \neq \underline{d} }
%
%\bigskip
%
%\noindent
%A non-local proof would be:
%
%\bigskip
%
%\noindent
%\infer[]
%{\m{\lnot A \lor \lnot B \lor \underline{\textcolor{red}{y}}=\textcolor{red}{x}}}
%{
	%\infer[]
		%{\m{\lnot A \lor \lnot B \lor \underline{g(\textcolor{blue}{l},\textcolor{red}{z})}=\textcolor{red}{x}}}
		%{
			%\m{\lnot A \lor \underline{a}=\textcolor{blue}{l}}
				%& 
			%\m{g(\underline{a},\textcolor{red}{z})=\textcolor{red}{x} \lor \lnot B} 
		%}
			%&
	%\infer[]
		%{\m{\underline{g(\textcolor{blue}{l},\textcolor{red}{z})}=\textcolor{red}{y} \lor \lnot A \lor \lnot B}}
		%{
			%\m{\lnot A \lor \underline{b}=\textcolor{blue}{l}}&
			%\m{g(\underline{b},\textcolor{red}{z})=\textcolor{red}{y} \lor \lnot B}
		%}
%}
%
%\bigskip
%
%\noindent
%\infer[]
%{\m{\underline{f(\textcolor{red}{x},\textcolor{blue}{m})\neq\textcolor{red}{w}} \lor \lnot \textcolor{blue}{M} \lor \lnot A \lor \lnot B}}
%{
	%\m{\lnot \textcolor{blue}{M} \lor \underline{d}=\textcolor{blue}{m}}
		%& 
	%\infer[]
	%{\m{\lnot A \lor \lnot B \lor \underline{f(\textcolor{red}{x},d)} \neq \textcolor{red}{w}}}
	%{
		%\m{\lnot A \lor \lnot B \lor \underline{\textcolor{red}{y}}=\textcolor{red}{x}}
			%&
		%\m{f(\underline{\textcolor{red}{y}},d) \neq \textcolor{red}{w} \lor \lnot B}
	%}
%}
%\bigskip
%
%\noindent
%\infer[]
%{\emptyClause}
%{
	%\infer*[]
	%{}
	%{
		%\infer[]
		%{\m{\lnot \textcolor{blue}{M} \lor \lnot A \lor \underline{\lnot B}  }}
		%{
			%\infer[]
			%{\m{\lnot \textcolor{blue}{M} \lor \lnot B \lor \underline{f(\textcolor{red}{x},\textcolor{blue}{m})=\textcolor{red}{w}}}}
			%{
				%\m{\lnot \textcolor{blue}{M} \lor \underline{c}=\textcolor{blue}{m}}
					%&
				%\m{f(\textcolor{red}{x},\underline{c})=\textcolor{red}{w} \lor \lnot B } 
			%}
					%&
			%\m{\underline{f(\textcolor{red}{x},\textcolor{blue}{m})\neq\textcolor{red}{w}} \lor \lnot \textcolor{blue}{M} \lor \lnot A \lor \lnot B }
		%}
	%}
%}
%
%
%\bigskip
%
%\noindent
%And the version with eager simplifications:
%
%
%\bigskip
%
%\infer[]
%{\m{\underline{\textcolor{red}{y}}=\textcolor{red}{x}}}
%{
	%\infer[]
	%{\m{\underline{g(\textcolor{blue}{l},\textcolor{red}{z})}=\textcolor{red}{x}}}
	%{
		%\m{
			%\infer[]
			%{\m{\underline{a}=\textcolor{blue}{l}}}
			%{\m{A} & \hcancel{\m{\lnot A \lor \underline{a}=\textcolor{blue}{l}}}}
		%}
			%&
		%\infer[]
		%{\hcancel{g(\underline{a},\textcolor{red}{z})=\textcolor{red}{x}}}
		%{\m{B} & \hcancel{\lnot B \lor \textcolor{red}{x}=\underline{g(a,\textcolor{red}{z})}}}
	%}
		%&
	%\infer[]
	%{\hcancel{\underline{g(\textcolor{blue}{l},\textcolor{red}{z})}=\textcolor{red}{y}}}
	%{
		%\m{
			%\infer[]
			%{\m{\underline{b}=\textcolor{blue}{l}}}
			%{\m{A} & \hcancel{\m{\lnot A \lor \underline{b}=\textcolor{blue}{l}}}}
		%}
			%&
		%\infer[]
		%{\hcancel{\underline{g(b,\textcolor{red}{z})}=\textcolor{red}{y}}}
		%{\m{B} & \hcancel{\lnot B \lor \textcolor{red}{y}=\underline{g(b,\textcolor{red}{z})}}}
	%}
%}
%\bigskip
%\bigskip
%
%\bigskip
%
%\noindent
%\infer[]
%{\m{f(\textcolor{red}{x},\textcolor{blue}{m}) = \textcolor{red}{w}}}
%{
	%\infer[]
	%{\m{\underline{c}=\textcolor{blue}{m}}}
	%{\m{\textcolor{blue}{M}} & \hcancel{\m{\lnot \textcolor{blue}{M} \lor \underline{c}=\textcolor{blue}{m}}}}
		%&
	%\infer[]
	%{\hcancel{\m{f(\textcolor{red}{x},\underline{c}) = \textcolor{red}{w}}}}
	%{
		%\m{B} & \hcancel{\lnot B \lor f(\underline{\textcolor{red}{x}},c) = \textcolor{red}{w}}
	%}
%}
%
%\bigskip
%
%\noindent
%\infer[]
%{\m{f(\textcolor{red}{x},\textcolor{blue}{m}) \neq \textcolor{red}{w}}}
%{
	%\infer[]
	%{\m{\underline{d}=\textcolor{blue}{m}}}
	%{\m{\textcolor{blue}{M}} & \hcancel{\m{\lnot \textcolor{blue}{M} \lor \underline{d}=\textcolor{blue}{m}}}}
		%&
	%\infer[]
	%{\hcancel{\m{f(\textcolor{red}{x},\underline{d}) \neq \textcolor{red}{w}}}}
	%{
		%\m{\underline{\textcolor{red}{y}}=\textcolor{red}{x}}
			%&
		%\infer[]
		%{\hcancel{\m{f(\underline{\textcolor{red}{y}},d) \neq \textcolor{red}{w}}}}
		%{\m{B} & \hcancel{\lnot B \lor f(\underline{\textcolor{red}{y}},d) \neq \textcolor{red}{w}}}
	%}
%}
%
%\bigskip
%
%\noindent
%\infer[]
%{\emptyClause}
%{
	%\m{f(\textcolor{red}{x},\textcolor{blue}{m}) = \textcolor{red}{w}}
		%&
	%\m{f(\textcolor{red}{x},\textcolor{blue}{m}) \neq \textcolor{red}{w}}
%}
%
%\bigskip
%
%
%
%\noindent
%Most intermediate conclusions in the above derivation are simple unit clauses.
%
%\bigskip
%
%\noindent
%The local proof without simplifications goes as follows:\\
%\noindent
%At \m{N^b}:
%
%\bigskip
%
%\noindent
%\infer[]
%{\m{a=b \in \Gamma}}
%{
	%\infer[]
	%{\m{a \in M}}
	%{\m{\lnot A \lor \underline{a}=\textcolor{blue}{l}}}
		%&
	%\infer[]
	%{\m{b \in M}}
	%{\m{\lnot A \lor \underline{b}=\textcolor{blue}{l}}}
%}
%
%\bigskip
%
%\noindent
%\infer[]
%{\m{c=d \in \Gamma}}
%{
	%\infer[]
	%{\m{c \in M}}
	%{\m{\lnot \textcolor{blue}{L} \lor \underline{c}=\textcolor{blue}{m}}}
		%&
	%\infer[]
	%{\m{d \in M}}
	%{\m{\lnot \textcolor{blue}{M} \lor \underline{d}=\textcolor{blue}{m}}}
%}
%
%\bigskip
%
%%\noindent
%%.....\\
%%\m{a=b=c=d \in P}
%%
%%\bigskip
%
%\noindent
%Then at \m{N^t}:
%
%\bigskip
%
%\noindent
%
%\bigskip
%
%\noindent
%\infer[]
%{\m{\lnot B \lor a \neq b \lor c \neq \underline{d} }}
%{
	%\m{
		%\infer[]
		%{\m{\lnot B \lor a \neq b \lor c \neq d \lor \textcolor{red}{w} \neq \textcolor{red}{w}}}
		%{
			%\m{\lnot B \lor \underline{f(\textcolor{red}{x},c)}=\textcolor{red}{w}}
				%&
%\infer[]
%{\m{\lnot B \lor a \neq b \lor \underline{f(\textcolor{red}{x},d)} \neq \textcolor{red}{w}}}
%{
	%\infer[]
	%{\m{\lnot B \lor a \neq b \lor \underline{\textcolor{red}{y}}=\textcolor{red}{x}}}
	%{
		%\m{\lnot B \lor \textcolor{red}{x}=\underline{g(a,\textcolor{red}{z})}} 
			%&
		%\m{\lnot B \lor \textcolor{red}{y}=\underline{g(b,\textcolor{red}{z})}} 
	%}
		%&
	%\m{\lnot B \lor f(\underline{\textcolor{red}{y}},d) \neq \textcolor{red}{w}}
%}
%%			\m{\lnot B \lor a \neq b \lor \underline{f(\textcolor{red}{x},d)} \neq \textcolor{red}{w}}
		%}
	%}
%}
%
%\bigskip
%
%
%
%\noindent
%And again at \m{N^b}:
%
%\bigskip
%
%\noindent
%\infer[]
%{\m{\lnot \textcolor{blue}{M} \lor \lnot B\lor a \neq \underline{b} \lor \textcolor{blue}{m} \neq \textcolor{blue}{m} }}
%{
	%\m{\lnot \textcolor{blue}{M} \lor \underline{c}=\textcolor{blue}{m}}
		%&
	%\m{
		%\infer[]
		%{\m{\lnot \textcolor{blue}{M} \lor \lnot B \lor a \neq b \lor \underline{c} \neq \textcolor{blue}{m} }}
		%{
			%\m{\lnot \textcolor{blue}{M} \lor \underline{d}=\textcolor{blue}{m}}
				%&
			%\m{\lnot B \lor a \neq b \lor c \neq \underline{d} }
		%}
	%}
%}
%
%
%\bigskip
%
%\noindent
%\infer[]
%{\m{\lnot \textcolor{blue}{M} \lor \lnot A \lor \lnot \underline{B} \lor \textcolor{blue}{l} \neq \textcolor{blue}{l} \lor \textcolor{blue}{m} \neq \textcolor{blue}{m} }}
%{
	%\m{\lnot A \lor \underline{a}=\textcolor{blue}{l}}
		%&
	%\m{
		%\infer[]
		%{\m{\lnot \textcolor{blue}{M} \lor \lnot A \lor \lnot B \lor \underline{a} \neq \textcolor{blue}{l} \lor \textcolor{blue}{m} \neq \textcolor{blue}{m} }}
		%{
			%\m{\lnot A \lor \underline{b}=\textcolor{blue}{l}}
				%&
			%\m{ \lnot \textcolor{blue}{M} \lor \lnot B \lor a \neq \underline{b} \lor \textcolor{blue}{m} \neq \textcolor{blue}{m} }
		%}			
%}
%}
%
%
%
%\bigskip
%
%
%\noindent
%\infer[]
%{\emptyClause}
%{
		%\m{\textcolor{blue}{M}}
		%&
		%\infer[]
		%{\m{\lnot \textcolor{blue}{M}}}
		%{
			%\infer[]
			%{\m{\m{\lnot \textcolor{blue}{M} \lor \underline{\textcolor{blue}{l} \neq \textcolor{blue}{l}}}}}
			%{
				%\infer[]
				%{\m{\m{\lnot \textcolor{blue}{M} \lor \textcolor{blue}{l} \neq \textcolor{blue}{l} \lor \underline{\textcolor{blue}{m} \neq \textcolor{blue}{m}} }}}
				%{
					%\m{A}
						%&
					%\infer[]
					%{\m{\lnot \textcolor{blue}{M} \lor \underline{\lnot A} \lor \textcolor{blue}{l} \neq \textcolor{blue}{l} \lor \textcolor{blue}{m} \neq \textcolor{blue}{m} }}
					%{\m{B} & \m{\lnot \textcolor{blue}{M} \lor \lnot A \lor \underline{\lnot B} \lor \textcolor{blue}{l} \neq \textcolor{blue}{l} \lor \textcolor{blue}{m} \neq \textcolor{blue}{m} }}
				%}
			%}	
		%}
%}
%\bigskip
%
%
%\noindent
%The local proof with simplifications proceeds as follows:
%
%\noindent
%\m{N^t} sends \s{A} to \m{N^b}.
%
%\bigskip
%
%\noindent
%At \m{N^b}:
%
%\bigskip
%
%\noindent
%\infer[]
%{\m{a=b\in \Gamma}}
%{
	%\infer[]
	%{\m{a \in M}}
	%{
		%\infer[]
		%{\m{\underline{a}=\textcolor{blue}{l}}}
		%{
			%\m{A}
				%&
			%\hcancel{\lnot A \lor \underline{a}=\textcolor{blue}{l}}
		%}
	%}
		%&
	%\infer[]
	%{\m{b \in M}}
	%{
		%\infer[]
		%{\m{\underline{b}=\textcolor{blue}{l}}}
		%{
			%\m{A}
				%&
			%\hcancel{\lnot A \lor \underline{b}=\textcolor{blue}{l}}
		%}
	%}
%}
%
%\bigskip
%
%\noindent
%\infer[]
%{\m{c=d \in \Gamma}}
%{
	%{
		%\infer[]
		%{\m{c \in M}}
		%{
			%\infer[]
			%{\m{\underline{c}=\textcolor{blue}{m}}}
			%{
				%\m{\textcolor{blue}{M}}
					%&
				%\hcancel{\lnot \textcolor{blue}{M} \lor \underline{c}=\textcolor{blue}{m}}
			%}
		%}
	%}
		%&
	%\infer[]
	%{\m{d \in M}}
	%{
		%\infer[]
		%{\m{\underline{d}=\textcolor{blue}{m}}}
		%{
			%\m{\textcolor{blue}{M}}
				%&
			%\hcancel{\lnot \textcolor{blue}{M} \lor \underline{d}=\textcolor{blue}{m}}
		%}
	%}
%}
%
%\bigskip
%
%%\noindent
%%.....\\
%%\m{a=b=c=d \in P}
%%
%%\bigskip
%
%\noindent
%Then at \m{N^t}:
%
%\bigskip
%
%\noindent
%\infer[]
%{\m{\lnot B \lor a \neq b \lor \underline{f(\textcolor{red}{x},d)} \neq \textcolor{red}{w}}}
%{
	%\infer[]
	%{\m{\lnot B \lor a \neq b \lor \underline{\textcolor{red}{y}}=\textcolor{red}{x}}}
	%{
		%\m{\lnot B \lor \textcolor{red}{x}=\underline{g(a,\textcolor{red}{z})}} 
			%&
		%\m{\lnot B \lor \textcolor{red}{y}=\underline{g(b,\textcolor{red}{z})}} 
	%}
		%&
	%\m{\lnot B \lor f(\underline{\textcolor{red}{y}},d) \neq \textcolor{red}{w}}
%}
%
%\bigskip
%
%\noindent
%\infer[]
%{\m{\lnot B \lor a \neq b \lor c \neq \underline{d} }}
%{
	%\infer[]
	%{\hcancel{\lnot B \lor a \neq b \lor c \neq d \lor \underline{\textcolor{red}{w} \neq \textcolor{red}{w}}}}
	%{
		%\m{\lnot B \lor \underline{f(\textcolor{red}{x},c)}=\textcolor{red}{w}}
			%&
		%\m{\lnot B \lor a \neq b \lor \underline{f(\textcolor{red}{x},d)} \neq \textcolor{red}{w}}
	%}
%}
%
%\bigskip
%
%
%
%\noindent
%And again at \m{N^b}:
%
%\bigskip
%\infer[]
%{\emptyClause}
%{
	%\infer[]
	%{\hcancel{\textcolor{blue}{l} \neq \textcolor{blue}{l}}}
	%{
		%\m{\underline{a}=\textcolor{blue}{l}}
			%&
		%\infer[]
		%{\hcancel{\underline{a} \neq \textcolor{blue}{l} }}
		%{
			%\m{\underline{b}=\textcolor{blue}{l}}
				%&
			%\infer[]
			%{\hcancel{\underline{b} \neq a}}
			%{
				%\infer[]
				%{\hcancel{a \neq \underline{b} \lor \textcolor{blue}{m} \neq \textcolor{blue}{m} }}
				%{
					%\m{\underline{c}=\textcolor{blue}{m}}
						%&
					%\infer[]
					%{\hcancel{a \neq b \lor \underline{c} \neq \textcolor{blue}{m} }}
					%{
						%\m{\underline{d}=\textcolor{blue}{m}}
							%&
						%\infer[]
						%{\hcancel{a \neq b \lor c \neq \underline{d} }}
						%{
							%\m{B}
								%&
							%\hcancel{\lnot B \lor a \neq b \lor c \neq \underline{d} }
						%}
					%}
				%}
			%}
		%}			
	%}
%}
%
%\bigskip
%
%\noindent
%
%\noindent
%We can see that, while the local proof benefits from eager simplification, it cannot be made as narrow as the non-local proof.
%Specifically, interpolation clauses can collect many literals that have to be propagated (in our setting) throughout the CFG
%until the CFG-node where they can be simplified.

%\subsection{Completeness proof}


\subsection{Ordering for Interpolation}
In this section we describe how we establish an adequate total order on ground terms for interpolation for the whole CFG.

Extending our interpolation calculus to a DAG CFG requires that we have a separating total order on all program constants, so that whenever a constant in scope at a CFG-node is not in scope in the successor, it is separated from the any constant that is in scope in the successor.\\
For a linear program, we can simply assign decreasing limit ordinals to each CFG-node, so that the root node has the maximal limit ordinal and each successor has a lower limit ordinal. We then assign each constant the weight \m{n\omega+1} where n is the limit ordinal index of the last CFG-node where it appears.\\
However, this solution does not extend immediately to tree-shaped CFGs, as different branches may imply different orderings. Consider, for example, figure \ref{snippet4.5.1}:
\begin{figure}[H]
\begin{lstlisting}
$\m{n_1:}$
	assume a$\neq$b
if (*)
	$\m{n_2:}$
		assume f(a)=f(b)
	$\m{n_3:}$
		assert f(a)=c
else
	$\m{n_4:}$
		assume f(a)$\neq$f(b)
	$\m{n_5:}$
		assert f(b)=c
\end{lstlisting}
\caption{Interpolation ordering for trees}
\label{snippet4.5.1}
\end{figure}

The \lstinline|then| branch implies that \m{b\succsep a} while the \lstinline|else|  branch implies that \m{a \succsep b}.
There is no total order with minimal scoping that respects the separating constraint for this program.

Our solution to this problem is to add a further DSA version for every constant at a branch that is in scope in at least one successor of the branch, unless it is in scope at the join of the branch.
For figure \ref{snippet4.5.1} the modified program is shown in figure \ref{snippet4.5.1b}:
\begin{figure}[H]
\begin{lstlisting}
$\m{n_1:}$
	assume a$\neq$b
if (*)
	$\m{n_2.0:}$
		assume a$_1$=a
		assume b$_1$=b
	$\m{n_2:}$
		assume f(a$_1$)=f(b$_1$)
	$\m{n_3:}$
		assert f(a$_1$)=c
else
	$\m{n_4.0:}$
		assume a$_2$=a
		assume b$_2$=b
	$\m{n_4:}$
		assume f(a$_2$)$\neq$f(b$_2$)
	$\m{n_5:}$
		assert f(b$_2$)=c
\end{lstlisting}
\caption{Interpolation ordering for trees - additional DSA versions}
\label{snippet4.5.1b}
\end{figure}

After this transformation, and after ensuring that the scope for each constant is contiguous (by renaming one instance of a constant that appears on two parallel branches but neither at branch nor join point), we can choose any topological order of the nodes and assign limit ordinal indices accordingly.

\subsection{Related and future work}
The main difference in our work from other works is the setting in which we work - verification in a DAG-shaped CFG.
Our work is closely related to \cite{BaumgartnerWaldmann13} and \cite{McMillan08}. The main difference from both papers, beside the setting, is that our calculus remains in the ground fragment for ground programs, while other works use a form of abstraction that adds quantified variables to a clause - the complexity of most proof steps is lower for ground clauses.

In \cite{BaumgartnerWaldmann13} (which is an extension of \cite{DBLP:journals/aaecc/BachmairGW94}), the authors present a superposition calculus that works modulo a theory. Their calculus essentially calculates interpolants between a set of clauses and a theory - the difference with our calculus, besdies that we remain in the ground fragment for a ground input, is that we can use equalities from the theory to direct interpolation, hence improving accuracy.

The theory is encoded as the rule

\bigskip

\noindent
\infer[]
{\emptyClause}
{\m{C_1 .... C_n}}

\bigskip

Where \m{C_1 .. C_n} are a set of clauses over the signature of the theory which are unsatisfiable in the theory.\\
The calculus of \cite{BaumgartnerWaldmann13} is similar to superposition, 
except that clauses that participate in the proof are converted using \emph{weak abstraction:}\\
Roughly, weak abstraction is application of the following transformation:\\
\m{\termRepAt{C}{t}{p} \rightarrow \textcolor{green}{X} \neq t \lor \termRepAt{C}{\textcolor{green}{X}}{p}} \\
where t is a term in the theory vocabulary (roughly equivalent to \langb{} in our case) that does not contain normal free variables and \textcolor{green}{X} is an \emph{abstraction variable}, which is roughly a free variable that can only be instantiated with theory terms - only terms that are in \langb{}.\\
Unification respects abstraction variables in the sense that, for the mgu $\sigma$, \m{\sigma(X) \in \langb{}} and \langb{} includes all abstraction variables.\\
For example \ref{example.3.1.1.3.1}, using superposition with weak abstraction the derivation is roughly equivalent to:

\bigskip

\noindent
\infer[\m{\sigma = \s{\textcolor{green}{Y} \mapsto \textcolor{green}{X}}}]
{\m{\textcolor{green}{X} \neq a \lor \textcolor{green}{X} \neq b \lor c=d}}
{
	\infer[]
		{\m{\textcolor{green}{X} \neq a \lor \underline{f(\textcolor{green}{X},\textcolor{red}{u})}=c}}
		{\m{\m{f(\underline{a},\textcolor{red}{x})=c}}}
	& 
	\infer[]
		{\m{\textcolor{green}{Y} \neq b \lor \underline{f(\textcolor{green}{Y},\textcolor{red}{u})}=d}}
		{\m{f(\underline{b},\textcolor{red}{x})=d}}
}
	
\bigskip

And then, for the theory \m{N^b}:

\bigskip

\noindent
\infer*[]
{\m{\emptyClause}}
{
	\m{\textcolor{green}{X} \neq a \lor \textcolor{green}{X} \neq b \lor c=d}
}
	
\bigskip

The main difference with our approach is that, if \m{N^b \not\models a=b},
we can avoid the above superposition derivation in \m{N^t} (if the approximation of \m{I^b} is precise enough) while their system does not take \m{N^b} into account in unification, and hence we can ensure completeness (for the ground fragment) while allowing strictly less derivations.

\noindent
In their system, using the above example, the following is \emph{not} a valid derivation:

\bigskip

\noindent
\infer[\m{\sigma = \s{\textcolor{green}{X} \mapsto \textcolor{red}{y}}}]
{\m{\textcolor{red}{y} \neq a \lor c=d}}
{
	\infer[]
		{\m{\textcolor{green}{X} \neq a \lor \underline{f(\textcolor{green}{X},\textcolor{red}{z})}=c}}
		{\m{\m{f(\underline{a},\textcolor{red}{z})=c}}}
	& 
	{\m{\underline{f(\textcolor{red}{y},\textcolor{red}{z})}=d}}
}
	
\bigskip

\noindent
Because the unifier maps an abstraction variable \textcolor{green}{X} to a term not in \\\langb.
Hence, their calculus avoids several unnecessary derivations (derivations that would be allowed if abstraction used normal variables as in  the original \cite{DBLP:journals/aaecc/BachmairGW94}) without losing completeness (for the ground fragment).\\
Abstraction is restricted (roughly) to top-background terms, meaning, in our context, terms over \langb{} whose direct super-term is not in \langb{} - for example for:\\
\m{\forall x \cdot g(f(b,c),f(h(a,x),\textcolor{red}{z}))=d}\\
The weak abstraction is:\\
\m{\textcolor{green}{X} \neq a \lor \textcolor{green}{Y} \neq f(b,c) \lor g(\textcolor{green}{Y},f(h(\textcolor{green}{X},x),\textcolor{red}{z})=d}\\
Where b,c, are not abstracted because they are not top \m{N^b} terms (f(b,c) is their direct super-term), 
and x is not abstracted because it is a (non-abstraction) variable.\\
\cite{BaumgartnerWaldmann13} shows a calculus for non-ground superposition with an opaque background theory, 
shows completeness in the ground case, and some conditions for completeness in the non-ground case (i.e. \m{N^t} includes non-ground clauses).\\
The main advantage of our method is that we use an approximation of the theory (in our case \m{N^b}) to restrict abstraction further 
(so that our communication is bi-directional), and we only perform the equivalent of abstraction when necessary for unification, 
so that we remain in the ground fragment for a ground problem - hence the interpolant also remains ground.
The authors discuss an improvement to their calculus that defines a sub-set of the theory terms (\langb) 
as \emph{domain elements} which are all known to be unequal to each other (e.g. numbers). 
They show that it is not necessary to abstract these elements (as, essentially, they are in their normal form already). 
In our calculus, if e.g. numbers are at the bottom of the ordering, we get the same effect.
The non-ground aspects of the comparison with the non-ground version of our calculus will is discussed in section \ref{section:fole:scoping}.


\bigskip

\noindent
In \cite{McMillan08}, the author presents a superposition calculus for deriving an interpolant between two sets of clauses, as in our setting.\\
The author uses a separating ordering similar to ours, and ensures (ground) completeness using the following \emph{procrastination} rule:

\bigskip

\noindent
\m{
	\vcenter{
		\infer[]
			{\m{f(\textcolor{blue}{l},\textcolor{red}{x})=c}}
			{\m{\underline{a}=\textcolor{blue}{l}} & \m{\underline{f(a,\textcolor{red}{x})=c}}}
	}
	\rightarrow
		\vcenter{
			\infer[]
				{\m{\forall y \cdot y \neq a \lor f(y,\textcolor{red}{x})=c \mid a \succ y}}
				{\m{\underline{f(a,\textcolor{red}{x})}=c}}
		}
	}


\bigskip

\noindent
Where y is a fresh normal variable, and the constraint \m{\mid a \succ y} is an ordering constraint (as in \cite{DBLP:books/el/RV01/NieuwenhuisR01} section 5). Roughly, the constraint means that any unifier \m{\sigma} used in an inference with the clause, it must hold that \m{a \succ \sigma{y} }. In this system, ordering constraints are used to ensure that the proof is local. The author mentions (discussion of lemma 3) that ordering constraints could be checked for feasibility and then discarded, rather than propagated (which adds a significant overhead), and the conclusion of non-local derivations discarded.\\
Note that the ordering constraints are, in a sense, a more precise variant on weak abstraction, as they constrain the newly introduced variables to be smaller than a given term, while essentially weak abstraction constrains the abstraction variables to be smaller than \m{l^t}.
We write the above replacement as the following rule in proofs:

\bigskip

\noindent
\[
\begin{array}{llll}
\infer[\m{s \succ l^t,l = \termAt{s}{p}}]
{\m{\forall x \cdot x \neq \termAt{s}{p} \lor C \lor D \lor \termRepAt{s}{x}{p} \bowtie t \mid \termAt{s}{p} \succ x}}
{
	\m{C \lor \underline{l}=r} &
	\m{s \bowtie t \lor D}
} &
\begin{array}{ll}
	\m{\sci{1} l \not\preceq r, \sci{2} l=r \not\preceq C}\\
	\m{\sci{3} s \not\prec t, \sci{4} s\bowtie t \not\prec D}\\
	\m{\sci{5} s \bowtie t \not\preceq l=r}\\
	\m{\sci{6} \mathbf{s \succeq l^t \succ l}}
\end{array}	\\

\end{array}
\]

\bigskip


\noindent
For example \ref{example.3.1.1.3.1}, the proof would be as follows:


\bigskip

\noindent
\infer[\m{\sigma=\s{z \mapsto y}}]
{\m{\forall y \cdot y \neq b \lor y \neq a \lor \underline{d}=c \mid a \succ y \land b \succ y}}
{
	\infer[]
	{\m{\forall y \cdot y \neq a \lor \underline{f(y,\textcolor{red}{x})}=c \mid a \succ y}}
	{
		\m{\underline{a}=\textcolor{blue}{l}}
			&
		\m{\underline{f(a,\textcolor{red}{x})}=c}
	}
		&
	\infer[]
	{\m{\forall z \cdot z \neq b \lor \underline{f(z,\textcolor{red}{x})}=d \mid b \succ z}}
	{
		\m{\underline{b}=\textcolor{blue}{l}}
			&
		\m{\underline{f(b,\textcolor{red}{x})}=d}
	}
}


\bigskip

\noindent
Where y,z are normal variables. We could replace \m{a \succ y \land b \succ y} with \m{a \succ y} as \m{b \succ a}.\\
This is followed, in \m{N^b} (dropping the ordering constraints):\\
\bigskip

\noindent
\infer[]
	{\emptyClause}
	{
		\infer[]
		{\m{\textcolor{blue}{l} \neq \textcolor{blue}{l}}}
		{
			\infer[]
				{\m{\forall y \cdot y \neq \textcolor{blue}{l} \lor y \neq \textcolor{blue}{l}}}
				{
					\m{\underline{a} = \textcolor{blue}{l}}
					& 
					{\infer[]
						{\m{\forall y \cdot \underline{a} \neq y \lor y \neq \textcolor{blue}{l}}}
						{
							\m{\underline{b} = \textcolor{blue}{l}}
							& 
							\infer[]
							{\m{\forall y \cdot \underline{b} \neq y \lor a \neq y}}
							{
								\m{\forall y \cdot y \neq b \lor y \neq a \lor \underline{d}=c}
									&
								\m{\underline{d \neq c}}
							}
						}
					}
				}
		}
	}


\bigskip


\noindent
The derivations above use the fact that a,b are maximal equality terms in order to drive abstraction, as opposed to \cite{BaumgartnerWaldmann13}
which (in a different setting) produces an abstraction for \emph{each} term in \langI.\\
Dropping the ordering constraints at \m{N^t} makes the abstraction not weak, similarly to \cite{DBLP:journals/aaecc/BachmairGW94} (i.e., in the above example, the variables y,z above can be unified with terms in \langtp such as \m{\textcolor{red}{x}} ).


The first difference of our approach is that we always remain in the ground fragment as long as the input is ground (and do not introduce fresh variables otherwise), and so we produce ground interpolants. 
One might consider adding an un-abstraction rule to interpolation clauses (similar to that suggested in \cite{BaumgartnerWaldmann13}) in order to produce ground interpolants:
\bigskip

\noindent
\[
\begin{array}{llll}
\vcenter{
\infer[]
{\m{C[t/x]}}
{
	\m{\forall x \cdot x \neq t \lor C}
}} &
\begin{array}{ll}
	\m{\sci{1} C,t \prec l^t}\\
	\m{\sci{2} x \not\trianglelefteq t}\\
\end{array}\\
\end{array}
\]

\bigskip

The second difference is that their approach abstracts each clause separately, even if it is not a candidate for any inference, while our approach performs essentially a combined step of abstraction (or procrastination), superposition and then unabstraction.
Our $\Gamma$ is, in a sense, a summary of all the possible abstractions of maximal terms.
%
%\noindent
%\textbf{Non convex $\Gamma$:}\\
%In our system, \m{\Gamma, \eqg} are parameters, and we can make them more or less accurate. 
%Importantly, we can make $\Gamma$ non-convex if we track, for each clause, which of the ground dis-equalities are assumptions (B in the conclusion in our rules) and so avoid mixing equality theories from different paths, as happens e.g. in example \ref{example_4.2.1.9_program}.
%In order to use a non-convex theory, we define $\Gamma$ as a set of sets of equations, and use equality constraints to track our assumptions, so that the rule \m{sup^i_=} is as follows:
%
%\[
%\begin{array}{llll}
%
	%\vspace{1cm}
	%
	%\m{sup^i_{=}} &
	%\vcenter{
		%\infer[]
		%{\m{C \lor D \lor \termRepAt{s}{r}{p}=t \mid B,B_1,B_2}}
		%{
			%\m{C \lor \underline{l}=r  \mid B_1} 
				%&
			%\m{\underline{s} = t \lor D \mid B_2}
		%}
	%}&
			%\begin{array}{ll}
				%\m{\sci{1} l \not\preceq_i r, \sci{2} l=r \not\preceq_i C}\\
				%\m{\sci{3} s \not\preceq_i t, \sci{4} s=t \not\preceq_i D}\\
				%\m{\sci{5} s=t \not\preceq_i l=r}\\
				%\m{\sci{6} l,s \succeq l^t}\\
				%\m{\sci{7} \m{\inidasg{B}{l}{\termAt{\mathbf{s}}{\mathbf{p}}}}}\\
				%\m{\sci{8} \mathbf{\exists S \in \Gamma \cdot B \cup B_1 \cup B_2 \subseteq S}}
			%\end{array}\\	
	%
	%\vspace{1cm}
	%
%\end{array}
%\]
%
%\noindent
%Where \m{B,B_1,B_2} are the equality constraints.
%
%\noindent
%So that \sci{8} does not hold in example \ref{example_4.2.1.9_program} 
%for unifying \m{h(c_1,c_2,\textcolor{red}{x})} with \\
%\m{h(d_1,d_2,\textcolor{red}{x})} because \\
%\m{\Gamma = \s{\s{a=b, c_1=d_1},\s{a=b, c_2=d_2}}} rather than\\
%\m{\Gamma = \s{\s{a=b, c_1=d_1,c_2=d_2}}}.
%
%\bigskip
%
%\noindent
%A second, minor difference is that we handle maximal equalities l=r where \m{l^t \succ l \succ r \succ l^b} differently from the case where
%\m{l^t \succ l \succ l^b \succ r} - so that our approximation is somewhat more precise - for example:\\
%If \m{N^b = \s{\textcolor{blue}{M} \lor \underline{c}=b,\textcolor{blue}{M} \lor \underline{d}=a}}, 
%we do not allow e.g. the inference
%
%\bigskip
%
%\noindent
%\infer[]
%{\m{c \neq d \lor e \neq e}}
%{
	%\m{f(c,\textcolor{red}{x})=e} &
	%\m{f(d,\textcolor{red}{x})\neq e}
%}
%
%\bigskip
%
%\noindent
%The procrastination rule will allow it by deriving:
%
%\bigskip
%
%\noindent
%\infer
%{\m{y \neq d \lor y \neq c \lor e \neq e \mid c \succ y,d\succ y}}
%{
	%\infer[]
	%{\m{y \neq c \lor f(y,\textcolor{red}{x})=e \mid c \succ y}}
	%{\m{\textcolor{blue}{M} \lor \underline{c}=b} & \m{f(c,\textcolor{red}{x})=e}}
		%&
	%\infer[]
	%{\m{z \neq d \lor f(z,\textcolor{red}{x})\neq e \mid d \succ z}}
	%{\m{\textcolor{blue}{M} \lor \underline{d}=a} & \m{z \neq d \lor f(z,\textcolor{red}{x})\neq e \mid d \succ z}}
%}
%
%\bigskip
%
%\noindent
%Where the conclusion has a satisfiable ordering constraint.\\
%Weak abstraction would work similarly.
%
%\bigskip
%
%Note that, in our setting, preventing useless derivations can be more important than in the general case, 
%as clause propagation can be a significant overhead - we emphasize more on reducing the proof search space than on the efficiency of each step in the proof search.
%
%\textbf{Variants of definitions for $\Gamma$:}\\
%In our system \m{\Gamma,\vdash_{\Gamma}} are parameters. We have shown one simple instance of these parameters, and it would be interesting to explore other, more precise, variants.
%
%Some useless derivations are not prevented in any of the three systems because of the over-approximation for super-terms - consider the example in figure \ref{example.4.1.3.3.1}:
%\begin{figure}[H]
%\m{N^t = \s{(1) \underline{f(b,\textcolor{red}{v})} = \textcolor{red}{u},(2)\underline{f(g(c),\textcolor{red}{v})} \neq \textcolor{red}{u},
%(3)\underline{f(h(c),\textcolor{red}{v})} \neq \textcolor{red}{u},(4)\underline{f(a,\textcolor{red}{v})} \neq \textcolor{red}{u}}}\\
%\m{N^b = \s{(5)\underline{b} = \textcolor{blue}{m},(6)\underline{c} = \textcolor{blue}{m}, (7)\underline{g(\textcolor{blue}{m})} = \textcolor{blue}{m} }}
%\caption{Example for the imprecision of different techniques}
%\label{example.4.1.3.3.1}
%\end{figure}
%
%\bigskip 
%
%\noindent
%Standard (non-local) superposition will allow the inferences between the pairs 
%(5)(1)=\m{f(\textcolor{blue}{m},\textcolor{red}{v}) = \textcolor{red}{u}} (the result we name (1a)),\\
%(6)(2)=\m{f(g(\textcolor{blue}{m}),\textcolor{red}{v}) \neq \textcolor{red}{u}}(2a),\\
%(6)(3)=\m{f(h(\textcolor{blue}{m}),\textcolor{red}{v}) \neq \textcolor{red}{u}}(3a), and then\\
%(7)(2a)= \m{f(\textcolor{blue}{m},\textcolor{red}{v}) \neq \textcolor{red}{u}}(2b) and finally\\
%(1a)(2b) = \emptyClause.\\
%Limiting these to \m{N^t}, essentially we have allowed a derivation between (1)(2).
%
%\bigskip
%
%\noindent
%Our system will allow the following inferences at \m{N^t}:
%
%\bigskip
%
%\noindent
%\infer[
%]
%{\m{b=g(c) \in \Gamma}}
%{
	%\infer[]
	%{\m{b \in M}}
	%{\m{\underline{b}=\textcolor{blue}{m}}}
		%&
	%\infer[
%\begin{tabular}{ll}
	%\m{c \triangleleft g(c) \trianglelefteq \maxTerms{N^t}}
	%\end{tabular}
	%]
	%{\m{g(c) \in M}}
	%{\m{\underline{c}=\textcolor{blue}{m}}}
%}
%
%\bigskip
%
%\noindent
%And similarly to derive \\
%\m{b\eqg h(c) \eqg g(c)}, but no equality for a,\\
%while only \m{b=g(c)} is actually needed for completeness.\\
%We will, hence, allow superposition derivations between each pair in (1)(2)(3) - 3 in total, although only (1)(2) is needed for a refutation.
%
%\bigskip
%
%\noindent
%For  ~\cite{McMillan08}, we get the following derivations:
%
%\bigskip
%
%\noindent
%\infer[]
%{\m{(2a)~\forall x \cdot c \neq x \lor \underline{f(g(x),\textcolor{red}{v})} = \textcolor{red}{u} \mid c \succ x}}
%{
	%{\m{\underline{c}=\textcolor{blue}{m}}}
		%&
	%\m{\underline{f(g(c),\textcolor{red}{v})} = \textcolor{red}{u}}
%}
%
%\bigskip
%\noindent
%And similarly we can derive \\
%{\m{(1a)~\forall y \cdot b \neq y \lor \underline{f(y,\textcolor{red}{v})} = \textcolor{red}{u} \mid b \succ y}} \\
%{\m{(3a)~\forall z \cdot c \neq z \lor \underline{f(h(z),\textcolor{red}{v})} = \textcolor{red}{u} \mid c \succ z}}
%
%\bigskip
%
%\noindent
%And now, in addition:
%
%\bigskip
%
%\noindent
%\infer[]
%{\m{(2b)~\forall x,w \cdot c \neq x \lor g(x) \neq w \lor \underline{f(w,\textcolor{red}{v})} = \textcolor{red}{u} \mid b \succ x, g(x) \succ w}}
%{
	%\m{\underline{g(\textcolor{blue}{m})} = \textcolor{blue}{m}}
		%&
	%\m{\forall x \cdot c \neq x \lor \underline{f(g(x),\textcolor{red}{v})} = \textcolor{red}{u} \mid b \succ x}
%}
%
%\bigskip
%
%\noindent
%Of the five clauses (1a)(2a)(2b)(3a)(4), both (1a),(2b) have each a valid superposition derivation with each other of the five clauses (7 in total).\\
%Superposition of (1)(1a) and similarly (2)(2a) and (3)(3a) will be performed but the conclusion rejected by the ordering constraints (so minimally unification is needed).\\
%Here, we have avoided derivations with (4) as in our system, but, additionally, we have avoided derivations with (3),(3a), 
%essentially as h(c) was abstracted to h(z) rather than to a variable, and g(c) was treated differently only because of the equation 
%\m{\underline{g(\textcolor{blue}{m})} = \textcolor{blue}{m}}, which the other two approaches are oblivious to.\\
%Note especially, that we needed to find \m{mgu(g(\textcolor{blue}{m}),g(x))} where \m{g(x)} appears only in \m{N^t} 
%but \m{g(\textcolor{blue}{m})} is not in scope in \m{N^t} - in our setting this would require a new mechanism for propagation.\\
%In the non-ground case we may need to find e.g. \m{mgu(g(\textcolor{blue}{m},y),g(x,\textcolor{red}{u}))} which is exactly what we are trying to avoid - 
%a mixed substitution.
%
%\bigskip
%
%\noindent
%Weak abstraction will derive:
%
%\bigskip
%
%\noindent
%\infer[]
%{\m{(2A)g(c) \neq \textcolor{green}{X} \lor \underline{f(\textcolor{green}{X},\textcolor{red}{v})} = \textcolor{red}{u}}}
%{
	%\m{\underline{f(g(c),\textcolor{red}{v})} = \textcolor{red}{u}}
%}
%
%\bigskip
%
%\noindent
%And similarly \\
%{\m{(1A)b \neq \textcolor{green}{Y} \lor \underline{f(\textcolor{green}{Y},\textcolor{red}{v})} = \textcolor{red}{u}}} \\
%{\m{(3A)h(c) \neq \textcolor{green}{Z} \lor \underline{f(\textcolor{green}{Z},\textcolor{red}{v})} = \textcolor{red}{u}}}\\
%{\m{(4A)a \neq \textcolor{green}{W} \lor \underline{f(\textcolor{green}{W},\textcolor{red}{v})} = \textcolor{red}{u}}}
%
%\bigskip
%
%\noindent
%Of the four clauses (1A)(2A)(3A)(4A), weak abstraction will allow superposition between each (abstracted) pair - 6 in total.
%
%\bigskip
%
%\noindent
%To summarize the comparison, we can see that our system remains in the ground fragment while the other two systems do not.
%Weak abstraction prevents unification with terms in \langtp{} by (slightly) changing the unification algorithm,
%while procrastination relies on ordering constraints for that purpose. 
%We use a system of ground equations, which allows easy bi-directional communication between \m{N^t} and \m{N^b}, 
%but our presented version of $\Gamma,\eqg$ uses a rough over-approximation for super-terms, 
%which we hope to improve in future work.\\
%
%One can view IC3/PDR as the most accurate abstraction (at the price of communication overhead) - in our example, 
%IC3 might find the model \\
%\m{b=c=g(b)=g(c)} for \m{N^b} and communicate it to \m{N^t}, 
%which will return (in the best case of generalization) \m{b \neq g(c)}, which is sufficient as an interpolant.\\
%However, as we have mentioned before, this does not extend readily to DAGs where the above model will have to be communicated from e.g. an assertion node 
%to the node which \m{N^t} represents, and if it is communicated through join-branch pairs, we will have to communicate (exponentially) more models (as per the number of paths) unless some abstraction is done on the model. 
%A combination of satisfaction and saturation would be interesting future work.


\subsubsection*{Other related work}

\textbf{Superposition modulo Shostak theories:} Another related work is \\
\cite{SuperpositionModuloShostak}, which shows a superposition calculus that works modulo a Shostak theory. 
A Shostak theory (\cite{Shostak84},\cite{DBLP:conf/frocos/BarrettDS02}) is an alternative approach to satisfiability modulo a theory, where a Shostak theory is composed of two elements: a canonizer which is a function $\sigma$ that rewrites a term to its normal form in the theory, and a solver, which essentially evaluates an equation,
returns whether the equation holds, does not hold, or a set of equalities on variables in the equation if the theory does not imply the equation or its negation. \\
The authors use \emph{blackbox path ordering} which is an instance of lexicographic path ordering (LPO),
that satisfies some additional properties, of  interest to us is that it is \emph{compatible with canonizer application} - that is, roughly, 
\m{t \succeq \sigma(t)} where $\sigma$ is the canonizer function and t an arbitrary term. As our EC-graphs can be thought of as rewriting to normal form, and as we have used some rewriting to normal form for the integer theory, it would be interesting to see how ttkbo interacts with such a canonizer. 

\textbf{Different ordering:} 
We require at most as many limit ordinals as there are CFG-nodes, and hence our ttkbo weights are always within $\omega^{\omega}$.\\
However, as our \m{l^t,l^b} are decreasing as we traverse the CFG in program execution direction, 
program extension (e.g. loop unrolling) might require us to shift all calculated ttkbo weights by some \m{n\omega}, which can be implemented easily.

In \cite{KovacsVoronkov09}, for two sets of clauses \m{N^t,N^b} over scopes \langt,\langb{} respectively ,
the search for an interpolant uses an ordering where \\
\m{\langtp \succsep \langI} and \\
\m{\langbp \succsep \langI} \\
while in our case and in \cite{McMillan08} \\
\m{\langtp \succsep \langI \succsep \langbp}

The main advantage of their work is that they do not need to modify the superposition calculus. 
Our main advantage is that our ordering extends to trees and DAGs - which is required for our verification algorithm.

The first promises completeness for ground local proofs as there cannot be any valid inference with premises in 
both \langtp and \langbp (proven in \cite{KovacsVoronkov09} for ground FOL with equality and linear arithmetic by showing by induction that such an inference is never valid). 
However, it does not extend readily for sequence (or tree or DAG) interpolants as each non-leaf, non-root CFG-node has at least two interfaces, with a predecessor and a successor, and so the ordering constraints cannot be all satisfied. Their solution is to run an interpolation instance per interface.


The second extends readily to sequence and tree, and DAG interpolants, however, SP is not complete as-is for local proofs for cases as we have seen above. For this reason we have added unification relative to an equational theory, ~\cite{McMillan08} adds procrastination and \cite{BaumgartnerWaldmann13} adds weak abstraction.


\textbf{Ordering:}\\
Our ordering is oriented and separating in the style of ~\cite{McMillan08}, meaning that each term that is in scope at a node but not at its successor is \emph{larger} in the ordering. The idea of separation using tkbo was presented in \cite{LudwigWaldmann07} motivated by hierarchic theorem proving and further discussed in \cite{KovacsMoserVoronkov11}.
TKBO is mentioned as a potential separating ordering for interpolation in \cite{KovacsVoronkov09}, while ~\cite{McMillan08} suggests a variant of recursive path ordering (RPO). We have found TKBO more appropriate for sequence, tree and DAG interpolants as it is easy to generate weights that satisfy the ordering constraints, for interface terms we can share the calculation of the weight 
(evaluation of $\prec_i$ is still done on a truncated weight - for each CFG-node there is a designated \m{l^t}), 
and often $\prec/\prec_i$ can be determined using just the weight comparison, without resorting to the recursive inspection of terms.

