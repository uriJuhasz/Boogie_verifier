\newpage
\section{Completeness proof for GFOLE interpolation}\label{appendix:interpolation:completeness}
We show completeness for this calculus using the standard model generation technique, adjusted to our setting.\\
The presentation follows mostly the proof in \cite{BachmairGanzingerSuperposition} with some modifications.


\begin{definition}{Set of clauses smaller than C - \m{\textcolor{blue}{S_C}}}
\label{def_4.3.0.2}

\noindent
For a set of clauses S and a clause C (not necessarily in S), we use:\\
\m{S_C \triangleq \s{D \in S \mid D \prec C}}
\end{definition}

\begin{definition}{Redundant clause \m{\textcolor{blue}{\red{C}{S}}}}
\label{def_4.3.0.3}

\noindent
A clause C is redundant in S (C not necessarily in S) iff:\\
\m{\red{C}{S} \equivdef S_C \models C}\\
That is, the clause implied by a set of smaller clauses in S.
\end{definition}

\begin{definition}{Derivability in one step \m{\textcolor{blue}{S \vdash_I^1 C}}}
\label{def_4.3.0.4}

\noindent
For a calculus I and a set of clauses S we write \m{S \vdash_I^1 C} if there is an instance of a rule in I s.t. all the premises are in S and the conclusion is C.
\end{definition}

\begin{definition}{Saturation up to redundancy}
\label{def_4.3.0.7}

\noindent
We say that a set of clauses S is saturated up to redundancy by a calculus I iff:\\
For every derivation in I with premises in S, where C is the conclusion, either \m{S_C \models C} or \m{C \in S}

\noindent
A set of clauses S is \textcolor{blue}{saturated up to redundancy w.r.t. a calculus I} if \\
\m{\forall C \cdot (S \vdash_I^1 C \Rightarrow (C \in S \lor \red{C}{S}))}

\end{definition}

\noindent
We assume from now on:\\
(A1) N is a coloured set of clauses saturated up to redundancy w.r.t. \m{SP_I}.\\
Hence \m{N = N^b \cup N^t}.


\noindent
We now give two lemmas that allow us to reason about the communication between \m{N^b} and \m{N^t}.\\
Our algorithm is parametric in \eqg and in \textbf{idas}, and the only property of \textbf{idas} used in the completeness proof is the lemma below.\\
Hence, any other definition of \textbf{idas} that satisfies the below lemma will give us a complete algorithm. Our unit algorithm essentially uses \eqg that is equivalent to \s{(s,t) \mid N^b \models s=t}, while the formulation we have given here is much less precise. We discuss in the future work section some other possibilities.

\noindent
\begin{lemma}{Completeness of equality theory}
\label{lemma_4.3.1.1}

\noindent
If \\
(a1) \m{s,t \trianglelefteq maxTerms(N^t)}\\
(a2) \m{s,t \succeq l^t}\\
(a3) \m{N^b \models s=t}

\noindent
Then there is a clause B s.t.\\
(c1) \m{s \eqg t}\\
(c2) \m{B = \idasg{s}{t}}\\
(c3) \m{B \in \langI}\\
(c4) \m{N^b \models \lnot B}

\noindent
The proof is straightforward from the definition of \eqg and \textbf{idasg}.
\end{lemma}
\bigskip

\begin{lemma}{Soundness of the equality theory}
\label{lemma_4.3.1.2}

\noindent
If\\
(a1) \m{B = \idasg{s}{t}}\\
Then\\
(c1) \m{(N^b \models B) \lor (N^b \models s=t)}

\noindent
The proof is straightforward
\end{lemma}


\begin{theorem}{Soundness of \m{SP_I}}
\label{theorem_4.3.1.1}

\noindent
For all S,C,I if \\
\m{S \vdash_{SP_I} C}\\
Then\\
\m{I \models S \Rightarrow I \models C}
 

\noindent
The proof is straightforward using lemma \ref{lemma_4.3.1.2}.
\end{theorem}


\subsubsection*{Model generation}
We will show that either \m{\emptyClause \in N} or N has a model, that we will construct.\\
We construct the model similarly to the standard model, except that we build separately the model for \m{N^b} using the standard method (as the standard calculus applies there) and on top of it the model for \m{N^t}.

\textbf{Models, structures and interpretations}\\
In our context we only discuss models of the following form:\\
Each model is characterized by a congruence relation $\cong$.\\
The universe (domain) of the model is the quotient of the Herbrand universe and the congruence - \m{\Ts{\Sigma} / \cong}, so that each element of the universe is an equivalence class of terms.\\
We use the following interpretation I for all models we consider, and hence discuss only models from here on:\\
Each term t maps to its equivalence class by $\cong$ :\\
\m{\llbracket t \rrbracket_I \triangleq [t]_{\cong}}.\\
The interpretation for functions is standard: \\
\m{\llbracket f \rrbracket_I (\llbracket \tup{s} \rrbracket_I ) \triangleq \llbracket \fa{f}{s} \rrbracket_I}.\\
As $\cong$ is a congruence this is well defined.\\
We only refer to the model from here on and assume the above interpretation.

\textbf{Term rewrite systems(TRS) and normal forms}\\
A \textcolor{blue}{term rewrite system (TRS)} is a set of rules of the form \m{u \rightarrow v} where u,v are ground terms.\\
A derivation in a TRS T is as follows: if \m{u \rightarrow v \in T} then \m{\termRepAt{s}{u}{p} \rightarrow_T \termRepAt{s}{v}{p}} for any term s and p a position in s.\\
\m{\rightarrow_T^{\ast}} is the transitive reflexive closure of \m{\rightarrow_T}.\\
We define derivability in up to n steps recursively as follows:\\
\m{\rightarrow_T^0 \triangleq id}\\
\m{\rightarrow_T^{n+1} \triangleq \rightarrow_T^{n} \circ \rightarrow_T}.\\
A TRS is \textcolor{blue}{terminating} if it only has finite derivation sequences.\\
A term t is \textcolor{blue}{reducible} by a TRS T iff there is some s different from t s.t. \m{t \rightarrow_T s}, and otherwise it is \textcolor{blue}{irreducible} by T.\\
A term t is in \textcolor{blue}{normal form w.r.t. T} if it is irreducible by T - in general a term can have more than one normal form, or none at all.\\
We use \m{\textcolor{blue}{\mathbf{irred_T}}} to denote the set of irreducible terms in T.\\
All the TRSs we discuss are \textcolor{blue}{oriented w.r.t. the ordering} $\succ$, meaning that \\
\m{\forall u \rightarrow v \in T \cdot u \succ v}.\\
A well known result is that a TRS oriented w.r.t. a well-founded ordering is \textcolor{blue}{terminating} 
- because the ordering is strict there are no derivation loops, and because the ordering is well-founded every derivation sequence terminates in a finite number of steps.\\
A TRS T is \textcolor{blue}{confluent} iff \\
\m{\forall s \cdot s \rightarrow_T^{\ast} t \land s \rightarrow_T^{\ast} u \Rightarrow \exists v \cdot t \rightarrow_T^{\ast} v \land u \rightarrow_T^{\ast} v} .\\
A TRS T is \textcolor{blue}{convergent} iff it is terminating and confluent.\\
In a convergent TRS T each term has a unique normal form - we will use the function \m{r^T} that returns the normal form of a term in T when we know T is convergent.\\
A TRS T is \textcolor{blue}{left reduced} if no derivation applies to the left-hand side of any rule - formally \\
\m{\forall s \rightarrow t,u \rightarrow v \in T \cdot s \not\trianglelefteq u}.\\
It is a well known result that a left-reduced TRS is confluent (as it is locally confluent) - 
hence any oriented, left-reduced TRS is convergent.\\
Each convergent TRS T defines a congruence relation \m{\textcolor{blue}{=_T}} as follows:\\
\m{s =_T t \equivdef r^T(s)=r^T(t)} - that is, two terms are congruent iff they have the same normal form.\\
Hence each convergent TRS defines a structure with the standard interpretation.\\
We will use the notation\\
\m{T \models s=t} to denote \m{s =_T t}.\\
A property of convergent TRSs is that, for each convergent T and terms s,u where \m{u = \termAt{s}{p}}:\\
\m{r^T(\termRepAt{s}{r^T(u)}{p})=r^T(\termRepAt{s}{u}{p})}

\textbf{Model for \m{N^b}:}\\
We assume we have a model \m{\textcolor{blue}{I^b}} generated for \m{N^b} in the standard method:\\
The model \m{I^b} is defined by a congruence relation on terms over \langb, we define it using the term rewrite system \m{R^b} that is constructed in the standard completeness proof - from which we get the following properties:\\
(B1) \m{R^b} is a TRS over \langb that is left reduced and oriented w.r.t. $\succ$ and hence convergent - formally:\\
(B2) Each rule \m{u \rightarrow v \in R^b} has the following properties:\\
(B2.1) There is some \m{C = C_0 \lor u=v \in N^b} s.t.\\
(B2.2) \m{u \succ v}\\
(B2.3) \m{u=v \succ C_0}\\
(B2.4) \m{u \prec l^t}\\
This motivates our construction of \eqg.\\
The congruence relation is \m{=_{R^b}} as defined above and the domain of the model is \m{\Ts{\Sigma} / =_{R^b}}. 

\noindent
As \m{R^b} is convergent each term has a unique normal form. We define a function \m{\textcolor{blue}{r^b}} that returns the normal form of the input - formally:\\
\m{r^b(t) \triangleq min_{\prec} \s{s \mid t \rightarrow_{R^b}^\ast s}}\\
We extend this function to atoms, literals, clauses and clause-sets and denote \m{r^b(x)} by \m{x'} for each of these objects. 
%For each term t, atom a, literal A, clause C and clause-set S in \m{N^t} we define 
%\m{(t',a',A',C',S')} as \m{r^b(t,a,A,C,S)}.

\bigskip 

\noindent
By the properties of the TRS \m{R^b}:\\
\m{(I^b \models s=t) \Leftrightarrow s'=t'}\\
And by standard completeness proof of the standard calculus:\\
\m{I^b \models N^b}

\bigskip 

\noindent
\textbf{Top interpretation}\\
We now want to build a similar \m{I^t,R^t,r^t} for \m{N^t}, which require some adjustments from the standard definitions.\\
The main difference is that, for each clause \m{C \in N^t}, we use \m{C'} rather than C in the construction.\\
We define first the partial models that we will show model all clauses smaller than a given clause - the definition is inductive and relies on the well-foundedness of the ordering \m{\prec} on normal forms (by \m{r^b}) of clauses.

\begin{definition}{Partial top models \m{\textcolor{blue}{N^t_C,R^t_C,r^t_C,E^t_C,I^t_C}}}
\label{def_4.3.1.2}

\noindent
For any clause C (not necessarily in \m{N^t}):\\ 
\m{N^t_C \triangleq \s{D \in N^t \mid D' \prec C'}} - (all clauses whose normal form is smaller than that of C in \m{N^t}).\\
\m{R^t_C} is a TRS defined by \m{N^t_C} defined below.\\
As \m{R^t_C} is by construction oriented w.r.t. $\prec$ and left reduced, we define \m{r^t_C} as its normal form function \m{r^{R^t_C}}.\\
The partial model \m{I^t_C} models all clauses smaller than C (as we show in the completeness proof).\\
The above symbols are defined inductively, based on the ordering on normal forms of clauses in \m{N^t}:\\
Each clause in \m{N^t} may contribute up to one rule to the TRS that defines the final model - we use the set of rules \m{E^t_C} to denote the (potential) rule contributed by the clause C.\\
\m{E^t_C \triangleq}\\
\indent If \\
	\indent \indent (1) \m{I_C \not\models C}\\
	\indent \indent (2) \m{C = C_0 \lor l=r} for some \m{C_0,l=r} s.t.\\
	\indent \indent (3) \m{r^t_C(l') = l'} (\m{l'} is irreducible by \m{R^t_C})\\
	\indent \indent (4) \m{l' \succ r'}\\
	\indent \indent (5) \m{l' = r' \succ C_0'}\\
	\indent \indent (6) \m{\forall s=t \in C_0 \cdot (l'=s' \Rightarrow I_C \models r \neq t)} \\
\indent	then \s{l' = r'}\\
\indent else \m{\emptyset}

\noindent
The TRS \m{R^t_C} is the partial model form all clauses whose normal form (in \m{R^b}) is smaller than that of C:\\
\m{R^t_C \triangleq \cup \s{E^t_D \mid D \in N^t_C}}\\
It is easy to see that \m{R^t_C} is left reduced (3) and oriented (4), and hence we can define its normal form function \m{r^t_C}:\\
\m{r^t_C(t') \triangleq min \s{s' \mid t' \rightarrow_{R^t_C}^{\ast} s'}}\\
We only apply \m{r^t_C} to terms reduced by \m{r^b}.

\noindent
A slightly delicate point is that, while $\succ$ is total, we order clauses by their normal form, so we can have 
\m{C \neq D} while \m{C' = D'}.\\
This is not a problem in our case as \m{C'=D' \Rightarrow E^t_C = E^t_D} in this case because the definition of \m{E^t_C} depends only on the normal form \m{C'}, and \m{R^t_C} is a set.

\noindent


\end{definition}

\begin{definition}{Global TRSs and global partial TRSs - \m{\textcolor{blue}{R^t,R,r^t,r,R_C,r_C}}}
\label{def_4.3.1.3}

\noindent
We define now the TRS that, combined with \m{R^b}, models \m{N}:\\
\m{R^t \triangleq \cup \s{E^t_C \mid C \in N^t}}\\
\m{R \triangleq R^t \cup R^b}\\
\m{R,R^t} are by definition left reduced.\\
By convergence we can define the normal form functions:\\
\m{r^t(t') \triangleq min \s{s' \mid t' \rightarrow_{R^t}^{\ast} s'}}\\
\m{r(t) \triangleq  min \s{s \mid t \rightarrow_{R}^{\ast} s}}\\
We only apply \m{r^t} to terms reduced by \m{r^b}. \\
By construction, \m{R^t_C} is also left reduced by \m{R^b} and hence we can define:\\
\m{R_C \triangleq R^b \cup R^t_C}\\
\m{r_C(t) \triangleq min \s{s \mid t \rightarrow_{R_C}^{\ast} s}}
\end{definition}

\begin{lemma}{Composition of normal form functions}

\noindent
It is easy to see that:\\
\m{\forall t \cdot r_C(t) = r^t_C(r^b(t))}\\
\m{\forall t \cdot r(t) = r^t(r^b(t))}
\end{lemma}

\begin{definition}{\m{\textcolor{blue}{R^t_{C+},R_{C+},r^t_{C+}}}}
\label{def_4.3.1.4}

\noindent
We will use:\\
\m{R^t_{C+} \triangleq R^t_C \cup E^t_C} and similarly\\
\m{R_{C+} \triangleq R_C \cup E^t_C}, and the corresponding function \\
\m{r^t_{C+}(t') \triangleq min \s{s' \mid t' \rightarrow_{R^t_{C+}}^\ast s'}}
\end{definition}

\begin{definition}{Models - \m{\textcolor{blue}{I^t_{C},I_{C},I^t,I^t_{C+},I_{C+},I}}}
\label{def_4.3.1.5}

\noindent
We now define the models as above:\\
\m{I^t_C \triangleq \s{s=t \mid r^t_C(s)=r^t_C(t)}}\\
\m{I^t_{C+} \triangleq \s{s=t \mid r^t_{C+}(s)=r^t_{C+}(t)}}\\
\m{I_C \triangleq \s{s=t \mid r_C(s)=r_C(t)}}\\
\m{I_{C+} \triangleq \s{s=t \mid r_{C+}(s)=r_{C+}(t)}}\\
\m{I^t \triangleq \s{s=t \mid r^t(s)=r^t(t)}}\\
\m{I \triangleq \s{s=t \mid r(s)=r(t)}}\\
I will be our final model (with the standard interpretation) for N.
\end{definition}


\begin{definition}{\m{\textcolor{blue}{max'_t,max'_l}}}
\label{def_4.3.1.6}

\noindent
We define \m{max'_t(C) = max_{\succ}\s{t' \mid t \triangleleft C}} the maximal normal form of a term in C, 
and similarly the maximal normal form of a term for a literal.\\
We also define the maximal normal form of a literal in a term as \\
\m{max'_l(C) = max_{\succ}\s{l' \mid l \in C}}.
\end{definition}

\begin{lemma}{Ordering and maximal terms and literals}
\label{lemma_4.3.1.5}

\noindent
By the properties of \m{\succ}:\\
Clause ordering is determined first by the maximal term:\\
\m{\forall C,D \in N \cdot C' \succ D' \Rightarrow max'_t(C) \succeq max'_t(D)}\\
\m{\forall C,D \in N \cdot (max'_t(C) \succ max'_t(D) \Rightarrow C' \succ D')}\\
And\\
\m{\forall C,D \in N \cdot C' \succ D' \Rightarrow max'_l(C) \succeq max'_l(D)}\\
\m{\forall C,D \in N \cdot (max'_l(C) \succ max'_l(D) \Rightarrow C' \succ D')}
\end{lemma}


\textbf{Persistence lemmas:}\\
We now give a few lemmas that will help us show that I is indeed a model for N.
The lemmas are mostly the standard lemmas, except that we use the normal forms (by \m{r^b}) of clauses rather than the original C.\\
We look at the sequence of partial models \m{I^t_C} arranged by increasing C, 
and show that for most literals \m{l \in C}, the truth of l persists in the sequence from \m{I^t_C} on, and hence also in \m{I^t}.\\
We also show some weaker properties for maximal literals and non-maximal literals that contain the maximal term.


\noindent
\begin{lemma}{Occurrences of a maximal term in a clause}
\label{lemma_4.3.1.6}

\noindent
If the maximal term appears in a positive literal, then it does not appear in a negative literal and
the maximal literal is positive:\\
For all C,s,t,u,v, if:\\
(a1) \m{C \in N}\\
(a2) \m{s = t \in C}\\
(a3) \m{s' = max'_t(C)}\\
(a4) \m{u \neq v \in C}\\
Then\\
(c1) \m{s' \succ u',v'}

\noindent
\textbf{Proof:}\\
By the definition of ordering for literals.
\end{lemma}

\begin{lemma}{Persistence of the normal form for non-maximal terms}
\label{lemma_4.3.1.7}

\noindent
If\\
(a1) \m{C,D \in N}\\
(a2) \m{s \triangleleft C}\\
(a3) \m{D' \succeq C'}\\
(a4) \m{max'_t(C) \succ s'}\\
Then\\
(c1) \m{r_C^t(s') = r_D^t(s') = r^t(s')}

\noindent
\textbf{Proof:}\\
The proof is immediate from the definition of \m{E^t_C} and \m{r^t_C}:\\
if \m{u=v \in E^t_D} for \m{D' \succ C'} then \m{u' \succeq max'_t(C) \succ s'}.
\end{lemma}

\noindent
Using the above lemmas, and the fact that \m{E^t_C} can only include a rewrite rule \m{l' \rightarrow r'} 
where \m{l' = max'_t(C)}, we see that, for any dis-equality appearing in a clause, 
the normal form for the dis-equality is the same for any \m{R_D} where \m{D' \succeq C'}:
\begin{lemma}{Persistence of dis-equalities}
\label{lemma_4.3.1.8}

\noindent
For any C,D,s,t, if :\\
(a1) \m{D \in N^t}\\
(a2) \m{s \neq t \in C }\\
(a3) \m{D' \succ C'}\\
Then\\
(c1) \m{I_C \models s \neq t \Leftrightarrow I_D \models s \neq t \Leftrightarrow I \models s' \neq t'}

\noindent
The proof is straightforward from lemmas \ref{lemma_4.3.1.6} and \ref{lemma_4.3.1.7} :\\
\m{\forall C,D \in N, s \neq t \in C \cdot D' \succeq C' \Rightarrow r^t_C(s',t') = r^t_D(s',t') =  r^t(s',t')}
\end{lemma}

\begin{lemma}{Persistence of equalities not containing the maximal term}
\label{lemma_4.3.1.9}

\noindent
Following the same reasoning as for disequalities, positive equalities persist if they do not contain the maximal term,
as otherwise \m{E^t_D} for some \m{D' \succeq C'} could rewrite the maximal term:

\bigskip

\noindent
For any C,D,s,t, if\\
(a1) \m{D \in N^t}\\
(a2) \m{s = t \in C}\\
(a3) \m{D' \succeq C'}\\
(a4) \m{s',t' \prec max'_t(C)}\\
Then\\
(c1) \m{I_C \models s = t \Leftrightarrow I_D \models s = t \Leftrightarrow I \models s = t}

\noindent
\textbf{Proof:}\\
%We use the fact that \m{R^b \subseteq R_C \subseteq R_D}.\\
%If \m{D \in N^b} then \m{I^b \models D} and hence \m{I_D \models D} and \m{I \models D}.\\
%If \m{D \in N^t} then, 
For each \m{B \in N^t} s.t. \m{D' \succeq B' \succ C'}, \m{max'_t(B) \succeq max'_t(C) \succ s',t'} and hence \\
\m{r^t_C(s',t') = r^t_D(s',t') =  r^t(s',t')} by lemma \ref{lemma_4.3.1.7}.
\end{lemma}

\noindent
Positive equalities always persist in the forward direction, as \\
\m{D' \succeq C' \Rightarrow R^D \supseteq R^C}:
\begin{lemma}
\label{lemma_4.3.1.10}
Persistence of true positive equalities\\
If a positive equality holds at an interpretation, it holds in all subsequent interpretations:

\noindent
For all C,D,s,t, if:\\
(a1) \m{D \in N^t}\\
(a2) \m{s = t \in C}\\
(a3) \m{D' \succeq C'}\\
(a4) \m{I^t_C \models s'=t'}\\
Then\\
(c1) \m{I^t_D \models s'=t'}

\noindent
\textbf{Proof:}\\
This holds as \m{R^t_D \supseteq R^t_C} and both are convergent.
\end{lemma}

\noindent
\begin{lemma}{Backward persistence by maximal term}
\label{lemma_4.3.1.11}

\noindent
For all C,D,s, if:\\
(a1) \m{C,D \in N^t}\\
(a2) \m{s' = max'_t(C)=max'_t(D)}\\
(a3) \m{r^t_D(s')=s'}\\
(a4) \m{C' \preceq D'}
Then\\
(c1) \m{R^t_D = R^t_C}

\bigskip

\noindent
Proof:\\
(s1) For any B s.t. \m{(C' \preceq B' \prec D')}:\\
(s2) \m{s' = max'_t(B)} by the definition of $\prec$.\\
(s3) \m{E^t_B \in \s{\emptyset,\s{s' \rightarrow t'}}} for some \m{t} s.t. \m{s' \succ t'} (s2) definition of \m{E^t_B}.\\
(s4) As \m{r^t_D(s')=s'} and \m{E^t_B \in R^t_D} we know that \m{E^t_B = \emptyset} for any such B.\\
(s5) \m{R^t_C = \cup \s{E^t_A \mid A \in N^t \land C \succ A}}\\
(s6) \m{R^t_D = \cup \s{E^t_A \mid A \in N^t \land D \succ A}}\\
(s7) \m{R^t_D = R^t_C \cup \s{E^t_A \mid A \in N^t \land C \preceq A \prec D} = R^t_C \cup \emptyset = R^t_C} (s5)(s6)

\end{lemma}

\noindent
We now want to characterize persistence for non-maximal equalities that include the maximal term.\\
These equalities are not, in general, persistent, however we have the following weaker property:

\begin{lemma}{Forward persistence of the non-maximal part of a productive clause}
\label{lemma_4.3.1.12}

\noindent
If a clause produces a rewrite rule, then all non maximal literals in the clause persist.

\noindent
For all \m{C,D,C_0,l,r,} if:\\
(a1) \m{C = C_0 \lor l=r \in N^t}\\
(a2) \m{E^t_C = \s{l' \rightarrow r'}}\\
\indent (a2a) \m{l' \succ r'}\\
\indent (a2b) \m{l'=r' \succ C_0'}\\
\indent (a2c) \m{I_C \not\models C}\\
\indent (a2d) \m{r^t_C(l')=l'}\\
\indent (a2e) \m{\forall s=t \in C_0 \cdot (l'=s' \Rightarrow I_C \models r \neq t)}\\
(a3) \m{D \succ C}\\
Then\\
(c1) \m{I_D \not\models C_0}\\
(c2) \m{I \not\models C_0}

\bigskip

\noindent
\textbf{Proof}:\\
%Given:\\
%(a1) \m{C = C_0 \lor l=r \in N^t}\\
%(a2) \m{E^t_C = \s{l' \rightarrow r'}}\\
%(a3) \m{D \succ C}\\
%\indent (a3a) \m{D' \succeq C'}\\
%\indent (a3b) \m{D' \succ C' \Rightarrow D \succ C}\\
%Proof steps:\\
(s1) For any \m{s \neq  t \in C_0} this is immediate from lemma \ref{lemma_4.3.1.8}\\
(s2) Select an arbitrary \m{\textcolor{blue}{s=t} \in C_0} \\
(s3) w.l.o.g \m{s' \succeq t'}\\
(s4) \m{s' \succ t'} by (a1) (a2c) and \m{I_C \supseteq I^b}\\
(s5) \m{I_C \not\models s=t} by (s2) (a2c)\\
(s6) We define \m{C_0 = \textcolor{blue}{C_1} \lor s=t}\\
(s7) \m{l' = s'} by (lemma \ref{lemma_4.3.1.9}) (a2a) (a2b) (s3)\\
(s8) \m{r' \succ t'} by (a1)(a2b) (s6)\\
(s9) \m{I_D \supseteq I_{C+}} by (a2)(a3) and the definition of \m{E^t_C}\\
(s10) \m{r^t_D(s') = r^t_D(l') = r^t_D(r') = r^t_C(r')} by (a2a)(s7)(s9) as \m{r^t_D} is convergent\\
(s11) \m{r^t_D(t') = r^t_C(t')} by (a3)(s4)(s5) (lemma \ref{lemma_4.3.1.7})\\
(s12) \m{r^t_D(s') = r^t_D(t')} by (s2)\\
(s13) \m{r^t_C(t') = r^t_C(r')} by (s10) (s11) (s12)\\
(s14) \m{r^t_C(t') \neq r^t_C(r')} by (a2e) (s2) (s3)\\
(s15) (s13) contradicts (s14)

$\blacksquare$

\end{lemma}

\begin{lemma}{Persistence of clauses in generated model}
\label{lemma_4.3.1.14}
	
\noindent
For all C,D, if\\
(a1) \m{C \in N^t}\\
(a2) \m{D \in N^t}\\
(a3) \m{D' \succ C'}\\
(a4) \m{I_{C+} \models C}\\
Then\\
(c1) \m{I_D \models C}\\
(c2) \m{I \models C}
		
		\bigskip
		\noindent
		\textbf{Proof:}\\
		(s1) \m{I_D \supseteq I_{C+}} (a3)\\
		(s2) We select a literal \m{\textcolor{blue}{s \bowtie t} \in C} s.t. \m{I_{C+} \models s \bowtie t} (a4) \\
		(s3) w.l.o.g. \m{s' \succeq t'}\\
		(4) If \m{s \bowtie t \equiv s \neq t}\\
		\indent (4.1) \m{I_D \models s \neq t},\m{I \models s \neq t} (lemma \ref{lemma_4.3.1.8})\\
		(5) If \m{s \bowtie t \equiv s = t}\\
		\indent (5.1) If (\m{s' \prec \max'_t(C)}) \\
		\indent \indent (5.1.1) \m{I_D \models s = t},\m{I \models s = t} (lemma \ref{lemma_4.3.1.9})\\
		\indent (5.2) If \m{s' = t'}\\
		\indent \indent (5.2.1) \m{I_D \models s = t},\m{I \models s = t} (lemma \ref{lemma_4.3.1.9}) by \m{I^b \models s=t,I_D \supseteq I^b}\\
		\indent (5.3) Else \m{s' \succ t'}\\
		\indent (5.4) \m{r^t_{C+}(s') = r^t_{C+}(t')} (s2)\\
		\indent (5.5) Hence \m{s' \succ r^t_{C+}(s')}\\
		\indent (5.6) There is a rule \m{\textcolor{blue}{u' \rightarrow v'} \in R^t_{C+}}, for some v, s.t.\\
		\indent \indent (5.6.1) \m{u' = \termAt{s}{p}} for some position p\\
		\indent (5.7) Each rule \m{l' \rightarrow r' \in R^t_D \setminus R^t_{C+}} satisfies \m{l' \not\trianglelefteq s',t'}\\
		\indent (5.8) Hence \m{r^t_D{s'} = r^t_{C+}(s') = r^t_{C+}(t') = r^t_D(t')}\\
		(6) \m{I_D \models C}\\
		(7) \m{I \models C} as \m{R_D \subseteq R} and they are both convergent\\
		\QED
\end{lemma}

\noindent
\subsubsection*{Completeness Proof}
We are now ready to give the completeness proof:

\begin{theorem}{Completeness of the ground interpolation calculus}


\noindent
If\\
(A1) N is a colored set of clauses.\\
(A2) N is saturated up to redundancy w.r.t. the calculus \m{SP_I}.\\
(A3) N does not include the empty clause.\\
Then\\
(C1) I, as defined above, is a model for N.
	

\begin{proof}
	We use induction on the ordering of bottom-normalized terms to show that each clause holds in I.\\ 
	We prove:\\
	\m{\forall C \in N^t \cdot I_{C+} \models C}\\
	And the induction hypothesis is:\\
	\m{\forall D \in N^t \cdot D' \prec C' \Rightarrow I_{D+} \models D}
	
\bigskip
	
	\noindent
	Given N, C s.t.\\
	(a1) \m{\emptyClause \notin N}\\
	(a2) N is coloured\\
	(a3) N is saturated up to redundancy\\
	(a4) \m{C \in N^t} (\m{\Rightarrow C \succ l^t})\\
	(a5) i.h. \m{\forall D \in N^t \cdot D' \prec C' \Rightarrow I_{D+} \models D}
	
	\noindent
	Our objective is to show:\\
	\m{I_{C+} \models C}
	
	\bigskip
	
	\noindent
	\textbf{Proof:}\\
	(1.1) \m{I_{C+} = \bigcup\limits_{D \preceq C} E_D} by (definition \ref{def_4.3.1.5})
%	(1.2) \m{R^t_{C+}} is left reduced by lemma \ref{} (1)

	\bigskip

	\noindent
	(2) If \m{I_C \models C}\\
	\indent (2.a.1) \m{I_{C+} = I_C} (definition \ref{def_4.3.1.5})\\
	\indent (2.a.2) \m{I_{C+} \models C}\\
	\indent (2.a.3) done\\
	(2e) We assume from here on that \m{I_C \not\models C}
	
	\bigskip

	\noindent
	\textbf{Maximal dis-equality}\\
	(3) If there is a maximal (normal form) literal of C that is a disequality \m{\textcolor{blue}{s \neq t} \in C}\\
	\indent (3.1) We define \m{C = \textcolor{blue}{C_0} \lor s \neq t}\\
	\indent (3.2) \m{s'\neq t' \succeq C_0} and w.l.o.g. \m{s' \succeq t'}\\
	\indent (3.3) If \m{s' = t'}\\
	\indent \indent (3.3.1) There is a clause \m{\textcolor{blue}{B}} s.t. (lemma \ref{lemma_4.3.1.1})(3.3)\\
	\indent \indent \indent (3.3.1.1) \m{B = \idasg{s}{t}}\\
	\indent \indent \indent (3.3.1.2) \m{I^b \not\models B}\\
	\indent \indent \indent (3.3.1.3) \m{B \prec l^t}\\
	\indent \indent (3.3.2) The following derivation is valid in N, we name the conclusion \textcolor{blue}{R}:

	\bigskip

	\noindent
	\indent \indent \indent $\vcenter{
			\infer[
%				\m{\vdash_p s=t \mid E}
			]
			{\m{B \lor C_0}}
			{
				\m{C_0 \lor \underline{s \neq t}}
			}
		}$
		
		\bigskip
		
		\noindent
		\indent\indent\indent
		\begin{tabular}{llll}
		(i)   & \m{[s \neq t \not\prec_i C_0]}  & (3.2)(lemma \ref{lemma_succ_i})\\
		(ii)  & \m{[s \succ l^t]}               & (a4)(3.1)(3.2)\\
		(iii) & \m{[B = \idasg{s}{t}]}          &  (3.3.1.1)
		\end{tabular}
		\bigskip

	\noindent
	\indent \indent (3.3.3) \m{R' \prec C'} (a4) (3.2) (3.3.1.3)\\
	\indent \indent (3.3.4) \m{I_C \not\models R} (2e) (3.1) (3.3.1.2)\\
	\indent \indent (3.3.5) \m{I_{R+} \not\models R} (3.3.3) (3.3.4) (lemma \ref{lemma_4.3.1.14})\\
	\indent \indent (3.3.6) Contradiction to (i.h.)(a3)(3.3.2)\\
	\indent (3.3e) \m{s' \neq t'} (3.3-3.3.6)\\
	\indent (3.4) \m{s' \succ t'} (3.2) (3.3)\\
	\indent (3.5) \m{r^t_C(s') = r^t_C(t')} (2e) (3.1)\\
	\indent (3.6) There is a rule \m{\textcolor{blue}{l' \rightarrow r'} \in R^t_C} s.t. (3.4)(3.5)(def \ref{def_4.3.1.2})\\
	\indent \indent (3.6.1) \m{l' = \termAt{s'}{p}} for some p\\
	\indent \indent (3.6.2) There is some \m{\textcolor{blue}{B}} s.t. (lemma \ref{lemma_4.3.1.1})(3.6.1)\\
	\indent \indent \indent (3.6.2.1) \m{B = \idasg{l}{\termAt{s}{p}}}\\
	\indent \indent \indent (3.6.2.2) \m{I^b \not\models B}\\
	\indent \indent \indent (3.6.2.3) \m{B \prec l^t}\\
	\indent \indent (3.6.3) \m{l' \succ r'} \\
	\indent \indent (3.6.4) There is a clause \m{\textcolor{blue}{D} = \textcolor{blue}{D_0} \lor l=r \in N^t} s.t. (3.6) (def \ref{def_4.3.1.2})\\
	\indent \indent \indent (3.6.4.1) \m{D' \prec C'}\\
	\indent \indent \indent (3.6.4.2) \m{l'=r' \succ D_0'}\\
	\indent \indent \indent (3.6.4.3) \m{I_D \not\models D}\\
	\indent \indent \indent (3.6.4.4) \m{E^t_D = \s{l' \rightarrow r'}}\\
	\indent (3.7) \m{I_C \not\models D_0} (lemma \ref{lemma_4.3.1.12})\\
	\indent (3.8) \m{I_C \models l=r} (3.6)
	
	\noindent
	\indent (3.9)  \hangindent=1.5cm The following instance of \m{sup^i_{\neq}} is valid in N,\\
	we name the conclusion \textcolor{blue}{R}:
	
	\bigskip
	
	\hangindent=0cm

	\indent \indent \indent $\vcenter{
		\infer[
%			\m{\vdash_p l = \termAt{s}{p} \mid E}
			]
		{\m{B \lor C_0 \lor D_0 \lor \termRepAt{s}{r}{p}\neq t}}
		{
			\m{D_0 \lor \underline{l}=r} 
				&
			\m{\underline{s} \neq t \lor C_0 }
		}
	}$
	\bigskip
	
		\noindent
		\indent
		\begin{tabular}{llll}
		(i)   & \m{[l \not\preceq_i r]}                & (3.6.3)(lemma \ref{lemma_succ_i})\\
		(ii)  & \m{[l=r \not\preceq_i D_0]}            & (3.6.4.2)(lemma \ref{lemma_succ_i})\\
		(iii) & \m{[s \not\preceq_i t]}                & (3.4)(lemma \ref{lemma_succ_i})\\
		(iv)  & \m{[s\neq t \not\preceq_i C_0]}        & (3.2)(lemma \ref{lemma_succ_i})\\
		(v)   & \m{[s\neq t \not\preceq_i l=r]}        & (3.6.1)(lemma \ref{lemma_succ_i})\\
		(vi)  & \m{[l,s \succeq l^t]}                  & (a4)(3.1)(3.2)(3.6.4)\\
		(vii) & \m{[B = \idasg{l}{\termAt{s}{p}}]}     & (3.6.2.1)
		\end{tabular}
		
		\bigskip

	\indent (3.10) \m{R' \prec C'} \\
	\indent \indent \begin{tabular}{llll}
	\m{[B']}                        & (3.6.2.3)\\
	\m{[C_0']}                      & (3.1)(3.2)\\ 
	\m{[D_0']}                      & (3.6.4)+(3.6.4.1)+(3.6.4.2) \\
	\m{[\termRepAt{s}{r}{p}\neq t]} &(3.1)(3.2)(3.6.3)
	\end{tabular}\\
	\indent (3.11) \m{I_C \not\models R} \\
	\indent \indent \begin{tabular}{llll}
	\m{[B]}                         & (3.6.2.2)\\
	\m{[C_0]}                       & (2e)\\
	\m{[D_0]}                       & (3.7)\\
	\m{[\termRepAt{s}{r}{p}\neq t]} & (2e)(3.8)(3.6.1)\\
	\end{tabular}\\
	\indent (3.12) \m{I_{R+} \not\models R} (3.11)(3.6.4.1) (lemma \ref{lemma_4.3.1.14})\\
	\indent (3.13) Contradiction to (i.h.) (a3)(3.9)\\
	\indent (3e) There is no maximal negative literal in C (3-3.13)\\
	\indent \indent (3e.1) All maximal literals in \m{C} are positive (3e)
	
	\bigskip
	
	\noindent
	\textbf{Maximal equality literal:}\\
	(4) We select a maximal literal \m{\textcolor{blue}{s=t} \in C}\\
	\indent (4.1) \m{C = C_0 \lor s=t}\\
	\indent (4.2) \m{s'=t' \succeq C_0'}\\
	\indent (4.3) w.l.o.g. \m{s' \succeq t'}\\
	\indent \indent (4.3.1) \m{s \succ l^t} (4.3)(4.2)(4.1)(a4)\\
	\indent (4.4) If \m{s' = t'}\\
	\indent \indent (4.4.1) \m{I^b \models s=t}\\
	\indent \indent (4.4.2) \m{I_C \models s=t}\\
	\indent \indent (4.4.3) Contradicts (2e)\\
	\indent (4.4e) \m{s' \succ t'}

	\bigskip

	
	\noindent
	\textbf{Maximal term is reducible in \m{R_C}:}\\
	(5) If \m{s' \succ r^t_C(s')}\\
	\indent (5.1) There is a rule \m{l' \rightarrow r' \in R^t_C} s.t.\\
	\indent \indent (5.1.1) \m{l' = \termAt{s'}{p}} for some p\\
	\indent \indent (5.1.2) There is some \m{\textcolor{blue}{B}} s.t. (lemma \ref{lemma_4.3.1.1})(5.1.1)\\
	\indent \indent \indent (5.1.2.1) \m{B = \idasg{l}{\termAt{s}{p}}}\\
	\indent \indent \indent (5.1.2.2) \m{I^b \not\models B}\\
	\indent \indent \indent (5.1.2.3) \m{B \prec l^t}\\
	\indent \indent (5.1.3) \m{l' \succ r'} \\
	\indent \indent (5.1.4) There is a clause \m{D = D_0 \lor l=r \in N^t} s.t. (definition \ref{def_4.3.1.2})\\
	\indent \indent \indent (5.1.4.1) \m{D' \prec C'}\\
	\indent \indent \indent (5.1.4.2) \m{l'=r' \succ D_0}\\
	\indent \indent \indent (5.1.4.3) \m{I_D \not\models D}\\
	\indent \indent \indent (5.1.4.4) \m{E^t_D = \s{l'\rightarrow r'}}\\
	\indent \indent \indent (5.1.4.5) \m{l \succ l^t} (5.1.4)(5.1.4.2)(5.1.3)\\
	\indent (5.2) \m{I_C \not\models D_0} (5.1.4.4)(lemma \ref{lemma_4.3.1.12})\\
	\indent (5.3) \m{I_C \models l=r} (5.1.4)

	\noindent
	\indent (5.4) \hangindent=1.5cm The following instance of \m{sup^i_{=}} is valid in N,\\
							we name the conclusion \textcolor{blue}{R}:

	\hangindent=0cm 
	
	\bigskip

	\noindent
	\indent \indent \indent $\vcenter{
		\infer[
		%\m{\vdash_p l = \termAt{s}{p} \mid E}
		]
		{\m{B \lor C_0 \lor D_0 \lor \termRepAt{s}{r}{p} = t}}
		{
			\m{D_0 \lor \underline{l}=r} 
				&
			\m{\underline{s} = t \lor C_0 }
		}
	}$

	\bigskip

		\noindent
		\indent
		\begin{tabular}{lll}
		(i)   & \m{[l \not\preceq_i r]}                & (5.1.3)(lemma \ref{lemma_succ_i})\\
		(ii)  & \m{[l=r \not\preceq_i D_0]}            & (5.1.4.2)(lemma \ref{lemma_succ_i})\\
		(iii) & \m{[s \not\preceq_i t]}                & (4.3)(lemma \ref{lemma_succ_i})\\
		(iv)  & \m{[s=t \not\preceq_i C_0]}            & (4.2)(lemma \ref{lemma_succ_i})\\
		(v)   & \m{[s=t \not\preceq_i l=r]}            & (4.2)(5.1.4)(5.1.4.1)(5.1.4.2)(lemma \ref{lemma_succ_i})\\
		(vi)  & \m{[l,s \succeq l^t]}                  & (4.3.1)(5.1.4.5)\\
		(vii) & \m{[B = \idasg{l}{\termAt{s}{p}}]}     & (5.1.2.1)
		\end{tabular}
		
		\bigskip

	\indent (5.5) \m{R' \prec C'}       \\
	\indent \indent \begin{tabular}{llll}
	\m{[B']}         & (5.1.2.3)\\
	\m{[C_0']}       & (4.2)\\
	\m{[D_0']}       & (5.1.4)+(5.1.4.1)+(5.1.4.2)\\
	\m{[\termRepAt{s}{r}{p} = t]} & (4.1)(4.2)(5.1.3) \\
	\end{tabular}\\
	\indent (5.6) \m{I_C \not\models R}\\
	\indent \indent \begin{tabular}{llll}
	\m{[B]}                       & (5.1.2.2)\\
	\m{[C_0]}                     & (2e)\\
	\m{[D_0]}                     & (5.2)\\
	\m{[\termRepAt{s}{r}{p} = t]} & (2e)(5.3)\\
	\end{tabular}\\
	\indent (5.7) \m{I_{R+} \not\models R} (5.3) (5.4) (lemma \ref{lemma_4.3.1.14})\\
	\indent (5.8) Contradiction to (i.h.) (a3)(5.4)\\
	(5e) \m{r^t(s') = s'} (5)-(5.8)

	\bigskip 
	
	\noindent
	\textbf{Maximal term occurs in a non-maximal equality literal}\\
	(6) If there is a literal \m{\textcolor{blue}{l=r} \in C_0} s.t.\\
	\indent \indent (6.0.1) \m{s'=l'}\\
	\indent \indent (6.0.2) \m{I_C \models t'=r'}\\
	\indent (6.1) We define \m{C_0 = \textcolor{blue}{C_1} \lor l=r}\\
	\indent (6.3) \m{t' \succeq r'} (4.2)\\
	\indent (6.4) \m{l' \succ r'} (2e)\\
	\indent (6.5) There is some \m{\textcolor{blue}{B}} s.t. (3.3) (lemma \ref{lemma_4.3.1.1})\\
	\indent \indent (6.5.1) \m{B = \idasg{l}{s}}\\
	\indent \indent (6.5.2) \m{I^b \not\models B}\\
	\indent \indent (6.5.3) \m{B \prec l^t}\\
	\indent (6.6) The following instance of \m{fact^i_{=}} is valid in N, \\
	\indent \indent  \indent we name the conclusion \textcolor{blue}{R}:

	\bigskip 

		\indent \indent \indent
		$\vcenter{
		\infer[
%		\m{\vdash_p l = s \mid E}
		]
		{\m{B \lor C_1 \lor r \neq t \lor s = r }}
		{
			\m{C_1 \lor l = r \lor \underline{s = t}}
		}
		}$
	
	\bigskip 

		\noindent
		\indent\indent\indent
		\begin{tabular}{llll}
		(i)   & \m{[s \not\preceq_i t]}              & (4.3)(lemma \ref{lemma_succ_i})\\
		(ii)  & \m{[s=t \not\preceq_i C_1 \lor l=r]} & (4.1)(4.2)(6.1)(lemma \ref{lemma_succ_i})\\
		(iii) & \m{[l \not\preceq_i r]}              & (6.4)(lemma \ref{lemma_succ_i})\\
		(iv)  & \m{[s \succ l^t]}                    & (4.3.1)\\
		(v)   & \m{[B = \idasg{l}{s}]}               & (6.5.1)\\
		\end{tabular}
		
		\bigskip

	\indent (6.6) \m{R' \prec C'}\\
	\indent \indent \begin{tabular}{llll}
	\m{[B]}          & (6.5.3)\\
	\m{[C_1]}        & (6.1)(4.2)\\
	\m{[r \neq t]}   & (4.2)(6.0.1)(6.4)\\
	\m{[s=r]}        & (4.2)(6.0.1)(6.3)(6.4)\\
	\end{tabular}\\
	\indent (6.7) \m{I_C \not\models R} \\
	\indent \indent \begin{tabular}{llll}
	\m{[B]}         & (6.5.2)\\
	\m{[C_1]}       & (2e)(6.1)\\
	\m{[r \neq t]}  & (6.0.2)\\
	\m{[s=r]}       & (2e)(6.0.2)\\
	\end{tabular}\\
	\indent (6.8) \m{I_{R+} \not\models R} (6.6) (6.7) (lemma \ref{lemma_4.3.1.14})\\
	\indent (6.9) Contradiction to (i.h.)(a3)(6.6)\\
	(6e) \m{ \forall l=r \in C_0 \cdot (l'=s' \Rightarrow I_C \not\models t' = r'}\\
	\indent (6e.1) \m{s'=t' \succ C_0}
	
	\bigskip

	\noindent
	(7) \m{E^t_C = \s{s' = t'}} (2e) (3e) (4.4e) (5e) (6e) (6e.1) (definition \ref{def_4.3.1.2})\\
	(7.1) \m{I_{C+} \models C} (4.1)(7)(definition \ref{def_4.3.1.2})
	
%	$\blacksquare$
\end{proof}
\end{theorem}


\textbf{Interpolation corollary:}\\
We can get a version of the (reverse) interpolation theorem for ground clausal first order logic with equality as a corollary:\\
For two sets of clauses \m{A,B} where \m{A \not\models \emptyClause}, \m{B \not\models \emptyClause} but 
\m{A,B \models \emptyClause}, we can show a set of clauses S s.t. \m{A \models S}, \m{S,B \models \emptyClause}, 
and \m{S} includes only constants that appear in both A and B:\\
We give the weight \m{2\omega+1} to each constant in \m{consts(A) \setminus consts(B)},\\
\m{1\omega+1} to each constant in \m{consts(A) \cap consts(B)} and \\
\m{1} to each constant in \m{consts(B) \setminus consts(A)}.\\
We set \m{l^b = \omega}, \m{l^t = 2\omega}.\\
As the calculus is complete, if we generate all valid inferences fairly, we will derive the empty clause eventually.\\
We define the interpolant J as \m{J = \s{C \in SP_I(A) \mid C \prec l^t}} - where \m{SP_I(A)} is the set of clauses we have derived with premises in A that are in \m{N^b}.\\
By the soundness of \m{SP_I} \m{A \models J} and by definition \m{\emptyClause \in SP(J \cup B)}.

Similarly we can build sequence interpolants using a separate \m{l^t,l^b} for each pair of nodes arranged accordingly,
and, as we will show later, with some reworking of the ordering we can also construct DAG interpolants.

