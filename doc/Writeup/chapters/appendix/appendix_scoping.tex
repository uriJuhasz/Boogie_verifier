\chapter{Improved justification extraction for scoping}\label{appendix:gamma_approximation}
In this section we show an alternative encoding of the approximation $\Gamma$ of the equivalence relation of successor CFG-nodes.
This alternative encoding uses a hierarchical structure of EC-graphs rather than a single EC-graph for approximation, and hence can help us reduce the size of the interpolants we produce.

In this sub-section we describe an improvement of the part of the algorithm that over-approximates $\Gamma$ - 
it mostly replaces \m{g_n^i} in deciding \eqg{} and especially in generating interpolation clauses (ensuring optimal clauses) and justifications.
We have seen sub-optimal interpolation clauses in the example in figure \ref{example_4.2.1.3}.
We assume a given EC-graph \m{g^t} and the set of equations $\Gamma$ over \m{I^t}, which over-approximates the equations of all suffix paths restricted to \m{I^t}.\\
\textbf{Basic idea:}\\
Instead of the graph \m{g_n^i}, we create a tree of EC-graphs, where the root is \m{g^t} and each tree-node is an EC-graph obtained from its parent by \lstinline|assuming| one relevant equation from $\Gamma$ - this equation is the label of the parent edge.
The leaves are the EC-graphs with no relevant equalities or that are in conflict (that is, an EC-node has a dis-equality self-loop).
At each tree-node we have an EC-graph g obtained from its tree-parent EC-graph by \lstinline|assuming| the edge label of its parent edge (the edge label is an equation of two EC-nodes of \m{g^t}, we follow the source-edges backwards from \m{g^p} to \m{g^t} (selecting one representative when there are several options) in order to translate the equation to be over two EC-nodes of \m{g^p}).
Each such tree-node g is associated with the sequence S of interface equations on the path leading to it from the tree-root.
We establish the source invariant between each EC-graph g and the EC-graph \m{g^p} of its parent tree-node.\\
\textbf{Extracting interpolants:}\\
For each such tree-node g we may output some interpolation clause-ECs:
The body is always the disjunction of the negation of all equations on the DAG-path leading to g.
If g is in conflict, we output only the body.
Otherwise, we basically want to output one interpolation clause for each pair of interface terms \m{s,t \in \terms{g^t}} s.t. \m{g \models s=t \land g^p \not\models s=t}. This set of pairs of terms can be infinite if g has cycles (e.g. a=f(a)), but we can output a sufficient number of clauses that imply any such equality.\\
The set of heads of Horn clauses for g is defined as follows:\\
The simple case is when two interface nodes \m{u,v \in g^p} are merged into one interface node \m{w \in g} by congruence closure - for example, in \ref{f.fs.ug.rc.1}, the tree has one tree-node beside the root - \m{g_{a=b}} (as a=b is the only relevant equality - in fact it is \m{[a]^t=[b]^t}). 
At \m{g^t}, \m{[c]^t \neq [d]^t} while at \m{g_1}, \m{[c]_{a=b} = [d]_{a=b}}, hence we output the interpolation clause 
\m{[a]^t \neq [b]^t \lor [c]^t = [d]^t}.\\
Looking at example \ref{example_4.2.1.3}, out tree will have four non-root nodes - \\
\m{g_{a=b},g_{a=b,c=d},g_{c=d},g_{c=d,a=b}}\\ 
We only consider here \m{g_{a=b},g_{c=d}} as we will show below how to avoid adding the other two.\\
At \m{g_{a=b}} we output \m{[a]^t \neq [b]^t \lor [e_1]^t = [e_2]^t} 
while at \m{g_{c=d}} we output \\
\m{[c]^t \neq [d]^t \lor [e_3]^t = [e_4]^t} - so we have produced the optimal interpolation clauses.\\
The second case for the head of an interpolation clause is when, for some EC-node \m{n \in g},
we have a pair of interface \GFAs{} \m{\fa{f}{s},\fa{g}{t} \in n}, s.t. at least one of the \GFAs{} has no source in \m{g^p} that is an interface gfa.\\
In example \ref{example_4.2.1.3_0}, at the graph \m{g_{a=b}} with the parent \m{g^t},
the EC-node \m{[e]_{a=b}} has two interface gfas, \s{f([e,\textcolor{red}{y}]),e()} where the first \GFA{} has just one source in \m{g^t} - \m{f([\textcolor{red}{y}])}, which is not an interface gfa, and hence we output the interpolation clause \m{[a]^t \neq [b]^t \lor [e]^t = [f(e)]^t}.
 Note that \m{[f(e)]^t} does not exist in \m{g^t} and has to be added.\\
The way we choose which EC-nodes to add to \m{g^t} is as follows - we define a total order on ground terms s.t. all interface terms are smaller than all non-interface terms (described later), each \GFA{} is represented by its minimal term and each EC-node is represented by the minimum on terms representing its gfas. If the representatives of our interface \GFAs{} at g do not appear in \m{g^t}, we add them. In the above example, \\
\m{\terms{[e]_{a=b}} = \s{f^k(e),f^k(\textcolor{red}{y}) \mid k \geq 0} \cup \s{h(a,\textcolor{red}{x}),h(b,\textcolor{red}{x})}} - the smallest interface term is e. \\
For the \GFA{} \m{f([e])} the representative is \m{f(\rep{[e]})=f([e])}, which has to be added to \m{g^t}.

\subsection*{Reducing redundancy using a DAG}
Evidently, as described above, the algorithm is very inefficient, as each set of interface equalities can be reached on several paths - one for each order of \lstinline|assuming| the equalities - we could use sets instead of sequences to characterize tree-nodes.\\
Hence we use a DAG rather than a tree, where each DAG-node that represents the set of equalities S will have, for each \m{u=v \in S}, 
a parent DAG-edge to the DAG-node that represents \m{S \setminus \s{u=v}}.
We avoid creating a child node if \emph{any} of its parents are in conflict, as it will not produce any new interpolation clauses.\\
Additionally, we only output interpolation clauses as above if the conditions for generating the clause hold for \emph{all} parents.\\
For example, in figure \ref{example_4.2.1.4}, the root of the DAG is the graph \m{g^t}.\\
The root has two depth-1 children - \m{g_{a=b},g_{c=d}}, as per the relevant equalities - producing the interpolation clauses 
\m{a\neq b \lor e_1=e_2,c\neq d \lor e_3=e_4} respectively.\\
Each of these children has a relevant equality, producing the DAG-node \\
\m{g_{a=b,c=d}}.
%and hence we are done. 
%Even if we had a reason to add the DAG-node \m{g_{a=b,c=d}} 
%(e.g. if we were to add \m{h(a,b,b,\textcolor{red}{x})\neq h(c,d,c,\textcolor{red}{x})} to \m{N^t}),
However, we will not produce the interpolation clauses \\
\m{a\neq b \lor c\neq d \lor e_1=e_2,a\neq b \lor c\neq d \lor e_3=e_4} because
\m{g_{a=b} \models e_1=e_2} and\\
 \m{g_{c=d} \models e_3=e_4}.

%\textbf{Extracting interpolants:}\\
%In order to extract interpolants, we use the definition of the sources function. For each DAG-node EC-graph \m{g_S}, for each parent DAG-node EC-graph \m{g_p} we can calculate the sources function between \m{g_S} and \m{g_p} as usual, except that in this case the multiple parents represent a meet rather than a join (we enforce the sequential invariant in each case). We refer to the sources of EC-nodes and \GFAs{} as defined in chapter \ref{chapter:ugfole}.\\
%Each EC-graph \m{g_S} at a DAG-node generates the following Horn clauses, each with the antecedent of the disjunction of the negation of all equalities in S:
%\begin{itemize}
	%\item If \m{g_S} is in conflict, we produce only the head-less Horn-clause, otherwise
	%\item For each EC-node \m{u \in g_S} with two \GFAs{} \m{\fa{f}{s} \neq \fa{g}{t} \in u} that each have an interface representative and, for each parent EC-graph \m{g_p} of \m{g_S}, the sources of the \GFAs{} are in separate EC-nodes, we produce the Horn clause with the head 
	%\m{\rep{\fa{f}{s}}=\rep{\fa{g}{t}}} (this covers examples e.g. \ref{example_4.2.1.4}, \ref{f.fs.ug.rc.1}).
	%\item For each EC-node \m{u \in g_S} with two \GFAs{} \m{\fa{f}{s} \neq \fa{g}{t} \in u} that each have an interface representative and, for each parent EC-graph \m{g_p} of \m{g_S}, the source of \fa{f}{s} has no interface representative, we produce the Horn clause with the head 
	%\m{\rep{\fa{f}{s}}=\rep{\fa{g}{t}}} (This covers e.g. example \ref{example_4.2.1.2_0}).
%\end{itemize}
%
%The algorithm described above will generate all possible optimal interpolation clauses for any \m{g^b}.

\subsection*{Reducing redundancy using disjoint parts}
Looking at example \ref{example_4.2.1.3}, we see that the equalities \m{a=b,c=d} are independent in the following sense:
If we \lstinline|assume(a=b)|, we have not modified the part of the EC-graph which is the (\GFA{}) downward closure of \m{[e_3],[e_4]} and vice versa, where \lstinline|assume(c=d)| does not affect \m{[e_1],[e_2]}.\\
In this case, although \m{g_{a=b}} has a relevant equality \m{c=d} and \m{g_{c=d}} has a relevant equality \m{a=b}, we will not gain any new interpolation clause by calculating \m{g_{a=b,c=d}} because each interpolation clause is subsumed by one of its parents.
In order to take advantage of this fact, for each EC-graph \m{g_{u=v}} of depth one in the DAG, we trim any EC-node that is not in the down-closure of the up-closure of u,v - that is any term that is not needed to represent any super-term of u,v.\\
In our example the sub-term closure of the super-term closure of \s{[a],[b]} is \s{[a],[b],[e_1],[e_2],[\textcolor{red}{x}]} 
and similarly for \s{[c],[d]} it is \s{[c],[d],[e_3],[e_4],[\textcolor{red}{x}]}.\\
After trimming \m{g_{a=b},g_{c=d}}, they both have no more relevant equalities, and hence we avoid adding the meet DAG-node \m{g_{a=b,c=d}}.\\
In order to maintain completeness, we must, besides super-term closure, also maintain closure w.r.t. \eqg - consider the example in figure \ref{example_4.2.1.6}:
\begin{figure}[H]
\m{N^t=\s{\textcolor{red}{y_0}=g(a,\textcolor{red}{x}),\textcolor{red}{y_1}=g(b,\textcolor{red}{x})=h(c,\textcolor{red}{x}),
\textcolor{red}{y_2}=h(d,\textcolor{red}{x}),\textcolor{red}{y_0} \neq \textcolor{red}{y_2}}}
\caption{Example for different upward closures\\
Without \eqg closure:\\
The graph \m{g_{a=b}} will include \s{[a],[b],[c],[\textcolor{red}{x}],[\textcolor{red}{y_0},\textcolor{red}{y_1}]}\\
The graph \m{g_{c=d}} will include \s{[b],[c],[d],[\textcolor{red}{x}],[\textcolor{red}{y_1},\textcolor{red}{y_2}]}\\
Both graphs will be leaves - we are missing the closure with \m{\textcolor{red}{y_0} \eqg \textcolor{red}{y_2}}.
}
\label{example_4.2.1.6}
\end{figure}


It is also sufficient to include in each DAG-node only the EC-nodes that can be merged by one application of congruence closure, which are sufficient for calculating the relevant equalities, and only transitive super-terms once congruence closure is performed.

\textbf{Reducing redundancy using path compression:}\\
One case we have not yet covered is an asymmetric meet - consider the example in figure \ref{example_4.2.1.7}
\begin{figure}[H]
\m{N^t=\s{b_0=g(a_0,\textcolor{red}{x}),b_1=g(a_1,\textcolor{red}{x}),
\textcolor{red}{y_0}=h(b_0,c,\textcolor{red}{x}),\textcolor{red}{y_1}=h(b_1,d,\textcolor{red}{x}),
\textcolor{red}{y_0}\neq \textcolor{red}{y_1}}}
\caption{Example for an asymmetric meet\\
The graph \m{g_{a_0=a_1}} will have a node \m{[b_0,b_1]} and hence is effectively also \m{g_{a_0=a_1,b_0=b_1}}\\
The graph \m{g_{b_0=b_1}} will have \m{g_{a_0=a_1}} as a child.\\
The interpolation clauses will be:\\
\s{a_0 \neq a_1 \lor b_0=b_1,b_0 \neq b_1 \lor c \neq d}
}
\label{example_4.2.1.7}
\end{figure}
Here we can compress the DAG-path \m{a_0=a_1.b_0=b_1} to \m{a_0=a_1} because of the implication.\\
When extracting interpolation clauses we can skip \m{a_0 \neq a_1 \lor c \neq d} because it is derivable from the other two 
- this is detected because \m{g_{b_0=b_1} \models c=d} and \m{g_{b_0=b_1}} is a parent to \m{g_{a_0=a_1}} which is, in fact, \m{g_{a_0=a_1,b_0=b_1}}. We do not reject the interpolation clause \m{a_0 \neq a_1 \lor b_0=b_1} although \m{g_{b_0=b_1} \models b_0=b_1} because \m{[b_0]=[b_1]} is a consequence of \m{[a_0]=[a_1]} in \m{g_{a_0=a_1}} and \emph{directly} assumed in \m{g_{b_0=b_1}}.

Another instance of path compression is shown in figure \ref{example_4.2.1.8}:
\begin{figure}[H]
\m{N^t=\s{c=g(a_0,b_0,\textcolor{red}{x}),d=g(a_1,b_1,\textcolor{red}{x})}}\\
\m{N^b=\s{a_0=a_1,b_0=b_1,c \neq d}}
\caption{Example for path compression\\
There is no need for the graphs \m{g_{a_0=a_1},g_{b_0=b_1}}, only \m{g_{a_0=a_1,b_0=b_1}}.
}
\label{example_4.2.1.8}
\end{figure}
\noindent
We can skip any DAG-node where no congruence closure has been performed in comparison with \emph{all} parents - any interpolation clause derived by transitive closure is not needed (e.g. if \m{g^t \models a=b,c=d} and \m{g^b \models b=c} then there is no need for the clause \m{b\neq c \lor a=d} because we use the clauses \m{a=b,c=d} which are on the interface (which is transitively closed).\\
In the incremental case, we might encounter a non-incremental interpolant - for example, in \ref{example_4.2.1.8}, 
if we were to remove \m{c\neq d} from \m{N^b}, derive the interpolation clause \m{a_0 \neq a_1 \lor b_0 \neq b_1 \lor c=d},
and later derive in \m{N^t} (in another fragment) \m{c=f(a_0,\textcolor{red}{x}),d=f(a_1,\textcolor{red}{x})}, then our interpolation clause is obsolete and should be replaced (subsumed) by \m{a_0 \neq a_1 \lor c=d} - as we have stated before, 
removing the old clause from all successors is not very efficient, but, in the case where we have several successors,
we cannot skip adding the smaller clause, as some successors might only entail \m{a_0=a_1} and not \m{b_0=b_1} e.g. \m{N^t = \s{a_0=a_1,c\neq d}}.
Hence, we add and propagate the shorter clause as well and rely on subsumption to eliminate the unnecessary clauses 
(although at the price of some redundant derivations). \\
We can, however, eliminate the DAG node \m{g_{a_0=a_1,b_0=b_1}} as it has been subsumed.\\
Using the above described path compression we ensure that, for each DAG-node, 
at least one pair of non-interface \GFAs{} is merged that was not merged in \emph{all} parents.

\subsection*{Complexity}
The DAG can have, in the worst case, a single DAG-node per any subset of the possible interface-node equalities - hence \bigO{2^n} complexity for the set of interpolation clauses.\\
An example that achieves exponential complexity:\\
\m{N^t =\s{\bigwedge\limits_{i\in I,j \in J} (f(a_{i,j},\textcolor{red}{x})=\textcolor{red}{y_i} \land f(b_i,\textcolor{red}{x})=\textcolor{red}{z_i}), c = f(\textcolor{red}{\tup{z}}), d = f(\textcolor{red}{\tup{y}})}}\\
Here the non-redundant interpolants are:\\
\s{\bigwedge\limits_{\tup{k} \in J^{\size{I}}} (c = d \lor \bigvee\limits_{i \in I} a_{i,{k_i}} \neq b_i) }\\
A total of \m{\theta(\size{J}^{\size{I}})} clauses for an input size of \bigO{\size{I} \times \size{J}}.

\textbf{Worst case lower space bound, non-unit case:}\\
The worst case complexity for any interpolant between two sets of ground (not necessarily unit) clauses is exponential, as expected from the propositional case (e.g. pigeon-hole principle).
Even if the first set of clauses includes only unit ground clauses, the worst case minimal CNF interpolant is still exponential - for example:\\
\m{N^t =\s{\bigwedge\limits_{i\in I,j \in J} (f(a_{i,j},\textcolor{red}{x})=\textcolor{red}{y_i} \land f(b_{i},\textcolor{red}{x})=\textcolor{red}{z_i}), c = f(\textcolor{red}{\tup{z}}), d = f(\textcolor{red}{\tup{y}})}}\\
\m{N^b =\s{\bigwedge\limits_{i \in I} (\bigvee\limits_{j\in J} a_{i,j} = b_{i}), c \neq d}}\\
Where the minimal CNF interpolant is:\\
\s{\bigwedge\limits_{\tup{k} \in J^n} (c = d \lor \bigvee\limits_{i \in I} a_{i,k_i} \neq b_i)}\\
As above, the minimal CNF interpolant is exponential in the input size.

\textbf{Worst case upper space bound, unit case, binary:}\\
The maximal space complexity of a minimal interpolant between two sets of unit ground clauses is polynomial, based on our above algorithm (the basic algorithm without the EC-graph DAG):\\
We let k be the number of EC-nodes in the larger between \m{g^t} and \m{g^b}. At each step we transfer at least one equality from \m{g^b} to \m{g^t}, hence decreasing the number of nodes by at at least one, and at least one equality from \m{g^t} to \m{g^b} - hence the number of literals of each interpolation clause (Horn clause) is at most k and we have at most k Horn clauses - hence the interpolant (conjunction of above Horn clauses) is of at most \m{k^2} size as a function of interface nodes (clause ECs).\\
An interpolant over terms rather than EC-nodes could be exponential - for example:\\
\m{N^t = \s{\bigwedge\limits_{i \in 1..n} \textcolor{red}{x_{i+1}} = f(\textcolor{red}{x_{i}},\textcolor{red}{x_{i}}),\textcolor{red}{x_0}=a,\textcolor{red}{x_n} \neq b}}\\
\m{N^b = \s{\bigwedge\limits_{i \in 1..n} \textcolor{blue}{m_{i+1}} = f(\textcolor{blue}{m_{i}},\textcolor{blue}{m_{i}}),\textcolor{blue}{m_{0}}=a,\textcolor{blue}{m_{n}} = b}}\\
Here, the size of the minimal representative term of \m{[\textcolor{red}{x_n}]^t} is exponential in n, 
and similarly for \m{[\textcolor{blue}{m_n}]^b}. However, if we allow shared representation (so that we count only distinct sub-terms), then the representation is polynomial.

\textbf{Worst case lower space bound, unit case, CFG:}\\
The problem of calculating an interpolant between a CFG node and all paths from it to assertions is equivalent to the non-unit case,
even if each CFG-node only contains unit clauses - we can simulate an n-literal CNF clause by an n-branch with each literal assumed in one arm of the branch, which is immediately joined. Hence in this case the complexity is exponential.

\textbf{Bounded fragments:}\\
In realistic scenarios, we do not expect the above worst case example to be common, as e.g. examples of the pigeon-hole principle are not common as a verification condition for propositional programs.\\
In order to ensure a time and space complexity bound we simply limit the depth of any path in the DAG - as the out-degree of each DAG-node of depth k is at most \m{\binom{\size{g^t} - k}{2}} (quadratic), it is easy to calculate the maximal size of the DAG up to any given depth n, 
with path compression we would not expect the necessary depth to be high for most realistic examples.

\subsection*{Unit disequalities}
The algorithm for verification with only ground unit equalities presented in chapter \ref{chapter:ugfole} assumed (for simplicity) that all dis-equalities occur at assertion nodes. We have seen how dis-equalities can be represented in the EC-graph using dis-equality edges between EC-nodes. Propagating dis-equalities using sources is simple - for a CFG-node n, and two EC-nodes u,v,  if in all predecessors there is a dis-equality edge between a source of u and a source of v, we can add the edge also between u and v. We also need to propagate dis-equalities on super-terms of EC-nodes with more than one source - but we omit the details here. 
In this section we are interested in the effect of scoping on the propagation of dis-equalities. 
We show that our technique for generating interpolants can be used to generate dis-equality interpolants that strengthens the EC-graph based verification algorithm by allowing it to propagate more dis-equalities.

For a unit dis-equality at a CFG-node n that includes a sub-term that is not in scope in a successor CFG-node s, we can have several options for a unit interpolant - for example:\\
\s{g(f^7(a),\textcolor{red}{x}) \neq g(f^6(a),\textcolor{red}{x})} \\
implies each of \s{f^{i+1}(a) \neq f^{i}(a) \mid 0\leq i <7}\\
While \\
\s{g(f^7(a),\textcolor{red}{x}) \neq g(f(a),\textcolor{red}{x})} \\
implies the following (direct implication graph):
\begin{figure}[H]
\begin{tikzpicture}
	\node (1)                      {$\m{f^7(a) \neq f(a)}$};
	
	\node (2a)[below       = 1cm of 1 ]     {$\m{f^6(a) \neq a}$};
	\node (2b)[below left  = of 1 ] {$\m{f^4(a) \neq f(a)}$};
	\node (2c)[below right = of 1 ]     {$\m{f^3(a) \neq f(a)}$};
	
	\node (3a)[below left  = 1cm and 0.2cm of 2a] {$\m{f^3(a) \neq a}$};
	\node (3b)[below right = 1cm and 0.2cm of 2a] {$\m{f^2(a) \neq a}$};
	
	\node (3c)[below       = 1cm of 2a] {$\m{f^2(a) \neq f(a)}$};

	\node (4a)[below       = 1cm of 3c] {$\m{f(a) \neq a}$};
	\draw[double,->] (1) to  (2a);
	\draw[double,->] (1) to  (2b);
	\draw[double,->] (1) to  (2c);
	\draw[double,->] (2a) to  (3a);
	\draw[double,->] (2a) to  (3b);
	\draw[double,->] (2b) to  (3a);
	\draw[double,->] (2b) to  (3c);
	\draw[double,->] (2c) to  (3b);
	\draw[double,->] (2c) to  (3c);
	\draw[double,->] (3a) to  (4a);
	\draw[double,->] (3b) to  (4a);
	\draw[double,->] (3c) to  (4a);
\end{tikzpicture}
\caption{
Unit disequalities
}
\label{example_4.2.5}
\end{figure}

Intuitively, the simplest inteprolant is \m{f^7(a) \neq f(a)}.
In the set of unit dis-equalities implied by one top unit dis-equality, 
the \emph{top disequalities} are those that are not implied by any other dis-equality in the set - in the above \\
\s{f^4(a) \neq f(a),f^3(a) \neq f(a),f^6(a) \neq a}

Consider the following example:\\
\m{\clauses{p} = \s{f^7(a) \neq f(a),f^4(a)=y,f^7(b) \neq f(b),f^4(b)=z}}\\
\m{\clauses{n} = \s{f^3(a)=a \lor f^2(b)=b}}\\
Here, the interpolants \s{f^7(a) \neq f(a),f^7(b) \neq f(b)} imply all the others.\\
For EC-graphs with a depth bound of 1 and relative depth (as described in chapter \ref{chapter:bounds}), 
we cannot represent \m{f^7(a),f^7(b)} at n, but we can represent \m{f^3(a)\neq a} and \m{f^2(b)\neq b} - 
hence we want to generate the top dis-equalities for each dis-equality that is too deep for the successor.

\textbf{Algorithm:}\\
We can easily use our DAG-based algorithm to find the unit disequalities as follows:\\
We develop our DAG for a depth of 1, using only dis-equalities as goals (that is, we do not use pairs in \m{I^{\Gamma}} in order to calculate G).\\
Each conflicting EC-graph at a DAG-node represents one unit disequality.\\
This can be done incrementally, using any variant of $\Gamma$ (over-approximation of equalities in transitive successors).

