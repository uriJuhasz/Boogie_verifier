\chapter{Preliminaries}\label{chapter:preliminaries}
In this chapter we present the formalisms with which we work.\\
Most of the chapter is a repetition of standard definitions, except for section \ref{section:preliminaries_programs} which discusses our assumptions about the input program, and transformations we perform on it before our algorithm begins.

\section*{Sets, multisets and sequences}
A \newdef{multiset} \m{m} over a set \m{S} is a function \m{m : S \rightarrow \mathbb{N}_0}.\\
We use 0 for the empty multiset.\\
Singleton multisets are defined as:\\
 \m{[x \mapsto n](y) \triangleq \ite{x\equiv y}{n}{0}}.\\
Multiset union is defined as:\\
 \m{(m \cup n)(x) \triangleq \lambda x \cdot m(x)+n(x)}.\\
Sub-multiset relation is defined as follows:\\
 \m{m \subseteq n \equivdef \forall i \in \mathbb{N}_0 \cdot m(i) \leq n(i)}

\noindent
An \newdef{equivalence relation} \m{\approx} over a set $\m{S}$ is a subset of $\m{S^2}$ s.t.\\
\m{\forall (x,y)\in\approx \cdot (y,x) \in \approx}\\
\m{\forall (x,y),(y,z) \in \approx \cdot (x,z) \in \approx}\\
\m{\forall x \in S \cdot (x,x) \in \approx}\\
For an equivalence relation $\m{\approx}$ over $\m{S}$, and a term $\m{x \in S}$ we denote by 
$\m{\ECOf{x}{\approx}}$ the equivalence class of $\m{x \in S}$ with respect to $\m{\approx}$:\\
\m{\ECOf{x}{\approx} \triangleq \s{y \in S \mid x \approx y}}

\noindent
A \newdef{partition} of a set $\m{S}$ - $\m{P \subseteq \powset{S} }$ - satisfies:\\
\m{\forall S_1,S_2 \in P \cdot S_1 \cap S_2 = \emptyset}\\
\m{\cup P = S}

\noindent
The \newdef{quotient set} of a set $\m{S}$ for the equivalence relation $\m{\approx}$, $\m{S/\approx}$ is the set:\\
\s{ \ECOf{x}{\approx} \mid x \in S}\\
and is a partition of \m{S}, and similarly a partition defines an equivalence relation.

A (finite) sequence of length n of elements of a set S is a function from \m{0..n-1} to S.
For a sequence s we use \m{s_i} for the ith element of the sequence - \m{s(i)}. 
\size{s} is the length of the sequence (\size{\dom{s}}). We sometimes use the notation \m{[i \mapsto e(i)]} to denote sequences, where \m{e(i)} is an expression that defines the ith element, and \m{<s>} for a singleton sequence. We use this notation where the domain is unambiguous.
The concatenation of the sequences s,t is denoted by \m{s.t}.
We use sequences to denote paths in the program CFG (as a sequence of CFG nodes) and we extend the notation for concatenation to include single CFG-nodes - so, for example, if n,p are nodes and P,Q are sequences of nodes, r.P.p.Q is the sequence \m{P.<p>.Q.<n>}.

\section*{Logic}

\subsection*{Syntax and notation}

A \newdef{language} is defined formally as follows:\\
A function or predicate symbol (denoted $\func{f},\func{g},\func{h}$) has a fixed arity ($\arity{f}\geq 0$).\\
A \newdef{signature} $\m{\Sigma=\Fs{\Sigma} \cup \Ps{\Sigma} \cup \Vs{\Sigma}}$
is a set of function symbols $\Fs{\sig}$, predicate symbols $\Ps{\sig}$ and variables $\Vs{\sig}$.\\
%(We will mention non-ground definitions here to be consistent with later chapters, but assume for this chapter that \m{\Ps{\sig}=\Vs{\sig}=\emptyset}). 
We only handle finite programs and sets of axioms, so that the sets $\Fs{\Sigma},\Ps{\Sigma}$ used in any VC are finite. 
However, in order to be able to handle theories with countable signatures, such as integer arithmetic, we allow both these sets to be countable.
The set of \newdef{constants} is $\m{\consts{}=\s{\func{f} \in \funcs{} \mid \arity{f}=0 }}$, we assume $\|\consts{}\|>0$ (otherwise the ground fragment is trivial).\\
We denote a (possibly empty) tuple by an overline - detailed in the following.\\
We use a standard definition for the language:
\begin{figure}[H]
$
\begin{array}{llllll}
	\mathbf{function}  & \mbox{f,g,h}   &     &                                       & \in \Fs{\sig} & \textrm{functions}\\
	\mathbf{predicate} & \mbox{P,Q,R}   &     &                                       & \in \Ps{\sig} & \textrm{predicates}\\
	\mathbf{variable}  & \mbox{x,y,z}   &     &                                       & \in \Vs{\sig} & \textrm{variables}\\
	\mathbf{term}      & \mbox{t,s,u,v} & ::= & \m{\fa{f}{t} \mid x}                  & \in \Ts{\sig} & \textrm{free term algebra }\\
	\mathbf{atom}      & \mbox{a}       & ::= & \m{s = t \mid \fa{P}{t}}              & \in \As{\sig} & \textrm{the atoms over } \Sigma\\
	\mathbf{literal}   & \mbox{A,B}     & ::= & \mbox{a} \mid \lnot \mbox{a}          & \in \Ls{\sig} & \textrm{the literals over } \Sigma\\
	\mathbf{clause}    & \mbox{C,D}     & ::= & \m{\emptyClause \mid A \mid A \lor C} & \in \Cs{\sig} & \textrm{the clauses over } \Sigma\\
\end{array}
$
\caption{language}
\end{figure}

\noindent
We use $\term{s,t,u,v}$ for terms, $\term{\tup{s},\tup{t},\tup{u},\tup{v}}$ for term tuples,
we also construct tuples from terms using parenthesis - e.g. $\term{(t,s)}$.\\
We occasionally treat an n-tuple as a sequence of n ground terms.\\
$\tupAt{t}{i}$ is the i-th element of the tuple $\tup{t}$ and $\tupL{t}$ is the number of terms (size or length) in a tuple. \\
We treat an equality atom as an unordered set and so $\term{(s=t)} \equiv \term{(t=s)}$.\\
We use $\term{s \neq t}$ to denote \term{\lnot s = t}.\\
As we do not manipulate negations syntactically, we consider $\m{\lnot \lnot a \equiv a}$.\\
We treat clauses as sets of literals whose semantics is the disjunction of these literals.\\
We denote the \newdef{empty clause} by \newdef{\emptyClause}.\\
We use $\term{\bowtie}$ to denote either $\term{=}$ or $\term{\neq}$.

\noindent
We define the set of terms \newdef{\terms{S}} of a set of clauses as follows:\\
$
\begin{array}{llll}
\terms{S}           & \triangleq & \bigcup\limits_{\m{C} \in \m{S}} \terms{C} \\ 
\terms{C}           & \triangleq & \bigcup\limits_{\m{l} \in \m{C}} \terms{l} \\ 
\terms{s \bowtie t} & \triangleq & \terms{s} \cup \terms{t} \\ 
\terms{\fa{P}{s}}   & \triangleq & \terms{\tup{s}} \\ 
\terms{\tup{s}}     & \triangleq & \m{\bigcup\limits_{i} \terms{\tupAt{s}{i}}} \\ 
\terms{\fa{f}{s}}   & \triangleq & \s{\fa{f}{s}} \cup \terms{\tup{s}} \\
\terms{x}           & \triangleq & \s{x} \\
\end{array}
$

\noindent
For a set $\mathrm{S}$ we denote by $\mathrm{F_n(S)} \triangleq \mathrm{S^n} \rightarrow \mathrm{S}$ the set of all functions of arity \m{n} over \m{S} and \m{F(S) \triangleq \bigcup\limits_{n \in  \mathbb{N}_0} F_n(S)} the set of all functions over \m{S}.\\
Similarly, we define relations over S as \m{R_n(S) \triangleq P(S^n)} and \m{R(S) \triangleq \bigcup\limits_{n \in  \mathbb{N}_0} \mathrm{R_n(S)}}.

\noindent
For the semantics we use functions from terms to a domain $\term{D}$, $\function{f} : \Ts{\sig} \rightarrow \term{D}$.
When applying such a function to a tuple $\tup{t}$ we mean the point-wise application of the function that returns a tuple in
$\term{D}^{\tupL{t}}$ - so $\term{f}(\tup{t})_{\m{i}} = \term{f}(\tupAt{t}{i})$.


\bigskip

\noindent
A \newdef{structure} $\m{\mathbf{S}=(D_S,F_S,P_S)}$ for a signature $\mathbf{\Sigma}$ includes:
\begin{itemize}
	\item A domain $\mathrm{D_S}$ which is a non-empty set
	\item An interpretation for function symbols $\m{F_S}$ which maps each function of $\sig{}$ to a function of the corresponding arity over $\mathrm{D_S}$ -
namely $\mathbf{F_S} \in \mathbf{F_{\sig{}}} \rightarrow \mathrm{F(D_S)}$ such that
$\forall \m{f} \in \mathbf{\sig} \cdot \m{F_S}(\m{f}) \in \m{F_{\arity{f}}(S)}$. 
	\item An interpretation $\m{\mathbf{P_S}}$ for predicate symbols that maps each predicate symbol of arity n to a relation of arity n over $\m{D_S}$ - namely $\mathbf{P_S} \in \mathbf{P_{\sig{}}} \rightarrow \mathrm{R(D_S)}$ such that
$\forall \m{P} \in \mathbf{\sig} \cdot \m{P_S}(\m{P}) \in \m{R_{\arity{P}}(S)}$. 
\end{itemize}

\noindent
An \newdef{interpretation} \m{\mathbf{I}=(S_I,\sigma_I)} is a structure \m{\mathbf{S_I}=(D_I,F_I,P_I)} and a variable assignment 
\m{\sigma_I : \Vs{\sig} \rightarrow D_I}.

\subsection*{Semantics}\label{section:preliminaries:semantics}
For a term $\term{t}$ and an interpretation \m{\mathbf{I}=(S_I,\sigma_I)} we denote by \m{\den{t}{I} \in \mathrm{D_I}} the interpretation of \term{t} in \m{\mathbf{I}} in the standard way, as defined below:\\
$
\begin{array}{lll}
	\den{\fa{f}{t}} {I} & \triangleq & \den{\term{f}}{I}(\den{\tup{t}}{I}) \\
	\den{x}         {I} & \triangleq & \m{\sigma_I(x)} \\
	\den{t=s}       {I} & \triangleq & \den{t}{I}=\den{s}{I}  \\
	\den{\fa{P}{t}} {I} & \triangleq & \m{\den{\tup{t}}{I} \in \den{P}{I}} \\
	\den{\lnot a}   {I} & \triangleq & \lnot \den{a}{I} \\
	\den{C}         {I} & \triangleq & \m{ \bigvee\limits_{A \in C} \den{A}{I}} \\
\end{array}
$

\noindent
We extend  $\den{\cdot}{I}$ point-wise to tuples, and use $\den{f}{I}$ for $\m{\mathrm{F_I}(f)}$ and $\den{P}{I}$ for $\m{\mathrm{P_I}(P)}$.\\
For the ground fragment, where $\Vs{\sig}=\emptyset$, an interpretation is essentially a structure.\\
Satisfiability in the ground fragment of this language is decidable and NP-complete:\\
We can reduce (in linear time and space) a propositional problem to our fragment by replacing each propositional atom $\term{A}$ with the GFOLE atom $\term{f_A()=T}$ where $\term{f_A}$ is a constant function and $\term{T}$ is a specially designated (fresh) true symbol.
In the other direction we have a polynomial reduction using the Ackermann transformation - basically, encode every term $\fa{f}{t}$ by a fresh variable  $\m{v_{\fa{f}{t}}}$, for any pair of terms \fa{f}{t}, \fa{f}{s} with the same function symbol we add the clause
$ \bigvee\limits_{\m{i}} \m{\m{v}_{t_i} \neq \m{v}_{s_i}} \lor \m{v}_{\fa{f}{t}=\fa{f}{s}}$ to encode congruence closure.\\
We are left with a set of CNF clauses over the fresh variables, with no non-constant function symbols, of at most square size.
We replace each atom $\m{v=u}$ by a propositional atom $\term{A_{v=u}}$, and for each triple of constants $\m{a,b,c}$ we add the clause 
$\m{\lnot A_{a=b} \lor \lnot A_{b=c} \lor A_{a=c}}$ to encode transitivity - we end up with an equi-satisfiable propositional set of clauses.
There are more efficient transformations that achieve the same, however we are interested mostly in the fact that the reduction is polynomial, as the best known algorithm for propositional CNF is exponential.


\subsection*{Terms}
\bigskip

\noindent
\textbf{Substitutions}\\
A \newdef{substitution} on a signature $\m{\Sigma}$ is a total function $\m{\sigma : \m{X} \rightarrow \Ts{\sig}}$ extended to terms as follows:\\
$
\begin{array}{lll}
\m{x\sigma}         & \triangleq & \m{\sigma(x)}\\
\m{\fa{f}{t}\sigma} & \triangleq & \m{f(\tup{t}\sigma)}\\
\m{\tup{t}\sigma_i} & \triangleq & \m{t_i\sigma}\\
\end{array}
$


\smallskip

\noindent
The substitution $\newdef{\m{[x \mapsto t]}}$ is defined as \\
\m{[x \mapsto t](y) \triangleq \ite{x\equiv y}{t}{y}}\\
A \newdef{composition of substitutions}, denoted {$\newdef{\m{\sigma_1\sigma_2}}$}, is defined as:\\
\m{(\sigma_1\sigma_2)(x) \triangleq \sigma_1(\sigma_2(x))}\\
We denote the set of substitutions for a signature $\Sigma{}$ as $\newdef{\subs{\Sigma}}$.

\bigskip

\noindent
\textbf{Term positions}\\
We denote by $\emptySeq$ the empty (integer) sequence and by \lstinline{i.s} the sequence constructed by prepending the integer \lstinline{i} before the sequence \lstinline{s}.\\
A \newdef{position} is a sequence of integers.\\
We denote a \newdef{sub-term of the term $\term{u}$ at position p} by \newdef{\restrict{u}{p}} - which is (partially) defined recursively as follows:\\
$
\begin{array}{lll}
	\termAt{u}{\emptySeq}    & \triangleq & \term{u} \\
	\termAt{f(\tup{r})}{i.s} & \triangleq & \termAt{\tupAt{r}{i}}{s}
\end{array}
$

\noindent
For example, for  $\term{t=f(g(a),h(b,g(a)))}$, \\
$\termAt{f(g(a),h(b,g(a)))}{\emptySeq}=\term{f(g(a),h(b,g(a)))}$, \\
$\termAt{f(g(a),h(b,g(a)))}{0}        =\termAt{f(g(a),h(b,g(a)))}{1.1}=\term{g(a)}$, \\
$\termAt{f(g(a),h(b,g(a)))}{0.0}      =\termAt{f(g(a),h(b,g(a)))}{1.1.0}=\term{a}$ \\
etc.

\noindent
For a term $\m{t}$ the set $\newdef{\m{\poss{t}}}$ is the set of all positions of $\m{t}$ defined as follows:\\
$
\begin{array}{lll}
	\poss{\fa{f}{s}} & \triangleq & \s{\emptySeq} \cup \s{i.p \mid p \in \poss{s_i}} \\
\end{array}
$\\
We also use all positions of a term \m{s} in a term \m{t}:\\
$
\begin{array}{lll}
	\posss{\fa{f}{s}}{t} & \triangleq & \s{\emptySeq \mid \fa{f}{s}\equiv t} \cup \s{i.p \mid p \in \posss{s_i}{t}} \\
\end{array}
$\\
Two positions $\m{p,q}$ are $\newdef{\mbox{disjoint}}$, denoted $\newdef{\disj{p}{q}}$, iff they do not share any common sub-term - formally:\\
\m{\disj{p}{q} \equivdef \exists i,j \cdot (p=i.p' \land q=j.q' \land (i\neq j \lor \disj{p'}{q'}))}\\
The set $\posss{t}{s}$ is pairwise disjoint.\\
By $\termRepAt{u}{t}{p}$ we denote a \newdef{replacement} of \termAt{u}{p} by $\term{t}$ at the position $\term{p}$ in term $\term{u}$ - formally:\\
$
\begin{array}{lll}
	\termRepAt{s        }{t}{\emptySeq} & \triangleq & \term{t} \\
	\termRepAt{\fa{f}{s}}{t}{i.p      } & \triangleq & \term{f}\left(\tupRepAt{s}{i}{\termRepAt{(\tupAt{s}{i})}{t}{p}}\right) \\
\end{array}
$\\
We extend this notion to simultaneous replacement on a pairwise disjoint set of positions \m{P}:\\
$
\begin{array}{lll}
	\termRepAt{s        }{t}{\emptyset}     & \triangleq & \term{s} \\
	\termRepAt{s        }{t}{\s{\emptySeq}} & \triangleq & \term{t} \\
	\termRepAt{\tup{s}  }{t}{P            } & \triangleq & \m{i \mapsto \termRepAt{s_i}{t}{\s{p \mid i.p \in P}}} \\
	\termRepAt{\fa{f}{s}}{t}{P            } & \triangleq & \m{f(\termRepAt{\tup{s}  }{t}{P            })} \\
\end{array}
$

\bigskip


\noindent
\textbf{Sub-terms}\\
A term $\m{s}$ is a \newdef{proper sub-term} of a term \m{t}, denoted \newdef{\m{s \lhd t}}, if:\\
\m{s \lhd t \equivdef \exists p \neq \emptySeq \cdot s = \termAt{t}{p} }\\
And a non-proper sub-term if:\\
\m{s \unlhd t \equivdef s=t \lor s \lhd t }\\
We extend the sub-term relation to tuples, literals, clauses and sets of clauses:\\
$
\begin{array}{lll}
\m{s \lhd \tup{t}}    & \equivdef & \m{\exists i \mid s \lhd t_i}\\
\m{s \lhd u \bowtie v} & \equivdef & \m{s \lhd (u,v)}\\
\m{s \lhd \fa{P}{t}}   & \equivdef & \m{s \lhd \tup{t}}\\
\m{s \lhd C}           & \equivdef & \m{\exists l \in C \cdot s \lhd l}\\
\end{array}
$




\subsubsection*{Orders}\label{section:preliminaties:ordering}
For a set \m{S}, a \newdef{strict partial order} \m{\succ \in S^2} on \m{S} is a binary relation on \m{S} satisfying:
\begin{itemize}
	\item Irreflexive: \m{\forall x \in S \cdot x \not\succ x}
	\item Transitivity: \m{\forall x,y,z \in S \cdot x \succ y \land y \succ z \Rightarrow x \succ z}
	\item Asymmetric: \m{\forall x,y \in S \cdot x\succ y \Rightarrow y \not\succ x }
\end{itemize}
For any strict partial order \m{\succ}, the corresponding \newdef{reflexive closure} \m{\succeq} is defined as:\\
\m{\forall x,y \in S \cdot x\succeq y \Leftrightarrow (x=y \lor x \succ y)}\\
A \newdef{strict total order} is a strict partial order where \\
\m{\forall x,y \in S  \cdot x=y \lor x \succ y \lor y \succ x} \\
and correspondingly a \newdef{total reflexive closure} satisfies:\\
\m{\forall x,y \in S  \cdot x \succeq y \lor y \succeq x}\\
A \newdef{well founded} strict partial order $\m{\succ}$ on $\m{S}$ has no infinite descending chains - formally:\\
\m{\lnot \exists f : \mathbb{N} \rightarrow S \cdot \forall i \in \mathbb{N} \cdot f(i) \succ f(i+1)} \\
An equivalent definition is that each subset has a minimum:\\
\m{\forall S' \subseteq S \cdot S' \neq \emptyset \Rightarrow \exists t \in S' \cdot \forall s \in S' \cdot s \succeq t}

\subsubsection*{Term Orderings}
A \newdef{simplification ordering} $\m{\succ}$ on a term algebra $\Ts{\sig}$ is a strict partial order on $\Ts{\sig}$ that satisfies:
\begin{itemize}
	\item Compatible with contexts (monotonic):\\
		\m{\forall s,t,c \in \Ts{\sig},p \cdot s \succ t \Rightarrow \termRepAt{c}{s}{p} \succ \termRepAt{c}{t}{p}}
	\item Stable under substitution: \\
		\m{\forall s,t \in \Ts{\sig},\sigma \in \subs{\Sigma} \cdot s \succ t \Rightarrow s\sigma \succ t\sigma}
	\item Sub-term compatible: \\
		\m{\rhd \subseteq \succ}
\end{itemize}
A \newdef{reduction ordering} is a well founded simplification ordering.\\
The lexicographic extension of an ordering on S to an ordering on $\m{S^n}$ for $\m{n>1}$ is defined as follows:\\
\m{\tup{s} \succ \tup{t} \equivdef \exists i \geq 0 \cdot s_i \succ t_i \land (\forall j<i \cdot s_j=t_j)}\\
For an ordering $\m{\succ}$ on S, we use the multiset extension of $\m{\succ}$, for a pair of finite multisets on $\m{S}, \m{m,n}$:\\
\m{m \succ n \equivdef \forall x \in S \cdot m(x) > n(x) \lor \exists y \succ x \cdot m(y) \succ n(y) }.\\
For a multiset \m{m} and an element \m{x \in S} we use:\\
\m{x \succ m \equivdef \forall y \in S \mid m(y)=0 \lor x \succ y}
%The multiset for an atom \m{s=t} is \m{s \mapsto 1,t \mapsto 1} and for \m{\fa{P}{t}} \s{

\noindent
We use a form of \emph{transfinite Knuth Bendix order} (\cite{WinklerZanklMiddeldorp12},\cite{KovacsMoserVoronkov11}).\\
We use $\newdef{\ords}$ for the set of ordinal numbers and $\newdef{\bigoplus,\bigotimes}$ for natural addition and multiplication on ordinals, respectively.

\noindent
The \newdef{Transfinite Knuth Bendix term ordering}, \newdef{\m{\succ_{tkbo}}} has two parameters:\\
A strict partial (potentially total) ordering \m{\succ} on the signature \m{F_\Sigma} (sometimes called a \emph{precedence}).\\
A $\newdef{\mbox{weight function}}$, $\newdef{\m{w:F_\Sigma \cup X_\Sigma \rightarrow \ords}}$ that satisfies:\\
\m{\forall f \in F_\Sigma \cdot w(f) > 0} and  \m{\forall x \in X_\Sigma \cdot w(x) = 1}.\\
Unless otherwise noted we will use a $\m{w}$ that satisfied\\
\m{\forall f \cdot \arity{f}>0  \Rightarrow \exists m \in \mathbb{N} \cdot w(f) = \omega\cdot m + 1}\\
\m{\forall f \cdot \arity{f}=0  \Rightarrow w(f) = 1}\\
that is, the function maps all constants to 1 and all non-constants only to direct successors of limit ordinals less than $\m{\omega^\omega}$. 
Note that it is not required that the precedence on function symbols agrees with the ordinal order on their weights.

\noindent
The $\newdef{\mbox{weight of a term}}$, $\newdef{\m{w(t)}}$, is defined recursively as:\\
\m{w(\fa{f}{s}) \triangleq w(f) + \bigoplus\limits_i w(s_i)}\\
For literals:\\
\m{w(s=t) \triangleq  w(s) \oplus w(t)}\\
\m{w(\fa{P}{t}) \triangleq  w(P) \oplus w(\tup{t})}

\noindent
We define the \newdef{multiset of variables of a term} \m{t}, \newdef{\m{\Vars{t}}}, recursively as follows:\\
$ 
\begin{array}{lll}
	\Vars{x}         & \triangleq & [x \mapsto 1]\\
	\Vars{\fa{f}{s}} & \triangleq & \bigcup\limits_\m{i} \Vars{s_i}\\
\end{array} 
$

\bigskip

\noindent
The transfinite Knuth Bendix ordering (tkbo) for terms we use is defined as follows:\\
\m{s \succ t} iff \m{\Vars{s} \supseteq \Vars{t}} and
\begin{itemize}
	\item \m{w(s) > w(t)} or
	\item \m{w(s) = w(t), s\equiv\fa{f}{s}, t\equiv\fa{g}{t}} and
		\subitem \m{f \succ g} or
		\subitem \m{f \equiv g} and \m{\tup{s} \succ \tup{t}}
\end{itemize}

\noindent
In order to extend the definition to literals and clauses, we extend the weight function to predicate symbols, and assume we have a precedence (total order) $\m{\succ}$ also on $\m{P_{\sigma} \cup \s{=}}$ - predicate symbols including the equality symbol.\\
This extension is total if $\succ$ is.\\
tkbo for literals is:\\
\m{l \succ l'} iff \m{\Vars{l} \supseteq \Vars{l'}} and
\begin{itemize}
	\item \m{w(l) > w(l')} or
	\item \m{w(l) = w(l')} where \m{l\equiv[\lnot]\fa{P}{s}, l'\equiv[\lnot]\fa{Q}{t}} and
	\begin{itemize}
		\item l is negative and \m{l\equiv \lnot l'} or
		\item \m{P \succ Q} or
		\item \m{P \equiv Q} and \m{\tup{s} \succ \tup{t}}
	\end{itemize}
\end{itemize}
In the above we used $\fa{P}{t}$ to denote also $\m{s=t}$ as $\m{=(s,t)}$.\\
We treat each clause as a multiset of literals and then use the multiset extension of $\succ$.\\
tkbo is total on ground terms.\\
tkbo also has the desirable property that it is \newdef{separating} for constants - 
that is, given a constant indexing function \m{ci(c):\consts{\Sigma} \rightarrow \mathbb{N}} we can assign \\
\m{\forall c \in \mathbf{const} \cdot w(c) = \omega\cdot \mathbf{ci}(c)+1}\\
Where, for any two term \m{s,t}, if the maximal constant index of \m{s} is greater than that of \m{t} then \m{s \succ t} regardless of size.
This property is important for completeness under scoping (for the ground fragment) as in ~\cite{KovacsVoronkov09},~\cite{McMillan08}.

\subsection*{Superposition}\label{section:preliminaries:superposition}
We have chosen to use superposition (\cite{BachmairGanzingerSuperposition}) as the underlying logical calculus.
The motivation for this choice is that superposition is a complete semi-decision procedure for FOLE (as opposed to many of the quantifier instantiation schemes used in SMT solvers), and it is known to be efficient in handling equalities in the presence of quantifiers.
An additional motivation is that it is possible to define fragments of superposition that, while not complete, 
have polynomial complexity. Most of our technique is also relevant for some other calculi, as we discuss in the relevant sections.
We present here only the ground fragment of superposition, and present the full superposition calculus when we discuss quantification.

\noindent
The main ideas of superposition can be described as follows:
The propositional part of superposition is based on ordered resolution for propositional logic.
Roughly, the main idea is to order the literals in each clause (and hence clauses by the multiset extension of the ordering) and for each pair of clauses with opposing maximal literals derive a smaller clause that encodes a case-split on the maximal literal.\\
Clauses implied by smaller clauses in the set are called redundant. When all case-split clauses have been derived and the empty clause has not been derived, a model of the set of clauses is the set of positive maximal literals of non-redundant clauses.

\noindent
For example, consider the following clause set (maximal literals are underlined):\\
$\s{C \lor \underline{A}, \underline{\lnot A} \lor D}$ \\
where $\m{A,\lnot A}$ are maximal in their respective clauses and do not occur in either of $\m{C,D}$.\\
We assume $\m{D \succ C}$ and hence $\m{\lnot A \lor D \succ C \lor A}$.\\
The clause $\m{C\lor D}$ derived by resolution on the maximal literal encodes the case split on $\m{A}$.\\
The new clause is smaller than both premises by the definition of multiset orderings.\\
The maximal literal $\m{l}$ of the new clause satisfies $\m{A \succ l}$ and $\m{l \in D}$ by the ordering.\\
Hence the model for the set of clauses is $\s{l,A}$.\\
The reason that we say the clause $\m{C \lor D}$ encodes the case-split is that if we had another singleton clause $\m{l'}$ in the set where $\m{l' \in C}$ then $\m{C \lor D,C \lor A}$ are redundant and hence the set of positive maximal literals of non-redundant clauses, $\s{l'}$, is a model.

\noindent
The complication added by equality is that an equality literal can conflict with a larger literal - for example, $\m{a=b}$ conflicts with the larger $\m{f(a)\neq f(b)}$, and the set $\m{a=b,b=c}$ conflicts with the larger $\m{f(a)\neq f(c)}$.\\
Superposition uses unfailing Knuth Bendix completion to ensure that in a clause set saturated for superposition the set of maximal positive literals of non-redundant clauses forms a convergent term rewrite system, and that all maximal terms of maximal literals of non-redundant clauses are reduced to their normal form by the  convergent term rewrite system defined by the maximal positive literals of all smaller clauses.\\
Resolution is replaced with term rewriting by maximal positive literals in order to ensure that the set of maximal literals of non-redundant clauses is a model.\\
The ground superposition calculus is shown in figure \ref{fig_ground_superposition}, we discuss the full calculus when we discuss quantification. The full calculus was shown sound and complete in \cite{BachmairGanzinger94}.

\bigskip

\noindent
A simple example of ground superposition (maximal terms are underlined):\\
For the set $\s{a=\underline{b},b=\underline{c},f(a)\neq \underline{f(c)}}$ with the ordering\\
$\m{f(c)\succ f(b) \succ f(a) \succ c \succ b \succ a}$ \\
superposition allows us to rewrite the term $\m{f(c)}$ to a smaller term using the equation $\m{c=b}$ - to get:\\
$\m{f(a) \neq \underline{f(b)}}$\\
And then, rewriting with the first clause:\\
$\m{f(a) \neq f(a)}$\\
And then\\
\emptyClause

\bigskip

\begin{figure}
$
\begin{array}[c]{llll}
%\vspace{10pt}
\mathrm{res_{=}} &\vcenter{\infer[]{\m{C       }                               }{\m{C \lor \underline{s\neq s}}                   }} & 
\parbox[c][1.8cm]{5cm}{}
\\
%\vspace{10pt}
\mathrm{sup_{=}} &\vcenter{\infer[]{\m{C \lor \termRepAt{s}{r}{p} =    t \lor D}}{\m{C \lor \underline{l}=r} & \m{\underline{s} =    t \lor D}}} & 
\parbox[c][1.8cm]{5cm}{
	\m{\sci{1}l = \termAt{s}{p}}\\
	\m{\sci{2}l \succ r,\sci{3}l=r \succ C}\\
	\m{\sci{4}s \succ t,\sci{5}s=t \succ D}\\
	\m{\sci{6}s=t \succ l=r}}\\
%\parbox[c][2cm]{4cm}{\sci{1}\m{l \succ r\sci{2}l=r \succ C}\\\sci{3}\m{s \succ t}\sci{4}\m{s=t \succ D}\\\sci{5}\m{s=t \succ l=r}}\\
%\vspace{10pt}
\mathrm{sup_{\neq}} &\vcenter{\infer[]{\m{C \lor \termRepAt{s}{r}{p} \neq t \lor D}}{\m{C \lor \underline{l}=r} & \m{\underline{s} \neq t \lor D}}} & 
\parbox[c][1.8cm]{5cm}{
	\m{\sci{1}l = \termAt{s}{p}}\\
	\m{\sci{2}l \succ r,\sci{3}l=r \succ C}\\
	\m{\sci{4}s \succ t,\sci{5}s=t \succ D}}\\
%\parbox[c][2cm]{4cm}{\m{l \succ r,l=r \succ C}\\\m{s \succ t,s \neq t \succ D}}\\
%\vspace{10pt}
\mathrm{fact} & \vcenter{\infer[]{\m{C \lor t \neq r \lor l=r }                }{\m{C \lor l = t \lor \underline{l} = r}}} & 
\parbox[c][1.8cm]{5cm}{
	\m{\sci{1}l \succ r,\sci{2}r \succ t}\\
	\m{\sci{3}l=r \succ C}}\\
%\parbox[c][2cm]{4cm}{\m{s \succ t,t \succ r}\\\m{s=t \succ C}}
%\vspace{10pt}
\end{array}
$
\caption{The ground superposition calculus \SPG\\
$\succ$ is a reduction ordering.\\
The numbered conditions on the right are the side conditions of each inference rule.\\
The calculus combines ordered resolution with unfailing Knuth Bendix completion.\\
Equality resolution ($\m{res_{=}}$) allows the elimination of maximal false literals.\\
Positive superposition ($\m{sup_{=}}$) ensures that the set of maximal positive literals of non-redundant clauses is a convergent rewrite system.\\
Negative superposition ($\m{sup_{\neq}}$) allows rewriting maximal dis-equalities by the term-rewrite system defined by maximal positive literals, and together with equality resolution is a generalization of ordered resolution for equality.\\
Equality factoring ($\m{fact}$) is a version of ordered factorting.
}
\label{fig_ground_superposition}
\end{figure}


\noindent
We use the notation $\newdef{\m{S \vdash_{X} C}}$ to denote that the clause $\m{C}$ is derivable in the calculus $\m{X}$ from the set of clauses $\m{S}$. When the calculus is clear from the context we use $\m{S \vdash C}$. In this section we only refer to the ground superposition calculus and hence we shorten $\m{\vdash_{\SPG}}$ to $\m{\vdash}$.

\subsubsection*{Redundancy elimination}
The superposition calculus is complete even when redundant clauses are eliminated according to a certain redundancy criterion.\\
The full superposition calculus was shown complete under the following redundancy criterion (here only a variant for ground clauses):\\
For a finite set of clauses $\m{S}$ and clause $\m{D}$, 
if $\m{S,D \vdash \emptyClause}$ and for some $\m{S' \subseteq S}$, $\m{S' \models D}$ and $\m{D \succ S'}$ (D is greater than all members of $\m{S'}$) then $\m{S \vdash \emptyClause}$ - that is, $\m{D}$ is redundant.

For the ground superposition calculus we use the simplifying inference rules shown in figure \ref{fig_superposition_simp}, all of which satisfy the above criterion. 
Most of the simplification rules are standard, and $\m{simp_{res},simp_{res2}}$ are chosen in order to handle the clauses that occur at join points in the program - we discuss these later.

\begin{figure}
$
\begin{array}[c]{llll}
%\vspace{10pt}
\m{unit} & \vcenter{\infer[]{\m{C}                            }{\m{\lnot A}  & \cancel{\m{C \lor A}}}} & \parbox[c][1.0cm]{3cm}{}\\
%\vspace{10pt}
\m{taut} & \vcenter{\infer[]{\m{}                             }{\cancel{\m{C \lor A \lor \lnot A}}}} & \parbox[c][1.0cm]{3cm}{}\\
%\vspace{10pt}
\m{taut_{=}} & \vcenter{\infer[]{\m{}                             }{\cancel{\m{C \lor s=s}}}} & \parbox[c][1.0cm]{3cm}{}\\
%\vspace{10pt}
\m{sub} & \vcenter{\infer[]  {\m{}                             }{\m{C} & \cancel{\m{C \lor D}}}} & \parbox[c][1.0cm]{3cm}{}\\
%\vspace{10pt}
\m{simp_{res}} & \vcenter{\infer[]{\m{C}                      }{\cancel{\m{C \lor A }} & \cancel{\m{C \lor \lnot A}}}} & \parbox[c][1.2cm]{3cm}{}\\
%\vspace{10pt}
\m{simp_{res2}} & \vcenter{\infer[]{\m{C \lor D}              }{\m{C \lor A } & \cancel{\m{C \lor D \lor \lnot A}}}} & \parbox[c][1.2cm]{3cm}{}\\
\m{simp_{=}} & \vcenter{\infer[]{\m{\termRepAt{C}{r}{p}}}{\m{l=r} & \cancel{\m{C}}}}   &
\parbox[c][1.2cm]{3cm}{\m{l=\termAt{C}{p}}\\\m{l \succ r}\\\m{C \succ l=r}}\\
\end{array}
$
\caption{simplification rules\\
$\cancel{\m{C}}$ denotes that the premise $\m{C}$ is redundant after the addition of the conclusion to the clause-set and hence can be removed.
}
\label{fig_superposition_simp}
\end{figure}

\subsection*{Congruence closure}
While superposition can decide the ground equality fragment, some techniques based on congruence closure are more efficient, 
and specifically efficient join algorithms have been developed for congruence closure.\\
We use two variants of the transitive reflexive congruence closure calculus \m{\mathbf{CC}} for unit ground\\ (dis)equalities.\\
The reason we mention two variants is that the first describes the operation of the graph structure we use and the second follows directly from the definition of congruence closure, and hence is used for completeness proofs.

\bigskip

\noindent
The first version $\m{\mathbf{CC}}$ is described in figure \ref{calculus_CC}. 
This calculus has a standard transitivity axiom and a version of equality resolution, 
but the less standard part is the congruence closure rule. 
This rule only allows instances of the general congruence closure rule if one of the terms in the conclusion already occurs in some clause.
The reason we use this version is that it describes the operation of a congruence closure (CC) graph (in the sense that the graph represents a set of clauses saturated w.r.t. the calculus) - performing congruence closure in a CC graph does not introduce new equivalence classes, although it may introduce new terms.

\begin{figure}
$
\begin{array}{lll}
%	\vspace{10pt}
	\m{tra_{\bowtie}} & \vcenter{\infer[]{\m{s \bowtie t}}{\m{s=u,u \bowtie t}}} & \parbox[c][1.5cm]{2cm}{}\\
%	\vspace{10pt}
	\m{res}           & \vcenter{\infer[]{\emptyClause }{\m{s=t, s \neq t}}} & \parbox[c][1.5cm]{2cm}{}\\
	\m{con}           & \vcenter{\infer[]{\fa{f}{s}=\fa{f}{t} }{\m{C} & \tup{s=t} }} &
\parbox[c][1.5cm]{2cm}{\m{\fa{f}{s} \lhd C}}\\
%	\vspace{10pt}
\end{array}
$
\caption{The $\m{\mathbf{CC}}$ calculus\\
We denote by $\m{\tup{s=t}}$ for two tuples $\m{\tup{s},\tup{t}}$ of the same arity the set of non-trivial equalities between corresponding elements of the tuples - formally:\\
$\m{\tup{s=t}}$ is the set $\s{s_i = t_i \mid i \in 0..\size{s}-1 \land s_i \not\equiv t_i}$.\\
The rule $\m{tra_{\bowtie}}$ is the transitivity rule.\\
The rule $\m{res}$ is similar to equality resolution.\\
The rule $\m{con}$ encodes standard congruence closure, 
except that the side condition $\m{\fa{f}{s} \lhd C}$, where C is any (dis)equality, ensures that no new equivalence classes are introduced in any derivation.
}
\label{calculus_CC}
\end{figure}

\noindent
For a set of unit ground equality clauses $\m{S}$, $\CC{S}$ is the closure of $\m{S}$ w.r.t. $\m{\mathbf{CC}}$ - we use a dedicated data structure to represent $\CC{S}$, described later. 

%\noindent
%The second version $\mathbf{CC_R}$ differs in that it does not introduce any new \emph{terms} (as opposed to no new equivalence cla, it differs from $\m{\mathbf{CC}}$ only in the congruence closure rule, whose version is described in figure \ref{calculus_CC_R}.
%
%\begin{figure}
%$
%\begin{array}{lllll}
%%	\vspace{10pt}
	%\m{con_R} & \vcenter{\infer[]{\fa{f}{s}=\fa{f}{t} }{\m{C} & \m{D} & \tup{s=t} }} & 
	%\parbox[c][1.5cm]{2cm}{\m{\fa{f}{s} \lhd C}\\\m{\fa{f}{t} \lhd D}}
%\end{array}
%$
%\caption{The \m{\mathbf{CC_R}} calculus\\
%The side condition ensures that no new terms are introduced in any derivation.
%}
%\label{calculus_CC_R}
%\end{figure}

\noindent
The second version, $\mathbf{CC_I}$, follows directly the definition of congruence closure. It differs only in the congruence closure rule , as described in figure \ref{calculus_CC_I}. We use this version in completeness proofs - a set of (dis)equalities is inconsistent iff it has a refutation in this calculus.
%We will discuss later how to determine the side condition for generating implied dis-equalities effectively, we only note here that for each \m{u \neq v} both \m{u,v} are sub-terms of \m{s,t} (at least one is proper) and for each of \m{s,t} at least one of \m{u,v} is a sub-term.


\begin{figure}
$
\begin{array}{llll}
	\vspace{10pt}
	\m{con_I} &
	\vcenter{\infer[]{\fa{f}{s}=\fa{f}{t} }{\tup{s}=\tup{t} }} & 
	\parbox[c][1.1cm]{4cm}{} \\
%	\vspace{10pt}
	%\m{das_I} &
	%\vcenter{\infer[]{\m{u \neq v}}{\m{s \neq t}}} & 
	%\parbox[c][1.1cm]{4cm}{\m{s \not\equiv t}\\\m{u \neq v \in das_u^{max}(s,t)}}\\
\end{array}
$
\caption{The \m{\mathbf{CC_I}} calculus\\
The rule \m{con_I} follows the definition of congruence closure.
%This calculus is complete for non-tautological consequences.\\
%$\m{das_u^{max}(s,t)}$ is the set containing the maximal (by entailment) dis-equalities implied by $\m{s\neq t}$.\\
%This set is the set of maximal unit disagreement sets of $\m{s \not\equiv t}$ - formally:\\
%$\m{das_u^{max}(s,t) \triangleq \s{ u \neq v \in das_u(s,t) \mid \forall u' \neq v' \in das_u(s,t) \cdot u' \neq v' \not\models u \neq v}}$\\
%And the unit disagreement sets are defined as:\\
%$\m{das_u(s,t) \triangleq \s{ u \neq v \mid u \not\equiv v \land s \neq t \models u \neq v }}$
}
\label{calculus_CC_I}
\end{figure}

\bigskip

\section*{Programs}\label{section:preliminaries_programs}
We assume as input a program in the Boogie (\cite{BarnettCDJL05}) intermediate verification language (or a similar IVL) that has been generated as a verification condition (VC) for some \newdef{source program} (and potentially some annotation).
We assume a low level Boogie representation that includes a DAG-shaped CFG (loops and method calls are removed using annotations) and the only statements are \lstinline|assume| and \lstinline|assert| (a passified program as described in \cite{Leino:2005:EWP:1066417.1710882}). CFG-nodes in the input represent basic blocks of the Boogie program, and often correspond to basic blocks of the source program.\\
We modify this input slightly by splitting each CFG-node at each assertion statement \lstinline|assert e| that occurs in it and replacing the assertion with an outgoing edge to a new leaf node with the statement \lstinline|assume $\lnot$e|. 
Now each CFG-node has only \lstinline|assume| statements and the order of statements within each CFG-node is unimportant - hence each CFG-node can be treated as a set of FOLE formulae. 
We convert this set of formulae per CFG-node (including the negated assertion nodes) to CNF form and now each CFG node is associated with a set of clauses. \\
For example, the source program in figure \ref{CFG_source_program} may be converted to the Boogie-style program \ref{CFG_Boogie} and is further converted to our representation \ref{CFG_ours}.
\begin{figure}
\begin{lstlisting}
$\m{n_0}$:
x:=0
y:=10
while (x<10)
	invariant x>=0 && x<=10 && x+y==10
	$\m{n_1}$:
	x:=x+1
	y:=y-1
$\m{n_2}$:
assert x+y<20
\end{lstlisting}
\caption{Example for VC encoding - source program}
\label{CFG_source_program}
\end{figure}

\begin{figure}
\begin{lstlisting}
$\m{n_0}$:
assume x$_0$=0
assume y$_0$=10
if (*)
	$\m{n_1}$:
	//loop head - assume loop condition
	assume x$_1$<10
	//assume loop invariant
	assume x$_1$>=0 && x$_1$<=10 && x$_1$+y$_1$==10
	//loop body
	assume x$_2$=x$_1$+1
	assume y$_2$=y$_1$-1
	//assert loop invariant (on current DSA versions)
	assert x$_2$>=0 && x$_2$<=10 && x$_2$+y$_2$==10
	//back edge is removed
	assume false
$\m{n_2}$:
//new DSA versions
//assume negated loop condition
assume !x$_3$<10
//assume loop invariant
assume x$_3$>=0 && x$_3$<=10 && x$_3$+y$_3$==10
assert x$_3$+y$_3$<20
\end{lstlisting}
\caption{Example for VC encoding - Boogie program\\
All variables have been split to DSA versions.\\
All assignments are converted to \lstinline|assume| statementes.\\
The loop return edge has been cut and the body begins with a fresh version for each variable,
an \lstinline|assume| of the invariant and negation of the loop condition, 
and ends with an \lstinline|assert| of the loop invariant on the latest DSA versions.\\
The code after the loop also uses a fresh DSA version of all variables and \lstinline|assume|s the invariant.
(we did not detail a modifies clause for the loop)
}
\label{CFG_Boogie}
\end{figure}

\begin{figure}
\begin{lstlisting}
assume x$_0$=0
assume y$_0$=10
if (*)
	$\m{n_1}$:
	assume x$_1$<10
	assume x$_1$>=0 
	assume x$_1$<=10
	assume x$_1$+y$_1$==10
	assume x$_2$=x$_1$+1
	assume y$_2$=y$_1$-1
	if (*)
		$\m{n_{1a}}$: //introduced assertion node
		assume $\lnot$x$_2$>=0 $\lor$ $\lnot$x$_2$<=10 $\lor$ $\lnot$x$_2$+y$_2$==10
		assert false
	assume false
else
	$\m{n_2}$: 
	assume !x$_3$<10
	assume x$_3$>=0 && x$_3$<=10 && x$_3$+y$_3$==10
	if (*)
		$\m{n_{2a}}$: //introduced assertion node
		assume $\lnot$x$_3$+y$_3$<20
		assert false
\end{lstlisting}
\caption{Example for VC encoding - our encoding\\
Showing that the post-states of both $\m{n_{1a},n_{2a}}$ is infeasible proves the Boogie program and hence the source program.
}
\label{CFG_ours}
\end{figure}

%The idea is that, for each CFG node $\m{n}$, for each path from the CFG-root to n, the set of all clauses on all CFG-nodes on the path is unsatisfiable iff the post-state of the statement represented at $\m{n}$ is unreachable in the Boogie program.

\noindent
We refer to the IVL program after our transformations as the \newdef{program} and the source language program as the source program. 

\subsection*{Structure}
The structure of our program is as follows:\\
A control flow graph - \newdef{CFG} - which is a directed acyclic graph with one root (the program entry point).\\
The leaf nodes of the CFG are the \newdef{goal nodes} - introduced per assertion. The goal of verification is to show them infeasible.\\
Each CFG-node $\m{n}$ is associated with a set of clauses - \newdef{\clauses{n}}.\\ 
The clauses at each non-leaf CFG-node represent an encoding of the transition relation of the original program, or some instrumentation used by the verification condition generator to generate the IVL program.\\
The clauses at each leaf node represent the negation of an assertion generated for the VC, as described above.

\noindent
We use the following functions to refer to the CFG structure - for a given CFG-node $\m{n}$:\\
\newdef{\succs{n},\succsto{n},\succst{n}} are the direct, transitive and reflexive-transitive successors of $\m{n}$, respectively.\\
Similarly, \newdef{\preds{n},\predsto{n},\predst{n}} are the corresponding sets for predecessors.

\subsubsection*{CFG paths}
For a program CFG $\m{G}$, a directed $\newdef{\m{path}}$ $\m{P}$ in the $\m{G}$ is a (possibly empty) sequence of nodes s.t. \\
$\m{\forall 0 \leq i < \size{P} \cdot P_{i+1} \in \succs{P_i}}$.\\
$\size{P}$ is the length of the path.\\
We use $\newdef{\paths{G}{}}$ for the set of all directed paths in $\m{G}$, starting at any node and ending at any transitive successor of the node, including one and zero length paths.\\
For a node n and transitive predecessor $\m{p \in \predst{n}}$,
$\newdef{\paths{p}{n}}$ are all the paths in $\m{G}$ that start at $\m{p}$ and end at $\m{n}$, including the case $\m{n=p}$.\\
$\newdef{\paths{n}{}}$ is short for $\paths{root}{n}$.

\noindent
For a path $\m{P}$, the set of all clauses in all nodes on the paths is denoted by $\newdef{\clauses{P}}$ - formally:\\
$\m{ \clauses{P} \triangleq \bigcup\limits_{n \in P} \clauses{n}}$.
%Note that the program point after the last statement in a CFG-node is reachable on the path $\m{P}$ iff $\clauses{P}$ is consistent.

\noindent
For a set of clauses $\m{S}$ we use $\newdef{\Eqs{S}}$ for the subset $\m{S}$ that is unit equalities, dis-equalities and the empty clause - formally:\\
$\m{\Eqs{S} \triangleq (S \cap \s{\emptyClause}) \cup \s{u \bowtie v \mid u \bowtie v \in S}}$


\subsection*{Semantics}
\subsubsection*{Traces}
A \newdef{trace} is a pair $\m{(P,M)}$ where $\m{P}$ is a path from the root-node to some node $\m{P_{end}}$ and $\m{M}$ is a model for the signature of the clauses of the program, s.t. for each node on the path, the clauses at the node are satisfied - formally:\\
$\m{\forall n \in P \cdot M \models \clauses{n}}$. $\m{M}$ encodes the values of program variables.

\subsubsection*{Validity}
For a given path $\m{P}$ from the CFG-root to a node $\m{n}$ and a clause $\m{C}$, 
we say that $\m{C}$ holds at $\m{n}$ on $\m{P}$ - $\newdef{\m{n \models_P C}}$ - if $\m{C}$ is entailed by $\clauses{P}$ - 
this means that $\m{C}$ holds at the post-state of $\m{n}$ for any trace that passes through $\m{P}$ - formally:\\
$\m{n \models_P C \equivdef n = P_{end} \land \clauses{P} \models C}$.

\noindent
A clause $\m{C}$ holds at a node $\m{n}$ - $\newdef{\m{n \models C}}$ - if $\m{C}$ holds holds on every trace reaching $\m{n}$ - formally:\\
\m{n \models C \triangleq \forall P \in \paths{root}{n} \cdot \clauses{P} \models C}\\
A node n is $\newdef{\m{infeasible}}$ iff $\m{n \models \emptyClause}$\\
A program is $\newdef{\m{valid}}$ iff all its assertion nodes are infeasible - denoted $\newdef{\m{\models P}}$ for a program $\m{P}$.

\subsubsection*{Program transformations}
Our verification algorithm works by manipulating the set of clauses at each CFG-node, and sometimes the CFG-structure itself, until there are no assertion nodes left. 
We describe a set of program transformations that include both the manipulation of the set of clauses at nodes and the CFG-structure.\\
In order for the verification algorithm to be sound, it must only apply invalidity-preserving transformations to the program, we call these \newdef{sound transformations} - formally:\\
A transformation T of a program $\m{P}$ is sound iff $\m{~\models P ~ \Leftarrow ~\models T(P)}$.\\
Conversely, a \newdef{complete transformations} preserve validity - intuitively not losing information - formally:\\
A transformation of a program $\m{P}$ to a program $\m{P'}$ is complete iff\\ $\m{~\models P ~ \Rightarrow ~\models T(P)}$.
For example, removing any CFG-node whose clause-set contains the empty clause is a sound and complete transformation, as is adding a clause to a node's set of clauses that is entailed by this set of clauses.

\noindent
All of our transformations satisfy a stronger property than soundness and completeness:\\
A transformation of $\m{P}$ to $\m{P'}$ is \newdef{conservative} if it is sound and complete and, for each CFG-node $\m{n}$ that occurs in both $\m{P}$ and $\m{P'}$ and each clause $\m{C}$ in the vocabulary of P (containing only symbols that occur in P), $\m{C}$ holds at $\m{n}$ in $\m{P}$ iff it holds at $\m{n}$ in $\m{P'}$.\\
The reason this property is interesting is that it allows incremental verification for some program and vocabulary extensions.
For example, if a node $\m{n}$ has exactly two successors $\m{p_1,p_2}$ and $\m{C \in \clauses{p_1} \cap \clauses{p_2}}$, 
we can modify $\clauses{n}$ by adding $\m{C}$ to it, which is both sound and complete, but not conservative, as $\m{n \models C}$ in $\m{P'}$ but not necessarily in $\m{P}$.

We use mostly two kinds of transformations:\\
\newdef{Inference} - this transformation, for a given node n, replaces the set $\m{S \triangleq \clauses{n}}$ with a new clause-set $\m{S'}$ s.t. $\m{S \vDash S'}$ and $\m{S' \vDash S}$. The inference transformation is conservative by definition. 
In all the cases we consider, $\m{S'}$ is the result of applying some inference rules from some logical calculus to $\m{S}$ (including simplification rules that remove redundant clauses).\\
\newdef{Propagation} - this transformation propagates clauses from the direct predecessors of a node to the node. 
For example, a CFG-node with one predecessor can add any clause in its predecessor's clause set to its own clause set while being conservative. For nodes with more than one predecessors, propagation can only add a clause to the node's clause-set if it is entailed by the clause-sets of \emph{all} its direct predecessors - we discuss such joins for each logical fragment we consider.

\subsection*{Joins}
A CFG-node with more than one direct predecessor is called a join node.
In general, it is not sound to add a clause to a join node that does not occur in all predecessors, hence we need some mechanism to propagate information at joins.
In order to be able to perform propagation in a sound and complete way for join nodes, we modify the Boogie program as follows:\\
We ensure that each branch and join in the program is binary - any n-ary branch or join is cascaded to a binary tree of binary branches or joins. This cascading of branches and joins is a conservatives program transformation.

\subsubsection*{Branch conditions}
For each binary branch b, we add a \newdef{branch condition atom} \m{P_b} which is a fresh nullary predicate symbol.\\
For a binary branch node $\m{b}$ with successors $\m{s_1,s_2}$ s.t. \m{s_1,s_2} have a common transitive successors, we add the clause $\m{P_b}$ to $\m{s_1}$ and $\m{\lnot P_b}$ to $\m{s_2}$.\\
Note that the transformation that adds branch conditions is conservative as it only \lstinline|assume|s new fresh literals.

For a binary join node $\m{j}$ with two predecessors $\m{p_1,p_2}$, if, for some path condition atom $\m{P_b}$, $\m{P_b \in \clauses{p_1}}$ and $\m{\lnot P_b \in \clauses{p_2}}$, and also for some clause $\m{C}$, $\m{C \in \clauses{p_1}}$, it is sound to add the clause $\m{\lnot P_b \lor C}$ to $\clauses{j}$. We call the clause $\m{\lnot P_b \lor C}$ a $\m{\newdef{relativized}}$ version of $\m{C}$.\\
As we show later, inference and propagation can form a complete verification procedure if the above condition holds for all join nodes - that is, for each join node some $\m{P_b}$ holds at the first predecessor and $\m{\lnot P_b}$ at the other.


\subsubsection*{Well-branching programs}
We define a class of programs for which completeness by propagation and inference can be shown - the class of well-branching programs.\\
Intuitively, a program is well-branching if each join joins exactly one branch.\\
Formally, a \newdef{good-join} is a binary join node $\m{j}$ with direct predecessors $\m{p_1,p_2}$ s.t. the set $\predsto{p_1} \cap \predsto{p_2}$ has a single maximum $\m{m}$ w.r.t. the topological order on the CFG, and each path from the root to $\m{j}$ passes through $\m{m}$. We call this maximum the \newdef{corresponding branch} of the join.\\
A \newdef{well branching program} is a program where all joins are good joins.\\
It is easy to see that a well branching program always satisfies the above condition for joins - namely, the branch condition of the corresponding branch is always at opposite polarities at the predecessors of a join.\\
When cascading branches and joins, we try to ensure that the resulting program is well-branching if possible.\\
In our experience, the VC of programs without exceptional control flow is often well-branching.\\
If a program is not well-branching, the addition of branch conditions still allows completeness of propagation, 
but the relativization of clauses is less efficient.


\subsubsection*{Path conditions}\label{section_path_condition}
For a well branching program, we define the \newdef{path condition} of a CFG-node, \newdef{\pc{n}}, as the set (conjunction) of the branch conditions that hold at the node - formally:\\
$\m{\pc{n} \triangleq \textbf{ if } n=\m{root} \textbf{ then } \emptyset \textbf{ else } \lpc{n} \cup \bigcap\limits_{p \in \preds{n}} \pc{p}}$\\
where \newdef{\lpc{n}} is the local path condition of \m{n}, which is $\m{P_b}$ for one successor of a branch b, ${\m{\lnot P_b}}$ for the other successor, and $\emptyset$ for all other nodes.

When a clause is propagated through several nodes, in some of which it is relativized, it can collect several branch literals on the way - the \textcolor{blue}{relative path condition} - $\newdef{\rpc{p}{n}}$ - is intuitively the set of branch literals added to the clause when it is relativized on the path from p to n - formally:\\
$\rpc{p}{n} \triangleq \pc{p} \setminus \pc{n}$

The path condition and relative path conditions can be defined also for non well-branching programs, but the definition is more complicated.




%\section*{Fragments}
%A \emph{program logic fragment} \m{F=(C_F,\vdash_F,\sqcup_F)} is:
%\begin{itemize}
	%\item A subset \m{C_F} of \Cs{\sig} for the \sig{}
	%\item A derivation relation \m{\vdash_F} on sets of \m{C_F}
	%\item A maximal join \m{\sqcup_F : P(C_F) \times \bigcup\limits_n (P(C_F))^n \rightarrow P(C_F)} \\
	%where n ranges between 0 and the maximal degree of a join (in our case always 2), that satisfies \m{\forall c \in \sqcup_F(C,P) \cdot \forall i \cdot ( \emptyClause \in P_i \lor C \cup P_i \vdash_F c)} - \\
	%basically this set includes any clause that is derivable by the fragment on all paths, but might include less
%\end{itemize}
%
%\noindent
%The fragment of ground unit equalities with strong join, \m{u}, is:
%\begin{itemize}
	%\item \m{C_{u} = \s{C \in \Cs{\sig} \mid C=\emptyClause \lor \exists s,t \cdot C=s \bowtie t }}
	%\item \m{\vdash_{u}} is \m{\vdash_{\mathbf{CC_I}}}
	%\item \m{\sqcup_{u}(C,P) = \s{c \in C_{u} \mid \forall i \cdot C \cup P_i \vdash_{u} \emptyClause \lor C \cup P_i \vdash_{u} c }} - 
	%that is, a full join
%\end{itemize}
%We will use also a weaker fragment, \m{w} as follows:
%\begin{itemize}
	%\item \m{C_{w} = \s{C \in \Cs{\sig} \mid C=\emptyClause \lor \exists s,t \cdot C=s \bowtie t }}
	%\item \m{\vdash_{w}} is \m{\vdash_{\mathbf{CC}}}
	%\item \m{\sqcup_{w}(C,P) = \s{c \in C_{w} \mid \forall i \cdot C \cup P_i \vdash_{CC} \emptyClause \lor C \cup P_i \vdash_{CC} c }} - 
	%roughly, \\this join can only infer equalities at the join for terms that appear in either the clauses at the join node or in all predecessors
%\end{itemize}
%
%\noindent
%We define a \emph{fragment DAG interpolant} for verifying a subset of the leaf nodes \m{A}:\\
%A fragment DAG interpolant I for a logical fragment F is a mapping from nodes to finite sets (conjunctions) of clauses, where \m{I_n} is the set of these clauses at node n (distinct from \clauses{n}), that satisfies, for each node n:\\
%$
%\m{I_n \in C_F(\sig{n})}\\
%\m{\sqcup_F(\clauses{n},\m{i \mapsto I_{\preds{n}_i}}) \vdash_F I_n}\\
%\m{\forall n \in A \cdot \emptyClause \in I_n}
%$\\
%Where \m{i \mapsto I_{\preds{n}_i}} is the sequence of interpolants of direct predecessors of the node n, for some arbitrary (but fixed) order on \preds{n}.\\
%\sig{\m{n}} is the signature at node \node{n}, until discussing scoping we assume all the nodes have the same signature 
%(the clauses \clauses{n} are always of the signature \sig{\m{n}}).
%For a given cfg and fragment F, the set of all interpolants in the fragment \m{F} is \m{I_F}.
%
%%\noindent
%%To see why we need a separate join operation, consider the following alternative fragment \m{w}:
%%\begin{itemize}
	%%\item \m{C_{w} = \s{C \in \Cs{\sig} \mid C=\emptyClause \lor \exists s,t \cdot C=s \bowtie t }}
	%%\item \m{\vdash_{w}} is \m{\vdash_{\mathbf{CC_R}}}
	%%\item \m{\sqcup_{=}(C,S) = \s{c \in C_{=} \mid \forall P \in S \cdot P \vdash_{w} c}} - 
	%%that is, a join only for common terms
%%\end{itemize}
%%This would be a simpler join that just takes the intersection of the closure of clauses on both sides, with respect to the calculus \m{\mathbf{CC_R}} - importantly it can only infer an equality at the join for terms that appear onb both sides, but is strictly weaker as we will show later.\\
%%In our experience, we have not found many cases where the stronger join produced additional information that actually helped proving programs, and the simple join is easier to implement - we will discuss this later.
%
%\subsubsection*{Provability in a fragment}
%For a given program logic fragment \m{F}:\\
%\m{n \models_F C \triangleq \exists I \in I_F \mid C \in I_n}\\
%And the program is \emph{within the fragment} if there is a fragment interpolant \m{I \in I_F} s.t. for each assertion node \m{n}, \m{\emptyClause \in I_n}.


\section*{Equivalence classes}
For a given set of clauses S, we overload the meaning of \Eqs{S} to denote also the congruence relation defined by the reflexive transitive congruence closure of the unit ground equalities in \Eqs{S}.\\
The set of equivalence classes of terms of a set of clauses is defined as:\\
\m{\ECs{S} \triangleq \terms{S}/\Eqs{S}}.\\
For a CFG-node $\node{n}$ we use $\ECs{n}$ for $\ECs{\clauses{n}}$ - the set of equivalence classes of terms that occur in clauses at n according to the congruence relation defined by unit clauses at n.

For a congruence relation R we use the notation \m{[t]_R} to denote the equivalence class of t in R. We drop the subscript when it is clear from the context.

\noindent
A desirable property of the calculus $\m{\mathbf{CC}}$ is that
$\size{\ECs{S}} \geq \size{\ECs{CC(S)}}$.
In fact, if $\m{C}$ is the result of a derivation with premises in $\m{S}$, and $\m{S'=S \cup \s{C}}$, then $\size{\ECs{S}} \geq \size{\ECs{S'}}$, so the set of equivalence classes does not grow from applying derivations in the calculus.
This property is immediate from the definition of the calculus, as for each rule, for each sub-term of the conclusion, either the sub-term occurs in the premises, or the conclusion equates it to a term that occurs in the premises.

\noindent
\subsubsection*{Atomic ECs}
%Our algorithm maintains a\newdef{partial equivalence relation}at each CFG-node. 
%A partial equivalence is defined by a set of ground equations E and a set ground terms T (that includes at least all terms that occur in the equations). R is a partial relation on $\Ts{\Sigma}$ that is only defined on $\m{T^2}$. 
%For a pair of terms either of which is not in T, the relation 

Our algorithm annotates each CFG-node with an approximation of an congruence relation, and the approximations at adjacent CFG-nodes are often similar (agree on many pairs of terms). We use the following concepts to describe the approximation and the relation between similar congruence relations

For a given congruence relation on ground terms we define the set of \newdef{atomic ECs} - \newdef{\AEC} - which are the smallest sets of terms out of which equivalence classes can be constructed, and the smallest unit that is potentially common with stronger congruence relation.\\
Given a congruence relation R and a set of terms T, an EC-tuple $\tup{s}$ is a tuple of equivalence classes of T in R - $\m{\tup{s} \in (T/R)^{\arity{f}}}$.\\
The atomic EC $\fa{f}{s}$ for an EC-tuple $\tup{s}$ is a set of terms defined as:\\
\m{\terms{\fa{f}{s}} \triangleq \s{\fa{f}{t} \mid \bigwedge\limits_i t_i \in s_i}} \\
The set of such atomic ECs for a congruence $\m{R}$ is \newdef{\AECs{R}}.\\
By the definition of congruence closure, all terms of an AEC are in the same EC of R - formally:\\
$\m{\forall \fa{f}{s} \in \AECs{R},\m{t} \in \terms{\fa{f}{s}} \cdot \fa{f}{s} \subseteq [t]_{R}}$.\\
However, an EC of R may include more than one AEC.\\
For example, in the congruence defined by $\m{S = \s{a=b,f(a)=g(c)}}$, the set of ECs of terms of S\\ ($\m{\terms{S}/R}$) are:\\
$\s{ \s{a,b}, \s{c}, \s{f(a),f(b),g(c)}}$ while the set of AECs of terms of S is \\
$\s{a(),b(),c(),f(\s{a,b}),g(\s{c})}$.\\
This hints also at another property of AECs - they allow us to share some of the representation of two similar congruence relations (that is, relations that agree on some subset of equalities). 
In our setting this is most often the case of the sets of possible AECs for the congruence relations that hold at two consecutive CFG-nodes - for example: \\
For the set S above and the set $\m{S' = S \cup \s{c=d}}$,
the set of AECs of $\m{S'}$ is\\ $\s{a(),b(),c(),d(),f(\s{a,b}),g(\s{c,d})}$.\\
If $\m{S}$ is the set of clauses of a node and $\m{S'}$ is the set of clauses of a direct successor (in the CFG) of that node, they can share the common AECs\\ $\m{a(),b(),c(),f(\s{a,b})}$ while they can only share the equivalence class $\m{\s{a,b}}$.\\
For a given congruence $\m{R}$, the sets of terms of AECs are disjoint and each equivalence class is a disjoint union of sets of terms of AECs.
Our congruence closure calculus $\m{\mathbf{CC}}$ does not generate any new AECs - the only rule that may introduce a new term (con) does not introduce a new AEC.\\
We will use the number of AECs as the main space complexity measure as our data structure is based on AECs and, for all the other congruence closure algorithms that we are aware of, the space complexity is at least the number of AECs, possibly more (this is similar to measuring the size of a fully reduced set of equations as in ~\cite{GulwaniTiwariNecula04}).
%In most cases we do not consider the (largest) function arity as a complexity factor as it does not change asymptotic behaviour, 
%but in the few cases where it does we mention it. 
%However, in our experience, function arity can strongly affect actual performance, and we will discuss it in the relevant section.\\
%For a set of functions \m{F} and a set of equivalence classes \m{E}, the maximal number of AECs is:\\
%\m{\sum\limits_{f \in F} \size{E}^{\arity{f}}} so if $n=max\s{\arity{f} \mid f \in F}$ then the upper bound is \\
%\bigO{\size{F}\size{E}^n},
%hence the size of the representation of each such AEC \fa{f}{s} is \arity{f} (references to term equivalence classes), so the total complexity is at most \\ \m{\sum\limits_{f \in F} \arity{f}\size{E}^{\arity{f}}}, which, by sharing such tuple ECs can be reduced to 
%\m{\sum\limits_{f \in F} \size{E}^{\arity{f}}}.\\
%However, new AECs are only introduced by introducing new clauses, not by congruence closure derivations, and is at most linear in the sum of sizes of clauses in \m{S}.\\
%For an arity \m{n} we expect many equivalence class tuples to participate in more than one AEC - for example:\\
%In \m{f(a,b)=g(a,b),a=c} the EC tuple \m{(\s{a,c},\s{b})} is used twice,
%we can also share EC tuples to reduce overall complexity, but the dominant part of the complexity measure is still \bigO{\size{E}^n}.\\
%Another important property of this complexity measure is that it is agnostic to the order of derivations - it only depends on the current set of clauses - for example, compare:\\
%First assuming \m{f(a)=c,f(b)=d} and then \m{a=b} - the final result will have only one AEC with the function symbol \m{f}, while some union find data structures will have two\\
%with - first assuming \m{a=b} and then \m{f(a)=c,f(b)=d} - many union data structures will only have one such function edge.

\subsection*{Proofs and models}
A \newdef{proof tree} for a logical calculus and a set of axioms is a tree with an instance of an inference rule from the calculus at each nodeץ, where the conclusions of the children of each node are the premises of the inference rule instance at the node. The leaves of the tree are axiom nodes.
A refutation tree is a proof tree where the conclusion of the root is a contradiction - in CNF form this is usually the empty clause.
A \newdef{proof DAG} is similar to a proof tree where the difference is that the conclusion of a node can be used as the premise of more than one parent. A non-redundant proof DAG is a proof DAG where no two nodes share the same conclusion.

For a given logical calculus, set of axioms and theorem, the minimal proof depth is the minimum of depth for all proof-trees (and, equivalently, proof-DAGs) of the theorem from the axioms in the calculus. The depth of a tree or DAG is the length of the longest path from the root to a leaf. The size of a proof-DAG is the number of nodes it contains. 
The minimal proof size for a given theorem, calculus and set of axioms is the minimal size of proof-DAG for the theorem from the axioms, and similarly for depth.

For each of the automated theorem proving techniques, when a refutation is obtained, a proof-DAG (possibly redundant, depending on the ATP technology) for the refutation can be extracted in the calculus used by the ATP. For example, the original DPLL algorithm produces proofs in tree from, where the minimal proof size can be exponentially larger than an equivalent non-redundant DAG-proof. CDCL produces DAG proofs (\cite{DBLP:conf/aaai/HertelBPG08}). The lower bound on time complexity of an ATP run on a problem is related to the minimal proof size in the ATP's calculus, but also to the size of the proof search space - preventing the prover from considering proofs with a highly redundant proof-DAG often accelerates proof search.

%
%\subsection*{Theories}
%Many computer programs and most program VCs make use of some form of linear integer arithmetic.
%Some programs that use floating point numbers are modeled using rational arithmetic, for which linear arithmetic has the most developed decision procedures.
%Arrays can be modeled using read-over-write axioms (\cite{Mccarthy62towardsa}), which suffice for a large class of programs.
%When array extensionality is required the situation is more complicated, and with integer indices decidability and complexity depends on the fragment of integer arithmetic used.
%Floating point arithmetic is sometimes modeled as rational arithmetic, for which the linear fragment has efficient decision procedures.
%Linear rational arithmetic finds also other uses, such as modeling permissions (\cite{Boyland:2014:CSA:2635631.2635847}).
%
%\subsubsection*{Complexity and decidability}
%The problem of deciding the validity of a formula in FOLE is semi-decidable while the problem for the ground fragment GFOLE is NP complete - the satisfiability of a conjunction of unit ground positive and negative equations (the unit ground theory of equality with uninterpreted functions) can be decided in \bigO{nlgn}time, while the unit quantified theory of equality with uninterpreted functions is undecidable even for unit clauses, but is semi-decidable.
%The theory of quantified linear integer arithmetic (LIA) is decidable but a lower bound for the decision problem is double exponential(complexity and decidability results are summarized in \cite{Bradley:2007:CCD:1324777} section 3.7). The unit ground (also called quantifier free conjunctive) theory of linear integer arithmetic (QF\_LIA) is NP complete - this theory is important for the verification of many computer programs and hence most SMT solvers include a decision procedure for it.
%The theory of quantified linear rational arithmetic (LRA) is decidable and an exponential lower bound has been shown.
%The unit ground version (QF\_LRA) has polynomial complexity and, while not directly useful in programs that do not manipulate rational (or, as an approximation, fixed point) numbers, it can be useful for modeling parts of the VC of programs such as permissions.
%The quantified theory of arrays without extensionality is undecidable and the unit ground theory is NP-complete - arrays are import as they are a common data structure in computer programs and they can sometimes be used to model memory. Extensionality for arrays is important in some verification contexts and the unit ground theory of arrays with extensionality is NP-complete (\cite{932480}).
%


\chapter{Background on Smartphones}
\label{chap:background}

In this chapter we introduce the architecture of modern smartphones with regard to both their hardware and software, as well as the mechanisms used to deploy applications onto them. Due to the amount of sensitive private data stored on smartphones, specific security measures have been implemented by vendors. In particular, smartphone vendors have implemented security mechanisms aimed at preventing malicious software from running on their operating systems or controlling what potentially malicious applications can do. These mechanisms have so far limited widespread malware infections that, in contrast, have plagued personal computers for decades. Additional mechanisms provide a way for users to manage which applications have access to which components and services of their smartphones. This form of control allows careful users to protect their privacy by choosing which private information to share with whom and which applications have access to restricted system components (e.g., access to the microphone, or to the pictures taken with the camera).

We note that different hardware and software vendors introduce slight variations into the basic concepts described in this chapter. While keeping the descriptions generic we will also highlight some differentiation factors between smartphone vendors. We will try to give an as up-to-date as possible view of the different system components. In this fast-evolving market, hardware and software vendors keep on increasing the security mechanisms used in their smartphones and in the whole ecosystem around them. Finally, we will focus our attention only on Android and iOS smartphones, the most used platforms in the current market (accounting to approximately 97\% of the global marketshare~\cite{marketshare}). Other popular smartphone platforms provide similar security architectures and mechanisms (i.e., Microsoft WindowsPhone~\cite{windowssecurity} and RIM's Blackberry~\cite{blackberrysecurity}).

Overall we try to present the reader with details that are of interest to the rest of this thesis and will help in understanding the following research chapters. The rest of this chapter is structured as follows, we first present the typical hardware architecture of a modern smartphone in Section~\ref{sec:bg_hardware}. We will see how we exploit some of the hardware components of smartphones in the first part of this thesis, where we use smartphones to secure our daily operations. We then introduce the different software components and highlight the security mechanisms provided by the operating system in Section~\ref{sec:bg_software}. Finally, we provide an overview of the two main distribution strategies for applications for Android and iOS devices in Section~\ref{sec:bg_markets}. In the second part of this thesis we will understand how attackers can overcome the security mechanisms implemented on current smartphones to carry out sophisticated attacks to steal users' credentials or private data.

\section{Smartphone Hardware}
\label{sec:bg_hardware}

A standard mobile device architecture (as shown in Figure~\ref{fig:bg_mobile})
has two processors. The \emph{application processor} runs the mobile OS (e.g.,
Android) and the applications on top of it. The vast majority of devices
use ARM as the architecture for their processors. This is due, mainly, to
ARM cores small footprint and reduced power consumption while offering powerful
multi-core options. Modern ARM processors, from ARMv7 onwards,
typically support both the ARM and the Thumb-2 instruction
sets~\cite{arminstructionset}. The original iPhone, in 2007, was one of the first smartphones to use an ARM processing core. Ever since, the majority of modern smartphones have used ARM cores as their main processing unit. The more recent ARM cores (since ARMv6) also support a system-wide security mechanism called ARM TrustZone. We defer the discussion on ARM TrustZone to Section~\ref{sec:ps_tee_mobile} as it will be the focus of that research chapter.

\begin{figure}[!ht]
    \centering
    \includegraphics[width=.9\linewidth]{figures/others/bg_smartphone}
    \caption[Architecture overview of a modern smartphone]{Architecture overview of a modern smartphone. The application processor is separated from the baseband processor that handles network operations and communicates with the SIM card. Applications run on top of a mobile operating system (e.g., Android or iOS).}
    \label{fig:bg_mobile}
\end{figure}

The mobile OS that runs on the application processor has direct access to the
peripherals found on the device and mediates this access to the unprivileged
applications running on top. A common set of peripherals found on modern
smartphones consists of a wireless adapter (implementing both WiFi as well as
Bluetooth functionality, such as the Broadcom BCM4354~\cite{broadcombcm}), a
GPS receiver (such as the Broadcom BCM47521~\cite{broadcomgps}), one or more
gyroscopes and accelerometers (such as the InvenSense's
ICM-20608-G~\cite{6axis}), one or two integrated cameras, and a set of
microphones and speakers. A physical keyboard and pointing device are typically
omitted and a software-based implementation that shows on the screen is
preferred. This plethora of peripherals is one feature that distinguishes
smartphones from other mobile devices and personal computers and enables many
different applications, some of which malicious, as well as some interesting
security solutions.

A \emph{baseband processor}, running the baseband OS, handles cellular
communication and mediates communication between the application processor and
the SIM card. Each SIM card has a unique identifier called IMSI (International
Mobile Subscriber Identity), used to negotiate with a base station to grant
access to the mobile network. The baseband OS is the responsible for
implementing the protocols that govern the mobile networking space. For
example, it must implement the stacks used for GSM~\cite{etsigsm},
GPRS~\cite{etsigprs}, EDGE~\cite{etsiedge}, LTE~\cite{etsilte}. The
application processor and the baseband processor interact by exchanging
messages, typically through a shared memory region. Device manufacturers are
free to integrate any baseband OS of their choosing. Typically such operating
systems are small microkernel-based real-time operating systems customized for
baseband processors. Notable examples are: Nucleus RTOS~\cite{nucleos},
ThreadX~\cite{threadx} and OKL4~\cite{okl4}. Due to the fact that baseband OSs are interfacing directly with the network providers and that they have full access to the smartphone hardware, they are typically well tested and undergo
strict code audits to ensure that they are free of bugs. While some attacks against baseband processors have been found~\cite{basebandwoot,basebandccc}, their numbers are relatively small compared to bugs found in more complex operating systems.

\section{Smartphone Software}
\label{sec:bg_software}

We now briefly describe the software architecture of Android and iOS operating systems. We then focus on their security features. We first introduce the features generically and then focus on more details for the Android OS, which is also used in the rest of this thesis as the main smartphone OS for discussion.

Smartphone operating systems are based on a monolithic kernel, such as the
Linux kernel~\cite{androidkernel} (for Android) or a hybrid kernel such as
Mach~\cite{machkernel} (for iOS). The kernel implements common functionality
(such as process and memory management, filesystem access, drivers to access
the peripherals). On top of the kernel, the OS features a layer of software
that is both the foundation for developers to develop their applications (so
called Software Development Kits, or SDKs) as well as a set of pre-installed
privileged utilities to manage the system. Examples of these utilities are a
network manager, a way to configure the many preferences of the system, a
centralized notification center, and so on. The development framework dictates
the running environment as well as the main development language of the
operating system. Finally, a number of applications come bundled with the OS.
Examples include an internet browser (on iOS an offspring of Safari, based on WebKit~\cite{webkit}, and on Android a mobile version of Chrome, based on Blink~\cite{blink}), an e-mail client, a calendar application, an address book application, and so on.

We now introduce how third party applications can be developed on both Android and iOS and then focus our attention on the security features provided by both architectures.

\subsubsection*{Android Application Development}

\begin{figure}[!t]
    \centering
    \includegraphics[width=.5\columnwidth]{figures/others/bg_android}
    \caption[Android software components]{Android software components. The Linux kernel sits at the lowest level and manages drivers and other OS components. The runtime environment runs Dalvik executables, which can interface with system libraries. The application framework allows applications to use standard components such as activities and services. Finally, on top, applications are the front end to the user.}
    \label{fig:bg_android_stack}
\end{figure}

Android applications are Dalvik executables~\cite{dalvik}, where Dalvik is a
small implementation of Java specifically tailored for ARM processors and
optimized for mobile platforms. Each application is developed in a type-safe
language very similar to Java, and is run in a virtual machine (the so-called
ART: Android runtime). Developers can also develop applications in C or C++ by
using the Android NDK (Native Development Kit) and providing an
interface to communicate data back and forth with the Dalvik application.
Developing against the NDK allows for fast ARM-optimized code and also allows
developers to use any C/C++ library, like opengl~\cite{androidopengl}, or
ProjectNe10~\cite{projectne10} used to access ARM Neon optimized routines.
Figure~\ref{fig:bg_android_stack} shows an overview of the main software
components of an Android device.

Applications are developed using a set of Dalvik classes and XML files which define a plethora of parameters (e.g., string values, localization information, color combinations) as well as GUI descriptions. These values are either compiled in the application at compilation time or parsed at runtime by the OS to, for example, draw the GUI of the application on the screen. Developers can also draw GUI elements programmatically. Applications can have a number of components, and, most notably, can be split into \emph{Activities}, the foreground processes that interact with the user, and \emph{Services}, the background processes that can be used to perform long running operations or are active when an application goes into background. The system, indeed, allows background applications to perform any kind of task, like monitoring the GPS coordinates, record sound through the microphone and perform any network activity. 
% Figure~\ref{fig:bg_android_lifecycle} illustrates the lifecycle of an Android application for both its activities and services, which can continue to run in the background once the main Activity loses the foreground.

% \begin{figure}[!ht]
%     \centering
%     \subfigure[] {
%     \includegraphics[width=.45\columnwidth]{figures/others/bg_android_activity_lifecycle}}
%     \subfigure[] {
%     \includegraphics[width=.45\columnwidth]{figures/others/bg_android_service_lifecycle}}
%     \caption[Android applications lifecycle]{Android applications lifecycle for an Activity (a) and a Service (b). In particular we see that applications are composed of multiple Activities and Services which can continue to run even when the application enters a background state.\todo{new images}}
%     \label{fig:bg_android_lifecycle}
% \end{figure}

We refer the interested reader to the Android developer portal for a complete overview of Android system components, development practices and possible applications~\cite{androidkernel}.

\subsubsection*{Apple iOS Application Development}

iOS applications are compiled binaries implemented in Objective-C, an
object-oriented dialect of C. Starting with iOS 7, applications can also be
developed in Swift, a new open-source object-oriented language developed by Apple~\cite{swift}. Through Objective-C glue code applications are able
to directly use C/C++ libraries and have direct access to ARM functions through
the direct use of assembly code. Developers make extensive use of the CocoaTouch
runtime framework to interact with system components and the user interface. The latter can be implemented either programmatically or
through XML-based files. Figure~\ref{fig:bg_ios_stack} shows an overview of the
main software components of an iOS device.

\begin{figure}[!t]
    \centering
    \includegraphics[width=.5\columnwidth]{figures/others/bg_ios}
    \caption[iOS software components]{iOS software components. The kernel is at the lowest level and manages drivers, OS components and the security mechanisms. The runtime environment runs Objective-C and Swift applications. Applications can use the public libraries and the application framework components.}
    \label{fig:bg_ios_stack}
\end{figure}

In contrast to Android applications, iOS applications are monolithic. In order to run longer-running tasks, each process can spawn multiple threads. The application lifecycle is also different from Android's in that, once in the background, applications are typically fully suspended or can continue to operate for a finite period of time (a number of seconds or minutes, at most), mostly in order to preserve battery life. A handful of exceptions to this rule are applications that require GPS updates (e.g., a mapping application or an activity logger), perform VoIP functionality (e.g., Skype) or play music (e.g., Spotify). With the recent release of iOS 9, Apple allows for two applications to run concurrently on some devices (e.g., the iPad Air 2, the iPad Pro and the iPad mini 4) and for both to display content on the screen. Apart from when they are in the foreground, applications can execute arbitrary code only upon receiving a \emph{silent} push notification. When this happens the application has approximately 30 seconds to, for example, fetch data from a server. 
% Figure~\ref{fig:bg_ios_lifecycle} illustrates the lifecycle of an iOS application.

% \begin{figure}[!ht]
%     \centering
%     \includegraphics[width=.8\columnwidth]{figures/others/bg_ios_lifecycle}
%     \caption[iOS applications lifecycle]{iOS applications lifecycle. In particular we see how applications are monolithic and enter a suspended state as they go into background.\todo{new image}}
%     \label{fig:bg_ios_lifecycle}
% \end{figure}

We refer the interested reader to the iOS developer portal for a complete overview of iOS system components, development practices and possible applications~\cite{iosdevelopment}.

\subsection{Security Features}

In terms of security, each OS features different mechanisms that we will now present in more detail.

\paragraph{Secure Boot.} Modern smartphones employ, in most cases, a standard
secure boot chain. Android manufacturers can develop their own version, with a
potentially slightly modified strucutre. Some enable a modified (and unsigned)
kernel to run right out of production (albeit typically voiding the phone
warranty), and others lock the platform and require hacks, so-called
\emph{device rooting}, before an unsigned kernel can be booted. Apple's iOS
devices, on the other hand, all follow the same procedure, which we now detail.
Upon device boot, the application processor runs code from the Boot ROM (a
read-only memory region burnt-in at hardware manufacturing time) which
consitutes the hardware root of trust. Apart from the boot code, the Boot ROM
also contains the Apple Root CA public key, which is used to verify the
integrity of the Low-Level Bootloader (LLB). The LLB, when it has finished
running its tasks, in turns verifies the integrity of the next bootloader
(iBoot) which finally verifies and boots the iOS kernel. For any unsigned
kernel to boot, the phone must be rooted. Exploits for each new kernel
version must be discovered in order for the modified kernel to be booted up
correctly. It is common that rooted devices modify the framework running on top
of the kernel (or some kernel extensions), rather than the kernel itself.

Similar to the application processor, the baseband processor and the code running in the trusted execution environment (if any) follow a similar procedure to make sure that the code that runs at the lowest level on a device is verified and has not been modified.

\paragraph{Application Sandboxing.} Each application running on top of the
operating system and developed using the system SDK is sandboxed in its own
execution environment. The sandbox makes sure that at runtime the application
cannot access code or data used by another application (memory isolation). In
Android, memory isolation is typically achieved by starting each application in
its own virtual machine. The system then performs two levels of access control
enforcement: \emph{(i)} the middleware component controls IPC calls and
\emph{(ii)} each application is assigned a locally unique Linux UID and the
kernel enforces access to low-level resources based on these. In contrast, iOS
performs memory isolation through the use of kernel-level process-based
isolation to protect the address space of each application and of other system
resources. Both platforms support address space layout randomization (ASLR), so
that memory regions are randomized at launch, both for system processes and
third party applications~\cite{androidsecurity,applesecurity}. Furthermore the
use of ARM's Execute Never (XN), which marks memory pages as non-executable
improves the overall memory protection. On iOS, only Apple-approved (and, to
the best of our knowledge, only Apple-developed) applications can overcome this
limitation to, for example, enable the JavaScript just-in-time compiler of the
system browser.

\paragraph{Storage Isolation.} Applications data integrity and confidentiality
is protected when at rest. On Android, each application is assigned its own
unique user identifier (Linux UID), as if it were a different user on the
system. Storage isolation is then implemented as file-system permissions.
Application files are created by default as owned by that particular user on
the system and are, hence, accessible only by it. An application can also
create world-readable and world-writable files that can then be accessed by any
other application. On systems that provide an external storage medium (e.g.,
smartphones that allow the user to expand the internal storage with an
SD-card), any data stored in the external storage can be accessed by any
application. This is mainly due to the fact that external storage is typically
formatted as FAT which does not support Unix access-control bits.

On both iOS and Android, third party applications can further store their data
in encrypted form through the use of special
APIs~\cite{applesecurity,androidsecurity} that make use of device-specific and
application-specific keys. On iOS devices the Data Protection framework
combines a device-specific key together with the user's passcode (i.e., either
a 4-digit PIN or a longer alpha-numeric password) to generate per-application
or per-file keys to keep stored data in an encrypted form. These operations
happen transparent to the developer aided by the OS as well as the hardware.
While an iOS application's data is stored in encrypted form automatically, on
Android devices the developers decide what to store encrypted and how. Android
provides a plethora of hardware-accelerated encryption routines available
through the Java API.

\paragraph{Permission-based Architecture.} Applications running on top of smartphone operating systems do not have direct access to peripherals or stored user's data. Instead, all access is mediated by the kernel, the operating system and the upper-layer framework. When mediating access to peripherals (such as the microphone, or the camera) as well as to data (such as contacts or GPS location) the OS performs access control checks to make sure that the application accessing the private information has indeed been granted access to it (what is called a permission). 

Android and iOS have a different approach to permissions. Android has roughly 138 permissions (at the time of this writing, for API level 21). For example, applications require permissions to access the internet, to read the contacts, to manipulate the pictures stored in the photogallery, or to receive location information. We refer the reader to the information available with Google for further details~\cite{androidsecurity}. In contrast, permissions on iOS devices are very coarse-grained. At this time, the system requires explicit permissions to access: the microphone, the camera, photos, location information, contacts, calendars and reminders, the motion activity sensor (on iPhone 5s and later), social media accounts, HomeKit and HealthKit and Bluetooth sharing~\cite{applesecurity}. Finally, the user has to explicitly allow applications to receive push notifications.

\begin{figure}[!t]
    \centering
    \subfigure[] {
    \includegraphics[height=220px]{figures/others/bg_android_permission}}
    \subfigure[] {
    \includegraphics[height=220px]{figures/others/bg_ios_permission}}
    \caption[The permission dialog for both Android and iOS shown to the user]{The permission dialog for both Android version 4.3 (a) and iOS version 7 (b) shown to the user. On Android the dialogue appears at install time. If the user does not approve the required permissions the application is not installed. On iOS the user is prompted with a dialogue to allow the access to a specific resource at run-time.}
    \label{fig:bg_permissions}
\end{figure}

Permissions are further handled differently by the two platforms both from the developer as well as from the user's perspective. On Android, the developer has to specify in the application's \emph{manifest} file (which also specifies the application ID, name, and other details) all the permissions required by its application to run. The manifest is parsed on the marketplace to show to the user which permissions are required. Upon installation it is parsed by the Android OS to grant the correct set of permissions to the process. Up until Android 6.0, released in late 2015, users could only accept or deny (the latter resulting in the application not being installed) all the permissions required by a specific application at install time. In order to help users make sense of the potentially long list of permissions required by an application, the OS bundles fine-grained permissions into broader categories (as shown in Figure~\ref{fig:bg_permissions}~(a)). This mechanism has changed recently and users can now revoke previously granted permissions to an application by going in the system settings. On iOS devices, the developer does not have to specify permissions while developing the application. At run-time, the operating system blocks the first access to any protected resource and prompts the user with a system dialog to deny or grant access to the resource (as shown in Figure~\ref{fig:bg_permissions}~(b)). Again, the user is able to revoke a previously granted access to a resource by going into the system settings.

\paragraph{Application Signatures.} On both Android and iOS, applications are signed packages. On the Apple platform, only software signed by a valid Apple-issued certificate will be permitted to be installed and run. This mechanism extends the chain of trust to third-party applications and prevents unsigned code and self-modifying code from executing. Developers are required to sign their applications with an Apple-issued certificate that is released only after verification of the individual or organization requesting it. This allows Apple to bind each application with a particular identity, discouraging the creation and distribution of malicious code through their marketplace. All the signature checks are performed at runtime by the iOS kernel before starting an application. On Android, in contrast, applications can be signed with self-signed certificates that anyone can produce. In fact, the main reason behind signing applications on Android, is not to prevent unsigned code from running on the platform (which is possible), but rather to maintain a same-origin principle for application updates. On Android, it is possible to ship an update to an application only if said update is signed by the same key used for the previous version.

\paragraph{Trusted Execution Environments.} Smartphones have different Trusted Execution Environments (TEEs) available. In general, a TEE is any environment that is able to store secrets (e.g., private keys, passphrases) and run code in isolation from the main operating system. All smartphones have a SIM card, which can run small applets~\cite{global} (known, since before the advent of smartphones, as SIM applications). Such applets come pre-installed on the SIM card and have to be endorsed by the carrier operator in order to be deployed to customers' SIM cards. 

Another option to perform operations in a trusted environment is ARM TrustZone, which is available on ARM cores since the Cortex-A series~\cite{ARMTrustZone,armTZslides}. We will go into more detail on how ARM TrustZone works in Chapter~\ref{chap:ps_tee}. As an overview, a TrustZone-enabled device supports two execution modes whose isolation is controlled and enforced in hardware. The \emph{normal world} executes the main operating system (e.g., Android or iOS), while the \emph{secure world} executes a smaller, typically more secure, operating system (e.g., an L4-variation on Apple's iPhones~\cite{applesecurity}, or a custom Trustonic OS on some versions of the Samsung Galaxy family~\cite{trustonicknox}). On previous phones and earlier smartphones, the TrustZone technology was more tied down by the device manufacturer and used for SIM locks and similar features~\cite{kostiainen2011codaspy}. Although the software running in the secure world potentially has access to all resources of the device (unlike software running on SIM cards), one of the main design goals is to keep this software small and verifiable in order to prevent bugs in this higher-privilege execution mode.

Although some proposals have been made in order to allow third-party developers to tap into the potential of smartphone TEEs~\cite{kostiainen09asiaccs,kostiainen2011acns,kari11stc}, at the time of this writing the software in the secure world is mostly controlled by device manufacturers. Apple uses the secure world (the \emph{secure enclave}, in Apple's terminology) to store encryption keys and to perform fingerprint matching from its Touch ID peripheral. Samsung proposed their KNOX platform~\cite{trustonicknox} to enable businesses to store credentials and encryption keys in the secure world of TrustZone-enabled devices.

\paragraph{Secure Peripherals.} Starting with the iPhone 5S and on some Android models (e.g., the Samsung Galaxy S5) hardware manufacturers have started to embed secure peripherals. By secure peripherals we mean peripherals that are not accessible from applications directly. For example a fingerprint reader that both provides added security to the device as well as is securely integrated with the rest of the hardware. In this space, Apple's Touch ID stores the scanned image(s) into the encrypted memory of the secure enclave and the memory region is wiped as soon as the Touch ID sensor is deactivated. Only software running in the secure enclave is able to process the scanned copies of the fingerprint which are never accessible from the rest of the system. While fingerprint-based access (and, in general biometric-based access control) is prone to false positives and negatives and can be circumvented~\cite{cccfingerprint}, the intention is to force users to enable passcodes on their phones, which in turns enables stronger secure storage. The general idea being that a larger population using a passcode (and more people using stronger passcodes, since they are required to type them in less frequently) is better than leaving devices unprotected for longer periods of time or missing a passcode altogether.

\section{Distribution Markets}
\label{sec:bg_markets}

Smartphone vendors have adopted a controlled system to let users install third-party applications in the form of \emph{marketplaces}. Apple's AppStore~\cite{appstore} and Google Play~\cite{googleplay} are the most secure way (and in the case of Apple the only way) to install applications on a smartphone. This distribution model has some security advantages which we now outline in brief.

First of all, applications developed by third parties are submitted to the marketplaces and vetted before publication. Apple has a tight control model in which applications are subjected to static and dynamic analysis to check that they do not contain any potentially malicious code or that they use undocumented or private APIs. Then they are tested by a team of people to make sure that applications conform to visual guidelines but also to test for obvious bugs or problems created by each application. The whole testing procedure takes between one and two weeks in most cases, and terminates with the application being prepared for download by customers or being rejected with some motivation. The Android market follows a similar procedure, although applications tend to be published more quickly. Static and dynamic analysis (there is evidence that applications are tested in an emulator~\cite{googlebouncer,bouncerdissect}) is performed in the background and the application is pulled from the market should anything malicious be detected. Although these mechanisms do not fully prevent malware from making its way to a large number of customers, they are providing a first line of defense against malware. In fact it is observed that the amount of malware present on smartphones is significantly smaller, compared to other (more open) systems~\cite{lever-ndss13,truong13}.

Second, the marketplace managers (i.e., Apple and Google, although smaller ones exist such as from Amazon~\cite{amazonappstore}), that have all applications in a single repository can perform large-scale analysis to detect potentially malicious applications. This has, for example, led to the discovery of some malware masquerading as legitimate banking applications and trying to steal user credentials~\cite{droid09}.

Another security advantage of the distribution model of smartphone applications is that it allows for continous and fast upgrades. This is true for third-party applications that can be updated (potentially fixing security bugs) and that will in turn be automatically downloaded and installed by the majority of the user base (this option is typically on by default but could be switched off, if desired by the user). Similarly, OS (and firmware) software updates are enabled over-the-air (OTA), something that again makes adoption of security fixes fast. For example, at the time of this writing, the iOS 8 (introduced in September 2014) adoption rate is 41\% and iOS 9 (introduced in September 2015) is at 52\%. On Android the numbers are lower, due to a more varied landscape in terms of devices: Android 4.4 (introduced in October 2013) is at 39\%, Android 5.0 (introduced in November 2014) is at 21\%.\footnote{The most up-to-date numbers can be found at \url{https://developer.apple.com/support/app-store/} for Apple devices and at \url{https://developer.android.com/about/dashboards/index.html} for Android devices.}

Finally, applications installed through marketplaces can be remotely removed from devices or disabled in case malicious activity is detected. Although some consider this activity a violation of user's privacy (the act of remotely disabling applications), it is also a strong security mechanism in case malicious software indeed finds its way through to users.

\paragraph{Sideloading.} Although installation of applications through marketplaces is the recommended and typical way for users to install software on their smartphones, on Android it is also possible to install unsigned software from other places. This operation, known as sideloading, is potentially an attack vector.

On Apple devices it is not possible to install any third party application unless it is signed and comes from the AppStore. The only possibility to install and run unsigned content is to root or jailbreak the one's device. This operation disables the security checks performed by the operating system at runtime and hence, similar to Android, is a potential security risk for end users.

\section{Summary}

We have given a broad overview of smartphone platforms both in terms of hardware and software. In particular, after a generic introduction, we focused on the details that will be most useful to better appreciate the rest of this thesis. We will see how the hardware configurations of current smartphones enable solutions where they are used as an effective and usable two-factor authentication mechanism for daily operations. The second part of this thesis will focus on the software security mechanisms deployed today by device vendors and how they can be circumvented to steal users' private data.


