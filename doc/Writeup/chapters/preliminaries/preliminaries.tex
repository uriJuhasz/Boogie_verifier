\chapter{Preliminaries}\label{chapter:preliminaries}
In this chapter we present the formalisms with which we work.\\
Most of the chapter is a repetition of standard definitions, except for section \ref{section:preliminaries_programs} which discusses our assumptions about the input program, and transformations we perform on it before our algorithm begins.

\section*{Sets, multisets and sequences}
A \newdef{multiset} \m{m} over a set \m{S} is a function \m{m : S \rightarrow \mathbb{N}_0}.\\
We use 0 for the empty multiset.\\
Singleton multisets are defined as:\\
 \m{[x \mapsto n](y) \triangleq \ite{x\equiv y}{n}{0}}.\\
Multiset union is defined as:\\
 \m{(m \cup n)(x) \triangleq \lambda x \cdot m(x)+n(x)}.\\
Sub-multiset relation is defined as follows:\\
 \m{m \subseteq n \equivdef \forall i \in \mathbb{N}_0 \cdot m(i) \leq n(i)}

\noindent
An \newdef{equivalence relation} \m{\approx} over a set $\m{S}$ is a subset of $\m{S^2}$ s.t.\\
\m{\forall (x,y)\in\approx \cdot (y,x) \in \approx}\\
\m{\forall (x,y),(y,z) \in \approx \cdot (x,z) \in \approx}\\
\m{\forall x \in S \cdot (x,x) \in \approx}\\
For an equivalence relation $\m{\approx}$ over $\m{S}$, and a term $\m{x \in S}$ we denote by 
$\m{\ECOf{x}{\approx}}$ the equivalence class of $\m{x \in S}$ with respect to $\m{\approx}$:\\
\m{\ECOf{x}{\approx} \triangleq \s{y \in S \mid x \approx y}}

\noindent
A \newdef{partition} of a set $\m{S}$ - $\m{P \subseteq \powset{S} }$ - satisfies:\\
\m{\forall S_1,S_2 \in P \cdot S_1 \cap S_2 = \emptyset}\\
\m{\cup P = S}

\noindent
The \newdef{quotient set} of a set $\m{S}$ for the equivalence relation $\m{\approx}$, $\m{S/\approx}$ is the set:\\
\s{ \ECOf{x}{\approx} \mid x \in S}\\
and is a partition of \m{S}, and similarly a partition defines an equivalence relation.

A (finite) sequence of length n of elements of a set S is a function from \m{0..n-1} to S.
For a sequence s we use \m{s_i} for the ith element of the sequence - \m{s(i)}. 
\size{s} is the length of the sequence (\size{\dom{s}}). We sometimes use the notation \m{[i \mapsto e(i)]} to denote sequences, where \m{e(i)} is an expression that defines the ith element, and \m{<s>} for a singleton sequence. We use this notation where the domain is unambiguous.
The concatenation of the sequences s,t is denoted by \m{s.t}.
We use sequences to denote paths in the program CFG (as a sequence of CFG nodes) and we extend the notation for concatenation to include single CFG-nodes - so, for example, if n,p are nodes and P,Q are sequences of nodes, r.P.p.Q is the sequence \m{P.<p>.Q.<n>}.

\section*{Logic}

\subsection*{Syntax and notation}

A \newdef{language} is defined formally as follows:\\
A function or predicate symbol (denoted $\func{f},\func{g},\func{h}$) has a fixed arity ($\arity{f}\geq 0$).\\
A \newdef{signature} $\m{\Sigma=\Fs{\Sigma} \cup \Ps{\Sigma} \cup \Vs{\Sigma}}$
is a set of function symbols $\Fs{\sig}$, predicate symbols $\Ps{\sig}$ and variables $\Vs{\sig}$.\\
%(We will mention non-ground definitions here to be consistent with later chapters, but assume for this chapter that \m{\Ps{\sig}=\Vs{\sig}=\emptyset}). 
We only handle finite programs and sets of axioms, so that the sets $\Fs{\Sigma},\Ps{\Sigma}$ used in any VC are finite. 
However, in order to be able to handle theories with countable signatures, such as integer arithmetic, we allow both these sets to be countable.
The set of \newdef{constants} is $\m{\consts{}=\s{\func{f} \in \funcs{} \mid \arity{f}=0 }}$, we assume $\|\consts{}\|>0$ (otherwise the ground fragment is trivial).\\
We denote a (possibly empty) tuple by an overline - detailed in the following.\\
We use a standard definition for the language:
\begin{figure}[H]
$
\begin{array}{llllll}
	\mathbf{function}  & \mbox{f,g,h}   &     &                                       & \in \Fs{\sig} & \textrm{functions}\\
	\mathbf{predicate} & \mbox{P,Q,R}   &     &                                       & \in \Ps{\sig} & \textrm{predicates}\\
	\mathbf{variable}  & \mbox{x,y,z}   &     &                                       & \in \Vs{\sig} & \textrm{variables}\\
	\mathbf{term}      & \mbox{t,s,u,v} & ::= & \m{\fa{f}{t} \mid x}                  & \in \Ts{\sig} & \textrm{free term algebra }\\
	\mathbf{atom}      & \mbox{a}       & ::= & \m{s = t \mid \fa{P}{t}}              & \in \As{\sig} & \textrm{the atoms over } \Sigma\\
	\mathbf{literal}   & \mbox{A,B}     & ::= & \mbox{a} \mid \lnot \mbox{a}          & \in \Ls{\sig} & \textrm{the literals over } \Sigma\\
	\mathbf{clause}    & \mbox{C,D}     & ::= & \m{\emptyClause \mid A \mid A \lor C} & \in \Cs{\sig} & \textrm{the clauses over } \Sigma\\
\end{array}
$
\caption{language}
\end{figure}

\noindent
We use $\term{s,t,u,v}$ for terms, $\term{\tup{s},\tup{t},\tup{u},\tup{v}}$ for term tuples,
we also construct tuples from terms using parenthesis - e.g. $\term{(t,s)}$.\\
We occasionally treat an n-tuple as a sequence of n ground terms.\\
$\tupAt{t}{i}$ is the i-th element of the tuple $\tup{t}$ and $\tupL{t}$ is the number of terms (size or length) in a tuple. \\
We treat an equality atom as an unordered set and so $\term{(s=t)} \equiv \term{(t=s)}$.\\
We use $\term{s \neq t}$ to denote \term{\lnot s = t}.\\
As we do not manipulate negations syntactically, we consider $\m{\lnot \lnot a \equiv a}$.\\
We treat clauses as sets of literals whose semantics is the disjunction of these literals.\\
We denote the \newdef{empty clause} by \newdef{\emptyClause}.\\
We use $\term{\bowtie}$ to denote either $\term{=}$ or $\term{\neq}$.

\noindent
We define the set of terms \newdef{\terms{S}} of a set of clauses as follows:\\
$
\begin{array}{llll}
\terms{S}           & \triangleq & \bigcup\limits_{\m{C} \in \m{S}} \terms{C} \\ 
\terms{C}           & \triangleq & \bigcup\limits_{\m{l} \in \m{C}} \terms{l} \\ 
\terms{s \bowtie t} & \triangleq & \terms{s} \cup \terms{t} \\ 
\terms{\fa{P}{s}}   & \triangleq & \terms{\tup{s}} \\ 
\terms{\tup{s}}     & \triangleq & \m{\bigcup\limits_{i} \terms{\tupAt{s}{i}}} \\ 
\terms{\fa{f}{s}}   & \triangleq & \s{\fa{f}{s}} \cup \terms{\tup{s}} \\
\terms{x}           & \triangleq & \s{x} \\
\end{array}
$

\noindent
For a set $\mathrm{S}$ we denote by $\mathrm{F_n(S)} \triangleq \mathrm{S^n} \rightarrow \mathrm{S}$ the set of all functions of arity \m{n} over \m{S} and \m{F(S) \triangleq \bigcup\limits_{n \in  \mathbb{N}_0} F_n(S)} the set of all functions over \m{S}.\\
Similarly, we define relations over S as \m{R_n(S) \triangleq P(S^n)} and \m{R(S) \triangleq \bigcup\limits_{n \in  \mathbb{N}_0} \mathrm{R_n(S)}}.

\noindent
For the semantics we use functions from terms to a domain $\term{D}$, $\function{f} : \Ts{\sig} \rightarrow \term{D}$.
When applying such a function to a tuple $\tup{t}$ we mean the point-wise application of the function that returns a tuple in
$\term{D}^{\tupL{t}}$ - so $\term{f}(\tup{t})_{\m{i}} = \term{f}(\tupAt{t}{i})$.


\bigskip

\noindent
A \newdef{structure} $\m{\mathbf{S}=(D_S,F_S,P_S)}$ for a signature $\mathbf{\Sigma}$ includes:
\begin{itemize}
	\item A domain $\mathrm{D_S}$ which is a non-empty set
	\item An interpretation for function symbols $\m{F_S}$ which maps each function of $\sig{}$ to a function of the corresponding arity over $\mathrm{D_S}$ -
namely $\mathbf{F_S} \in \mathbf{F_{\sig{}}} \rightarrow \mathrm{F(D_S)}$ such that
$\forall \m{f} \in \mathbf{\sig} \cdot \m{F_S}(\m{f}) \in \m{F_{\arity{f}}(S)}$. 
	\item An interpretation $\m{\mathbf{P_S}}$ for predicate symbols that maps each predicate symbol of arity n to a relation of arity n over $\m{D_S}$ - namely $\mathbf{P_S} \in \mathbf{P_{\sig{}}} \rightarrow \mathrm{R(D_S)}$ such that
$\forall \m{P} \in \mathbf{\sig} \cdot \m{P_S}(\m{P}) \in \m{R_{\arity{P}}(S)}$. 
\end{itemize}

\noindent
An \newdef{interpretation} \m{\mathbf{I}=(S_I,\sigma_I)} is a structure \m{\mathbf{S_I}=(D_I,F_I,P_I)} and a variable assignment 
\m{\sigma_I : \Vs{\sig} \rightarrow D_I}.

\subsection*{Semantics}\label{section:preliminaries:semantics}
For a term $\term{t}$ and an interpretation \m{\mathbf{I}=(S_I,\sigma_I)} we denote by \m{\den{t}{I} \in \mathrm{D_I}} the interpretation of \term{t} in \m{\mathbf{I}} in the standard way, as defined below:\\
$
\begin{array}{lll}
	\den{\fa{f}{t}} {I} & \triangleq & \den{\term{f}}{I}(\den{\tup{t}}{I}) \\
	\den{x}         {I} & \triangleq & \m{\sigma_I(x)} \\
	\den{t=s}       {I} & \triangleq & \den{t}{I}=\den{s}{I}  \\
	\den{\fa{P}{t}} {I} & \triangleq & \m{\den{\tup{t}}{I} \in \den{P}{I}} \\
	\den{\lnot a}   {I} & \triangleq & \lnot \den{a}{I} \\
	\den{C}         {I} & \triangleq & \m{ \bigvee\limits_{A \in C} \den{A}{I}} \\
\end{array}
$

\noindent
We extend  $\den{\cdot}{I}$ point-wise to tuples, and use $\den{f}{I}$ for $\m{\mathrm{F_I}(f)}$ and $\den{P}{I}$ for $\m{\mathrm{P_I}(P)}$.\\
For the ground fragment, where $\Vs{\sig}=\emptyset$, an interpretation is essentially a structure.\\
Satisfiability in the ground fragment of this language is decidable and NP-complete:\\
We can reduce (in linear time and space) a propositional problem to our fragment by replacing each propositional atom $\term{A}$ with the GFOLE atom $\term{f_A()=T}$ where $\term{f_A}$ is a constant function and $\term{T}$ is a specially designated (fresh) true symbol.
In the other direction we have a polynomial reduction using the Ackermann transformation - basically, encode every term $\fa{f}{t}$ by a fresh variable  $\m{v_{\fa{f}{t}}}$, for any pair of terms \fa{f}{t}, \fa{f}{s} with the same function symbol we add the clause
$ \bigvee\limits_{\m{i}} \m{\m{v}_{t_i} \neq \m{v}_{s_i}} \lor \m{v}_{\fa{f}{t}=\fa{f}{s}}$ to encode congruence closure.\\
We are left with a set of CNF clauses over the fresh variables, with no non-constant function symbols, of at most square size.
We replace each atom $\m{v=u}$ by a propositional atom $\term{A_{v=u}}$, and for each triple of constants $\m{a,b,c}$ we add the clause 
$\m{\lnot A_{a=b} \lor \lnot A_{b=c} \lor A_{a=c}}$ to encode transitivity - we end up with an equi-satisfiable propositional set of clauses.
There are more efficient transformations that achieve the same, however we are interested mostly in the fact that the reduction is polynomial, as the best known algorithm for propositional CNF is exponential.


\subsection*{Terms}
\bigskip

\noindent
\textbf{Substitutions}\\
A \newdef{substitution} on a signature $\m{\Sigma}$ is a total function $\m{\sigma : \m{X} \rightarrow \Ts{\sig}}$ extended to terms as follows:\\
$
\begin{array}{lll}
\m{x\sigma}         & \triangleq & \m{\sigma(x)}\\
\m{\fa{f}{t}\sigma} & \triangleq & \m{f(\tup{t}\sigma)}\\
\m{\tup{t}\sigma_i} & \triangleq & \m{t_i\sigma}\\
\end{array}
$


\smallskip

\noindent
The substitution $\newdef{\m{[x \mapsto t]}}$ is defined as \\
\m{[x \mapsto t](y) \triangleq \ite{x\equiv y}{t}{y}}\\
A \newdef{composition of substitutions}, denoted {$\newdef{\m{\sigma_1\sigma_2}}$}, is defined as:\\
\m{(\sigma_1\sigma_2)(x) \triangleq \sigma_1(\sigma_2(x))}\\
We denote the set of substitutions for a signature $\Sigma{}$ as $\newdef{\subs{\Sigma}}$.

\bigskip

\noindent
\textbf{Term positions}\\
We denote by $\emptySeq$ the empty (integer) sequence and by \lstinline{i.s} the sequence constructed by prepending the integer \lstinline{i} before the sequence \lstinline{s}.\\
A \newdef{position} is a sequence of integers.\\
We denote a \newdef{sub-term of the term $\term{u}$ at position p} by \newdef{\restrict{u}{p}} - which is (partially) defined recursively as follows:\\
$
\begin{array}{lll}
	\termAt{u}{\emptySeq}    & \triangleq & \term{u} \\
	\termAt{f(\tup{r})}{i.s} & \triangleq & \termAt{\tupAt{r}{i}}{s}
\end{array}
$

\noindent
For example, for  $\term{t=f(g(a),h(b,g(a)))}$, \\
$\termAt{f(g(a),h(b,g(a)))}{\emptySeq}=\term{f(g(a),h(b,g(a)))}$, \\
$\termAt{f(g(a),h(b,g(a)))}{0}        =\termAt{f(g(a),h(b,g(a)))}{1.1}=\term{g(a)}$, \\
$\termAt{f(g(a),h(b,g(a)))}{0.0}      =\termAt{f(g(a),h(b,g(a)))}{1.1.0}=\term{a}$ \\
etc.

\noindent
For a term $\m{t}$ the set $\newdef{\m{\poss{t}}}$ is the set of all positions of $\m{t}$ defined as follows:\\
$
\begin{array}{lll}
	\poss{\fa{f}{s}} & \triangleq & \s{\emptySeq} \cup \s{i.p \mid p \in \poss{s_i}} \\
\end{array}
$\\
We also use all positions of a term \m{s} in a term \m{t}:\\
$
\begin{array}{lll}
	\posss{\fa{f}{s}}{t} & \triangleq & \s{\emptySeq \mid \fa{f}{s}\equiv t} \cup \s{i.p \mid p \in \posss{s_i}{t}} \\
\end{array}
$\\
Two positions $\m{p,q}$ are $\newdef{\mbox{disjoint}}$, denoted $\newdef{\disj{p}{q}}$, iff they do not share any common sub-term - formally:\\
\m{\disj{p}{q} \equivdef \exists i,j \cdot (p=i.p' \land q=j.q' \land (i\neq j \lor \disj{p'}{q'}))}\\
The set $\posss{t}{s}$ is pairwise disjoint.\\
By $\termRepAt{u}{t}{p}$ we denote a \newdef{replacement} of \termAt{u}{p} by $\term{t}$ at the position $\term{p}$ in term $\term{u}$ - formally:\\
$
\begin{array}{lll}
	\termRepAt{s        }{t}{\emptySeq} & \triangleq & \term{t} \\
	\termRepAt{\fa{f}{s}}{t}{i.p      } & \triangleq & \term{f}\left(\tupRepAt{s}{i}{\termRepAt{(\tupAt{s}{i})}{t}{p}}\right) \\
\end{array}
$\\
We extend this notion to simultaneous replacement on a pairwise disjoint set of positions \m{P}:\\
$
\begin{array}{lll}
	\termRepAt{s        }{t}{\emptyset}     & \triangleq & \term{s} \\
	\termRepAt{s        }{t}{\s{\emptySeq}} & \triangleq & \term{t} \\
	\termRepAt{\tup{s}  }{t}{P            } & \triangleq & \m{i \mapsto \termRepAt{s_i}{t}{\s{p \mid i.p \in P}}} \\
	\termRepAt{\fa{f}{s}}{t}{P            } & \triangleq & \m{f(\termRepAt{\tup{s}  }{t}{P            })} \\
\end{array}
$

\bigskip


\noindent
\textbf{Sub-terms}\\
A term $\m{s}$ is a \newdef{proper sub-term} of a term \m{t}, denoted \newdef{\m{s \lhd t}}, if:\\
\m{s \lhd t \equivdef \exists p \neq \emptySeq \cdot s = \termAt{t}{p} }\\
And a non-proper sub-term if:\\
\m{s \unlhd t \equivdef s=t \lor s \lhd t }\\
We extend the sub-term relation to tuples, literals, clauses and sets of clauses:\\
$
\begin{array}{lll}
\m{s \lhd \tup{t}}    & \equivdef & \m{\exists i \mid s \lhd t_i}\\
\m{s \lhd u \bowtie v} & \equivdef & \m{s \lhd (u,v)}\\
\m{s \lhd \fa{P}{t}}   & \equivdef & \m{s \lhd \tup{t}}\\
\m{s \lhd C}           & \equivdef & \m{\exists l \in C \cdot s \lhd l}\\
\end{array}
$




\subsubsection*{Orders}\label{section:preliminaties:ordering}
For a set \m{S}, a \newdef{strict partial order} \m{\succ \in S^2} on \m{S} is a binary relation on \m{S} satisfying:
\begin{itemize}
	\item Irreflexive: \m{\forall x \in S \cdot x \not\succ x}
	\item Transitivity: \m{\forall x,y,z \in S \cdot x \succ y \land y \succ z \Rightarrow x \succ z}
	\item Asymmetric: \m{\forall x,y \in S \cdot x\succ y \Rightarrow y \not\succ x }
\end{itemize}
For any strict partial order \m{\succ}, the corresponding \newdef{reflexive closure} \m{\succeq} is defined as:\\
\m{\forall x,y \in S \cdot x\succeq y \Leftrightarrow (x=y \lor x \succ y)}\\
A \newdef{strict total order} is a strict partial order where \\
\m{\forall x,y \in S  \cdot x=y \lor x \succ y \lor y \succ x} \\
and correspondingly a \newdef{total reflexive closure} satisfies:\\
\m{\forall x,y \in S  \cdot x \succeq y \lor y \succeq x}\\
A \newdef{well founded} strict partial order $\m{\succ}$ on $\m{S}$ has no infinite descending chains - formally:\\
\m{\lnot \exists f : \mathbb{N} \rightarrow S \cdot \forall i \in \mathbb{N} \cdot f(i) \succ f(i+1)} \\
An equivalent definition is that each subset has a minimum:\\
\m{\forall S' \subseteq S \cdot S' \neq \emptyset \Rightarrow \exists t \in S' \cdot \forall s \in S' \cdot s \succeq t}

\subsubsection*{Term Orderings}
A \newdef{simplification ordering} $\m{\succ}$ on a term algebra $\Ts{\sig}$ is a strict partial order on $\Ts{\sig}$ that satisfies:
\begin{itemize}
	\item Compatible with contexts (monotonic):\\
		\m{\forall s,t,c \in \Ts{\sig},p \cdot s \succ t \Rightarrow \termRepAt{c}{s}{p} \succ \termRepAt{c}{t}{p}}
	\item Stable under substitution: \\
		\m{\forall s,t \in \Ts{\sig},\sigma \in \subs{\Sigma} \cdot s \succ t \Rightarrow s\sigma \succ t\sigma}
	\item Sub-term compatible: \\
		\m{\rhd \subseteq \succ}
\end{itemize}
A \newdef{reduction ordering} is a well founded simplification ordering.\\
The lexicographic extension of an ordering on S to an ordering on $\m{S^n}$ for $\m{n>1}$ is defined as follows:\\
\m{\tup{s} \succ \tup{t} \equivdef \exists i \geq 0 \cdot s_i \succ t_i \land (\forall j<i \cdot s_j=t_j)}\\
For an ordering $\m{\succ}$ on S, we use the multiset extension of $\m{\succ}$, for a pair of finite multisets on $\m{S}, \m{m,n}$:\\
\m{m \succ n \equivdef \forall x \in S \cdot m(x) > n(x) \lor \exists y \succ x \cdot m(y) \succ n(y) }.\\
For a multiset \m{m} and an element \m{x \in S} we use:\\
\m{x \succ m \equivdef \forall y \in S \mid m(y)=0 \lor x \succ y}
%The multiset for an atom \m{s=t} is \m{s \mapsto 1,t \mapsto 1} and for \m{\fa{P}{t}} \s{

\noindent
We use a form of \emph{transfinite Knuth Bendix order} (\cite{WinklerZanklMiddeldorp12},\cite{KovacsMoserVoronkov11}).\\
We use $\newdef{\ords}$ for the set of ordinal numbers and $\newdef{\bigoplus,\bigotimes}$ for natural addition and multiplication on ordinals, respectively.

\noindent
The \newdef{Transfinite Knuth Bendix term ordering}, \newdef{\m{\succ_{tkbo}}} has two parameters:\\
A strict partial (potentially total) ordering \m{\succ} on the signature \m{F_\Sigma} (sometimes called a \emph{precedence}).\\
A $\newdef{\mbox{weight function}}$, $\newdef{\m{w:F_\Sigma \cup X_\Sigma \rightarrow \ords}}$ that satisfies:\\
\m{\forall f \in F_\Sigma \cdot w(f) > 0} and  \m{\forall x \in X_\Sigma \cdot w(x) = 1}.\\
Unless otherwise noted we will use a $\m{w}$ that satisfied\\
\m{\forall f \cdot \arity{f}>0  \Rightarrow \exists m \in \mathbb{N} \cdot w(f) = \omega\cdot m + 1}\\
\m{\forall f \cdot \arity{f}=0  \Rightarrow w(f) = 1}\\
that is, the function maps all constants to 1 and all non-constants only to direct successors of limit ordinals less than $\m{\omega^\omega}$. 
Note that it is not required that the precedence on function symbols agrees with the ordinal order on their weights.

\noindent
The $\newdef{\mbox{weight of a term}}$, $\newdef{\m{w(t)}}$, is defined recursively as:\\
\m{w(\fa{f}{s}) \triangleq w(f) + \bigoplus\limits_i w(s_i)}\\
For literals:\\
\m{w(s=t) \triangleq  w(s) \oplus w(t)}\\
\m{w(\fa{P}{t}) \triangleq  w(P) \oplus w(\tup{t})}

\noindent
We define the \newdef{multiset of variables of a term} \m{t}, \newdef{\m{\Vars{t}}}, recursively as follows:\\
$ 
\begin{array}{lll}
	\Vars{x}         & \triangleq & [x \mapsto 1]\\
	\Vars{\fa{f}{s}} & \triangleq & \bigcup\limits_\m{i} \Vars{s_i}\\
\end{array} 
$

\bigskip

\noindent
The transfinite Knuth Bendix ordering (tkbo) for terms we use is defined as follows:\\
\m{s \succ t} iff \m{\Vars{s} \supseteq \Vars{t}} and
\begin{itemize}
	\item \m{w(s) > w(t)} or
	\item \m{w(s) = w(t), s\equiv\fa{f}{s}, t\equiv\fa{g}{t}} and
		\subitem \m{f \succ g} or
		\subitem \m{f \equiv g} and \m{\tup{s} \succ \tup{t}}
\end{itemize}

\noindent
In order to extend the definition to literals and clauses, we extend the weight function to predicate symbols, and assume we have a precedence (total order) $\m{\succ}$ also on $\m{P_{\sigma} \cup \s{=}}$ - predicate symbols including the equality symbol.\\
This extension is total if $\succ$ is.\\
tkbo for literals is:\\
\m{l \succ l'} iff \m{\Vars{l} \supseteq \Vars{l'}} and
\begin{itemize}
	\item \m{w(l) > w(l')} or
	\item \m{w(l) = w(l')} where \m{l\equiv[\lnot]\fa{P}{s}, l'\equiv[\lnot]\fa{Q}{t}} and
	\begin{itemize}
		\item l is negative and \m{l\equiv \lnot l'} or
		\item \m{P \succ Q} or
		\item \m{P \equiv Q} and \m{\tup{s} \succ \tup{t}}
	\end{itemize}
\end{itemize}
In the above we used $\fa{P}{t}$ to denote also $\m{s=t}$ as $\m{=(s,t)}$.\\
We treat each clause as a multiset of literals and then use the multiset extension of $\succ$.\\
tkbo is total on ground terms.\\
tkbo also has the desirable property that it is \newdef{separating} for constants - 
that is, given a constant indexing function \m{ci(c):\consts{\Sigma} \rightarrow \mathbb{N}} we can assign \\
\m{\forall c \in \mathbf{const} \cdot w(c) = \omega\cdot \mathbf{ci}(c)+1}\\
Where, for any two term \m{s,t}, if the maximal constant index of \m{s} is greater than that of \m{t} then \m{s \succ t} regardless of size.
This property is important for completeness under scoping (for the ground fragment) as in ~\cite{KovacsVoronkov09},~\cite{McMillan08}.

\subsection*{Superposition}\label{section:preliminaries:superposition}
We have chosen to use superposition (\cite{BachmairGanzingerSuperposition}) as the underlying logical calculus.
The motivation for this choice is that superposition is a complete semi-decision procedure for FOLE (as opposed to many of the quantifier instantiation schemes used in SMT solvers), and it is known to be efficient in handling equalities in the presence of quantifiers.
An additional motivation is that it is possible to define fragments of superposition that, while not complete, 
have polynomial complexity. Most of our technique is also relevant for some other calculi, as we discuss in the relevant sections.
We present here only the ground fragment of superposition, and present the full superposition calculus when we discuss quantification.

\noindent
The main ideas of superposition can be described as follows:
The propositional part of superposition is based on ordered resolution for propositional logic.
Roughly, the main idea is to order the literals in each clause (and hence clauses by the multiset extension of the ordering) and for each pair of clauses with opposing maximal literals derive a smaller clause that encodes a case-split on the maximal literal.\\
Clauses implied by smaller clauses in the set are called redundant. When all case-split clauses have been derived and the empty clause has not been derived, a model of the set of clauses is the set of positive maximal literals of non-redundant clauses.

\noindent
For example, consider the following clause set (maximal literals are underlined):\\
$\s{C \lor \underline{A}, \underline{\lnot A} \lor D}$ \\
where $\m{A,\lnot A}$ are maximal in their respective clauses and do not occur in either of $\m{C,D}$.\\
We assume $\m{D \succ C}$ and hence $\m{\lnot A \lor D \succ C \lor A}$.\\
The clause $\m{C\lor D}$ derived by resolution on the maximal literal encodes the case split on $\m{A}$.\\
The new clause is smaller than both premises by the definition of multiset orderings.\\
The maximal literal $\m{l}$ of the new clause satisfies $\m{A \succ l}$ and $\m{l \in D}$ by the ordering.\\
Hence the model for the set of clauses is $\s{l,A}$.\\
The reason that we say the clause $\m{C \lor D}$ encodes the case-split is that if we had another singleton clause $\m{l'}$ in the set where $\m{l' \in C}$ then $\m{C \lor D,C \lor A}$ are redundant and hence the set of positive maximal literals of non-redundant clauses, $\s{l'}$, is a model.

\noindent
The complication added by equality is that an equality literal can conflict with a larger literal - for example, $\m{a=b}$ conflicts with the larger $\m{f(a)\neq f(b)}$, and the set $\m{a=b,b=c}$ conflicts with the larger $\m{f(a)\neq f(c)}$.\\
Superposition uses unfailing Knuth Bendix completion to ensure that in a clause set saturated for superposition the set of maximal positive literals of non-redundant clauses forms a convergent term rewrite system, and that all maximal terms of maximal literals of non-redundant clauses are reduced to their normal form by the  convergent term rewrite system defined by the maximal positive literals of all smaller clauses.\\
Resolution is replaced with term rewriting by maximal positive literals in order to ensure that the set of maximal literals of non-redundant clauses is a model.\\
The ground superposition calculus is shown in figure \ref{fig_ground_superposition}, we discuss the full calculus when we discuss quantification. The full calculus was shown sound and complete in \cite{BachmairGanzinger94}.

\bigskip

\noindent
A simple example of ground superposition (maximal terms are underlined):\\
For the set $\s{a=\underline{b},b=\underline{c},f(a)\neq \underline{f(c)}}$ with the ordering\\
$\m{f(c)\succ f(b) \succ f(a) \succ c \succ b \succ a}$ \\
superposition allows us to rewrite the term $\m{f(c)}$ to a smaller term using the equation $\m{c=b}$ - to get:\\
$\m{f(a) \neq \underline{f(b)}}$\\
And then, rewriting with the first clause:\\
$\m{f(a) \neq f(a)}$\\
And then\\
\emptyClause

\bigskip

\begin{figure}
$
\begin{array}[c]{llll}
%\vspace{10pt}
\mathrm{res_{=}} &\vcenter{\infer[]{\m{C       }                               }{\m{C \lor \underline{s\neq s}}                   }} & 
\parbox[c][1.8cm]{5cm}{}
\\
%\vspace{10pt}
\mathrm{sup_{=}} &\vcenter{\infer[]{\m{C \lor \termRepAt{s}{r}{p} =    t \lor D}}{\m{C \lor \underline{l}=r} & \m{\underline{s} =    t \lor D}}} & 
\parbox[c][1.8cm]{5cm}{
	\m{\sci{1}l = \termAt{s}{p}}\\
	\m{\sci{2}l \succ r,\sci{3}l=r \succ C}\\
	\m{\sci{4}s \succ t,\sci{5}s=t \succ D}\\
	\m{\sci{6}s=t \succ l=r}}\\
%\parbox[c][2cm]{4cm}{\sci{1}\m{l \succ r\sci{2}l=r \succ C}\\\sci{3}\m{s \succ t}\sci{4}\m{s=t \succ D}\\\sci{5}\m{s=t \succ l=r}}\\
%\vspace{10pt}
\mathrm{sup_{\neq}} &\vcenter{\infer[]{\m{C \lor \termRepAt{s}{r}{p} \neq t \lor D}}{\m{C \lor \underline{l}=r} & \m{\underline{s} \neq t \lor D}}} & 
\parbox[c][1.8cm]{5cm}{
	\m{\sci{1}l = \termAt{s}{p}}\\
	\m{\sci{2}l \succ r,\sci{3}l=r \succ C}\\
	\m{\sci{4}s \succ t,\sci{5}s=t \succ D}}\\
%\parbox[c][2cm]{4cm}{\m{l \succ r,l=r \succ C}\\\m{s \succ t,s \neq t \succ D}}\\
%\vspace{10pt}
\mathrm{fact} & \vcenter{\infer[]{\m{C \lor t \neq r \lor l=r }                }{\m{C \lor l = t \lor \underline{l} = r}}} & 
\parbox[c][1.8cm]{5cm}{
	\m{\sci{1}l \succ r,\sci{2}r \succ t}\\
	\m{\sci{3}l=r \succ C}}\\
%\parbox[c][2cm]{4cm}{\m{s \succ t,t \succ r}\\\m{s=t \succ C}}
%\vspace{10pt}
\end{array}
$
\caption{The ground superposition calculus \SPG\\
$\succ$ is a reduction ordering.\\
The numbered conditions on the right are the side conditions of each inference rule.\\
The calculus combines ordered resolution with unfailing Knuth Bendix completion.\\
Equality resolution ($\m{res_{=}}$) allows the elimination of maximal false literals.\\
Positive superposition ($\m{sup_{=}}$) ensures that the set of maximal positive literals of non-redundant clauses is a convergent rewrite system.\\
Negative superposition ($\m{sup_{\neq}}$) allows rewriting maximal dis-equalities by the term-rewrite system defined by maximal positive literals, and together with equality resolution is a generalization of ordered resolution for equality.\\
Equality factoring ($\m{fact}$) is a version of ordered factorting.
}
\label{fig_ground_superposition}
\end{figure}


\noindent
We use the notation $\newdef{\m{S \vdash_{X} C}}$ to denote that the clause $\m{C}$ is derivable in the calculus $\m{X}$ from the set of clauses $\m{S}$. When the calculus is clear from the context we use $\m{S \vdash C}$. In this section we only refer to the ground superposition calculus and hence we shorten $\m{\vdash_{\SPG}}$ to $\m{\vdash}$.

\subsubsection*{Redundancy elimination}
The superposition calculus is complete even when redundant clauses are eliminated according to a certain redundancy criterion.\\
The full superposition calculus was shown complete under the following redundancy criterion (here only a variant for ground clauses):\\
For a finite set of clauses $\m{S}$ and clause $\m{D}$, 
if $\m{S,D \vdash \emptyClause}$ and for some $\m{S' \subseteq S}$, $\m{S' \models D}$ and $\m{D \succ S'}$ (D is greater than all members of $\m{S'}$) then $\m{S \vdash \emptyClause}$ - that is, $\m{D}$ is redundant.

For the ground superposition calculus we use the simplifying inference rules shown in figure \ref{fig_superposition_simp}, all of which satisfy the above criterion. 
Most of the simplification rules are standard, and $\m{simp_{res},simp_{res2}}$ are chosen in order to handle the clauses that occur at join points in the program - we discuss these later.

\begin{figure}
$
\begin{array}[c]{llll}
%\vspace{10pt}
\m{unit} & \vcenter{\infer[]{\m{C}                            }{\m{\lnot A}  & \cancel{\m{C \lor A}}}} & \parbox[c][1.0cm]{3cm}{}\\
%\vspace{10pt}
\m{taut} & \vcenter{\infer[]{\m{}                             }{\cancel{\m{C \lor A \lor \lnot A}}}} & \parbox[c][1.0cm]{3cm}{}\\
%\vspace{10pt}
\m{taut_{=}} & \vcenter{\infer[]{\m{}                             }{\cancel{\m{C \lor s=s}}}} & \parbox[c][1.0cm]{3cm}{}\\
%\vspace{10pt}
\m{sub} & \vcenter{\infer[]  {\m{}                             }{\m{C} & \cancel{\m{C \lor D}}}} & \parbox[c][1.0cm]{3cm}{}\\
%\vspace{10pt}
\m{simp_{res}} & \vcenter{\infer[]{\m{C}                      }{\cancel{\m{C \lor A }} & \cancel{\m{C \lor \lnot A}}}} & \parbox[c][1.2cm]{3cm}{}\\
%\vspace{10pt}
\m{simp_{res2}} & \vcenter{\infer[]{\m{C \lor D}              }{\m{C \lor A } & \cancel{\m{C \lor D \lor \lnot A}}}} & \parbox[c][1.2cm]{3cm}{}\\
\m{simp_{=}} & \vcenter{\infer[]{\m{\termRepAt{C}{r}{p}}}{\m{l=r} & \cancel{\m{C}}}}   &
\parbox[c][1.2cm]{3cm}{\m{l=\termAt{C}{p}}\\\m{l \succ r}\\\m{C \succ l=r}}\\
\end{array}
$
\caption{simplification rules\\
$\cancel{\m{C}}$ denotes that the premise $\m{C}$ is redundant after the addition of the conclusion to the clause-set and hence can be removed.
}
\label{fig_superposition_simp}
\end{figure}

\subsection*{Congruence closure}
While superposition can decide the ground equality fragment, some techniques based on congruence closure are more efficient, 
and specifically efficient join algorithms have been developed for congruence closure.\\
We use two variants of the transitive reflexive congruence closure calculus \m{\mathbf{CC}} for unit ground\\ (dis)equalities.\\
The reason we mention two variants is that the first describes the operation of the graph structure we use and the second follows directly from the definition of congruence closure, and hence is used for completeness proofs.

\bigskip

\noindent
The first version $\m{\mathbf{CC}}$ is described in figure \ref{calculus_CC}. 
This calculus has a standard transitivity axiom and a version of equality resolution, 
but the less standard part is the congruence closure rule. 
This rule only allows instances of the general congruence closure rule if one of the terms in the conclusion already occurs in some clause.
The reason we use this version is that it describes the operation of a congruence closure (CC) graph (in the sense that the graph represents a set of clauses saturated w.r.t. the calculus) - performing congruence closure in a CC graph does not introduce new equivalence classes, although it may introduce new terms.

\begin{figure}
$
\begin{array}{lll}
%	\vspace{10pt}
	\m{tra_{\bowtie}} & \vcenter{\infer[]{\m{s \bowtie t}}{\m{s=u,u \bowtie t}}} & \parbox[c][1.5cm]{2cm}{}\\
%	\vspace{10pt}
	\m{res}           & \vcenter{\infer[]{\emptyClause }{\m{s=t, s \neq t}}} & \parbox[c][1.5cm]{2cm}{}\\
	\m{con}           & \vcenter{\infer[]{\fa{f}{s}=\fa{f}{t} }{\m{C} & \tup{s=t} }} &
\parbox[c][1.5cm]{2cm}{\m{\fa{f}{s} \lhd C}}\\
%	\vspace{10pt}
\end{array}
$
\caption{The $\m{\mathbf{CC}}$ calculus\\
We denote by $\m{\tup{s=t}}$ for two tuples $\m{\tup{s},\tup{t}}$ of the same arity the set of non-trivial equalities between corresponding elements of the tuples - formally:\\
$\m{\tup{s=t}}$ is the set $\s{s_i = t_i \mid i \in 0..\size{s}-1 \land s_i \not\equiv t_i}$.\\
The rule $\m{tra_{\bowtie}}$ is the transitivity rule.\\
The rule $\m{res}$ is similar to equality resolution.\\
The rule $\m{con}$ encodes standard congruence closure, 
except that the side condition $\m{\fa{f}{s} \lhd C}$, where C is any (dis)equality, ensures that no new equivalence classes are introduced in any derivation.
}
\label{calculus_CC}
\end{figure}

\noindent
For a set of unit ground equality clauses $\m{S}$, $\CC{S}$ is the closure of $\m{S}$ w.r.t. $\m{\mathbf{CC}}$ - we use a dedicated data structure to represent $\CC{S}$, described later. 

%\noindent
%The second version $\mathbf{CC_R}$ differs in that it does not introduce any new \emph{terms} (as opposed to no new equivalence cla, it differs from $\m{\mathbf{CC}}$ only in the congruence closure rule, whose version is described in figure \ref{calculus_CC_R}.
%
%\begin{figure}
%$
%\begin{array}{lllll}
%%	\vspace{10pt}
	%\m{con_R} & \vcenter{\infer[]{\fa{f}{s}=\fa{f}{t} }{\m{C} & \m{D} & \tup{s=t} }} & 
	%\parbox[c][1.5cm]{2cm}{\m{\fa{f}{s} \lhd C}\\\m{\fa{f}{t} \lhd D}}
%\end{array}
%$
%\caption{The \m{\mathbf{CC_R}} calculus\\
%The side condition ensures that no new terms are introduced in any derivation.
%}
%\label{calculus_CC_R}
%\end{figure}

\noindent
The second version, $\mathbf{CC_I}$, follows directly the definition of congruence closure. It differs only in the congruence closure rule , as described in figure \ref{calculus_CC_I}. We use this version in completeness proofs - a set of (dis)equalities is inconsistent iff it has a refutation in this calculus.
%We will discuss later how to determine the side condition for generating implied dis-equalities effectively, we only note here that for each \m{u \neq v} both \m{u,v} are sub-terms of \m{s,t} (at least one is proper) and for each of \m{s,t} at least one of \m{u,v} is a sub-term.


\begin{figure}
$
\begin{array}{llll}
	\vspace{10pt}
	\m{con_I} &
	\vcenter{\infer[]{\fa{f}{s}=\fa{f}{t} }{\tup{s}=\tup{t} }} & 
	\parbox[c][1.1cm]{4cm}{} \\
%	\vspace{10pt}
	%\m{das_I} &
	%\vcenter{\infer[]{\m{u \neq v}}{\m{s \neq t}}} & 
	%\parbox[c][1.1cm]{4cm}{\m{s \not\equiv t}\\\m{u \neq v \in das_u^{max}(s,t)}}\\
\end{array}
$
\caption{The \m{\mathbf{CC_I}} calculus\\
The rule \m{con_I} follows the definition of congruence closure.
%This calculus is complete for non-tautological consequences.\\
%$\m{das_u^{max}(s,t)}$ is the set containing the maximal (by entailment) dis-equalities implied by $\m{s\neq t}$.\\
%This set is the set of maximal unit disagreement sets of $\m{s \not\equiv t}$ - formally:\\
%$\m{das_u^{max}(s,t) \triangleq \s{ u \neq v \in das_u(s,t) \mid \forall u' \neq v' \in das_u(s,t) \cdot u' \neq v' \not\models u \neq v}}$\\
%And the unit disagreement sets are defined as:\\
%$\m{das_u(s,t) \triangleq \s{ u \neq v \mid u \not\equiv v \land s \neq t \models u \neq v }}$
}
\label{calculus_CC_I}
\end{figure}

\bigskip

\section*{Programs}\label{section:preliminaries_programs}
We assume as input a program in the Boogie (\cite{BarnettCDJL05}) intermediate verification language (or a similar IVL) that has been generated as a verification condition (VC) for some \newdef{source program} (and potentially some annotation).
We assume a low level Boogie representation that includes a DAG-shaped CFG (loops and method calls are removed using annotations) and the only statements are \lstinline|assume| and \lstinline|assert| (a passified program as described in \cite{Leino:2005:EWP:1066417.1710882}). CFG-nodes in the input represent basic blocks of the Boogie program, and often correspond to basic blocks of the source program.\\
We modify this input slightly by splitting each CFG-node at each assertion statement \lstinline|assert e| that occurs in it and replacing the assertion with an outgoing edge to a new leaf node with the statement \lstinline|assume $\lnot$e|. 
Now each CFG-node has only \lstinline|assume| statements and the order of statements within each CFG-node is unimportant - hence each CFG-node can be treated as a set of FOLE formulae. 
We convert this set of formulae per CFG-node (including the negated assertion nodes) to CNF form and now each CFG node is associated with a set of clauses. \\
For example, the source program in figure \ref{CFG_source_program} may be converted to the Boogie-style program \ref{CFG_Boogie} and is further converted to our representation \ref{CFG_ours}.
\begin{figure}
\begin{lstlisting}
$\m{n_0}$:
x:=0
y:=10
while (x<10)
	invariant x>=0 && x<=10 && x+y==10
	$\m{n_1}$:
	x:=x+1
	y:=y-1
$\m{n_2}$:
assert x+y<20
\end{lstlisting}
\caption{Example for VC encoding - source program}
\label{CFG_source_program}
\end{figure}

\begin{figure}
\begin{lstlisting}
$\m{n_0}$:
assume x$_0$=0
assume y$_0$=10
if (*)
	$\m{n_1}$:
	//loop head - assume loop condition
	assume x$_1$<10
	//assume loop invariant
	assume x$_1$>=0 && x$_1$<=10 && x$_1$+y$_1$==10
	//loop body
	assume x$_2$=x$_1$+1
	assume y$_2$=y$_1$-1
	//assert loop invariant (on current DSA versions)
	assert x$_2$>=0 && x$_2$<=10 && x$_2$+y$_2$==10
	//back edge is removed
	assume false
$\m{n_2}$:
//new DSA versions
//assume negated loop condition
assume !x$_3$<10
//assume loop invariant
assume x$_3$>=0 && x$_3$<=10 && x$_3$+y$_3$==10
assert x$_3$+y$_3$<20
\end{lstlisting}
\caption{Example for VC encoding - Boogie program\\
All variables have been split to DSA versions.\\
All assignments are converted to \lstinline|assume| statementes.\\
The loop return edge has been cut and the body begins with a fresh version for each variable,
an \lstinline|assume| of the invariant and negation of the loop condition, 
and ends with an \lstinline|assert| of the loop invariant on the latest DSA versions.\\
The code after the loop also uses a fresh DSA version of all variables and \lstinline|assume|s the invariant.
(we did not detail a modifies clause for the loop)
}
\label{CFG_Boogie}
\end{figure}

\begin{figure}
\begin{lstlisting}
assume x$_0$=0
assume y$_0$=10
if (*)
	$\m{n_1}$:
	assume x$_1$<10
	assume x$_1$>=0 
	assume x$_1$<=10
	assume x$_1$+y$_1$==10
	assume x$_2$=x$_1$+1
	assume y$_2$=y$_1$-1
	if (*)
		$\m{n_{1a}}$: //introduced assertion node
		assume $\lnot$x$_2$>=0 $\lor$ $\lnot$x$_2$<=10 $\lor$ $\lnot$x$_2$+y$_2$==10
		assert false
	assume false
else
	$\m{n_2}$: 
	assume !x$_3$<10
	assume x$_3$>=0 && x$_3$<=10 && x$_3$+y$_3$==10
	if (*)
		$\m{n_{2a}}$: //introduced assertion node
		assume $\lnot$x$_3$+y$_3$<20
		assert false
\end{lstlisting}
\caption{Example for VC encoding - our encoding\\
Showing that the post-states of both $\m{n_{1a},n_{2a}}$ is infeasible proves the Boogie program and hence the source program.
}
\label{CFG_ours}
\end{figure}

%The idea is that, for each CFG node $\m{n}$, for each path from the CFG-root to n, the set of all clauses on all CFG-nodes on the path is unsatisfiable iff the post-state of the statement represented at $\m{n}$ is unreachable in the Boogie program.

\noindent
We refer to the IVL program after our transformations as the \newdef{program} and the source language program as the source program. 

\subsection*{Structure}
The structure of our program is as follows:\\
A control flow graph - \newdef{CFG} - which is a directed acyclic graph with one root (the program entry point).\\
The leaf nodes of the CFG are the \newdef{goal nodes} - introduced per assertion. The goal of verification is to show them infeasible.\\
Each CFG-node $\m{n}$ is associated with a set of clauses - \newdef{\clauses{n}}.\\ 
The clauses at each non-leaf CFG-node represent an encoding of the transition relation of the original program, or some instrumentation used by the verification condition generator to generate the IVL program.\\
The clauses at each leaf node represent the negation of an assertion generated for the VC, as described above.

\noindent
We use the following functions to refer to the CFG structure - for a given CFG-node $\m{n}$:\\
\newdef{\succs{n},\succsto{n},\succst{n}} are the direct, transitive and reflexive-transitive successors of $\m{n}$, respectively.\\
Similarly, \newdef{\preds{n},\predsto{n},\predst{n}} are the corresponding sets for predecessors.

\subsubsection*{CFG paths}
For a program CFG $\m{G}$, a directed $\newdef{\m{path}}$ $\m{P}$ in the $\m{G}$ is a (possibly empty) sequence of nodes s.t. \\
$\m{\forall 0 \leq i < \size{P} \cdot P_{i+1} \in \succs{P_i}}$.\\
$\size{P}$ is the length of the path.\\
We use $\newdef{\paths{G}{}}$ for the set of all directed paths in $\m{G}$, starting at any node and ending at any transitive successor of the node, including one and zero length paths.\\
For a node n and transitive predecessor $\m{p \in \predst{n}}$,
$\newdef{\paths{p}{n}}$ are all the paths in $\m{G}$ that start at $\m{p}$ and end at $\m{n}$, including the case $\m{n=p}$.\\
$\newdef{\paths{n}{}}$ is short for $\paths{root}{n}$.

\noindent
For a path $\m{P}$, the set of all clauses in all nodes on the paths is denoted by $\newdef{\clauses{P}}$ - formally:\\
$\m{ \clauses{P} \triangleq \bigcup\limits_{n \in P} \clauses{n}}$.
%Note that the program point after the last statement in a CFG-node is reachable on the path $\m{P}$ iff $\clauses{P}$ is consistent.

\noindent
For a set of clauses $\m{S}$ we use $\newdef{\Eqs{S}}$ for the subset $\m{S}$ that is unit equalities, dis-equalities and the empty clause - formally:\\
$\m{\Eqs{S} \triangleq (S \cap \s{\emptyClause}) \cup \s{u \bowtie v \mid u \bowtie v \in S}}$


\subsection*{Semantics}
\subsubsection*{Traces}
A \newdef{trace} is a pair $\m{(P,M)}$ where $\m{P}$ is a path from the root-node to some node $\m{P_{end}}$ and $\m{M}$ is a model for the signature of the clauses of the program, s.t. for each node on the path, the clauses at the node are satisfied - formally:\\
$\m{\forall n \in P \cdot M \models \clauses{n}}$. $\m{M}$ encodes the values of program variables.

\subsubsection*{Validity}
For a given path $\m{P}$ from the CFG-root to a node $\m{n}$ and a clause $\m{C}$, 
we say that $\m{C}$ holds at $\m{n}$ on $\m{P}$ - $\newdef{\m{n \models_P C}}$ - if $\m{C}$ is entailed by $\clauses{P}$ - 
this means that $\m{C}$ holds at the post-state of $\m{n}$ for any trace that passes through $\m{P}$ - formally:\\
$\m{n \models_P C \equivdef n = P_{end} \land \clauses{P} \models C}$.

\noindent
A clause $\m{C}$ holds at a node $\m{n}$ - $\newdef{\m{n \models C}}$ - if $\m{C}$ holds holds on every trace reaching $\m{n}$ - formally:\\
\m{n \models C \triangleq \forall P \in \paths{root}{n} \cdot \clauses{P} \models C}\\
A node n is $\newdef{\m{infeasible}}$ iff $\m{n \models \emptyClause}$\\
A program is $\newdef{\m{valid}}$ iff all its assertion nodes are infeasible - denoted $\newdef{\m{\models P}}$ for a program $\m{P}$.

\subsubsection*{Program transformations}
Our verification algorithm works by manipulating the set of clauses at each CFG-node, and sometimes the CFG-structure itself, until there are no assertion nodes left. 
We describe a set of program transformations that include both the manipulation of the set of clauses at nodes and the CFG-structure.\\
In order for the verification algorithm to be sound, it must only apply invalidity-preserving transformations to the program, we call these \newdef{sound transformations} - formally:\\
A transformation T of a program $\m{P}$ is sound iff $\m{~\models P ~ \Leftarrow ~\models T(P)}$.\\
Conversely, a \newdef{complete transformations} preserve validity - intuitively not losing information - formally:\\
A transformation of a program $\m{P}$ to a program $\m{P'}$ is complete iff\\ $\m{~\models P ~ \Rightarrow ~\models T(P)}$.
For example, removing any CFG-node whose clause-set contains the empty clause is a sound and complete transformation, as is adding a clause to a node's set of clauses that is entailed by this set of clauses.

\noindent
All of our transformations satisfy a stronger property than soundness and completeness:\\
A transformation of $\m{P}$ to $\m{P'}$ is \newdef{conservative} if it is sound and complete and, for each CFG-node $\m{n}$ that occurs in both $\m{P}$ and $\m{P'}$ and each clause $\m{C}$ in the vocabulary of P (containing only symbols that occur in P), $\m{C}$ holds at $\m{n}$ in $\m{P}$ iff it holds at $\m{n}$ in $\m{P'}$.\\
The reason this property is interesting is that it allows incremental verification for some program and vocabulary extensions.
For example, if a node $\m{n}$ has exactly two successors $\m{p_1,p_2}$ and $\m{C \in \clauses{p_1} \cap \clauses{p_2}}$, 
we can modify $\clauses{n}$ by adding $\m{C}$ to it, which is both sound and complete, but not conservative, as $\m{n \models C}$ in $\m{P'}$ but not necessarily in $\m{P}$.

We use mostly two kinds of transformations:\\
\newdef{Inference} - this transformation, for a given node n, replaces the set $\m{S \triangleq \clauses{n}}$ with a new clause-set $\m{S'}$ s.t. $\m{S \vDash S'}$ and $\m{S' \vDash S}$. The inference transformation is conservative by definition. 
In all the cases we consider, $\m{S'}$ is the result of applying some inference rules from some logical calculus to $\m{S}$ (including simplification rules that remove redundant clauses).\\
\newdef{Propagation} - this transformation propagates clauses from the direct predecessors of a node to the node. 
For example, a CFG-node with one predecessor can add any clause in its predecessor's clause set to its own clause set while being conservative. For nodes with more than one predecessors, propagation can only add a clause to the node's clause-set if it is entailed by the clause-sets of \emph{all} its direct predecessors - we discuss such joins for each logical fragment we consider.

\subsection*{Joins}
A CFG-node with more than one direct predecessor is called a join node.
In general, it is not sound to add a clause to a join node that does not occur in all predecessors, hence we need some mechanism to propagate information at joins.
In order to be able to perform propagation in a sound and complete way for join nodes, we modify the Boogie program as follows:\\
We ensure that each branch and join in the program is binary - any n-ary branch or join is cascaded to a binary tree of binary branches or joins. This cascading of branches and joins is a conservatives program transformation.

\subsubsection*{Branch conditions}
For each binary branch b, we add a \newdef{branch condition atom} \m{P_b} which is a fresh nullary predicate symbol.\\
For a binary branch node $\m{b}$ with successors $\m{s_1,s_2}$ s.t. \m{s_1,s_2} have a common transitive successors, we add the clause $\m{P_b}$ to $\m{s_1}$ and $\m{\lnot P_b}$ to $\m{s_2}$.\\
Note that the transformation that adds branch conditions is conservative as it only \lstinline|assume|s new fresh literals.

For a binary join node $\m{j}$ with two predecessors $\m{p_1,p_2}$, if, for some path condition atom $\m{P_b}$, $\m{P_b \in \clauses{p_1}}$ and $\m{\lnot P_b \in \clauses{p_2}}$, and also for some clause $\m{C}$, $\m{C \in \clauses{p_1}}$, it is sound to add the clause $\m{\lnot P_b \lor C}$ to $\clauses{j}$. We call the clause $\m{\lnot P_b \lor C}$ a $\m{\newdef{relativized}}$ version of $\m{C}$.\\
As we show later, inference and propagation can form a complete verification procedure if the above condition holds for all join nodes - that is, for each join node some $\m{P_b}$ holds at the first predecessor and $\m{\lnot P_b}$ at the other.


\subsubsection*{Well-branching programs}
We define a class of programs for which completeness by propagation and inference can be shown - the class of well-branching programs.\\
Intuitively, a program is well-branching if each join joins exactly one branch.\\
Formally, a \newdef{good-join} is a binary join node $\m{j}$ with direct predecessors $\m{p_1,p_2}$ s.t. the set $\predsto{p_1} \cap \predsto{p_2}$ has a single maximum $\m{m}$ w.r.t. the topological order on the CFG, and each path from the root to $\m{j}$ passes through $\m{m}$. We call this maximum the \newdef{corresponding branch} of the join.\\
A \newdef{well branching program} is a program where all joins are good joins.\\
It is easy to see that a well branching program always satisfies the above condition for joins - namely, the branch condition of the corresponding branch is always at opposite polarities at the predecessors of a join.\\
When cascading branches and joins, we try to ensure that the resulting program is well-branching if possible.\\
In our experience, the VC of programs without exceptional control flow is often well-branching.\\
If a program is not well-branching, the addition of branch conditions still allows completeness of propagation, 
but the relativization of clauses is less efficient.


\subsubsection*{Path conditions}\label{section_path_condition}
For a well branching program, we define the \newdef{path condition} of a CFG-node, \newdef{\pc{n}}, as the set (conjunction) of the branch conditions that hold at the node - formally:\\
$\m{\pc{n} \triangleq \textbf{ if } n=\m{root} \textbf{ then } \emptyset \textbf{ else } \lpc{n} \cup \bigcap\limits_{p \in \preds{n}} \pc{p}}$\\
where \newdef{\lpc{n}} is the local path condition of \m{n}, which is $\m{P_b}$ for one successor of a branch b, ${\m{\lnot P_b}}$ for the other successor, and $\emptyset$ for all other nodes.

When a clause is propagated through several nodes, in some of which it is relativized, it can collect several branch literals on the way - the \textcolor{blue}{relative path condition} - $\newdef{\rpc{p}{n}}$ - is intuitively the set of branch literals added to the clause when it is relativized on the path from p to n - formally:\\
$\rpc{p}{n} \triangleq \pc{p} \setminus \pc{n}$

The path condition and relative path conditions can be defined also for non well-branching programs, but the definition is more complicated.




%\section*{Fragments}
%A \emph{program logic fragment} \m{F=(C_F,\vdash_F,\sqcup_F)} is:
%\begin{itemize}
	%\item A subset \m{C_F} of \Cs{\sig} for the \sig{}
	%\item A derivation relation \m{\vdash_F} on sets of \m{C_F}
	%\item A maximal join \m{\sqcup_F : P(C_F) \times \bigcup\limits_n (P(C_F))^n \rightarrow P(C_F)} \\
	%where n ranges between 0 and the maximal degree of a join (in our case always 2), that satisfies \m{\forall c \in \sqcup_F(C,P) \cdot \forall i \cdot ( \emptyClause \in P_i \lor C \cup P_i \vdash_F c)} - \\
	%basically this set includes any clause that is derivable by the fragment on all paths, but might include less
%\end{itemize}
%
%\noindent
%The fragment of ground unit equalities with strong join, \m{u}, is:
%\begin{itemize}
	%\item \m{C_{u} = \s{C \in \Cs{\sig} \mid C=\emptyClause \lor \exists s,t \cdot C=s \bowtie t }}
	%\item \m{\vdash_{u}} is \m{\vdash_{\mathbf{CC_I}}}
	%\item \m{\sqcup_{u}(C,P) = \s{c \in C_{u} \mid \forall i \cdot C \cup P_i \vdash_{u} \emptyClause \lor C \cup P_i \vdash_{u} c }} - 
	%that is, a full join
%\end{itemize}
%We will use also a weaker fragment, \m{w} as follows:
%\begin{itemize}
	%\item \m{C_{w} = \s{C \in \Cs{\sig} \mid C=\emptyClause \lor \exists s,t \cdot C=s \bowtie t }}
	%\item \m{\vdash_{w}} is \m{\vdash_{\mathbf{CC}}}
	%\item \m{\sqcup_{w}(C,P) = \s{c \in C_{w} \mid \forall i \cdot C \cup P_i \vdash_{CC} \emptyClause \lor C \cup P_i \vdash_{CC} c }} - 
	%roughly, \\this join can only infer equalities at the join for terms that appear in either the clauses at the join node or in all predecessors
%\end{itemize}
%
%\noindent
%We define a \emph{fragment DAG interpolant} for verifying a subset of the leaf nodes \m{A}:\\
%A fragment DAG interpolant I for a logical fragment F is a mapping from nodes to finite sets (conjunctions) of clauses, where \m{I_n} is the set of these clauses at node n (distinct from \clauses{n}), that satisfies, for each node n:\\
%$
%\m{I_n \in C_F(\sig{n})}\\
%\m{\sqcup_F(\clauses{n},\m{i \mapsto I_{\preds{n}_i}}) \vdash_F I_n}\\
%\m{\forall n \in A \cdot \emptyClause \in I_n}
%$\\
%Where \m{i \mapsto I_{\preds{n}_i}} is the sequence of interpolants of direct predecessors of the node n, for some arbitrary (but fixed) order on \preds{n}.\\
%\sig{\m{n}} is the signature at node \node{n}, until discussing scoping we assume all the nodes have the same signature 
%(the clauses \clauses{n} are always of the signature \sig{\m{n}}).
%For a given cfg and fragment F, the set of all interpolants in the fragment \m{F} is \m{I_F}.
%
%%\noindent
%%To see why we need a separate join operation, consider the following alternative fragment \m{w}:
%%\begin{itemize}
	%%\item \m{C_{w} = \s{C \in \Cs{\sig} \mid C=\emptyClause \lor \exists s,t \cdot C=s \bowtie t }}
	%%\item \m{\vdash_{w}} is \m{\vdash_{\mathbf{CC_R}}}
	%%\item \m{\sqcup_{=}(C,S) = \s{c \in C_{=} \mid \forall P \in S \cdot P \vdash_{w} c}} - 
	%%that is, a join only for common terms
%%\end{itemize}
%%This would be a simpler join that just takes the intersection of the closure of clauses on both sides, with respect to the calculus \m{\mathbf{CC_R}} - importantly it can only infer an equality at the join for terms that appear onb both sides, but is strictly weaker as we will show later.\\
%%In our experience, we have not found many cases where the stronger join produced additional information that actually helped proving programs, and the simple join is easier to implement - we will discuss this later.
%
%\subsubsection*{Provability in a fragment}
%For a given program logic fragment \m{F}:\\
%\m{n \models_F C \triangleq \exists I \in I_F \mid C \in I_n}\\
%And the program is \emph{within the fragment} if there is a fragment interpolant \m{I \in I_F} s.t. for each assertion node \m{n}, \m{\emptyClause \in I_n}.


\section*{Equivalence classes}
For a given set of clauses S, we overload the meaning of \Eqs{S} to denote also the congruence relation defined by the reflexive transitive congruence closure of the unit ground equalities in \Eqs{S}.\\
The set of equivalence classes of terms of a set of clauses is defined as:\\
\m{\ECs{S} \triangleq \terms{S}/\Eqs{S}}.\\
For a CFG-node $\node{n}$ we use $\ECs{n}$ for $\ECs{\clauses{n}}$ - the set of equivalence classes of terms that occur in clauses at n according to the congruence relation defined by unit clauses at n.

For a congruence relation R we use the notation \m{[t]_R} to denote the equivalence class of t in R. We drop the subscript when it is clear from the context.

\noindent
A desirable property of the calculus $\m{\mathbf{CC}}$ is that
$\size{\ECs{S}} \geq \size{\ECs{CC(S)}}$.
In fact, if $\m{C}$ is the result of a derivation with premises in $\m{S}$, and $\m{S'=S \cup \s{C}}$, then $\size{\ECs{S}} \geq \size{\ECs{S'}}$, so the set of equivalence classes does not grow from applying derivations in the calculus.
This property is immediate from the definition of the calculus, as for each rule, for each sub-term of the conclusion, either the sub-term occurs in the premises, or the conclusion equates it to a term that occurs in the premises.

\noindent
\subsubsection*{Atomic ECs}
%Our algorithm maintains a\newdef{partial equivalence relation}at each CFG-node. 
%A partial equivalence is defined by a set of ground equations E and a set ground terms T (that includes at least all terms that occur in the equations). R is a partial relation on $\Ts{\Sigma}$ that is only defined on $\m{T^2}$. 
%For a pair of terms either of which is not in T, the relation 

Our algorithm annotates each CFG-node with an approximation of an congruence relation, and the approximations at adjacent CFG-nodes are often similar (agree on many pairs of terms). We use the following concepts to describe the approximation and the relation between similar congruence relations

For a given congruence relation on ground terms we define the set of \newdef{atomic ECs} - \newdef{\AEC} - which are the smallest sets of terms out of which equivalence classes can be constructed, and the smallest unit that is potentially common with stronger congruence relation.\\
Given a congruence relation R and a set of terms T, an EC-tuple $\tup{s}$ is a tuple of equivalence classes of T in R - $\m{\tup{s} \in (T/R)^{\arity{f}}}$.\\
The atomic EC $\fa{f}{s}$ for an EC-tuple $\tup{s}$ is a set of terms defined as:\\
\m{\terms{\fa{f}{s}} \triangleq \s{\fa{f}{t} \mid \bigwedge\limits_i t_i \in s_i}} \\
The set of such atomic ECs for a congruence $\m{R}$ is \newdef{\AECs{R}}.\\
By the definition of congruence closure, all terms of an AEC are in the same EC of R - formally:\\
$\m{\forall \fa{f}{s} \in \AECs{R},\m{t} \in \terms{\fa{f}{s}} \cdot \fa{f}{s} \subseteq [t]_{R}}$.\\
However, an EC of R may include more than one AEC.\\
For example, in the congruence defined by $\m{S = \s{a=b,f(a)=g(c)}}$, the set of ECs of terms of S\\ ($\m{\terms{S}/R}$) are:\\
$\s{ \s{a,b}, \s{c}, \s{f(a),f(b),g(c)}}$ while the set of AECs of terms of S is \\
$\s{a(),b(),c(),f(\s{a,b}),g(\s{c})}$.\\
This hints also at another property of AECs - they allow us to share some of the representation of two similar congruence relations (that is, relations that agree on some subset of equalities). 
In our setting this is most often the case of the sets of possible AECs for the congruence relations that hold at two consecutive CFG-nodes - for example: \\
For the set S above and the set $\m{S' = S \cup \s{c=d}}$,
the set of AECs of $\m{S'}$ is\\ $\s{a(),b(),c(),d(),f(\s{a,b}),g(\s{c,d})}$.\\
If $\m{S}$ is the set of clauses of a node and $\m{S'}$ is the set of clauses of a direct successor (in the CFG) of that node, they can share the common AECs\\ $\m{a(),b(),c(),f(\s{a,b})}$ while they can only share the equivalence class $\m{\s{a,b}}$.\\
For a given congruence $\m{R}$, the sets of terms of AECs are disjoint and each equivalence class is a disjoint union of sets of terms of AECs.
Our congruence closure calculus $\m{\mathbf{CC}}$ does not generate any new AECs - the only rule that may introduce a new term (con) does not introduce a new AEC.\\
We will use the number of AECs as the main space complexity measure as our data structure is based on AECs and, for all the other congruence closure algorithms that we are aware of, the space complexity is at least the number of AECs, possibly more (this is similar to measuring the size of a fully reduced set of equations as in ~\cite{GulwaniTiwariNecula04}).
%In most cases we do not consider the (largest) function arity as a complexity factor as it does not change asymptotic behaviour, 
%but in the few cases where it does we mention it. 
%However, in our experience, function arity can strongly affect actual performance, and we will discuss it in the relevant section.\\
%For a set of functions \m{F} and a set of equivalence classes \m{E}, the maximal number of AECs is:\\
%\m{\sum\limits_{f \in F} \size{E}^{\arity{f}}} so if $n=max\s{\arity{f} \mid f \in F}$ then the upper bound is \\
%\bigO{\size{F}\size{E}^n},
%hence the size of the representation of each such AEC \fa{f}{s} is \arity{f} (references to term equivalence classes), so the total complexity is at most \\ \m{\sum\limits_{f \in F} \arity{f}\size{E}^{\arity{f}}}, which, by sharing such tuple ECs can be reduced to 
%\m{\sum\limits_{f \in F} \size{E}^{\arity{f}}}.\\
%However, new AECs are only introduced by introducing new clauses, not by congruence closure derivations, and is at most linear in the sum of sizes of clauses in \m{S}.\\
%For an arity \m{n} we expect many equivalence class tuples to participate in more than one AEC - for example:\\
%In \m{f(a,b)=g(a,b),a=c} the EC tuple \m{(\s{a,c},\s{b})} is used twice,
%we can also share EC tuples to reduce overall complexity, but the dominant part of the complexity measure is still \bigO{\size{E}^n}.\\
%Another important property of this complexity measure is that it is agnostic to the order of derivations - it only depends on the current set of clauses - for example, compare:\\
%First assuming \m{f(a)=c,f(b)=d} and then \m{a=b} - the final result will have only one AEC with the function symbol \m{f}, while some union find data structures will have two\\
%with - first assuming \m{a=b} and then \m{f(a)=c,f(b)=d} - many union data structures will only have one such function edge.

\subsection*{Proofs and models}
A \newdef{proof tree} for a logical calculus and a set of axioms is a tree with an instance of an inference rule from the calculus at each nodeץ, where the conclusions of the children of each node are the premises of the inference rule instance at the node. The leaves of the tree are axiom nodes.
A refutation tree is a proof tree where the conclusion of the root is a contradiction - in CNF form this is usually the empty clause.
A \newdef{proof DAG} is similar to a proof tree where the difference is that the conclusion of a node can be used as the premise of more than one parent. A non-redundant proof DAG is a proof DAG where no two nodes share the same conclusion.

For a given logical calculus, set of axioms and theorem, the minimal proof depth is the minimum of depth for all proof-trees (and, equivalently, proof-DAGs) of the theorem from the axioms in the calculus. The depth of a tree or DAG is the length of the longest path from the root to a leaf. The size of a proof-DAG is the number of nodes it contains. 
The minimal proof size for a given theorem, calculus and set of axioms is the minimal size of proof-DAG for the theorem from the axioms, and similarly for depth.

For each of the automated theorem proving techniques, when a refutation is obtained, a proof-DAG (possibly redundant, depending on the ATP technology) for the refutation can be extracted in the calculus used by the ATP. For example, the original DPLL algorithm produces proofs in tree from, where the minimal proof size can be exponentially larger than an equivalent non-redundant DAG-proof. CDCL produces DAG proofs (\cite{DBLP:conf/aaai/HertelBPG08}). The lower bound on time complexity of an ATP run on a problem is related to the minimal proof size in the ATP's calculus, but also to the size of the proof search space - preventing the prover from considering proofs with a highly redundant proof-DAG often accelerates proof search.

%
%\subsection*{Theories}
%Many computer programs and most program VCs make use of some form of linear integer arithmetic.
%Some programs that use floating point numbers are modeled using rational arithmetic, for which linear arithmetic has the most developed decision procedures.
%Arrays can be modeled using read-over-write axioms (\cite{Mccarthy62towardsa}), which suffice for a large class of programs.
%When array extensionality is required the situation is more complicated, and with integer indices decidability and complexity depends on the fragment of integer arithmetic used.
%Floating point arithmetic is sometimes modeled as rational arithmetic, for which the linear fragment has efficient decision procedures.
%Linear rational arithmetic finds also other uses, such as modeling permissions (\cite{Boyland:2014:CSA:2635631.2635847}).
%
%\subsubsection*{Complexity and decidability}
%The problem of deciding the validity of a formula in FOLE is semi-decidable while the problem for the ground fragment GFOLE is NP complete - the satisfiability of a conjunction of unit ground positive and negative equations (the unit ground theory of equality with uninterpreted functions) can be decided in \bigO{nlgn}time, while the unit quantified theory of equality with uninterpreted functions is undecidable even for unit clauses, but is semi-decidable.
%The theory of quantified linear integer arithmetic (LIA) is decidable but a lower bound for the decision problem is double exponential(complexity and decidability results are summarized in \cite{Bradley:2007:CCD:1324777} section 3.7). The unit ground (also called quantifier free conjunctive) theory of linear integer arithmetic (QF\_LIA) is NP complete - this theory is important for the verification of many computer programs and hence most SMT solvers include a decision procedure for it.
%The theory of quantified linear rational arithmetic (LRA) is decidable and an exponential lower bound has been shown.
%The unit ground version (QF\_LRA) has polynomial complexity and, while not directly useful in programs that do not manipulate rational (or, as an approximation, fixed point) numbers, it can be useful for modeling parts of the VC of programs such as permissions.
%The quantified theory of arrays without extensionality is undecidable and the unit ground theory is NP-complete - arrays are import as they are a common data structure in computer programs and they can sometimes be used to model memory. Extensionality for arrays is important in some verification contexts and the unit ground theory of arrays with extensionality is NP-complete (\cite{932480}).
%


\section*{Background}
We discuss here briefly some of the main existing technologies for theorem proving for first order logic with equality (FOLE) and some of its sub-fragments, as FOLE has proven to be sufficiently expressive for many verification tasks and there are practical automatic theorem provers that support it. 
Many provers also support directly extensions of FOLE that are either not directly axiomatizable in FOLE or for which no efficient axiom system has been found - specifically, rational and integer linear arithmetic and also bit-vectors, arrays and strings.

A notable exception to the expressiveness of FOLE is that it cannot express transitive closure and least fixed points useful for describing heap structures - however, transitive closure logics are beyond the scope of this work, and most are even not semi-decidable (see e.g. \cite{DBLP:journals/aml/GradelOR99},\cite{DBLP:conf/csl/ImmermanRRSY04}). Verification systems for heap manipulating programs often use either recursive data structures or restricted forms of quantification (\cite{MuellerSchwerhoffSummers16b}) 
in order to utilize the more mature field of FOLE theorem proving.

\subsection*{Automated theorem provers}
Most current theorem provers for FOLE or its fragments are based on some form of resolution, either on the DPLL algorithm (\cite{DBLP:journals/cacm/DavisLL62}) or on superposition (\cite{BachmairGanzingerSuperposition}) and ultimately resolution with unification (\cite{Robinson:1965:MLB:321250.321253}). Virtually all current theorem provers prove the validity of a formula by showing that its negation is unsatisfiable.

\subsubsection*{SMT solvers}
DPLL based SMT solvers (e.g. \cite{DBLP:conf/cav/BarrettCDHJKRT11},\cite{DBLP:conf/tacas/MouraB08}) generally work by trying to build a model for a formula. The model is constructed for a propositional formula by \textit{deciding}, at each step, the propositional value of one ground literal that occurs in the original formula (a literal is an atom or its negation, where the definition of an atom depends on the logic used - for FOLE an atom is an equality on terms). Once a literal has been decided, the formula is simplified by replacing all occurrences of the literal with the selected Boolean value and simplifying the formula accordingly (the formula is often kept in CNF form in which case simplification is simply unit propagation) - simplification may produce further unit clauses that are also propagated. 
If a contradiction is reached at any point (for CNF - a contradiction is the empty clause), the algorithm backtracks to the last decision and \textit{learns} the negation of the decision - adding the negated literal to the set of clauses and simplifying clauses accordingly. If learning the negated literal also produces a contradiction the previous decision is reversed. If the negated top decision produces a conflict then the formula is not satisfiable.
If all literals have been decided and no contradiction has been found then the set of decisions and simplified clauses define a model for the formula - at any point in the algorithm the set of decided and derived literals is called the candidate model. 

\textbf{Conflict driven clause learning:} A major improvement in the DPLL algorithm which has made it practical is conflict driven clause learning (CDCL) (\cite{GRASP}). CDCL improves upon DPLL by maintaining a graph that keeps track of the implications between literals that have been decided, and hence allows the algorithm to \textit{learn} a new clause when a conflict is reached that includes only the negation of the decisions that have caused the conflict. The algorithm backtracks to the first decision involved in the conflict, rather than to the last decision as in DPLL. CDCL was shown to be exponentially more efficient than the original DPLL (\cite{DBLP:conf/aaai/HertelBPG08}).

An important property of the propositional DPLL algorithm is that the space requirement is at most proportional to the number of literals in the input formula (the sequence of decided and derived literals). On the other hand, the algorithm can progress - learn new information - only when a conflict is reached. For propositional logic and GFOLE the depth of the search is bounded by the size of the input, but when quantifiers are involved the depth of the search is not bounded.

Nelson and Oppen have added the efficient handling of equality to the ground DPLL procedure above 
and the handling of some ground theories for which satisfiability is decidable (\cite{DBLP:journals/toplas/NelsonO79}).
The algorithm handles equality by using a congruence closure (CC) data structure which allows deciding and backtracking equality literals.
The congruence closure data structure detects contradictions by maintaining dis-equality edges and is, at any stage of the algorithm, the candidate model for equality. Ground theories are decided by sending each theory decision procedure the subset of the candidate model that is relevant, with the non-theory symbols abstracted away and, if no contradiction is found, all equality literals between terms in the original formula are decided until either a conflict is found or a full model is found which is agreed by all theories. Several improvements have been proposed to this algorithm, notably allowing a theory decision procedure, when it reports a set of literals as satisfiable, to also report the set of (dis)equalities that are implied by the set of literals (over the terms that occur in the literals), in a similar fashion to unit propagation.

\textbf{Shostak theories:} An alternative method for combining ground propositional and equality reasoning with a theory was introduced by Shostak (\cite{Shostak84}).
The Shostak method requires, instead of a just a satisfiability decision procedure, a \textit{canonizer} that rewrites any theory term to a normal form (where two theory terms are equal iff they have the same normal form) and a \textit{solver} that decides satisfiability for a set of literals in normal form, and outputs a \textit{solved form} of the input literals, which intuitively means that it expresses one of the input variables as a function of the rest of the variables (variables for the theory are the top non-theory ground terms that occur in the theory literals). The Shostak method is used in the theorem prover Alt-Ergo (\cite{DBLP:journals/entcs/ConchonCKL08}).


\textbf{DPLL for first order logic:}
The original DPLL algorithm applied to quantified FOL formula by repeatedly generating ground instances of the quantified clauses and applying the DPLL algorithm incrementally to the set of ground instances - this extension to quantified FOL has not seen much practical use. Current SMT solvers usually use some heuristics to create ground instances of quantified clauses (e.g. \cite{DBLP:conf/cade/ReynoldsTGKDB13},\cite{DBLP:conf/cav/GeM09}). One common heuristic uses patterns (also called triggers), where, for each quantified clause, a set of terms over the free variables that occur in the clause is selected - the \textit{triggers}, and a ground instance of the clause is generated only if the ground instance of each trigger is E-unifiable (unifiable modulo the equality theory encoded in the CC data structure) with a term represented by the CC data structure.\\
Another approach for handling quantifier in SMT solvers is model-based quantifier instantiation (MBQI) (\cite{DBLP:conf/cav/GeM09}) which determines a set of relevant ground instances (potentially infinite) for each quantified clause and incrementally tries to find a model for a subset of the relevant instances.
If any such subset is unsatisfiable then the original formula is unsatisfiable, otherwise a model is found which can be used to guide further instantiation. 
Current SMT solvers tend to be very efficient on problems involving propositional and ground first order logic, and the ground linear integer and rational theories. Problems including quantifiers are sometimes not handled as efficiently by SMT solvers, and especially the performance of the prover is very sensitive to small changes in the input. 

\textbf{Incremental or Lazy CNF conversion:}
Another technique that contributes significantly to the success of SMT solvers is incremental or lazy CNF conversion (\cite{demoura2007relevancy},\cite{Detlefs:2005:STP:1066100.1066102}).
The basic algorithms of DPLL and its CDCL variant operate on a formula in CNF form (a set (conjunction) of clauses (disjunctions of literals)). In order to convert an arbitrary propositional logic (PL) formula including arbitrary logical connectives in polynomial time and space, it is sometimes necessary to introduce new literals that represent the (propositional) value of sub-formulae (\cite{Baaz2001273}), and add clauses to the result that encode the equivalence between the newly introduced literal and the sub-formula - for example: \\
$A \lor (B \land C)$ \\
is converted to\\
$(A \lor L_1) \land (L_1 \Leftrightarrow (B \land C))$\\
which is then converted to CNF as\\
$\{A \lor L1, L_1 \lor \lnot B \lor \lnot C, \lnot L_1 \lor B, \lnot L_1 \lor C\}$.\\
$L_1$ is the newly introduced literal and the formula $L_1 \Leftrightarrow (B \land C)$ is the definition of $L_1$.

The original DPLL algorithm converts the negation of a formula to a CNF set of clauses and then performs steps of deciding a literal, simplifying and backtracking when a conflict is detected.\\
Roughly, the idea of lazy CNF conversion in DPLL is to begin with a set of clauses that includes only the negation of the literal that represents the entire formula, and only add definition clauses to the clause-set when the relevant literal is decided with the relevant polarity. The advantage of this technique is that it prevents interference between clauses that may not be on the same decision branch.
Consider, for example, the program in figure \ref{snippet_1.3}.\\
\begin{figure}
\begin{lstlisting}
n$_0$: if (a>0)
	...
	if (b=2*a+1)
		....
	else
		....
else
	n$_2$: b:=a-1
n$_3$: assert (a+b)$^2$>0
\end{lstlisting}
\caption{Example for lazy CNF conversion}
\label{snippet_1.3}
\end{figure}

Depending on the encoding of the VC and the order of decisions by the SMT solver, eager CNF conversion may cause the prover to produce the following candidate model (showing only theory relevant literals):\\
$\{b=2a+1, b=a-1, \lnot a>0, (a+b)^2>0$.\\
The literals $b=2a+1, b=a-1$ should not be part of the same candidate model as they occur on different branches - the choice of whether $b=2a+1$ or not is irrelevant for the proof of the outer else path.
%This candidate model is sent to the theory solver for arithmetic.\\
%The two literals $b=2a+1, b=a-1, a>0$ should never be sent together to the theory solver as they are not on the same branch
%
%\noindent
%A possible VC for the example is\\
%$(\mathbf{let}~~ n_3Ok = (a+b)^2>0) ~~ \mathbf{in}$\\
%$((a>0 \Rightarrow (b=2a+1 \Rightarrow n_3ok)) \land$\\
%$(a\not>0 \Rightarrow (b=a-1 \Rightarrow n_3ok)))$\\
%The $\mathbf{let}$ expressions allows the VC to include several copies of a formula, 
%without enlarging the input text.\\
%The definitions for the lazy CNF version are:\\
%$(n_3ok \Leftrightarrow (a+b)^2>0)$\\
%$(L_0 \Leftrightarrow (b=2a+1 \Rightarrow n_3ok))$\\
%$(L_1 \Leftrightarrow (a>0 \Rightarrow L_0))$\\
%$(L_2 \Leftrightarrow (b=a-1 \Rightarrow n_3ok))$\\
%$(L_3 \Leftrightarrow (a\not>0 \Rightarrow L_2))$\\
%$(L_4 \Leftrightarrow (L_1 \land L_3)))$\\
%And the VC is \\
%$L_4$.
%
%\noindent
%The proof process proceeds as follows:\\
%The clause set is initialized with $\{\lnot L_4\}$.\\
%The definition of $L_4$ is expanded and we get\\
%$\{\lnot L_4, \lnot L_1 \lor \lnot L_3\}$\\
%The prover can choose to decide either $\lnot L_1$ or $\lnot L_3$.\\
%If $\lnot L_1$ is chosen we expand definitions and get:\\
%$\{\lnot L_4, \lnot L_1, a>0, L_0\}$\\
%And then:\\
%$\{\lnot L_4, \lnot L_1, L_0, a>0, b=2a+1, \lnot n_3ok\}$\\
%And finally:\\
%$\{\lnot L_4, \lnot L_1, L_0, \lnot n_3Ok, a>0, b=2a+1, (a+b)^2\not>0\}$\\
%The theory solver is given the problem \\
%$\{a>0, b=2a+1, (a+b)*(a+b)\not>0\}$\\
%which it reports unsatisfiable.\\
%Next the SMT solver reverses the last decision, $\lnot L_1$, and we get, after definitions expansion:\\
%$\{\lnot L_4, L_1, \lnot L_3, L_2, \lnot n_3ok, a\not>0, b=a-1, (a+b)^2\not>0\}$\\
%The problem sent to the theory solver is \\
%$\{a\not>0, b=a-1, (a+b)*(a+b)\not>0\}$\\
%which it reports unsatisfiable. As there is no decision to backtrack, the formula is proven.
%
%The key property of the above proof is that at no point in the proof-search process the prover had to consider both literals 
%$b=2a+1$ and $b=a-1$.\\
%Eager CNF conversion will produce the following CNF for the negated VC (after simplification, assuming the same sub-formula renaming):\\
%$\{\lnot L_1 \lor \lnot L_3,$\\
%$L_1 \lor a>0, L_1 \lor \lnot L_0,$\\
%$L_0 \lor b=2a+1, L_0 \lor \lnot n_3ok,$\\
%$L_3 \lor a\not>0, L_3 \lor \lnot L_2,$\\
%$L_2 \lor b=a-1, L_2 \lor \lnot n_3ok,$\\
%$n_3ok \lor (a+b)^2>0\}$\\
%Here the prover has more choices for deciding literals, and specifically can choose to decide both
%$b=2a+1$ and $b=a-1$ - for example, reaching the candidate model:\\
%$\{\lnot L_1, \lnot L_0, \lnot n_3ok, b=2a+1, b=a-1, a>0, (a+b)^2>0\}$\\
%And then send the problem\\
%$\{b=2a+1, b=a-1, a>0, (a+b)^2>0$\\
%To the theory solver, where the literal $b=a-1$ is redundant.
%The deepest branch of any decision tree for lazy CNF was 1 while the eager CNF admits deeper decision trees 
%(in this simple example a simple analysis of literal polarities can restrict the decision tree depth also for eager CNF, but does not hold in the general case).


For a large program with many branches such redundancy can make a significant performance difference (as reported in \cite{DBLP:conf/cav/BarrettDS02},\cite{DBLP:journals/jacm/DetlefsNS05}). Even more significant is the fact that when pattern-based quantified instantiation is used, 
and the above encoding is used for the VC, lazy CNF conversion ensures that quantifiers are instantiated only with sets of terms that occur on some path in the program, and specifically not on opposing branches.


\subsubsection*{Superposition}
Another approach for equational theorem proving is superposition, based ultimately on resolution \\(\cite{Robinson:1965:MLB:321250.321253}) and unfailing Knuth-Bendix completion\\
(\cite{DBLP:journals/logcom/BachmairG94}).
Ordered resolution (for propositional logic) can be seen also as a search with candidate models: given a formula in CNF (a conjunction of disjunctions of literals), we order the clauses according to some total order on literals and its extension for clauses.\\
A clause is redundant if it is entailed by smaller clauses. 
The candidate model is the set of maximal literals of non-redundant clauses where the maximal literal is positive.\\
The candidate is not a model iff some non-redundant clause has a maximal negative literal that occurs positively in the model, in which case we resolve the two clauses on their maximal literals and the conclusion is a smaller (at least from the bigger premise) clause that encodes the case-split on the maximal literal for the two clauses (it entails that one of the premises holds, regardless of the value of the maximal literal). If the empty clause is derived the set is unsatisfiable, otherwise the process terminates when no new resolutions can be performed.\\
For GFOLE and FOLE equalities and dis-equalities are used instead of propositional literals. The difficulty is that a set of maximal equalities can be inconsistent with a different dis-equality (e.g. $a=b$ is inconsistent with $f(a)\neq f(b)$). 
To solve this difficulty, resolution is replaced with a restriction of ordered paramodulation (rewriting the maximal term of one clause with the maximal equality of another clause) and additional inference rules, ensuring that the candidate model forms a convergent rewrite system and each maximal term in a maximal literal is rewritten to normal form. For example, the terms \m{C \lor \underline{l}=r, \underline{f(l)}=t \lor D} where the maximal terms are underlined, produce \m{C \lor D \lor f(r)=t} with a superposition inference - this clause encodes a case-split, either \m{C} holds or \m{l=r} holds and then we can rewrite the right premise to \m{D \lor f(r)=t}.
For quantified clauses, unification is used to restrict the number of possible inferences and the candidate model is constructed from the (often countable) set of ground instances of clauses.\\
Saturation based provers such as these based on superposition often work by maintaining two sets of clauses, 
the first set is clauses that are inter-saturated (so all valid inferences with all premises in this set have been performed) and the second set of not yet saturated clauses.

The strength of superposition based provers is in handling quantifiers while the main weaknesses are handling large Boolean problems and handling theories for which no efficient axiomatization has been found. A key difference between saturation based provers (that derive new clauses from existing clauses, such as superposition) and DPLL based provers is that DPLL based provers must find a conflict in order to derive new (learned) clauses, and hence progress, and the complexity of finding one conflict is not bounded when quantifiers are involved, while saturation based provers can generate new clauses (and hence progress) at a known complexity bound (e.g. perform all valid inferences between a not-yet saturated clause and the saturated set and add the clause to the set - with complexity dependent on the size of the saturated set).

\subsubsection*{Instantiation based theorem proving}
Another approach for theorem proving in instantiation based (\cite{Korovin2008}, with equality \cite{DBLP:conf/cade/KorovinS10}).
The approach uses unification, as in resolution based provers, in order to find substitutions, but, instead of performing resolution, the instantiated versions of the premises are added to the clause set - for example:\\
For the premises $Q(x) \lor P(x,b)$, $\lnot P(a,y) \lor R(y)$, resolution produces $\{Q(a) \lor R(b)\}$ while instantiation produces $\{Q(f(a)) \lor P(a,b),\lnot P(a,b) \lor R(b)\}$. The attraction of the approach is that it avoids producing large clauses as in superposition, but it requires a different solution for the propositional part of the proof.\\
After some steps of instantiation, the prover temporarily substitutes a new constant (not part of the vocabulary), for all free variables in all clauses and uses a SAT solver (or SMT solver with equality) to check for satisfiability.
Instantiation for FOL with equality is implemented in the iProver theorem prover (\cite{DBLP:conf/cade/KorovinS10}).

\subsubsection*{St\aa lmarck's method}
A somewhat different approach for propositional satisfiability, that also handles ground equalities, is St\aa lmarck's method (\cite{DBLP:conf/fmcad/SheeranS98}). This method has seen somewhat less research as it is protected by patent laws.\\
The interesting aspect of the method, in our context, is the approach for handling case-splits.
The algorithm maintains the formula for which satisfiability is to be checked in a specific form, somewhat similar to lazy CNF,  
and saturates the formula w.r.t. a set of simplification rules (a somewhat similar concept to unit propagation in DPLL).
The main difference in formula representation and simplification is that literal equivalence can be represented natively - for example,
the simplification rules can learn that $A \Leftrightarrow B$ without determining the actual truth values for $A$ and $B$.
When simplification is insufficient for refutation, the algorithm performs a case split on some literal or atomic formula, 
and simplifies instances of the formula for both polarities of the literals split on. 
If neither branch finds a contradiction, a form of join is performed which adds any fact learned separately on both branches to the pre-split formula, the two formula instances used for the case-split branches are discarded.
The algorithm proceeds by case-splitting on all relevant atomic formulae - which includes equality between two literals and between a literal and a Boolean constant.
If no contradiction is found, the algorithm proceeds by performing case splits of depth 2 - for each relevant atomic formula $f$, first choose a value $V$, simplify an instance of the formula, and then recursively perform the depth-1 case-splitting, and similarly for $f=\lnot V$.
We find this method interesting because it shares with DPLL the property that the space needed is approximately proportional to the depth of the decision tree, but, as opposed to DPLL and CDCL, it ensures progress regardless of finding a conflict, and has a well defined hierarchy of fragments (by the case-split depth) that can be applied incrementally.
An extension for first order logic is given in \cite{Björk2009}.
We have not tried to adapt this technique to our setting, but the kinds of joins used, both in the PL and FOLE versions, are related to the joins we perform in join nodes in the CFG.

\subsubsection*{Model evolution}
An algorithm for theorem proving related to DPLL is that of \textit{model evolution} (\cite{BaumgartnerPelzerTinelli12}) implemented in the Darwin theorem prover (\cite{Fuchs:Darwin:Thesis:2004}). Model evolution behaves like DPLL but decides also non-ground literals where, essentially, unification is used to select substituted versions of quantified literals to decide upon. The model evolution calculus has the advantage that quantifiers are handled directly, and unification is used for directing instantiation. 
However, as with the methods for handling quantifiers in SMT solvers, a conflict must be reached in order to learn new information, and the search depth is not bounded.



