\chapter{Advanced}

\section{Theories}

\noindent
\textbf{Interaction with other fragments:}\\
Another aspect of the choice of communicating terms has to do with the interaction with other fragments.\\
Consider the following example:
\begin{figure}[H]
\begin{lstlisting}
$\node{n_1}:$ 
assume $\m{f\neq g}$
assume $\m{h\neq g}$
// x.f := 5
assume $\m{mr(h_0,x,f)=5}$
$\node{n_2}:$ 
// y.g := 6
assume $\m{h_1 = mw(h_0,y,g,6)}$
$\node{n_3}:$ 
// assert x.f := 5
assert $\m{mr(h_1,x,f)=5}$
  //negated $\m{\textcolor{gray}{mr(h_1,x,f)\neq 5}}$
\end{lstlisting}
\caption{sources fragment interaction}
\label{snippet3.16i}
\end{figure}
Here \m{mr} is the map-read function, \m{mw} is the map-write function, and \m{h_0,h_1} are DSA version of the heap.\\
As we will describe in the next chapter, we can handle this program either with a dedicated procedure for the map theory, 
or by axiomatizing the map theory and using superposition or quantifier instantiation.\\
\textbf{Dedicated procedure:}\\
For heaps it would suffice to use a non-extensional theory for updateable maps which rewrites read-over-write terms to earlier DSA versions (somewhat in the spirit of ~\cite{ChangLeino2005}) -  
so that \m{mr(mw(h,x,f,v),y,g)} is rewritten to \m{mr(h,y,g)} if we have established \m{x\neq y} or \m{f \neq g}.\\
Here, we are allowed by the invariant to have, at \m{g_{n_2}}, \m{mr(h_1,x,f)} either as a \gfa in the graph or as an \rgfa, as it cannot, in itself, communicate any equality information.\\
However, our procedure would need to ....

