\subsection{Terminology}
We do not represent ground term equivalence classes using a representative but rather a dedicated object.
We also include an object per ground tuple equivalence class, although it is essentially redundant - we will discuss the reasoning behind this later.

We use the following sets and kinds of objects:
\begin{itemize}
	\item \GTEC : A ground term equivalence class
	\item \GTTEC : A ground tuple equivalence class
of a \GTTEC
	\item \sig{n} : The signature of node \node{n}, specifically \Fn{n} is the set of functions in \sig{n}
	\item \ECs{n} : The per node congruence closure data structure for node \node{n} that includes, among other details described later:
	\begin{itemize}
		\item \GTECs{n} : The set of ground term equivalence classes represented at the node \node{n}
		\item \GTTECs{n}: The set of ground term tuple equivalence classes represented at \node{n}
%		\item \GFAECs{n}: The set of ground function application equivalence classes represented at \node{n}
	\end{itemize}
	\item We use the following functions:
		\subitem \gts{n}: $\cup \GTECs{n}$ - \GTECs{n} are a partition of \gts{n}
		\subitem \Eqs{n} : The set of equalities that \GTECs{n} represents - that is
		$\s{\term{t1=t2} \mid t1 \not\equiv t2 \land \exists \term{gt} \in \GTECs{n} \cdot (\term{t1,t2 \in gt})}$ -
		this may be infinite if e.g. $\term(a=f(a)) \in \Eqs{n}$ which implies, for all k, $\term(a=f^k(a)) \in \Eqs{n}$
		\subitem \Eqst{n} : The set of equalities implied by \node{n} and its predecessors at \node{n} -
		defined recursively: \\
		\s{t1=t2 \mid t1,t2 \in \Ts{\sig{n}} \cap (\bigcap\limits_{p \in \preds{n}} ( \mathbf{CC}(\Eqs{n} \cup \Eqst{p}) )}
\end{itemize}

We represent the congruence closure of a set of equalities \term{E} by \CC{E} (implicitly on the signature of \term{E}) and the congruence closure of \term{E} restricted to the signature \sig as $\CC{E}{F}=\left.\CC{E}\right|_{\Ts{\sig}}$.\\
We sometimes treat each equivalence class object as a set of its members, for example, when we create a new term from a \GTTEC gtt and a function \function{f} we define the new equivalence class as \s{f(tt) \mid tt \in gtt}.
We say a term equivalence class \term{t} \emph{exists} at a node \node{n} if $\term{t} \in \GTECs{n}$

Our per-node data structure supports the following basic operations for the node \node{n}:
\begin{itemize}
	\item \lstinline{makeTerm(f : $\Fn{n}$,gtt : $\GTTECs{n}$)} : \GTEC \\
		Create the \GTEC \term{f(gtt)} from the existing \term{gtt} in \GTTECs{n} and add it to \GTECs{n}.
	\item \lstinline{makeTuple(gts : $\GTTECs{n}*$)} : \GTTEC
		Create the t\GTTEC \term{gts} from existing ground term ECs, and add it it to \GTTECs{n}
	\item \lstinline{merge(gt1,gt2 : \GTEC)} : \GTEC \\
		Merge the \GTEC{s} \term{gt1} and \term{gt2}, and perform congruence closure.
	\item \lstinline{synchronize()} \\
		Synchronize the node \node{n} with relevant information not yet propagated from predecessors.
\end{itemize}


\subsection{Sequential Nodes}
In figure \ref{ec_graph_example_sequential_1}  we show a merge operation on a node \node{n} with one predecessor \node{p}.
Note that any number of new nodes can be created in \node{n} but only nodes that are transitive superterms of the merged nodes can be affected.

\begin{figure}[!htp]
\centering
\begin{subfigure}[t]{\textwidth}
	\framebox[\textwidth]{
	\raisebox{0pt}[0.4\textheight][0pt]
	{
		\begin{tikzpicture}
			\node(O) at (0,0){};
			\node[gttn] (1)  at (2,2.5)                         {$()$};
			\node[gtn]  (2)  [above left =0.5cm and 0.7cm of 1] {\s{a}};
			\node[gtn]  (3)  [above right=0.5cm and 0.7cm of 1] {\s{b}};
			\node[gttn] (4)  [above      =0.5cm           of 2] {$\s{(a)}$};
			\node[gttn] (5)  [above      =0.5cm           of 3] {$\s{(b)}$};
			\node[gtn]  (6)  [above      =0.5cm           of 4] {\s{f(a)}};
			\node[gtn]  (7)  [above      =0.5cm           of 5] {\svb{f(b)}{g(f^n(f(a)))}};
			\node[gttn] (8)  [above      =0.5cm           of 6] {$\s{(f(a))}$};

			\node[gl,align=left] (1l) [below = 0cm of 1] {p};
						
			\draw[gfa] (2)     to node[el] {\term{a}} (1);
			\draw[gfa] (3)     to node[el] {\term{b}} (1);
			\draw[gfa] (6)     to node[el] {\term{f}} (4);
			\draw[gfa] (7)     to node[el] {\term{f}} (5);
			\draw[gfa] (6.225) to[out=225,in=135,looseness=2] node[el] {\term{f}} (8.135);
			\draw[gfa] (7.210) to[out=210,in=45 ,looseness=3] node[el] {\term{g}} (8.45);

			\draw[sgtt] (4) to node[el] {\term{0}} (2);
			\draw[sgtt] (5) to node[el] {\term{0}} (3);
			\draw[sgtt] (8) to node[el] {\term{0}} (6);

			\node[gttn] (11)  [right      =          5.0cm of  1]   {$()$};
			\node[gtn]  (13)  [above right=0.5cm and 0.5cm of 11]   {\s{b}};
			\node[gtn]  (19)  [above left =0.5cm and 0.5cm of 11]   {\s{c}};
			\node[gttn] (21)  [above      =0.5cm           of 19]   {$\s{(c)}$};
			\node[gtn]  (17)  [above      =0.5cm           of 21]   {\s{f(c)}};

			\node[gl,align=left] (11l) [below = 0cm of 11] {n};
						
			\draw[gfa] (13) to node[el] {\term{b}} (11);
			\draw[gfa] (19) to node[el] {\term{c}} (11);
			\draw[gfa] (17) to node[el] {\term{f}} (21);

			\draw[sgtt] (21) to node[el] {\term{0}} (19);

		\end{tikzpicture}
	}}
		\caption{before the \lstinline{merge} operation}
		\label{ec_graph_example_sequential_1a}
\end{subfigure}

\begin{subfigure}[t]{\textwidth}
	\framebox[\textwidth]{
	\raisebox{0pt}[0.4\textheight][0pt]
	{
		\begin{tikzpicture}
			\node(O) at (0,0){};
			\node[gttn] (1)  at (2,2.5)                       {$()$};
			\node[gtn]  (2)  [above left =0.5cm and 0.7cm of 1]   {\s{a}};
			\node[gtn]  (3)  [above right=0.5cm and 0.7cm of 1]   {\s{b}};
			\node[gttn] (4)  [above      =0.5cm           of 2]   {$\s{(a)}$};
			\node[gttn] (5)  [above      =0.5cm           of 3]   {$\s{(b)}$};
			\node[gtn]  (6)  [above      =0.5cm           of 4]   {\s{f(a)}};
			\node[gtn]  (7)  [above      =0.5cm           of 5]   {\svb{f(b)}{g(f^n(f(a)))}};
			\node[gttn] (8)  [above      =0.5cm           of 6]   {$\s{(f(a))}$};

			\node[gl,align=left] (1l) [below = 0cm of 1] {p};
						
			\draw[gfa] (2)     to node[el] {\term{a}} (1);
			\draw[gfa] (3)     to node[el] {\term{b}} (1);
			\draw[gfa] (6)     to node[el] {\term{f}} (4);
			\draw[gfa] (7)     to node[el] {\term{f}} (5);
			\draw[gfa] (6.225) to[out=225,in=135,looseness=2] node[el] {\term{f}} (8.135);
			\draw[gfa] (7.210) to[out=210,in=45 ,looseness=3] node[el] {\term{g}} (8.45);

			\draw[sgtt] (4) to node[el] {\term{0}} (2);
			\draw[sgtt] (5) to node[el] {\term{0}} (3);
			\draw[sgtt] (8) to node[el] {\term{0}} (6);

			\node[gttn] (11)  [right      =          5.0cm of  1]   {$()$};
			\node[gtn]  (12)  [above left =0.5cm and 0.5cm of 11]   {\s{a}};
			\node[gtn]  (13)  [above right=0.5cm and 0.5cm of 11]   {\s{b,c}};
			\node[gttn] (14)  [above      =0.5cm           of 12]   {$\s{(a)}$};
			\node[gttn] (15)  [above      =0.5cm           of 13]   {$\s{(b),(c)}$};
			\node[gtn]  (16)  [above      =0.5cm           of 14]   {\s{f(a)}};
			\node[gtn]  (17)  [above      =0.5cm           of 15]   {\svb{f(b),f(c)}{g(f^n(f(a)))}};
			\node[gttn] (18)  [above      =0.5cm           of 16]   {$\s{(f(a))}$};

			\node[gl,align=left] (11l) [below = 0cm of 11] {n};
						
			\draw[gfa] (12)     to[bend right]node[el] {\term{a}} (11);
			\draw[gfa] (13)     to            node[el] {\term{b}} (11);
			\draw[gfa] (13)     to[bend left] node[el] {\term{c}} (11);
			\draw[gfa] (16)     to            node[el] {\term{f}} (14);
			\draw[gfa] (17)     to            node[el] {\term{f}} (15);
			\draw[gfa] (17.210) to[out=210,in=45, looseness=2] node[el] {\term{g}} (18.45);
			\draw[gfa] (16.225) to[out=225,in=135,looseness=2] node[el] {\term{f}} (18.135);

			\draw[sgtt] (14) to node[el] {\term{0}} (12);
			\draw[sgtt] (15) to node[el] {\term{0}} (13);
			\draw[sgtt] (18) to node[el] {\term{0}} (16);

		\end{tikzpicture}
	}}
	\caption{after \lstinline{n.merge(\{b\},\{c\})}}
	\label{ec_graph_example_sequential_1b}
\end{subfigure}
\caption{Sequential \lstinline{merge}}
\label{ec_graph_example_sequential_1}
\end{figure}

In figure \ref{ec_graph_example_sequential_2} we show a \lstinline{makeTerm} operation on a node \node{n} with one predecessor \node{p}.
Note that no pre-existing node in \node{n} can be merged by the operation, but new merged (and cyclic) nodes can be created.
Note also that in order to determine which nodes in \node{p} are relevant, \node{n} had to lookup
a super-edge with label \term{f} of node \s{(c)} at \node{p}, then it would need to look at all sub-edges of \s{g(b),f(c)} at \node{p},
find the edge labled \term{g} and connect it to the corresponding \term{a,b}, and then again query \node{p} for the super-edge \term{g} of \term{a},
after which it would get the cyclic node from \node{p} and have to add the cyclic edge labeled \term{h}.
This up and down process (in the sense of the arrows in the diagram) shows that any number of edges and nodes may need to be added in this process.

\begin{figure}[!htp]
\centering
\begin{subfigure}[t]{\textwidth}
	\framebox[\textwidth]{
	\raisebox{0pt}[0.4\textheight][0pt]
	{
		\begin{tikzpicture}
			\node(O) at (0,0){};
			\node[gttn] (0)  at (-2,1)                           {$()$};
			\node[gtn]  (1)  [above left =1.0cm and 1.0cm of 0] {\s{a}};
			\node[gtn]  (3)  [above      =0.85cm          of 0] {\s{b}};
			\node[gtn]  (5)  [above right=1.0cm and 1.0cm of 0] {\s{c}};
			\node[gttn] (2)  [above      =0.5cm           of 1] {$\s{(a)}$};
			\node[gttn] (4)  [above      =0.5cm           of 3] {$\s{(b)}$};
			\node[gttn] (6)  [above      =0.5cm           of 5] {$\s{(c)}$};
			\node[gtn]  (7)  [above      =0.5cm           of 2] {\svb{g(a)}{h^n(g(a))}};
			\node[gtn]  (9)  [above      =0.5cm           of 6] {\s{g(b),f(c)}};
			\node[gttn] (8)  [above      =0.5cm           of 7] {$(\svb{g(a)}{h^n(g(a))})$};

			\node[gl,align=left] (0l) [below = 0cm of 0] {p};
						
			%\draw[gfa] (1) to node[el] {\term{a}} (0);
			%\draw[gfa] (3) to node[el] {\term{b}} (0);
			%\draw[gfa] (5) to node[el] {\term{c}} (0);
			%\draw[gfa] (7) to node[el] {\term{g}} (2);
			%\draw[gfa] (9) to node[el] {\term{g}} (4);
			%\draw[gfa] (9) to node[el] {\term{f}} (6);
			%\draw[gfa] (7.225) to[out=210,in=120,looseness=2] node[el,anchor=west] {\term{h}} (8.135);

			\draw[gfa] (1) to node[el] {\term{a}} (0)
			           (3) to node[el] {\term{b}} (0)
			           (5) to node[el] {\term{c}} (0)
			           (7) to node[el] {\term{g}} (2)
			           (9) to node[el] {\term{g}} (4)
			           (9) to node[el] {\term{f}} (6)
			           (7.225) to[out=210,in=120,looseness=2] node[el,anchor=west] {\term{h}} (8.135);

			\draw[sgtt] (2) to node[el] {\term{0}} (1);
			\draw[sgtt] (4) to node[el] {\term{0}} (3);
			\draw[sgtt] (6) to node[el] {\term{0}} (5);
			\draw[sgtt] (8) to node[el] {\term{0}} (7);

%%%%%%%%%%%%%%%%%%%%%%%%%%%%%%%%%%%%%%%%%%%%%%%%%%%%%%%%%%%%%%%%%%%%%%%%%%%
			\node[gttn] (10)  [right      =          5.0cm of  0] {$()$};
			\node[gtn]  (11)  [above left =1.0cm and 0.7cm of 10] {\s{a,b}};
			\node[gtn]  (15)  [above right=1.0cm and 0.7cm of 10] {\s{c}};
			\node[gttn] (16)  [above      =0.5cm           of 15] {$\s{(c)}$};

			\node[gl,align=left] (10l) [below = 0cm of 10] {n};
						
			\draw[gfa] (11) to[bend right] node[el] {\term{a}} (10);
			\draw[gfa] (11) to[bend left]  node[el] {\term{b}} (10);
			\draw[gfa] (15) to             node[el] {\term{c}} (10);

			\draw (16) edge[sgtt] node[el] {\term{0}} (15);

		\end{tikzpicture}
	}}
		\caption{before the \lstinline{makeTerm} operation}
		\label{ec_graph_example_sequential_2a}
\end{subfigure}

\begin{subfigure}[t]{\textwidth}
	\framebox[\textwidth]{
	\raisebox{0pt}[0.4\textheight][0pt]
	{
		\begin{tikzpicture}
			\node(O) at (0,0){};
			\node[gttn] (0)  at (-2,1.5)                           {$()$};
			\node[gtn]  (1)  [above left =1.0cm and 1.0cm of 0] {\s{\term{a}}};
			\node[gtn]  (3)  [above      =0.85cm          of 0] {\s{\term{b}}};
			\node[gtn]  (5)  [above right=1.0cm and 1.0cm of 0] {\s{\term{c}}};
			\node[gttn] (2)  [above      =0.5cm           of 1] {$\s{\term{(a)}}$};
			\node[gttn] (4)  [above      =0.5cm           of 3] {$\s{\term{(b)}}$};
			\node[gttn] (6)  [above      =0.5cm           of 5] {$\s{\term{(c)}}$};
			\node[gtn]  (7)  [above      =0.5cm           of 2] {$\s{\term{h^n(g(a))}}$};
			\node[gtn]  (9)  [above      =0.5cm           of 6] {$\s{\term{g(b),f(c)}}$};
			\node[gttn] (8)  [above      =0.5cm           of 7] {$\s{\term{(h^n(g(a)))}}$};

			\node[gl,align=left] (0l) [below = 0cm of 0] {p};
						
			\draw[gfa] (1) to node[el] {\term{a}} (0);
			\draw[gfa] (3) to node[el] {\term{b}} (0);
			\draw[gfa] (5) to node[el] {\term{c}} (0);
			\draw[gfa] (7) to node[el] {\term{g}} (2);
			\draw[gfa] (9) to node[el] {\term{g}} (4);
			\draw[gfa] (9) to node[el] {\term{f}} (6);
			\draw[gfa] (7.210) edge[gfa,out=210,in=120,looseness=2.5] node[el] {\term{h}} (8.135);

			\draw[sgtt] (2) to node[el] {\term{0}} (1);
			\draw[sgtt] (4) to node[el] {\term{0}} (3);
			\draw[sgtt] (6) to node[el] {\term{0}} (5);
			\draw[sgtt] (8) to node[el] {\term{0}} (7);

%%%%%%%%%%%%%%%%%%%%%%%%%%%%%%%%%%%%%%%%%%%%%%%%%%%%%%%%%%%%%%%%%%%%%%%%%%%
			\node[gttn] (10)  [right      =          5.0cm of  0] {$()$};
			\node[gtn]  (11)  [above left =1.0cm and 0.7cm of 10] {\s{a,b}};
			\node[gtn]  (15)  [above right=1.0cm and 0.7cm of 10] {\s{c}};
			\node[gttn] (14)  [above      =0.5cm           of 11] {$\s{({a},{b})}$};
			\node[gttn] (16)  [above      =0.5cm           of 15] {$\s{(c)}$};
			\node[gtn]  (17)  [above      =3.0cm           of 10] {$\svb{h^n(g({a,b}))}{h^n(f(c))}$};
			\node[gttn] (18)  [above      =0.5cm           of 17] {$\svb{(h^n(g({a,b})))}{(h^n(f(c)))}$};

			\node[gl,align=left] (10l) [below = 0cm of 10] {n};
						
			\draw (11) edge[gfa,bend right] node[el] {\term{a}} (10);
			\draw (11) edge[gfa,bend left]  node[el] {\term{b}} (10);
			\draw (15) edge[gfa]            node[el] {\term{c}} (10);
			\draw (17) edge[gfa]            node[el] {\term{f}} (16);
			\draw (17) edge[gfa]            node[el] {\term{g}} (14);
			\draw (17.330) edge[gfa,out=330,in=45,looseness=3] node[el] {\term{h}} (18.45);

			\draw (14) edge[sgtt] node[el] {\term{0}} (11);
			\draw (16) edge[sgtt] node[el] {\term{0}} (15);
			\draw (18) edge[sgtt] node[el] {\term{0}} (17);

		\end{tikzpicture}
	}}
	\caption{after the \lstinline{makeTerm(f,(\{c\}))} operation}
	\label{ec_graph_example_sequential_2b}
\end{subfigure}
\caption{Sequential \lstinline{makeTerm}}
\label{ec_graph_example_sequential_2}
\end{figure}


In figure \ref{ec_graph_example_sequential_3} we show a \lstinline{synchronize} operation on a node \node{n} with one predecessor \node{p}.
In figure \ref{ec_graph_example_sequential_3a} we see the state before \node{p} has performed a merge where the two nodes are synchronized.
In figure \ref{ec_graph_example_sequential_3b} \node{p} has performed a merge and now node \node{n} is out of sync - it does not contain all equality information for its terms that \node{p} has.
In figure \ref{ec_graph_example_sequential_3c} \node{n} has performed the synchronization operation and now it has all a merge and now node \node{n} is out of sync - it does not contain all equality information for its terms that \node{p} has.
Note how a cycle was formed that was not in either node before.

\begin{figure}[!htp]
\centering
\begin{subfigure}[t]{\textwidth}
	\framebox[\textwidth]{
	\raisebox{0pt}[0.2\textheight][0pt]
	{
		\begin{tikzpicture}
			\node(O) at (0,0){};
			\node[gttn] (0)  at (2,1)                           {$()$};
			\node[gtn]  (1)  [above left =0.5cm and 0.5cm of 0] {\s{a}};
			\node[gtn]  (3)  [above right=0.5cm and 0.5cm of 0] {\s{b}};
%			\node[gttn] (2)  [above      =0.5cm           of 1] {$\s{(a)}$};
%			\node[gttn] (4)  [above      =0.5cm           of 3] {$\s{(b)}$};

			\node[gl,align=left] (0l) [below = 0cm of 0] {p};
						
			\draw (1) edge[gfa] node[el] {\term{a}} (0);
			\draw (3) edge[gfa] node[el] {\term{b}} (0);
%			\draw (3.225) edge[gfa,out=210,in=120,looseness=2] node[el,anchor=west] {\term{g}} (2.135);

%			\draw (2) edge[sgtt] node[el] {\term{0}} (1);
%			\draw (4) edge[sgtt] node[el] {\term{0}} (3);

%%%%%%%%%%%%%%%%%%%%%%%%%%%%%%%%%%%%%%%%%%%%%%%%%%%%%%%%%%%%%%%%%%%%%%%%%%%
			\node[gttn] (10)  [right      =          5.0cm of  0] {$()$};
			\node[gtn]  (11)  [above left =0.5cm and 0.5cm of 10] {\s{a}};
			\node[gtn]  (13)  [above right=0.5cm and 0.5cm of 10] {\s{b}};
			\node[gttn] (12)  [above      =0.5cm           of 11] {$\s{(a)}$};
%			\node[gttn] (14)  [above      =0.5cm           of 13] {$\s{(b)}$};

			\node[gl,align=left] (10l) [below = 0cm of 10] {n};
						
			\draw (11) edge[gfa] node[el] {\term{a}} (10);
			\draw (13) edge[gfa] node[el] {\term{b}} (10);
			\draw (13.215) edge[gfa,out=215,in=45,looseness=1] node[el,anchor=west] {\term{g}} (12.45);

			\draw (12) edge[sgtt] node[el] {\term{0}} (11);
%			\draw (14) edge[sgtt] node[el] {\term{0}} (13);

		\end{tikzpicture}
	}}
		\caption{before the \lstinline{p.merge} operation}
		\label{ec_graph_example_sequential_3a}
\end{subfigure}

\begin{subfigure}[t]{\textwidth}
	\framebox[\textwidth]{
	\raisebox{0pt}[0.2\textheight][0pt]
	{
		\begin{tikzpicture}
			\node(O) at (0,0){};
			\node[gttn] (0)  at (2,1)                           {$()$};
			\node[gtn]  (1)  [above left =0.5cm and 0.5cm of 0] {\s{a}};
			\node[gtn]  (3)  [above right=0.5cm and 0.5cm of 0] {\s{b}};
%			\node[gttn] (2)  [above      =0.5cm           of 1] {$\s{(a)}$};
			\node[gttn] (4)  [above      =0.5cm           of 3] {$\s{(b)}$};

			\node[gl,align=left] (0l) [below = 0cm of 0] {p};
						
			\draw (1) edge[gfa] node[el] {\term{a}} (0);
			\draw (3) edge[gfa] node[el] {\term{b}} (0);
			\draw (1.325) edge[gfa,out=325,in=135,looseness=2] node[el,anchor=west] {\term{f}} (4.135);

%			\draw (2) edge[sgtt] node[el] {\term{0}} (1);
			\draw (4) edge[sgtt] node[el] {\term{0}} (3);

%%%%%%%%%%%%%%%%%%%%%%%%%%%%%%%%%%%%%%%%%%%%%%%%%%%%%%%%%%%%%%%%%%%%%%%%%%%
			\node[gttn] (10)  [right      =          5.0cm of  0] {$()$};
			\node[gtn]  (11)  [above left =0.5cm and 0.5cm of 10] {\s{a}};
			\node[gtn]  (13)  [above right=0.5cm and 0.5cm of 10] {\s{b}};
			\node[gttn] (12)  [above      =0.5cm           of 11] {$\s{(a)}$};
%			\node[gttn] (14)  [above      =0.5cm           of 13] {$\s{(b)}$};

			\node[gl,align=left] (10l) [below = 0cm of 10] {n};
						
			\draw (11) edge[gfa] node[el] {\term{a}} (10);
			\draw (13) edge[gfa] node[el] {\term{b}} (10);
			\draw (13.215) edge[gfa,out=215,in=45,looseness=1] node[el,anchor=west] {\term{g}} (12.45);

			\draw (12) edge[sgtt] node[el] {\term{0}} (11);
%			\draw (14) edge[sgtt] node[el] {\term{0}} (13);

		\end{tikzpicture}
	}}
		\caption{after the operations \lstinline{p.makeTuple((b)),p.makeTerm(f(b)),p.merge(a,f(b))}}
		\label{ec_graph_example_sequential_3b}
\end{subfigure}

\begin{subfigure}[t]{\textwidth}
	\framebox[\textwidth]{
	\raisebox{0pt}[0.2\textheight][0pt]
	{
		\begin{tikzpicture}
			\node(O) at (0,0){};
			\node[gttn] (0)  at (2,1)                           {$()$};
			\node[gtn]  (1)  [above left =0.5cm and 0.5cm of 0] {\s{a}};
			\node[gtn]  (3)  [above right=0.5cm and 0.5cm of 0] {\s{b}};
%			\node[gttn] (2)  [above      =0.5cm           of 1] {$\s{(a)}$};
			\node[gttn] (4)  [above      =0.5cm           of 3] {$\s{(b)}$};

			\node[gl,align=left] (0l) [below = 0cm of 0] {p};
						
			\draw (1) edge[gfa] node[el] {\term{a}} (0);
			\draw (3) edge[gfa] node[el] {\term{b}} (0);
			\draw (1.325) edge[gfa,out=325,in=135,looseness=2] node[el,anchor=west] {\term{f}} (4.135);

%			\draw (2) edge[sgtt] node[el] {\term{0}} (1);
			\draw (4) edge[sgtt] node[el] {\term{0}} (3);

%%%%%%%%%%%%%%%%%%%%%%%%%%%%%%%%%%%%%%%%%%%%%%%%%%%%%%%%%%%%%%%%%%%%%%%%%%%
			\node[gttn] (10)  [right      =          5.0cm of  0] {$()$};
			\node[gtn]  (11)  [above left =0.5cm and 0.5cm of 10] {\s{a}};
			\node[gtn]  (13)  [above right=0.5cm and 0.5cm of 10] {\s{b}};
			\node[gttn] (12)  [above      =0.5cm           of 11] {$\s{(a)}$};
			\node[gttn] (14)  [above      =0.5cm           of 13] {$\s{(b)}$};

			\node[gl,align=left] (10l) [below = 0cm of 10] {n};
						
			\draw (11) edge[gfa] node[el] {\term{a}} (10);
			\draw (13) edge[gfa] node[el] {\term{b}} (10);
			\draw (11.325) edge[gfa,out=325,in=135,looseness=1] node[el,anchor=west] {\term{f}} (14.135);
			\draw (13.215) edge[gfa,out=215,in= 45,looseness=1] node[el,anchor=east] {\term{g}} (12.45);

			\draw (12) edge[sgtt] node[el] {\term{0}} (11);
			\draw (14) edge[sgtt] node[el] {\term{0}} (13);

		\end{tikzpicture}
	}}
		\caption{after the \lstinline{n.synchronize} operation}
		\label{ec_graph_example_sequential_3c}
\end{subfigure}

\caption{Sequential \lstinline{synchronize}}
\label{ec_graph_example_sequential_3}
\end{figure}

The \lstinline{makeTuple} operation cannot create any more term and tuples than the one tuple that is created (if it does not already exist).
However, it can modify the caches of several predecessors.


\subsection{Join Nodes}
For binary join nodes the
\newpage
\subsection{Formal Operations}
\begin{itemize}
	\item \lstinline{makeTerm(f : $\Fn{n}$,gtt : $\GTTECs{n}$)} : \GTEC \\
	Create the \GTEC \term{f(gtt)} from the existing \term{gtt} in \GTTECs{n} and add it to \GTECs{n}.
	We define $\term{gfa} = \s{f(tt) \mid tt \in gtt}$ and then the result represents the minimal equivalence class that includes all of \term{gfa} and respects all the equalities in \ECs{n} (and in \node{n}'s predecessors).
	If such \GTEC exists in the prestate then nothing is modified.
	If such a \GTEC does not exist in the prestate, the result equivalence class may include more than just \term{gfa} - if this is implied by the some equalities at predecessors of \node{n}, which may imply the creation of more terms and tuples.
	Regardless of the arguments, no new equality information is added to existing terms in \gts{n} and globally no new equality information is added to any existing term in any node.
	Note that by construction, for any \GTEC \term{t} in \GTECs{n},
	$\s{f(tt) \mid tt \in gtt} \subseteq \term{t} \lor \s{f(tt) \mid tt \in gtt} \cap \term{t} = \emptyset$
	%\requires \inSync{\predst{n}} \\
	%\requires $\term{f} \in \scope{n}$ \\
	%\requires $\term{gtt} \in \terms{n}$ \\
	%\ensures  $\term{f(gtt)} \in \termsp{n}$ \\
	%\ensures  $\forall m \cdot \inSync{m} \Rightarrow \inSyncp{m}$
	
	\item \lstinline{makeTuple(gts : $\GTTECs{n}*$} : \GTTEC \\
	Create the tuple EC \term{gts} from existing ground term ECs. \\
	As before, if this \GTTEC exists it is simply returned, otherwise it is created. No new information is added for existing terms or tuples locally or globally (although some caches are modified - described later).
	\s{tt \in \Ts{n}^n \mid \forall i \cdot tt[i] \in gts[i]}
	%\requires \inSync{\predst{n}} \\
	%\requires $\dom{\term{gts}} \subseteq \terms{n}$ \\
	%\ensures  $\term{gts} \in \termsp{n}$ \\
	%\ensures  $\forall m \cdot \inSync{m} \Rightarrow \inSyncp{m}$

	\item \lstinline{merge(gt1,gt2 : \GTEC)} : \GTEC \\
	Merge the \GTEC \term{gt1} and \term{gt2}, and perform congruence closure.\\
	This operation adds equality information and, depending on the current state, may cause the creation of new \GTECs and so extend \gts{n}.
	In the poststate \ECs{n}' we get the minimal set of equalities that are implied by the prestate \ECs{n} and \term{gt1=gt2} (and the prestates of all transitive predecessors of \node{n}). No new equality information is added to any predecessor.
	The new equality information might be relevant to some successors, and these will have to synchronize (see below) before they can perform any other operation.
	%\requires \inSync{\predst{n}} \\
	%\requires $\term{t1,t2} \in \terms{n}$ \\
	%\ensures  $\term{t1=t2} \in \Eqsp{n}$ \\
	%\ensures  $\forall m \notin \succsto{n} \cdot \inSync{m} \Rightarrow \inSyncp{m}$

	\item \lstinline{synchronize()} \\
		Synchronize the node \node{n} with information not yet propagated from predecessors.\\
		In the poststate each of \GTECs{n} have all the members implied by the transitive predecessors.
%	\requires \inSync{\predsto{n}} \\
%	\ensures  \inSync{\predst{n}} \\
%	\ensures  $\forall m \notin \succsto{n} \cdot \inSync{m} \Rightarrow \inSyncp{m}$
\end{itemize}

In addition, all operations satisfy the following monotonicity requirements:\\
\begin{tabular}{ll}
No terms deleted:        & $\forall \node{m} \cdot \terms{m} \subseteq \termsp{m}$ \\
No equalities forgotten: & $\forall \node{m} \cdot \Eqs{m}   \subseteq \Eqsp{m}$ \\
\end{tabular}

In addition, for all nodes at all times, \terms{n} is always subterm-closed (and subtuple-closed)
 and \Eqs{n} is congruence closed for \terms{n}.

It is based on the following algorithmic choices:
\begin{itemize}
	\item Each node maintains a set of \emph{relevant terms} (subterm closed), for which it maintains full congruence information (up to scope and radius/size limitations) - including inequalities
	\item When a term becomes relevant in a node, the node would \emph{query} its predecessors (recursively) to get all known (in)equality information in its transitive predecessors - this might, in turn, \emph{enable derivations} in some transitive predecessors. The result is that each node on each path between the node where the (in)equality was established and where the (in)equality is used will cache the (in)equality
	\item In order to speed up propagation, each node maintains, for each relevant term, the set of \emph{corresponding terms} in its predecessors and the set of \emph{failed successor queries} that found no new information - so that transitive queries can often be avoided
	\item When a node discovers a new (in)equality as a result of a derivation (e.g. resolution), it does not update its successors immediately, rather, before a node can perform any operation it must synchronize with its (transitive) predecessors - hence many such updates can be bunched up in one operation
\end{itemize}


Our data structure represents terms as equivalence classes, rather than the widely used union-find data structure - the reason is that we do not need to be able to backtrack (monotone), and that it simplifies the propagation, sharing and caching of information.

Another divergence from the standard structure is that our equivalence class graph is heterogeneous - it contains nodes both for term equivalence classes and for term-tuple equivalence classes. The reason is that this simplifies the communication between nodes and the caching of queries.

We have actually gone a step further, and represent also atoms, literals and clauses as equivalence classes - this has the benefit of much faster unification and easier caching and communication between nodes, but at the prices of more expensive updates when a new equality is discovered - hence we try to collect several new ground equalities before updating the data structure to account for all those equalities in one step.

Formally, a k-\emph{tuple-node} $\term{ttn} \in \mathbf{tts}(G)$ in the graph has $\|\mathrm{ttn}\|=\arity{ttn}=k$ ordered outgoing edges to $k$ \emph{term-nodes}, represented by \term{tt[0]}..\term{tt[k-1]}.
A \emph{term-node} $\term{tn}\in \mathbf{ts}(G)$ has one or more outgoing \emph{function-application} edges each targeted at a tuple-node and labeled with a function symbol with matching arity.
We represent these edges as $\mathbf{fas}(\term{tn}) \subseteq \mathbf{ts} \times \mathbf{F} \times \mathbf{tts}$, with the label as $\fnc{fa}$ and the source,target as $\src{fa}$, $\trg{fa}$, respectively. The set of all function-application-edges of $\mathrm{G}$ is $\fas{G}$.

For a tuple-node \term{ttn} we denote by $\incoming{n}{ttn}$ the set of edges targeted at it: $\incoming{n}{ttn} = \{\mathrm{fa} \in \fas{G(n)} \mid \trg{fa}=\term{ttn} \}$.

We define a \emph{term-path} in the graph $\mathrm{G}$, that represents one member of an equivalence class, as a tree where each node \term{t} in the tree is labeled with a function-application edge $\mathrm{fa}(\term{t})=\term{e}$, and each such node has $\mathrm{n_t}=\arity{\func{e}}$ ordered edges \term{t_0..t_{n_t-1}} where, for $i \in 0..n_t-1$, $\trg{e}[i]=\src{\mathrm{fa}(\mathrm{target}(\mathrm{t_i}))}$ - that is, each node label represents a choice of one of the function application edges of the corresponding equivalence class, and each edge connects to the corresponding term-node for the tuple of that function application. All leaves are nodes where $\trg{\term{e}}=()$ - the empty tuple.
The term that is represented by such a tree can be defined recursively: each leaf node \node{t} represents the constant $\mathbf{term}(\mathrm{t})=\fnc{\mathrm{fa}(\term{t})}$ and each non-leaf node \node{t} represents the term $\mathbf{term}(\mathrm{t})=\fnc{\mathrm{fa}(\term{t})}(\mathbf{term}(\mathbf{target}(t_0))...\mathbf{term}(\mathbf{target}(t_{t_n-1})))$.

We designate all term paths of the graph $\mathrm{G}$ as $\mathrm{tps(G)}$.

For each term-node \term{t} in the equivalence class graph $\mathrm{G}$ we denote by $\mathbf{terms}(\term{tn})$ the set of terms it represents - $\mathbf{terms}(\term{tn})=\{ \mathbf{term}(\mathrm{T}) \mid \mathrm{T} \in \mathrm{tps(G)} \land \src{\mathrm{fa(root(T))}}=\term{tn}$.\\
Accordingly we define, for a tuple-node \term{ttn},\\
 $\mathbf{tuples}(\term{ttn})=\{ \mathrm{tt} \in \mathbf{T(\Sigma)}^{\|\term{tt}\|} \mid \forall \mathrm{i} \in 0..\|\term{tt}\|-1 \mid \mathrm{tt}[\mathrm{i}] \in \mathbf{terms}(\mathrm{ttn}[\mathrm{i}]) \}$.


We maintain several invariants on the above equivalence class graph $\mathrm{G}$:
\begin{itemize}
	\item The only leaf of the graph is the empty tuple $\mathbf{()}$
	\item Each term-node in the graph represents one or more terms, and hence has at least one term-path
	\item The graph is fully congruence closed: $\forall \mathrm{e_1,e_2} \in \mathrm{fas(G)} \mid ((\fnc{e_1}=\fnc{e_2} \land \trg{e_1}=\trg{e_2} ) \Rightarrow e_1=e_2)$ (each tuple has at most one incoming edge for each function symbol)
	\item Each tuple is represented by exactly one tuple-node: $\forall \mathrm{ttn_1,ttn_2} \in \mathrm{tts(G)} \mid ((\|\term{ttn_1}\|=\|\term{ttn_2}\| \land (\forall \mathrm{i} \in 0..\|\term{ttn_1}\|-1 \mid \term{ttn_1}[\mathrm{i}]=\term{ttn_2}[\mathrm{i}])) \Rightarrow \term{ttn_1}=\term{ttn_2})$
\end{itemize}

This implies that the set of terms (resp. tuples) represented by each pair of term-nodes (resp. tuple-nodes) of $\mathrm{G}$ is disjoint.

We also maintain some invariants between the equivalence class graph $\mathrm{G(n)}$ of node \node{n} and that of its predecessors. These invariants require some extra state - namely the cache for equality queries:\\
For a node \node{n}, for each tuple-node $\term{ttn} \in \tts{G(n)}$, we maintain a set of function symbols $\rfs{n}{ttn}$ (rejected function applications) of the correct arity (applicable to the tuples represented by the equivalence class).
This allows us to encode three mutually exclusive possible options for each such function \term{f} and tuple-equivalence-class \term{ttn}:
\begin{itemize}
	\item There is a known term-equivalence-class for this function application -
	$\mathrm{f} \in \incoming{n}{ttn}$
	\item There is no known term-equivalence-class for the function application in \node{n} or any of its transitive predecessors - $\mathrm{f} \in \rfs{n}{ttn}$
	\item It is unknown whether this function application has any term-equivalence class in \node{n}'s transitive predecessors - $\mathrm{f} \notin \rfs{n}{ttn} \cup \incoming{n}{ttn}$
\end{itemize}

We will later refine this definition to account for term-radius and scope.

The invariants that hold between nodes are:
\begin{itemize}
	\item Each node \node{n} has all the equivalence information from its predecessors for each term in its graph:
	$\forall \term{t} \in \terms{G(n)} \mid [\term{t}]_{\node{n}} \supseteq \cap_{\node{p} \in \preds{n}} [\term{t}]_{\node{p}} $
\end{itemize}

In general, for a node \node{n}, the knowledge about the equality relation between two terms $\term{t},\term{t'} \in \sScope{n}$ has several options:
\begin{itemize}
	\item $\term{t},\term{t'} \in \terms{n}$ and $[\term{t}]_{\node{n}}=[\term{t'}]_{\node{n}}$ and not $\term{t} \neq_{\node{n}} \term{t'}$- the terms are known to be equal at \node{n}
	\item $\term{t},\term{t'} \in \terms{n}$ and $[\term{t}]_{\node{n}} \neq [\term{t'}]_{\node{n}}$ and not $\term{t} \neq_{\node{n}} \term{t'}$ - the terms are not known to be equal or unequal at \node{n} and at any transitive predecessor cut of \node{n}
	\item $\term{t},\term{t'} \in \terms{n}$ and $[\term{t}]_{\node{n}} \neq [\term{t'}]_{\node{n}}$ and $\term{t} \neq_{\node{n}} \term{t'}$ - the terms are known to be unequal at \node{n}
	\item $\term{t},\term{t'} \in \terms{n}$ and $[\term{t}]_{\node{n}} = [\term{t'}]_{\node{n}}$ and $\term{t} \neq_{\node{n}} \term{t'}$ - the terms are known to be equal and unequal at \node{n} - the node is infeasible
	\item $\term{t} \in \terms{n}$, $\term{t'} \notin \terms{n}$ - the terms are not known to be equal at \node{n} or any transitive predecessor cut, but may be known to be unequal at some such cut
	\item $\term{t},\term{t'} \notin \terms{n}$ - it is not known at \node{n} whether the terms are equal or not, but it may be known in some transitive predecessor cut
\end{itemize}

We will first discuss the propagation of (in)equality information among nodes.
We will present several refinements of a system for the propagation of \\
(in)equality information along the CFG.

We refer first to propagating unit (in)equalities , that could come either directly from the encoding of the original program, or could be derived using e.g. resolution  - a unit (in)equality can, in turn, allow the derivation of new clauses which could derive new
(in)equalities and so on - in this section we only handle derivations within the fragment of unit
(in)equalities .

We denote by \relevantTerms{n} the set of terms that are deemed potentially relevant at node \node{n} at a given time during verification, this includes terms that are relevant only because of transitive successors of \node{n} - the exact definition of relevancy will be given later.
We denote by \terms{n} the set of terms \emph{represented} at a given time at node \node{n} - this would always a superset of (or equal to) \relevantTerms{n}.

For a term \term{t} represented at node \node{n}, we denote by $[\term{t}]_{\node{n}} \subseteq \terms{n}$ the currently \emph{known} equivalence class of \term{t} at node \node{n} and for another term $\term{t'} \in \terms{n}$ we denote by $\term{t} =_{\node{n}} \term{t'}$, $\term{t} \neq_{\node{n}} \term{t'}$ that at \node{n} it is known that \term{t} is equal, not equal to \term{t'}.

A transitive predecessor cut of a node \node{n} is a cut in the CFG between the \node{root} and \node{n} - a clause that holds in such a cut also holds at \node{n} - this is somewhat of a generalization of the notion of a dominator of \node{n}, which would preclude the case where, e.g. an assertion does not hold in any dominator of \node{n} but holds in two direct predecessors of \node{n} joined at \node{n}.

We use the following definitions to describe the stage of deriving an
(in) equality (or any other clause) at a given node:
\begin{itemize}
	\item A clause \clause{C} \emph{holds at} node \node{n} if $\postcond{n} \vDash \clause{C}$
	\item A clause \clause{C} is \emph{known at} node \node{n} if $\exists \clause{D} \in \clauses{n} \mid \clause{D} \subseteq \clause{C}$
	\item A clause \clause{C} is \emph{derived for} node \node{n} if there is a cut $\mathrm{T}$ in the CFG dominating \node {n}, such that $\forall \node{p} \in \mathrm{T} \mid \clause{C} \in \clauses{p}$ (this definition would need some refinement to account for scoping etc) - this basically means that only propagation is needed for \clause{C} to be known at \node{n}
	\item A clause \clause{C} is \emph{propagated for} node \node{n} if $\clauses{n} \vdash \clause{C}$ but $\clause{C} \notin \clauses{n}$ - that is, sufficient information has been propagated to \node{n} in order to derive \clause{C} in our calculus, but it has not yet been derived
\end{itemize}

\newpage

\noindent
Example 3:
\begin{lstlisting}[caption=congruence closure eager interpolant arity,label=snippet3.29]
$p_i$:
assume P($a_0$,$a_6$,...,$a_{6n}$)=T
....

$p_b$:
if (*)
	$p_t$:
	assume $a_0$=$a_2$
	assume $a_2$=$a_4$
	...
	assume $a_{6m-2}$=$a_{6m}$
	$p_{ta}$:
		assert P($a_0$,$a_0$,...,$a_0$)=T
else
	$p_e$:
	assume $a_0$=$a_3$
	assume $a_3$=$a_6$
	...
	assume $a_{6m-3}$=$a_{6m}$
	$p_{ea}$:
		assert P($a_0$,$a_0$,...,$a_0$)=T
\end{lstlisting}

