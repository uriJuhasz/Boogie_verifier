\chapter{Closing Remarks}
\label{chap:conclusion}

Smartphones are becoming ever more present in users' daily lives. Their always-connected nature and versatility have enabled a number of applications that make them the most personal device that people carry constantly. This leads to an ever-increasing amount of personal and business data being stored on these devices. It is therefore a natural consequence that, despite the security mechanisms that have been introduced to protect smartphone operating systems and applications, an increasing number of attacks target these devices. In this thesis we focused on two aspects of smartphone security. We first looked at how smartphones can be used to strengthen the security of everyday activities from web authentication to payments at points of sale. We then focused on attacks against smartphones and in particular at application phishing and collusion attacks.

In the first part of this thesis we introduced Sound-Proof, the first two-factor authentication mechanism for the web that is both transparent to the user and deployable with current browser technologies (without needing extra software to be installed on the user's computer). In Sound-Proof, two short simultaneous audio recordings from the phone and the computer are compared to provide proof of proximity between the two devices. Only when the comparison yields a positive result is the login attempt successful. We tested our solution in a variety of environments from busy cafes and train stations to more quiet offices and homes. We showed that Sound-Proof works without the user having to interact with his smartphone, even when the phone is kept in a pocket or purse. Finally, in a user study aimed at comparing the usability of Sound-Proof and Google 2-Step Verification, we discovered that users found Sound-Proof more usable. More importantly, the majority of users expressed their intention to use Sound-Proof in scenarios where two-factor authentication is optional. Finally, participants appreciated that to complete the two-factor authentication procedure with Sound-Proof took four to five times faster than with a code-based solution.

We then looked at how to strengthen the security of payments at points of sale in order to prevent a large number of fraudulent transactions caused by cloned or counterfeit credit cards. In this new setting we considered a stronger attacker model in which the adversary is able to also compromise the victim's mobile OS. With this new attacker model we showed how the location provided by a smartphone can be used in a timely, accurate and secure manner as a second authentication factor when performing a payment. We exploited the availability of mobile trusted execution environments and presented two novel enrollment schemes that work even when an adversary has compromised the victim's device before the enrollment takes place. We validated our solution through a series of prototype implementations that showed both deployability and usability with today's platforms.

In both cases we always kept usability and deployability as primary goals of our solutions. We strongly believe that any security solution that is proposed should gauge how hard it is to apply and how difficult it is for users to adopt. No matter how good a solution is from a security standpoint, no benefits can be gained if the user adoption is small.

In the second part of this thesis we looked at open security challenges on smartphone platforms. We started with application phishing attacks, where a malicious application masquerades as a legitimate one to steal user credentials. We analyzed different ways in which such attacks are possible on current platforms and possible countermeasures. We concluded that no solution was fit to counter all types of attacks. Still, borrowing the idea of personalized security indicators from the web, where they were found to be of little help in defeating phishing attacks, we asked ourselves if this was still the case in the new context of smartphone applications. We carried out a user study to understand their effectiveness in preventing application phishing attacks. We found that although they cannot be considered a silver bullet, they are much more effective in this scenario than previously found on the web. While further studies are needed to perfect their usage, our results show that they can be used as an effective method to counter phishing attacks.

Finally, we looked at application collusion attacks. Through the use of covert and overt channels malicious smartphone applications can evade the permissions-based security mechanisms employed on modern smartphones. That is, since the user approves the installation of one application independently from others, he is led to believe that the permissions granted reflect which resources the application has access to. Through application collusion attacks this is not the case. We implement and analyze a variety of overt and covert channels that can be implemented on today's smartphones. Some channels have high throughput and are easier to detect and prevent. Other covert channels, have lower throughput rates and are harder to detect. They are therefore an open threat to the user's private data (such as GPS location or contacts).

We showcased how smartphones are not free of attack vectors and hence how the data stored on them is at risk, even today. Further research is required to better the security of current smartphones and to find new applications where smartphones themselves can be used to strengthen the security of other systems. This thesis is a step in that direction.