\chapter{Related Work}
\label{chap:sp_relatedwork}

In this chapter we review related work in the area of smartphone attacks and countermeasures. We report both academic and industry work in the area of smartphone malware analysis and technologies to detect and prevent attacks.

\section{Smartphone Malware Attacks}

Malicious mobile applications typically exploit platform vulnerabilities (e.g., for privilege escalation) or use system APIs that provide access to sensitive user data. A malicious application can, for example, leak the user location or send SMS messages to premium numbers without user consent. Recently, more sophisticated attacks have emerged where malicious applications cooperated in botnets or mined cryptocurrencies~\cite{lookout2014}

As of today there have been two major ways through which malware spread on mobile platforms. The majority of malware is shipped as a repackaged legitimate application. The authors of~\cite{zhou-jiang12sp} found that in a set of 1260 malware samples, 86\% of them are repackaged applications. In these attacks malware authors inject malicious payloads in legitimate applications. The majority of repackaged applications that contain malicious payloads are found on alternative marketplaces (75\%). Users download these applications unaware of the embedded malware and the payloads are executed as the user starts the application.

Recently, malware authors have leveraged another infection mechanism. Instead of repackaging applications directly, they have infected the programs used to compile and package the applications. This technique has been used in late 2015 to generate malicious applications for the iOS platform~\cite{xcodeghost}. Developers using an infected version of XCode (i.e., the iOS IDE used to develop, compile and package applications to submit to the AppStore) have packaged malicious software in 39 applications that were then approved for distribution on the AppStore.

The research community has focused their attention on understanding the infection rates of malware on smartphones. On the one hand multiple malware families have been identified. In~\cite{zhou-jiang12sp} the authors classify malware samples in 48 distinct families all targeting the Android platorm. The samples have also been made available for further analysis~\cite{malwaregenome}. In~\cite{felt11w2sp}, the authors analyze 46 different malware samples for Android, iOS and Symbian platforms. Given the amount of malware samples collected one would think that their infection rates would be high. 

On the other hand, recent research has found little traces of malware in the wild. In~\cite{lever-ndss13} the authors analyze the traces collected over a three-months period by a US mobile network provider. In particular, the traces are scanned in search of DNS domains lookups associated to malware samples. Only 0.0009\% of the population if found to be infected with known malware. Furthermore, the authors note that a malware campaign that lasted for months was stopped even before the malware sample was identified. The authors of~\cite{truong13} also try to quantify the malware infection rates. Their approach differs from the previous work in that they collect signatures of applications installed on approximately 55000 mobile devices. The authors report malware infection rates on Android devices by comparing recorded traces on the phones against two malware datasets. The infection rates are slightly higher than the previous work, depending on which malware dataset is used, namely 0.28\% and 0.29\%. The authors further propose a solution to identify potentially compromised devices by looking at the applications installed as an indicator of how likely they are to be or become infected.

\subsection{Sophisticated Attacks}

While the research community has found little evidence of malware being a widespread problem on mobile platforms, we now present advanced techniques through which malware can exfiltrate users' private data.

\paragraph{Application Phishing.}
Phishing attacks have been first studied in the web context, where a user is redirected to a malicious webpage typically following a hyperlink sent over e-mail or other means~\cite{dhamija06chi,Fette2007,Garera2007}. In an application phishing attack, a malicious application presents the user a user interface associated with another application. The user is led to believe that the malicious application is the original one and enters his credentials that are then sent to the attackers. 

Application phishing attacks have been reported in the wild~\cite{droid09, digitaltrends, securelist,macrumors,forbes} leading to not only credential theft but also economical losses. In the context of smartphones, phishing attacks have been first hinted at in~\cite{chin11mobisys, zu-woot12, felt11w2sp} and more recently extensively studied, for the Android platform, in~\cite{bianchi15sp}. In this latest work, the authors present a systematic evaluation of application phishing attacks and use static analysis techniques to detect applications that use APIs that enable certain classes of application phishing attacks. We will extend the analysis on application phishing attacks and countermeasures as well as present a first user-study on personalized security indicators in the context of mobile platforms in Chapter~\ref{chap:sp_phishing}.

\paragraph{Application Collusion.}
Application collusion attacks are the most recent instantiation of decades-old attacks presented, at first, to exfiltrate data in multi-tier architectures~\cite{Lampson:note_on_confinment}. In these attacks two colluding processes on the same platform can exchange information through the file system. More recent work focused on timing channels, where two processes can exchange bits of information by timing particular operations on the same machine. Cache-based~\cite{Reducing_timing_channels_with_fuzzy_time} and memory-bus based attacks~\cite{hu-sp92,wu-usenix12} have been presented for the x86 architecture. More recently, channels that exploit virtual memory deduplication~\cite{xiao-dsn13}, and finally, covert channels that exploit the heat dissipation of processor cores~\cite{ramyaheat} have also been shown to work.

In the context of smartphones, Soundcomber~\cite{soundcomber-ndss} is the first work that looks at colluding applications. In Soundcomber, the authors use the smartphone microphone to harvest sensitive information, such as credit card numbers, by detecting voice and tone patterns. The recording application, then transmits the information to an application that has access to the network, so that the credit card number can be transmitted over the internet. The channels presented use globally-available settings (vibration, volume, screen lock, etc.) or file locks. In Chapter~\ref{chap:sp_appcollusion} we will focus on application collusion attacks and present a variety of overt and covert channels that can be created on mobile platforms. Recently in~\cite{lalande13}, the authors have further extended our list of covert communication channels.

\paragraph{Side Channels.}
Application collusion attacks require two applications to synchronize and exchange information. Side channels, in contrast, can be leveraged by malicious applications to extract information out of other applications. In particular, in the context of smartphones, researchers have explored how side channels can be used to infer private information. In~\cite{touchlogger11,taplogger12,accessory12} the authors propose to use accelerometers, gyroscopes and orientation sensors, available on the majority of smartphones, to infer which characters a user is typing on a keyboard and learn a user's passwords or PIN codes.

In Memento~\cite{memento12}, the authors describe a new side channel attack on smartphones. By monitoring memory and CPU usage, a malicious application can infer which webpage is being loaded in another process (e.g., a web browser) as well as finer-grained information.

\paragraph{Root Exploits.}
There have been a number of root exploits that target mobile platforms. Root exploits allow a malicious application to escalate its privileges and execute at the highest possible privilege (i.e., root). Root exploits typically exploit techniques that have been known for decades for x86 platforms. Examples include buffer overflows~\cite{stackprofit}, format strings vulnerabilities~\cite{stringformat} or integer overflows~\cite{integeroverflow}. Many of these techniques can be used to exploit also ARM-based systems, such as the majority of modern smartphones, and therefore a large number of attacks have been shown to be successful both on Android~\cite{androidexploit15,androidexploit15ipc} and iOS~\cite{iosmemory13,iosmemory15}.

\section{Smartphone Malware Countermeasures}

We have shown how a number of malware samples have been discovered in the wild and analyzed by both industry and researchers alike. We also presented a number of research work focusing on sophisticated attack techniques that attackers can exploit on current platforms. We now survey work focusing on mobile malware countermeasures.

\paragraph{Permission-Based Security.}
A significant amount of work has been performed, in the past few years, on the Android platform and specifically on the permission-based model~\cite{barrera:android,androidsurgery,DBLP:conf/pldi/Chaudhuri09,DD2SW010,Android_malware_Oberhide,App_centric_android_security,Smobile_android_market_Analysis}. 

Barrera et al. present an empirical methodology for the analysis and visualization of its permission-based model, which can help in refining the permission system~\cite{barrera:android}. In particular, the authors find that developers only use a set of available permissions and that some permissions are overly broad. In~\cite{Felt2011}, the authors build a tool to detect overprivileged applications. In their analysis, they find that approximately 33\% of analyzed applications have more permissions than needed. Focusing on the same subject, the authors of~\cite{Chia2012} found that popular applications ask for more permissions than other applications.

The Kirin tool~\cite{Enck:2009:LMP:1653662.1653691} uses predefined security rule templates to match dangerous combinations of permissions requested by applications. Saint~\cite{App_centric_android_security} allows run-time control over communication among applications according to their permissions. In~\cite{androidsurgery}, the authors discuss possible unchecked information flows due to applications that use Broadcast Intents without proper permissions checking.

Moving away from developers and focusing on users' understanding of permissions, in~\cite{Felt12}, the authors find that users have a poor understanding of permissions and that their display in the Android system yields to poor judgement when installing applications. Now that Android supports permissions revocation after installation new studies should be performed to understand how users behave. In~\cite{Chia2012}, the authors also look at the effectiveness of user-consent permissions systems for Facebook, Chrome and Android applications. They found that users do not have enough context to judge the privacy and security of an application when installing it. Finally, in~\cite{whyper}, the authors analyze the description of applications presented to user using natural language processing tools. Their goal is to infer if the permissions requested by applications are motivated through the description of the application behavior.

\paragraph{Information Leakage.}

Information leakage attacks target both users' private data and business data stored on smartphones. In~\cite{Bugiel2011}, the authors provide a framework that enables the creation of two separate domains on the Android platform. Their framework protects against inter-domain communication at different levels of the stack (e.g., kernel, middleware). In~\cite{TUD-CS-2011-0127,newxmandroid}, the authors present a security framework that can be implemented on Android to tame confused deputy attacks and application collusion attacks. They introduce a runtime IPC monitor to identify malicious calls that lead to confused deputy attacks. Furthermore, they build a Mandatory Access Control (MAC) framework to detect and prevent covert channels. We will analyze their work in more detail in Chapter~\ref{chap:sp_appcollusion} testing it against our covert channels implementation.

In order to preven information leakage from private data sources (e.g., the address book or the GPS coordinates of a user), the authors of TaintDroid~\cite{taintdroid} implement an information flow tracking mechanism. They identify a number of data sources and sinks and track data transmission across the system from a source to a sink. When the system detects that private data is leaving the system the user can be alerted or the operation can be blocked to prevent the data leak. We will test TaintDroid against our implemented overt and covert channels in Chapter~\ref{chap:sp_appcollusion}. With Edgeminer~\cite{edgeminer}, the authors improve the detection rate of information-flow solutions by implementing a solution that tracks data through implicit control flow transitions. Finally, in PiOS~\cite{pios} the authors provide a static analysis tool to detect data leakage in iOS applications, and found that half of the analyzed applications leaked the unique device identifier. This allows developers to link users' data and usage patterns across multiple applications.

\paragraph{Control Flow.}
In~\cite{mocfi}, the authors augment the work presented in~\cite{pios} to enforce application control flow integrity (CFI) and prevent control flow attacks. The solution works against return oriented programming and attacks that circumvent address-space layout randomization (ASLR) techniques. Their prototype implementation works for iOS applications but is also applicable to other ARM-based devices. In comparison to the extended version of the Google Native Client~\cite{naclusenix}, PiOS does not require access to the source-code of the application.

In Xifer~\cite{xifer}, the authors propose a new ASLR solution that works both for x86 and ARM systems. This solution is effective against code reuse attacks and works by randomizing all code blocks across the available address space and across multiple runs. In particular, the proposed solution remains compatible to code signing, which is used on both iOS and Android platforms.

\section{Summary}

In the last years researchers and industry have analyzed malicious software on mobile platforms. Despite the fact that low infection rates of malware have been found by different studies, the research community has proven how sophisticated attacks are possible on current platforms. We also presented a large number of research proposals aimed at preventing or detecting mobile malware.

In the following chapters we will focus in detail on application phishing and application collusion attacks. Research as well as industry proposed countermeasures do not fully protect users against such attacks. We will evaluate available countermeasures and study in detail personalized security indicators to counter application phishing attacks. We will further analyze in detail overt and covert channels that can be implemented on today's platforms.