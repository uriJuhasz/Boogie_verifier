\chapter{Introduction}
\label{chap:sp_introduction}

In the previous part of this thesis, we have shown how smartphones can be used to enhance the security of many daily operations. In Sound-Proof we showcased a solution for web authentication that is resilient to a remote attacker who cannot compromise his victim's mobile OS. In the scenario of payments at points of sale, our solution can withstand a stronger attacker, who is able to compromise the victim's mobile OS. Although mobile platforms implement many security mechanisms they are still vulnerable to attacks~\cite{lookout2014,xcodeghost,malwaregenome,zhou-jiang12sp,Felt11}.

In this part of the thesis we look at the open challenges that the security community faces when focusing on attacks against modern smartphones. Due to their ubiquity and the abundance of information stored on them, smartphones have become the target of various attacks. In particular, through the usage of infected applications, attackers try to exfiltrate users' information, such as location, contacts and application interactions. Malicious applications are typically shipped to customers as a repackaged version of a benign application. The attacker adds the malicious payload and makes the application available as a direct download or through third-party marketplaces.

The infection rates of malicious applications vary considerably depending on the source reporting them. Researchers found low infection rates (as low as 0.0009\%) in an empirical study based on domain-resolution traces~\cite{lever-ndss13}, and slightly higher ones in a direct study leveraging reports from Android devices (approximately 0.27\%)~\cite{truong13}. In contrast, the Lookout report from 2014~\cite{lookout2014} shows an increase in malware infection rates year over year of 75\%, going from 4\% to 7\%. This report also informs the readers of new malware types ranging from ransomware (malware that locks the user out of his smartphone unless a payment is issued to the attacker, such as ScareMeNot and Koler) to bitcoin mining on infected devices (such as CoinKrypt).

Irrespective of which report more accurately represents the real malware infection rate of smartphones, it is clear that the threat posed by malicious applications is real. We focus our attention on two types of attacks that can exfiltrate users' data. We first analyze the threat posed by application phishing attacks. In these attacks a malicious mobile application masquerades as a legitimate one in order to steal user credentials. Next, we look at how two malicious applications installed on the user's device can communicate covertly to evade the permission-based security architecture of Android.

In Chapter~\ref{chap:sp_phishing}, we provide a summary of application phishing attacks and possible countermeasures. Our analysis shows that personalized security indicators may help users to detect phishing attacks but rely on the user's alertness. Previous studies in the context of website phishing have shown that users tend to ignore personalized security indicators and fall victim to attacks despite indicators deployment. Consequently, the research community has deemed personalized security indicators an ineffective phishing detection mechanism. We revisit the question of personalized security indicator effectiveness and evaluate them in the previously unexplored, and increasingly important context of mobile applications. We designed and developed a mobile banking application that deployed personalized security indicators. We then conducted a user study with 221 participants and found that the deployment of personalized security indicators decreased the phishing attack success rate from 100\% to 50\%. Personalized security indicators can, therefore, help phishing detection in mobile applications and their reputation as an anti-phishing mechanism should be reconsidered.

While an attentive user can detect application phishing attacks and prevent his credentials from being stolen, we then look at application collusion attacks on mobile platforms. In these attacks, two applications that appear harmless when analyzed individually can collude and escalate their privileges. In particular, users are implicitly led to believe that by approving the installation of each application independently, they can limit the damage that an application can cause. 

In Chapter~\ref{chap:sp_appcollusion} we implement and analyze a number of covert and overt communication channels that enable applications to collude and indirectly escalate their permissions. Furthermore, we present and implement a covert channel between an installed application and a web page loaded in the system browser. We measure the throughput of all these channels as well as their bit-error rate and required synchronization for successful data transmission. The measured throughput of covert channels ranges from 3.70 bps to 3.27 kbps on a Nexus One phone and from 0.47 bps to 4.22 kbps on a Samsung Galaxy S phone; such throughputs are sufficient to efficiently exchange users' sensitive information (e.g., GPS coordinates or contacts). We test two popular research tools that track information flow or detect communication channels on mobile platforms, and confirm that even if these tools detect some channels, they still do not detect all of them and fail to fully prevent application collusion. Attacks using covert communication channels remain therefore, a real threat to smartphone security and an open problem for the research community.
