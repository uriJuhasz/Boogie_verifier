\chapter{Summary}
\label{chap:ps_summary}

Two-factor authentication is an effective mechanism to prevent attackers from accessing user's accounts and data or performing fraudulent credit card transactions. Deployed solutions have seen little adoption. Users find it cumbersome to change their behaviors when authenticating to a website or performing a payment at a point of sale. Requiring changes to the deployed systems further slows adoption and limits the potential security benefits of any proposed solution. We exploited the widespread use of smartphones to propose two solutions that enhance the security of online and payments systems without changing the current user interaction models.

We proposed Sound-Proof, a two-factor authentication mechanism for web logins that does not require the user to interact with his phone. Our solution leverages the vicinity of the user's smartphone and the computer from where he is logging in as the second authentication factor. In particular two simultaneous recordings of the surrounding ambient audio are performed on the two devices and compared to test for their proximity. Sound-Proof is deployable today and works with major browsers. The user does not have to interact with his smartphone upon login and we have shown how our system works also if the phone is the user's pocket or purse and in a wide variety of scenarios. In comparison to Google 2-Step Verification, the participants in our user study found Sound-Proof to be more usable. More importantly, they said that they would use Sound-Proof for online services for which two-factor authentication is optional. We see the possibility to foster large-scale adoption of two-factor authentication for the web with a solution that is both usable and deployable today.

Looking at payments at points of sale, we proposed a practical solution that adds the required security to recently proposed location-based two-factor authentication mechanisms. We have identified the necessary requirements for a deployable solution, which includes no changes to the user experience and to the infrastructure that is already in place. We have proposed a new mechanism to secure the readings of the smartphone GPS sensor in order to strengthen the security of our solution. Furthermore, we have proposed two novel enrollment schemes that enable secure enrollment and convenient device migration despite a compromised mobile OS. With this solutions we can protect customers transactions at points of sale even in the presence of a motivated adversary that performs targeted attacks against a particular user. Through prototype implementations on current hardware we have shown that our solution is indeed deployable. We further tested the proposed location-based mechanism in a field study and showed that the location verification operations causes an acceptable delay during a payment transaction. Finally, we presented other use-cases where our solution can be used to enhance their security such as entrance to buildings and ticketing.

Overall, we have presented two-factor authentication techniques that leverage smartphones and enhance the security of some of our daily operations. We focused on logins to websites and payments at points of sale. In both cases, we strived to achieve deployable and usable solutions that could see user adoption, something that is needed for any two-factor authentication mechanism.